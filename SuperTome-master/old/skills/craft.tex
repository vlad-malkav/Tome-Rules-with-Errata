%%%%%%%%%%%%%%%%%%%%%%%%%
\skillentry{Craft}{(Int)}
%%%%%%%%%%%%%%%%%%%%%%%%%

A Craft skill is specifically focused on creating something. If nothing is created by the endeavor, it probably falls under the heading of the \linkskill{Profession} skill.

Like with \linkskill{Perform}, Craft is a skill that has multiple styles. Each rank you put into this skill allows you to get one crafting style, though you don't have to select all of them right away. When you're using craft towards any of the styles that you have selected then you use your full ranks. If you're using craft in a style that you don't have then your ranks don't apply to your check, you are effectively untrained. A crafting style is usually based on the material used. In the case of items that are made of more than one material, usually the majority material will suffice. Metal armor has leather straps, but Craft(Metalworking) lets you make a complete suit of metal armor anyway. You can also just select an item category for items that come in a variety of materials (armor and weapons mostly).

\textbf{Example:} Clara has 3 ranks in Craft, and she's got Alchemy and Woodworking as her styles. When she goes to make a wooden chair or an acid flask she gets +3 from her ranks. When she attempt to make a horseshoe (metalworking) she doesn't get a bonus at all. However, given a little time she could easily learn the metalworking style. If she wanted to learn a fourth style after that then she'd need to put another rank into Craft.

Suggested styles include:
\begin{itemize*}
\item Alchemy
\item Armor
\item Bone
\item Cloth/Fabric
\item Clothing
\item Gemstone/Crystal
\item Jewelry
\item Leather
\item Metalwork
\item Stone/Clay
\item Traps
\item Weapons
\item Woodworking
\end{itemize*}

%%%
\subsubsection{Automatic Effects}
%%%

If you are trained in one or more crafting styles then you can generally work in a town or city as a crafter without needing to make a check. You make a weekly wage according to your craft's average wages (See: Economicon - The Service Economy). If you're the only one in the area with your crafting skills then you might be able to make more money than average, but if you honestly are the only one in the area you probably live in the middle of nowhere and nobody has any money to pay you with anyway. It pretty much averages out.

%%%
\subsubsection{Tools}
%%%

Crafting almost inevitably involves tools of some kind. Even a pit trap calls for a shovel. A check assumes that you have the proper tools and possibly even facilities for your craft. If you have no tools at all then you take -5 to your check and the task probably takes at least twice as long as normal to complete, if not more (GM's discretion). If you at least have improvised tools then you take a -2 to your check, but your task probably doesn't take any longer than normal. If you've got high quality tools and facilities then you get a +2 to your check (but it still takes the normal time).

%%%
\subsubsection{Check}
%%%

Make a craft check whenever you want to make an item. The DC of the check is determined by the complexity of the object.

Some suggested DCs are as follows:
\begin{basictable}{Suggested Craft DCs}{l c}
\textbf{Item} & \textbf{DC}\\
Bludgeoning weapon (club, shotput) & 8\\
Slashing / Piercing weapon (sword, spear) & 10\\
Weapon with moving parts (flail, bow) & +3\\
Shield & 11\\
Great Shield & 13\\
Light armor & 15\\
Medium armor & 17\\
Heavy armor & 20\\
Simple object & 5\\
Typical object & 10\\
Complicated object & 15\\
Very complicated object & 20\\
Masterwork object* & +5\\
Minor Magical Item Effect** (5k gp or less) & 21\\
Moderate Magical Item Effect** (5k to 15k) & 25\\
Major Magical Item Effect** (more than 15k gp) & 29\\
Intelligent item / Construct*** & 15+CR\\
Scroll / Potion & 10+Spell Level\\
\end{basictable}

*An item must be crafted as Masterwork to begin with, it can't be converted to masterwork later. Masterwork items cost +300gp for weapons, +150gp for shields and armor, and x2 to x10 the normal price for other items (depending on the item). The price change of making an item Masterwork doesn't increase the amount of raw materials you need.

**You can normally only add a magical property to a Masterwork item. The crafting of the magical component is a separate process from the crafting of the base item. The style required when crafting the magical component is the same as the base item's. Most magical items require you to have specific spells or other conditions during creation, in addition to just making the craft check.

***For intelligent magical items, the intelligence is generally added after the magic item is created. For stand-alone constructs, you can build the construct's body and animate it with a single check. Either way, the required materials involved are generally going to be Wish Economy materials.

The time it takes to craft an item is also based on the item being crafted. Suggestions are as follows:
\begin{basictable}{Suggested Craft Times}{l c}
\textbf{Item} & \textbf{Time Taken}\\
Bludgeoning weapon (club, shotput) & 1 hour\\
Slashing / Piercing weapon (sword, spear) & 1 day\\
Weapon with moving parts (flail, bow) & +1 day\\
Shield & 2 days\\
Great Shield & 2 days\\
Light armor & 3 days\\
Medium armor & 4 days\\
Heavy armor & 5 days\\
Simple object & 1 hour\\
Typical object & 1 day\\
Complicated object & 2 days\\
Very complicated object & 3 days\\
Masterwork Object & +50\% of base\\
Minor Magical Item Effect (5k gp or less) & 1 day\\
Moderate Magical Item Effect (5k to 15k) & 5 days\\
Major Magical Item Effect (more than 15k gp) & 10 days\\
Intelligent item / Construct & (1/2*CR) days\\
Scroll / Potion & 1 day\\
\end{basictable}

The times given assume that you have the proper raw materials for your task. You need raw materials equal to one half the market price of the item, but how much that actually means in terms of pounds of astral silk or dragon teeth is pretty arbitrary. If the final market price of the item is over 15k gold (thus making it a Wish Economy item) then you must of course use Wish Economy quality raw materials to create it.

For each 1gp of materials that you're missing it generally takes 1 additional day of foraging before you can make your check. This is an obvious abstraction, and assumes that the crafter is traveling, mining, negotiating with distant merchants, whatever it takes. In most cases this abstraction is not useful at all, because a skill check that takes hundreds of days is essentially an entire adventure that should be played out on its own. And you could probably get faster progress by just killing someone, taking their stuff, and then buying the materials. However, for an NPC or montage you may care to have a number without assuming combat and such, particularly if they're ageless like a vampire or demon. So there you go. Now you have a number.

Generally you just make a single craft check no matter the target object, because too much dice rolling is dumb. If you fail your craft check by 4 or less then you make no progress and lose an amount of time equal to the time it takes to create the item (no more than a week though). If you fail your craft check by 5 or more then you make no progress and you also ruin 50\% of your materials in the process.

%%%
\subsubsection{Spell Interactions}
%%%

If you've got the raw materials on hand, then \linkspell{Fabricate} lets you make a Craft check to convert them into a finished product at a highly accelerated rate. You seriously have to still make the Craft check though, regardless of what you're making, or it'll just rip up your materials and stuff without making anything useful. The fabrication process takes 1 round per 1 cubic foot of material being affected, and you can affect up to 10 cubic feet per caster level with a single casting. Fabricate cannot add magical properties to items, it only converts raw materials into mundane finished goods.

If you've got an item made of wood you can cast the \linkspell{Ironwood} spell on it to make it have the hardness, hit points, and not-catching-on-fire-ness of an iron item for the duration of the spell (one day per caster level). If you've just got raw wood, you can make it into an item as part of the casting of Ironwood by making a Craft check (basically just like with Fabricate).

You have to make a Craft check as part of the casting of \linkspell{Minor Creation}, \linkspell{Major Creation}, or \linkspell{True Creation}. A failed check causes the spell to fail and nothing gets created.

%%%
\subsubsection{Repairing An Item}
%%%

You can repair a damaged or broken item with a Craft check.

Repairing a damaged item has a DC 5 less than that required to create the item, and no material cost. This restores all of the item's hit points, and usually takes only 1 minute per missing hit point. A failure deals 1 damage to the item.

Repairing a totally broken item (it has been reduced to 0 hit points) requires a check with the same DC as to create the item. It costs 25\% of the item's base price in new materials, and the time taken is half as long as normal to create the item. A failure wastes your new materials, but you can try again. Repairing a broken magical item does not normally restore the magical portion of the item (though see the Master Artisan feat).

You can also remove any warping, such as by, \linkspell{Warp Wood}. The DC is 10 higher than to create the item normally, but it only takes an hour (no material cost). A failed check deals 1 damage to the item.

If an item has been hit with a \linkspell{Disintegrate} effect then you can try to undo it, unlikely as it may seem. The DC is 50 higher than to create the item, but the attempt only takes an hour (no material cost). If you fail, the dust is ruined and you can't recover anything from it.

The spells \linkspell{Make Whole} and \linkspell{Mending} allow you to effect repairs to items without needing to make Craft check at all, as described in the spells.

%%%
\subsubsection{Item Adjustments}
%%%

Sometimes you need an item to be changed around to suit you. Usually this comes up when you find armor of the improper size, but other situations might come up as well (perhaps you grew wings or a tail suddenly). The DC is normal for the item, but the time taken is half normal, and you don't need to pay for any new materials if the item is going to be the same size as before. If the resulting item is going to be larger than before, you need to pay for the difference in market value (see the "Armor For Unusual Creatures" table in the Equipment section). Even though "nonstandard" armor normally has a higher market value, you don't need to pay for additional materials (the higher value has to do with rarity rather than materials used).

If the resulting item is smaller than before, you have a chance to salvage some of the now-unused materials (see below).

%%%
\subsubsection{Item Salvage}
%%%

If you've got parts and scraps of some sort, you can make a DC 15 check to salvage the good bits and recover some usable materials out of it. You can't Take 10 on a salvage check, and a failed check ruins the stuff that you're trying to salvage.

The amount of time taken is totally arbitrary, depending on what you're trying to salvage and how much, anywhere from minutes to hours to days. If it's a broken apart item, you can usually recover up to half of the item's market value, though often less based on the condition of the scrap (again, totally arbitrary GM's call).

Materials ruined from a failed craft check can't themselves be salvaged (that's the whole point after all).
