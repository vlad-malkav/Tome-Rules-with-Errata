%%%%%%%%%%%%%%%%%%%%%%%%%%%%%%%%%%%%%%%%%%%%%%%%%%
\raceentry{Hobgoblin}
%%%%%%%%%%%%%%%%%%%%%%%%%%%%%%%%%%%%%%%%%%%%%%%%%%
\tagline{"That's some tough talk from a man who wears a basket on his head."}

Hobgoblins are totally awesome at everything they do. They don't have any telling weaknesses, and their strengths are general enough that they excel at \textit{everything} they put their mind to. And like Humans, this can seem like they are overpowered. But the thing is, each character is made separately. While many of the armies of the world are created of a single race, each player character can be something unique and crazy. So for the Hobgoblin \textit{people} there is a substantial advantage to being good at any class. But a player character probably never sees that. A Hobgoblin [anything] is a viable character, but if you want your mouth to taste like velveeta you'd make your Rogue a Deep Halfling, you'd make your Wizard a Gray Elf, and you'd make your Fighter a Dwarf.

But there's more to being a Hobgoblin than being able to ably fill any party role without overpowering the world. You get to have orange or gray skin, sharp teeth, and depending upon which version of Hobgoblin you're using -- either radically more or radically less body hair than a human. So what does that mean? It means that an influential Hobgoblin character in your campaign is going to be played by Robin Williams. But while that means that Hobgoblins \textit{can} be portrayed in a humorous light, chances are that the humor is going to be more like that in \emph{The Big White} or \emph{Death to Smoochy}. These guys have an incredibly baroque system of laws and an interlocking system of fealties that are actually 
a parody of Feudal Japan. 
