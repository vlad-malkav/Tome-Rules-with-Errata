%%%%%%%%%%%%%%%%%%%%%%%%%%%%%%%%%%%%%%%%%%%%%%%%%%
\raceentry{Feytouched}
%%%%%%%%%%%%%%%%%%%%%%%%%%%%%%%%%%%%%%%%%%%%%%%%%%
\tagline{"All my life, I have never fit in. Not in town, not in the forest. In some integral fashion I am unlike those around me, and I believe it is my fate to live and die alone."}

Now if you're one of the people who wonders why a product of fairies and humans, who both conspicuously lack an immunity to mind affecting magic, would have an immunity to mind affecting magic -- you aren't alone. That question comes up about as often as any other with regards to the fey touched. Of course, not all of those born to fey and human stock are immune to mind affecting magic, as you might expect from a group so diverse that some have bug parts and others are simply beautiful humans, while still others look like crazy rock men with teeth sticking out all kinds of places, the powers that a fey-touched is born with are extremely random. The powers of the fairies are more than a little bit chaotic in nature, and no two babes born to these couplings are the same. Unfortunately, these mulish offspring are also interesting both in the general sense and, much more to their detriment, to other fairies in particular. The unmitigated interest of the fey is hard on a small child, so fey touched who are not immune to compulsion effects are going to find themselves at the bottom of a pond or jumping out of a tall tree long before they reach adulthood. Indeed, feytouched immune to compulsion effects are the only ones that ever reach maturity -- the well meaning but deadly interest of the fairy family members simply weeds out any other possible results.
