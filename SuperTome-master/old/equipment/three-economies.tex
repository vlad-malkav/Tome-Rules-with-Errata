%%%%%%%%%%%%%%%%%%%%%%%%%%%%%%%%%%%%%%%%%%%%%%%%%%
\section{The Three Economies}
%%%%%%%%%%%%%%%%%%%%%%%%%%%%%%%%%%%%%%%%%%%%%%%%%%
\tagline{"100 pounds of gold for a house? How does anyone make rent without a wheelbarrow?"}

%%%%%%%%%%%%%%%%%%%%%%%%%
\subsection{The Turnip Economy}
%%%%%%%%%%%%%%%%%%%%%%%%%

turnipz

%%%%%%%%%%%%%%%%%%%%%%%%%
\subsection{The Gold Economy}
%%%%%%%%%%%%%%%%%%%%%%%%%

bitcoinz

%%%
\subsubsection{Trade Goods}
%%%

%%%
\subsubsection{Gems}
%%%

%%%
\subsubsection{Darkwood}
%%%

%%%
\subsubsection{Mithral}
%%%

%%%
\subsubsection{Adamantine}
%%%

%%%%%%%%%%%%%%%%%%%%%%%%%
\subsection{The Wish Economy}
%%%%%%%%%%%%%%%%%%%%%%%%%

Powerful people have access to a spell called \linkspell{Wish}, and it can generate a magical item worth 15,000 gp or less in a split second. It can also generate most other things you care to name worth less than 15,000 gp as well, including all the special materials talked about above. When you use many Wishes in a row you can even build things out of lots of small value items put together, like a life-sized statue of a dragon made of solid adamantine. Even with all this seemingly unlimited wealth, there are things that Wish \textit{can't} generate out of thin air. Specifically, you can't Wish for any magical items with a market price of 15,001 gp or more. These are the things that powerful people care about. Everything else is just chump change to them.

Now, in addition to Wishing for the things you want, you could go out and build the things you want. Except that to build things you need materials, and to build things you can't Wish for, you also need to use materials that you can't Wish for. It's a real pain like that. Items you can't Wish for are "Wish Economy items", and so naturally the materials that you can't wish for are called "Wish Economy materials". Some examples are given here, but it's easy enough to invent your own.

%%%
\subsubsection{Souls}
%%%

The souls of powerful creatures can be trapped in gems, and the soul trade is brisk on the outer planes. Once a soul is in a gem, the gem itself is of little to no value, but the soul goes for 100 gp times the square of the CR of the creature whose soul is trapped.

%%%
\subsubsection{Concentration}
%%%

Ideas take form on the outer planes, and really pernicious or stellar ideas can be so powerful that they take a while to form. In the before-time, they can be found as an amber-like substance that is extremely valued on Mechanus, and by extension every single other outer plane as well. Concentration is actually made out of ideas, and while it looks like a solid object it is actually a liquid that flows so slowly that you could watch it for a year and only a construct could tell you how far the flow had taken it. A pound of concentration goes for 50,000 gp to an interested party.

%%%
\subsubsection{Hope}
%%%

Hope is funny stuff, it has lots of inertia, but those who carry it are not weighed down in the least. It has mass, but not weight. Even the smallest piece of Hope sheds light like a \linkspell{Daylight} spell (effective spell level 7). Hope is measured in kilograms rather than pounds, and a kilo of Hope goes for 100,000 gp to those who want it.

%%%
\subsubsection{Raw Chaos}
%%%

The plane of Limbo is filled with possibility and change. Usually this manifests as a continuous creation and destruction that is awe inspiring and terrifying at the same time. Sometimes, for whatever reason, this possibility doesn't become anything, and just stays as Raw Chaos. Raw Chaos can have any dimensions and any amount of mass, but from a practical standpoint you either have it or you don't. If you have Raw Chaos and someone else doesn't you can give it to them, and it is generally considered good form for them to give you magical items or planar currency worth 200,000 gp in exchange.

%%%%%%%%%%%%%%%%%%%%%%%%%
\subsection{Getting Paid In Favors}
%%%%%%%%%%%%%%%%%%%%%%%%%

foovers

%%%%%%%%%%%%%%%%%%%%%%%%%
\subsection{Wartime Economies Make for Shortages}
%%%%%%%%%%%%%%%%%%%%%%%%%

Many people wonder why a masterwork dagger goes for more than its weight in gold. That's a pretty valid question to ask; certainly I'm not going to attempt to justify the 600 gp price tag on a masterwork walking stick -- that's just an example of simplistic game mechanics run amok. But to an \textit{extent} the crazy prices can be justified by the fact that every settlement on every plane is on a war footing \textit{all the time}. The idea that Peace is somehow a natural state is a fairly recent one, and based on the frequency of wars all over the world -- it's obviously just wishful thinking anyway. War is the default position of every major economy in the world, and that means that weapons have an immediate, and desperate, clientele. Iron is still relatively cheap, because you can't kill people with it \textit{right now}, but actual weapons and armor are crazy expensive.

That doesn't explain the fact that the game charges you over a quarter Oz. of gold just to get a backpack, and it doesn't explain the fact that the markup on masterworking a buckler is the same as the markup on masterworking a breastplate -- that's just a game simplification that makes no real-world sense. Ah, well.

%%%%%%%%%%%%%%%%%%%%%%%%%
\subsection{Bringing the World out of the Dark Ages}
%%%%%%%%%%%%%%%%%%%%%%%%%

It is historical fact that you can take a ridiculous and crumbling imperium with serfs and horse-drawn carts managed by a tyrannical and squabbling aristocracy and boot strap it into being a technologically sophisticated global power that can win the space race and such in a single generation even while being invaded by an evil and genocidal empire. The people at the top don't even need to be nice \textit{or sane}, they just have to understand that economics is an entirely voodoo science, and the limits of production can be broken by thousands of percentage points by getting everyone to buy on credit, work on projects that people looking at the big picture tell them to work on, continuously invest in productive capital, and believe in the future.

Right. That's called Communism, and it ends the dark ages immediately even if it isn't run well. Presumably if it was being run by Paladins who actually \textit{radiate goodness} and Wizards who are inhumanly intelligent and can cast powerful divinations to determine projected needs and goods could be distributed to the masses with teleportals -- it would work substantially better. That sort of thing is not outside the capabilities of your characters. It's not outside the capabilities of the people in the village your characters are saving from gnollish invasion. It's not even technically complicated. But it isn't done.

Partly it isn't done because that's just not the game we're playing. While it is true that you \textit{can} fix the world's ills in a much more tangible fashion by industrializing the production of grain and arranging a non-gold based distribution system such that staple food stuffs are available to all, thereby freeing up potential productive labor for use in blah blah blah\ldots{} the fact is that to a very real degree we play this game because telling stories about slaying evil necromancers and swinging on chandeliers is \textit{awesome}. But the other reason is that the society really isn't ready for a modern or futuristic social setup. No one is going to understand how they are supposed to interact with Socialism, Capitalism, or Fascism. Things are Feudal and people \textit{understand} that. Wealth is exchanged for goods and services on the grounds that people on both sides of the exchange aren't sure that they would win the resulting combat if they tried to take the goods or wealth by force of arms.

Rome had steam engines. Actual difference engines that propelled a metal device with the power of a combustion reaction through the medium of the expansion of heated water. Really. They never built rail roads because slaves were \textit{cheaper than donkeys} and the concept of investing in labor saving devices was preposterous. The idea of having an economy based around trust in the government and labor/wealth equivalences is similarly preposterous. It's not that the idea wouldn't work, it's that every man, woman, and child in society would simply laugh you out of the room if you tried to explain it to them.

%%%%%%%%%%%%%%%%%%%%%%%%%
\subsection{Bad Money Drives Out Good: The Penalties of Paper}
%%%%%%%%%%%%%%%%%%%%%%%%%

People from the modern world are generally pretty perplexed by this idea of handing back and forth actual metal as a medium of exchange. It is an undeniable truth in our lives that the idea of currency is just that: an \textit{idea}. As long as whatever I'm trading for goods and services can be traded for goods and services, it doesn't actually matter if the exchange commodity has any ascribed intrinsic worth. Paper descriptions of value or even ephemeral electronic representations are not only adequate, they're \textit{convenient}. But more than that, using valuable commodities as a medium of exchange inhibits the growth of the economy. As long as a certain portion of the wealth is locked up in currency, the economy is strangled coming and going: not only is there a completely arbitrary limit on how many goods and services can be exchanged (the gold supply), but there is also a limit on the kinds of industry and artistic expression that can occur (in that if you use gold for anything \textit{but} currency you're actually shrinking the money supply and producing negative GDP).

So\ldots{} you're going to solve that by instituting a paper-based exchange system where initially the paper is exchangeable for gold and that eventually gets phased out when the Plebes realize that handing actual gold back and forth is inconvenient and dumb, right? Wrong. Remember that this is the Iron Age, and people haven't invented Nationalism yet. The cornerstone of the Greenback currency is a belief in the nation that prints it -- and nations simply don't exist. You've got empires, and you've got kingdoms, and you've got tribes, and you've got unincorporated villages\ldots{} and that's it as far as civilization goes. When you look at a map and a colored region has a name on it, that's the name of the \textit{region}. Possibly it's even the name of some guy \textit{in} the region. The point is, that it's not a country in the modern sense of the word, so if some new guy walks in who's bad enough the next cartographer will put \textit{his} name on the region instead.

And that means that "The Full Faith and Credit of the Kingdom of Daxall" is worth precisely \textit{nothing}. And while King Daxall can, through force of arms, take all the gold away from all the peasants and get them to trade pieces of paper for goods and services in its place -- no one will actually \textit{believe} that the paper is currency. They're literally trading promises by King Daxall that he'll let them have their money back if they leave town. And since the serfs can't even leave town, even that promise is meaningless to them. A serf accepts paper for goods and services only because he'll be beheaded if he doesn't. The black market value of these pieces of paper is pretty close to zero. Worse, nearby governments will see this as a blatant attempt to sequester all the gold in King Daxall's pants and will probably declare war on him (in addition to the fact that no one outside the reach of King Daxall's pikemen will accept Daxall Dollars).
