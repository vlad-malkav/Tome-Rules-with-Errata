%%%%%%%%%%%%%%%%%%%%%%%%%%%%%%%%%%%%%%%%%%%%%%%%%%
\classentry{Wizard}
%%%%%%%%%%%%%%%%%%%%%%%%%%%%%%%%%%%%%%%%%%%%%%%%%%
\tagline{"And as you can see, when I wiggle my left pinky just like this\ldots{} and now your whole house is on fire. Isn't that fantastic?"}

\textbf{Alignment:} A wizard can be of any alignment. Though the science of magic follows many rules, wizards are just as likely to be fickle as not.

\textbf{Races:} Wizards tend to come from places with the civilization to support wizard colleges, so a large number of Wizards come from Human, Elf, or Dwarven lands. However, members of other races can often simply travel to a wizard college if they want to learn the arts. Particularly, many Gnomes often feel the call to become Illusionists.

\textbf{Starting Gold:} 2d6x10 gp (70 gold)

\textbf{Starting Age:} As Wizard.

\textbf{Hit Die:} d4

\textbf{Class Skills:} The wizard's class skills (and the key ability for each skill) are \linkskill{Concentration} (Con), \linkskill{Craft} (Int), \linkskill{Decipher Script} (Int), \linkskill{Knowledge} (Any) (Int), \linkskill{Profession} (Wis), and \linkskill{Spellcraft} (Int)

\textbf{Skills/Level:} 2 + Intelligence Bonus, plus 2 additional skill points that can only be spent on Knowledge skills.

\poorbab{}
\poorfor{}
\poorref{}
\goodwil{}

\begin{fullcastingclasstable}
\levelone{\multicolumn{1}{p{3cm}}{\raggedright{}Summon Familiar, Scribe Scroll} & 1 & -- & -- & -- & -- & -- & -- & -- & --}
\leveltwo{-- & 2 & -- & -- & -- & -- & -- & -- & -- & --}
\levelthree{-- & 2 & 1 & -- & -- & -- & -- & -- & -- & --}
\levelfour{-- & 3 & 2 & -- & -- & -- & -- & -- & -- & --}
\levelfive{Bonus Feat & 3 & 2 & 1 & -- & -- & -- & -- & -- & --}
\levelsix{-- & 3 & 3 & 2 & -- & -- & -- & -- & -- & --}
\levelseven{-- & 4 & 3 & 2 & 1 & -- & -- & -- & -- & --}
\leveleight{-- & 4 & 3 & 3 & 2 & -- & -- & -- & -- & --}
\levelnine{-- & 4 & 4 & 3 & 2 & 1 & -- & -- & -- & --}
\levelten{Bonus Feat & 4 & 4 & 3 & 3 & 2 & -- & -- & -- & --}
\leveleleven{-- & 4 & 4 & 4 & 3 & 2 & 1 & -- & -- & --}
\leveltwelve{-- & 4 & 4 & 4 & 3 & 3 & 2 & -- & -- & --}
\levelthirteen{-- & 4 & 4 & 4 & 4 & 3 & 2 & 1 & -- & --}
\levelfourteen{-- & 4 & 4 & 4 & 4 & 3 & 3 & 2 & -- & --}
\levelfifteen{Bonus Feat & 4 & 4 & 4 & 4 & 4 & 3 & 2 & 1 & --}
\levelsixteen{-- & 4 & 4 & 4 & 4 & 4 & 3 & 3 & 2 & --}
\levelseventeen{-- & 4 & 4 & 4 & 4 & 4 & 4 & 3 & 2 & 1}
\leveleighteen{-- & 4 & 4 & 4 & 4 & 4 & 4 & 3 & 3 & 2}
\levelnineteen{-- & 4 & 4 & 4 & 4 & 4 & 4 & 4 & 3 & 3}
\leveltwenty{Bonus Feat & 4 & 4 & 4 & 4 & 4 & 4 & 4 & 4 & 4}
\end{fullcastingclasstable}

\classfeatures

\textbf{Weapon and Armor Proficiency:} Wizards are proficient with the club, dagger, heavy crossbow, light crossbow, and quarterstaff, but not with any type of armor or shield. Armor of any type interferes with a wizard's movements, which can cause her spells with somatic components to fail.

\textbf{Spells:} A wizard casts arcane spells which are drawn from the sorcerer/ wizard spell list. A wizard must choose and prepare her spells ahead of time (see below).

To learn, prepare, or cast a spell, the wizard must have an Intelligence score equal to at least 10 + the spell level. The Difficulty Class for a saving throw against a wizard's spell is 10 + the spell level + the wizard's Intelligence modifier.

Like other spellcasters, a wizard can cast only a certain number of spells of each spell level per day. Her base daily spell allotment is given on Table: The Wizard. In addition, she receives bonus spells per day if she has a high Intelligence score.

Unlike a bard or sorcerer, a wizard may know any number of spells. She must choose and prepare her spells ahead of time by getting a good night's sleep and spending 1 hour studying her spellbook. While studying, the wizard decides which spells to prepare.

\textbf{Cantrips:} In addition to their normal allotment of spells per day, a Wizard can prepare a number of 0th level spells, known as "cantrips". A wizard can prepare four cantrips per day, and can cast any prepared cantrip an unlimited number of times.

\textbf{Familiar (Ex):} A Wizard can obtain a Familiar exactly like a Sorcerer can.

\textbf{Scribe Scroll:} At 1st level, a wizard gains Scribe Scroll as a bonus feat.

\textbf{Bonus Feats:} At 5th, 10th, 15th, and 20th level, a wizard gains a bonus feat. At each such opportunity, she can choose a metamagic feat, an item creation feat, or Spell Mastery. If the wizard is 10th level or more, they can convert their familiar into an Improved Familiar instead of selecting a bonus feat.

\textbf{Spellbooks:} A wizard must study her spellbook each day to prepare her spells. She cannot prepare any spell not recorded in her spellbook, except for \linkspell{Read Magic}, which all wizards can prepare from memory.

A wizard begins play with a spellbook containing all 0-level wizard spells (except those from her prohibited school or schools, if any; see School Specialization, below) plus three 1st-level spells of your choice. For each point of Intelligence bonus the wizard has, the spellbook holds one additional 1st-level spell of your choice. At each new wizard level, she gains two new spells of any spell level or levels that she can cast (based on her new wizard level) for her spellbook. At any time, a wizard can also add spells found in other wizards' spellbooks to her own.

%%%%%%%%%%%%%%%%%%%%%%%%%
\subsubsection{School Specialization}
%%%%%%%%%%%%%%%%%%%%%%%%%

A school is one of eight groupings of spells, each defined by a common theme. If desired, a wizard may specialize in one school of magic (see below). Specialization allows a wizard to cast extra spells from her chosen school, but she then never learns to cast spells from some other schools.

A specialist wizard can prepare one additional spell of her specialty school per spell level each day. She also gains a +2 bonus on \linkskill{Spellcraft} checks to learn the spells of her chosen school.

The wizard must choose whether to specialize and, if she does so, choose her specialty at 1st level. At this time, she must also give up two other schools of magic (unless she chooses to specialize in divination; see below), which become her prohibited schools.

A wizard can never give up divination to fulfill this requirement.

Spells of the prohibited school or schools are not available to the wizard, and she can't even cast such spells from scrolls or fire them from wands. She may not change either her specialization or her prohibited schools later.

The eight schools of arcane magic are abjuration, conjuration, divination, enchantment, evocation, illusion, necromancy, and transmutation.

Spells that do not fall into any of these schools are called universal spells.

\textit{Abjuration:} Spells that protect, block, or banish. An abjuration specialist is called an \gameterm{Abjurer}.

\textit{Conjuration:} Spells that bring creatures or materials to the caster. A conjuration specialist is called a \gameterm{Conjurer}.

\textit{Divination:} Spells that reveal information. A divination specialist is called a \gameterm{Diviner}. Unlike the other specialists, a diviner must give up only one other school.

\textit{Enchantment:} Spells that imbue the recipient with some property or grant the caster power over another being. An enchantment specialist is called an \gameterm{Enchanter}.

\textit{Evocation:} Spells that manipulate energy or create something from nothing. An evocation specialist is called an \gameterm{Evoker}.

\textit{Illusion:} Spells that alter perception or create false images. An illusion specialist is called an \gameterm{Illusionist}.

\textit{Necromancy:} Spells that manipulate, create, or destroy life or life force. A necromancy specialist is called a \gameterm{Necromancer}.

\textit{Transmutation:} Spells that transform the recipient physically or change its properties in a more subtle way. A transmutation specialist is called a \gameterm{Transmuter}.

\textit{Universal:} Not a school, but a category for spells that all wizards can learn. A wizard cannot select universal as a specialty school or as a prohibited school. Only a limited number of spells fall into this category.
