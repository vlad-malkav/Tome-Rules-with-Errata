%%%%%%%%%%%%%%%%%%%%%%%%%%%%%%%%%%%%%%%%%%%%%%%%%%
\classentry{Samurai}
%%%%%%%%%%%%%%%%%%%%%%%%%%%%%%%%%%%%%%%%%%%%%%%%%%
\tagline{"My ancestors had words for people who would do such a thing. They were not nice words."}

The war torn countrysides of many a land are held together with tremendous amounts of dedicated work by masters of weaponry and honor. These martial specialists represent the structure of society by their very existence, and beat the living crap out of any who would dare to stand against it.

At least, that's the idea. In truth, there are those who take upon the mantle of the Samurai only to betray its ideals. Those who trample the fabric of the social network they are supposedly sworn to protect. 

\textbf{Alignment:} The Samurai can be of any alignment. All of them must maintain the veneer of honor and civility, though there is no reason that they have to actually \textit{be} Lawful.

\textbf{Races:} The Samurai is primarily found in societies where the law of the land is held in primacy over other things. So while a Samurai himself does not have to be Lawful, Lawful races bear the vast majority of Samurai.

\textbf{Starting Gold:} 4d6x10 gp (140 gold), plus one masterwork weapon

\textbf{Starting Age:} As Fighter.

\textbf{Hit Die:} d8

\textbf{Class Skills:} The Samurai's class skills (and the key ability for each skill) are \linkskill{Balance} (Dex), \linkskill{Climb} (Str), \linkskill{Concentration} (Con), \linkskill{Craft} (Int), \linkskill{Diplomacy} (Cha), \linkskill{Escape Artist} (Dex), \linkskill{Hide} (Dex), \linkskill{Jump} (Str), \linkskill{Knowledge} (any) (Int), \linkskill{Listen} (Wis), \linkskill{Move Silently} (Dex), \linkskill{Perform} (Cha), \linkskill{Profession} (Wis), \linkskill{Sense Motive} (Wis), \linkskill{Swim} (Str), and \linkskill{Tumble} (Dex).

\textbf{Skills/Level:} 4 + Intelligence Bonus

\goodbab{}
\poorfor{}
\poorref{}
\goodwil{}

\begin{classtable}
\levelone{Ancestral Weaponry, Pledge of Loyalty}
\leveltwo{Horde Breaker}
\levelthree{Kiai!}
\levelfour{Whirlwind Attack}
\levelfive{Ancestral Guidance}
\levelsix{Blindfighting, Terrible Blows}
\levelseven{Iaijutsu}
\leveleight{Parry Magic}
\levelnine{Subtle Cut}
\levelten{Blade of Devastation}
\leveleleven{Iaijutsu Focus}
\leveltwelve{Cut Magic}
\levelthirteen{Deny Arcane Defenses}
\levelfourteen{Final Cut}
\levelfifteen{Iaijatsu Master}
\levelsixteen{Reflect Magic}
\levelseventeen{Blade of Souls}
\leveleighteen{Deny Armor}
\levelnineteen{Iaijutsu Grandmaster}
\leveltwenty{Scrolls of Wisdom}
\end{classtable}

\classfeatures

\textbf{Weapon and Armor Proficiency:} Samurai are proficient with all simple and martial weapons, as well as a single Exotic weapon appropriate to the Samurai's tradition. Samurai are proficient with Light and Medium Armor, but not with shields of any kind.

\textbf{Ancestral Weaponry (Ex):} Every culture with a Samurai tradition has a signature weapon that Samurai from that culture use. Whether it is the Spiked Chain of the Hobgoblin Khanate of Khadun or the Katana of the Human Empire of Rokugan, the weapon serves as a symbol of the office and prowess of the Samurai. 

A samurai can only have one weapon designated as his Ancestral Weapon at a time, and this weapon must be a masterwork weapon exalted by the Samurai's warrior culture. He must perform a 24 hour ritual to call his ancestral spirits into the weapon and designate it as his Ancestral Weapon. This ritual costs 100 gp in incense and offerings, and once performed grants the following abilities:

\begin{itemize*}
\item Counts as his Ancestral Weapon for all Samurai abilities.
\item Counts as a Magic weapon (+1/3rd the character's level, rounded up, as an enhancement bonus to attack and damage).
\item Counts as Ghost Touch (no miss chance against incorporeal creatures).
\item The Ancestral weapon has double HP and +10 Hardness. Apply this bonus first, then add in the additional bonus HP and hardness from having the enhancement bonus above.
\end{itemize*}

\textbf{Pledge of Loyalty (Ex):} Samurai pledge their loyalty to a Lord, a figure of temporal power and head of a noble family or clan. To retain this Lord, they must follow this Lord's orders and uphold any Code of Conduct the Lord obeys. As long as a Samurai does these two things, he cannot be forced to act against his Lord or Lord's family by mind-affecting effects. 

Samurai who have broken their vows to their lord are called ronin, while samurai who have never been pledged to a lord or are unwilling to do so are weaponmasters called kensai, sword saints, or simply "master swordsman" or other descriptive title. Regardless of their name, Samurai without a Lord receive a +4 bonus against mind-affecting effects.

\textbf{Horde Breaker:} A Samurai gains Horde Breaker as a bonus feat at 2nd level. If the Samurai already has this feat, he may choose a [Combat] Feat instead, but only if he meets the prerequisites of that feat.

\textbf{Kiai! (Ex)}: At 3rd level, a Samurai may convert a successful strike into a confirmed critical hit. He may use this ability a number of times per day equal to his half his Samurai level +2. This ability is a free action that is declared after the strike is rolled and confirmed as a hit, but before damage is rolled. This ability cannot be used on Attacks of Opportunity.

\textbf{Whirlwind Attack:} A Samurai gains Whirlwind Attack as a bonus feat at 4th level.

\textbf{Ancestral Guidance (Sp):} At 5th level, a Samurai may seek guidance from his ancestors. This counts as a \linkspell{Commune} effect that can be used once a day. A samurai can also seek guidance from other peoples' ancestors if they are available. This works like a \linkspell{Speak With Dead} effect that may be used once per day.

\textbf{Blindfighting:} A Samurai gains Blindfighting as a bonus feat at 6th level. If the Samurai already has this feat, he may choose a [Combat] Feat instead, but only if he meets the prerequisites of that feat.

\textbf{Terrible Blows (Su)}: At 6th level, a Samurai's Ancestral Weapon bypasses Damage Reduction and ignores Hardness.

\textbf{Iaijutsu (Ex):} When a 7th level Samurai has the Edge on an opponent, he may take an attack of opportunity against that opponent as an immediate action at any time. 

\textbf{Parry Magic (Su):} At 8th level, a Samurai may use his Ancestral Weapon to parry magic targeted at him. When the Samurai is targeted by a spell or supernatural ability, he may take an Attacks of Opportunity against the targeted effect. If he can make an attack roll against an AC equal to the spell or effect's DC with this Attack of Opportunity, the effect does not affect him.

\textbf{Subtle Cut} A Samurai gains Subtle Cut as a bonus feat at 9th level. If the Samurai already has this feat, he may choose another [Combat] Feat instead, but only if he meets the prerequisites of that feat.

\textbf{Blade of Devastation (Su):} At 10th level, the Samurai may attack enemies within his reach through objects and walls, his Ancestral Weapon automatically destroying any unattended object or wall in the way with a Hardness less than 20. As a result, enemies do not gain cover bonuses against an attacking Samurai. This effect can also pierce Force effects.

As a standard action, he can also destroy unattended objects of any Hardness with a successful Ancestral weapon attack, or dispel up to a 10' by 10' section of a Force effect.

\textbf{Iaijutsu Focus (Ex):} At 11th level, a Samurai may make up to his per round limit of Attacks of Opportunity against any opponent(s) that he threatens as an immediate action. 

\textbf{Cut Magic (Su):} At 12th level, a Samurai may attack ongoing spell effects by attacking the square they are in for Area of Effect effects or the object or person for targeted effects (which does damage as normal to the object or person). This attack is handled like the Samurai's Parry Magic ability, but it only dispels a 10' by 10' section of an Area of Effect spell or spell-like ability.

\textbf{Deny Caster Defenses (Ex):} At 13th level, a Samurai attacking with his Ancestral Weapon ignores any AC bonuses on his targets that come from spells or spell-like abilities.

\textbf{Final Cut (Ex):} At 14th level, a Samurai's Ancestral Weapon gains the Vorpal Special quality, even if it is not a slashing weapon. 

\textbf{Iaijatsu Master (Ex):} At 15th level, any enemy struck by a Samurai's Attacks of Opportunity must make a Fortitude save against a DC equal to 10 + half the Samurai's HD + the Samurai's Wisdom bonus or be dazed for one round. A successful save against this effect makes the enemy immune to this effect for five rounds. 

\textbf{Reflect Magic (Su):} At 16th level, an spell effect that would be dispelled by a successful use of the Samurai's Parry Magic ability can instead be reflected back on the caster, as per a \linkspell{Spell Turning} effect. 

\textbf{Blade of Souls (Su):} At 17th level, any enemy killed by a Samurai's Ancestral Weapon has his soul sucked into it, and this enemy cannot be raised, resurrected, or otherwise returned to life until the Ancestral Blade is broken. Each time the Ancestral Weapon takes a soul, the Samurai gains a single bonus use of his Kiai! ability that must be used before the next sunrise.

\textbf{Deny Armor (Su):} At 18th level, a Samurai attacking with his Ancestral Weapon ignores any AC bonuses on his targets that come from armor or natural armor.

\textbf{Iaijutsu Grandmaster (Ex):} At 19th level, any Attack of opportunity made by the Samurai is automatically a confirmed critical hit.

\textbf{Scrolls of Wisdom (Ex):} At 20th level, the Samurai has reached the pinnacle of his art, and he may compose a treatise of his collected wisdom. If he spends one month composing this treatise, he may distribute it and spread his wisdom. Anyone who reads this treatise gains a +2 competence bonus to attack characters with Samurai levels, but a -8 penalty to attack the writer of the treatise. In addition, a Samurai who writes his treatise no longer takes ability penalties for aging as long as one copy of the treatise exists. 
