%%%%%%%%%%%%%%%%%%%%%%%%%%%%%%%%%%%%%%%%%%%%%%%%%%
%%%%%%%%%%%%%%%%%%%%%%%%%%%%%%%%%%%%%%%%%%%%%%%%%%
\chapter{Introduction}
%%%%%%%%%%%%%%%%%%%%%%%%%%%%%%%%%%%%%%%%%%%%%%%%%%
%%%%%%%%%%%%%%%%%%%%%%%%%%%%%%%%%%%%%%%%%%%%%%%%%%

%%%%%%%%%%%%%%%%%%%%%%%%%%%%%%%%%%%%%%%%%%%%%%%%%%
\section{What Is This?}
%%%%%%%%%%%%%%%%%%%%%%%%%%%%%%%%%%%%%%%%%%%%%%%%%%
\tagline{To write a story together, everyone must be on the same page.}

Tome is a cooperative storytelling game set in a fantasy world of magic, monsters, knights, and wizards. One or more players each control one character (the Player Characters, or "PCs"), and then the final player is known as Mister Cavern (the MC, sometimes called the GM or DM). The MC is a combination referee, narrator, and roleplayer of last resort for antagonists and minor characters in the story. Players describe their actions, the MC adjudicates the results based on the situation and the game rules and possibly some dice rolls, and everyone goes back and forth like that for as long as folks want to keep going.

Cooperative storytelling can be done without any products at all, as with collaborative writing or just Cops and Robbers. Tome provides structure and conflict resolution in the form of a set of setting and mechanics to determine the results of actions with the help of dice. In this way, players of a Tome game can bypass many of the hangups of both collaborative fiction and Cops and Robbers: most notably the "I shot you/ No you did not" problem. It is hoped that the setting and mechanics will be sufficiently evocative as to give players of protagonists and MCs ample launching points for stories of their own.

Tome is based upon the System Reference Document (SRD). The SRD describes a pretty good game, but it's got some flaws, most notable of which is that "caster" and "non-caster" character types aren't well balanced against each other. However, most monsters and situations in the game are already well balanced against caster types. Tome was started by just two players, and there were far too many monsters and adventures already printed for two players to adjust them all, so instead they just added new and stronger options in the areas that the game was lacking. Over time, others have also added their own Tome options to the pool, and it has become quite a lot to sort through, so this PDF attempts to organize as much as possible as a single reference book. For a full list of credits, check out Appendix E near the end. That all said, Tome is still not entirely balanced or anything like that, but it's closer.

%%%%%%%%%%%%%%%%%%%%%%%%%%%%%%%%%%%%%%%%%%%%%%%%%%
\section{Things Needed To Play}
%%%%%%%%%%%%%%%%%%%%%%%%%%%%%%%%%%%%%%%%%%%%%%%%%%
\tagline{"Assuming flippant things like 'food, water, and shelter' are out of the way."}

You need some way to keep a record of your character, and your record will have to be something you can change a lot. Usually you can use a pencil and paper, but you can also use a spreadsheet or something if you're playing online.

During play, players will be rolling a lot of dice for attacks, damages, and saving throws. Each player should probably have their own dice just so that you don't have to keep passing them around.

Tome is a somewhat tactically oriented game, so you'll also need some space to represent the battlefield during fights, and some tokens to move around each other to show positioning and such.

%%%%%%%%%%%%%%%%%%%%%%%%%%%%%%%%%%%%%%%%%%%%%%%%%%
\section{A Note About Pronouns}
%%%%%%%%%%%%%%%%%%%%%%%%%%%%%%%%%%%%%%%%%%%%%%%%%%

The female pronoun (she, her, hers) is used exclusively throughout these rules. We hope that this won't be taken as an attempt to exclude anyone of other genders from the game though.

%%%%%%%%%%%%%%%%%%%%%%%%%%%%%%%%%%%%%%%%%%%%%%%%%%
\section{The Basics}
%%%%%%%%%%%%%%%%%%%%%%%%%%%%%%%%%%%%%%%%%%%%%%%%%%

%%%%%%%%%%%%%%%%%%%%%%%%%
\subsection{The Core Mechanic}
%%%%%%%%%%%%%%%%%%%%%%%%%
\index{Core Mechanic}
Whenever you attempt an action that has some chance of failure, you roll a twenty-sided die (d20). To determine if your character succeeds at a task you do this:
\begin{itemize*}
\item Roll a d20.
\item Add any relevant modifiers.
\item Compare the result to a target number.
\end{itemize*}
If the result equals or exceeds the target number, your character succeeds. If the result is lower than the target number, you fail.

%%%%%%%%%%%%%%%%%%%%%%%%%
\subsection{Dice}
%%%%%%%%%%%%%%%%%%%%%%%%%
\index{Dice}
Dice rolls are described with expressions such as "3d4+3", which means "roll three four-sided dice and add three" (resulting in a number between 6 and 15). The first number tells you how many dice to roll (adding the results together). The number immediately after the "d" tells you the type of die to use. Any number after that indicates a quantity that is added or subtracted from the result. Sometimes you add several dice together that aren't all the same number of sides, such as "1d8+1d10", but it's the same sort of thing.

\textbf{d\%:} Percentile dice work a little differently. You generate a number between 1 and 100 by rolling two different ten-sided dice. One (designated before you roll) is the tens digit. The other is the ones digit. Two 0s represent 100.

%%%%%%%%%%%%%%%%%%%%%%%%%
\subsection{Rounding Fractions}
%%%%%%%%%%%%%%%%%%%%%%%%%
\index{Rounding}
In general, if you wind up with a fraction, round down, even if the fraction is one-half or larger. The only time you should round up is when the formula in question explicitly says so, in which case you do so even when the fractional part is less than 1 (such as "one third your level (rounded up)" being a result of 1 at 1st level). Certain rolls also have a minimum of 1, even when you round down.

%%%%%%%%%%%%%%%%%%%%%%%%%
\subsection{Multiplying}
%%%%%%%%%%%%%%%%%%%%%%%%%
\index{Multiplying}
Sometimes a rule makes you multiply a number or a die roll. As long as you're applying a single multiplier, multiply the number normally. However, when two or more multipliers apply to any abstract value (such as a modifier or a die roll), combine them into a single multiple, with each extra multiple adding 1 less than its value to the first multiple. Thus, a double (x2) and a double (x2) applied to the same number results in a triple (x3, because 2+1=3). If it helps to make sense of it, keep in mind that each doubling is intended to represent +100\%, and so when combined two "doubles" (2 * +100\%) is a "triple" (+200\%). Similarly, higher multipliers or additional multipliers use the same concept. A x4 and a x3 are a x6 when combined (4+2), and three x2s are a x4 (2+1+1).

When applying multipliers to real-world values (such as weight or distance), normal rules of math apply instead. A creature whose size doubles (thus multiplying its weight by 8) and then is turned to stone (which would multiply its weight by a factor of roughly 3) now weighs about 24 times normal, not 10 times normal. Similarly, a blinded creature attempting to negotiate difficult terrain would count each square as 4 squares (doubling the cost twice, for a total multiplier of x4), rather than as 3 squares (adding 100\% twice).

%%%%%%%%%%%%%%%%%%%%%%%%%%%%%%%%%%%%%%%%%%%%%%%%%%
\section{Ability Scores}
%%%%%%%%%%%%%%%%%%%%%%%%%%%%%%%%%%%%%%%%%%%%%%%%%%
\index{Ability Scores}
Creatures have six ability scores that many game mechanics are influenced by. They are, in order: \indexthis{Strength}, \indexthis{Dexterity}, \indexthis{Constitution}, \indexthis{Intelligence}, \indexthis{Wisdom}, and \indexthis{Charisma}.

\glsdesc{strength}

\glsdesc{dexterity}

\glsdesc{constitution}

\glsdesc{intelligence}

\glsdesc{wisdom}

\glsdesc{charisma}

When an ability score changes, all attributes associated with that score (attack bonuses, AC, hit points, etc) change accordingly. However, a character does not retroactively get additional skill points for previous levels if she increases her intelligence.

%%%%%%%%%%%%%%%%%%%%%%%%%
\subsection{Ability Modifiers}
%%%%%%%%%%%%%%%%%%%%%%%%%
\index{Ability Modifier}
Each ability has a modifier associated with it. Starting out this will usually be in the range of -5 to +5, though ability modifiers above +5 are possible. The table shows ability modifiers up to +13. For ability scores not shown, simply continue the progression (Every 2 full ability points above 10 gives +1 ability modifier).

The modifier is the number you apply to the die roll when your character tries to do something related to that ability. You also use the modifier with some numbers that aren't die rolls. A positive modifier is called a bonus, and a negative modifier is called a penalty.

\begin{table}[htb]
\rowcolors{1}{white}{offyellow}
\caption{Ability Modifiers and Bonus Spells}
\centering
\begin{tabular}{*{11}{c}}
\textbf{Stat} & \textbf{Mod} & \textbf{1st} & \textbf{2nd} & \textbf{3rd} & \textbf{4th} & \textbf{5th} & \textbf{6th} & \textbf{7th} & \textbf{8th} & \textbf{9th}\\
1 & -5 & -- & -- & -- & -- & -- & -- & -- & -- & -- \\
2-3 & -4 & -- & -- & -- & -- & -- & -- & -- & -- & -- \\
4-5 & -3 & -- & -- & -- & -- & -- & -- & -- & -- & -- \\
6-7 & -2 & -- & -- & -- & -- & -- & -- & -- & -- & -- \\
8-9 & -1 & -- & -- & -- & -- & -- & -- & -- & -- & -- \\
10-11 & +0 & 0 & 0 & 0 & 0 & 0 & 0 & 0 & 0 & 0\\
12-13 & +1 & 1 & 0 & 0 & 0 & 0 & 0 & 0 & 0 & 0\\
14-15 & +2 & 1 & 1 & 0 & 0 & 0 & 0 & 0 & 0 & 0\\
16-17 & +3 & 1 & 1 & 1 & 0 & 0 & 0 & 0 & 0 & 0\\
18-19 & +4 & 1 & 1 & 1 & 1 & 0 & 0 & 0 & 0 & 0\\
20-21 & +5 & 2 & 1 & 1 & 1 & 1 & 0 & 0 & 0 & 0\\
22-23 & +6 & 2 & 2 & 1 & 1 & 1 & 1 & 0 & 0 & 0\\
24-25 & +7 & 2 & 2 & 2 & 1 & 1 & 1 & 1 & 0 & 0\\
26-27 & +8 & 2 & 2 & 2 & 2 & 1 & 1 & 1 & 1 & 0\\
28-29 & +9 & 3 & 2 & 2 & 2 & 2 & 1 & 1 & 1 & 1\\
30-31 & +10 & 3 & 3 & 2 & 2 & 2 & 2 & 1 & 1 & 1\\
32-33 & +11 & 3 & 3 & 3 & 2 & 2 & 2 & 2 & 1 & 1\\
34-35 & +12 & 3 & 3 & 3 & 3 & 2 & 2 & 2 & 2 & 1\\
36-37 & +13 & 4 & 3 & 3 & 3 & 3 & 2 & 2 & 2 & 2\\
%38-39 & +14 & 4 & 4 & 3 & 3 & 3 & 3 & 2 & 2 & 2\\
%40-41 & +15 & 4 & 4 & 4 & 3 & 3 & 3 & 3 & 2 & 2\\
%42-43 & +16 & 4 & 4 & 4 & 4 & 3 & 3 & 3 & 3 & 2\\
%44-45 & +17 & 5 & 4 & 4 & 4 & 4 & 3 & 3 & 3 & 3\\
%46-47 & +18 & 5 & 5 & 4 & 4 & 4 & 4 & 3 & 3 & 3\\
%48-49 & +19 & 5 & 5 & 5 & 4 & 4 & 4 & 4 & 3 & 3\\
%50-51 & +20 & 5 & 5 & 5 & 5 & 4 & 4 & 4 & 4 & 3\\
\end{tabular}
\end{table}

%%%%%%%%%%%%%%%%%%%%%%%%%
\subsection{Abilities and Spellcasters}
%%%%%%%%%%%%%%%%%%%%%%%%%
\index{Bonus Spells}
Spellcasters have much of their spellcasting power influenced by their ability scores. The exact scored used varies from class to class, but the following rules apply to all spellcasters regardless of the ability score they cast with.

To cast a spell, a spellcaster must have an ability score of 10 + the level of the spell in their spellcasting stat. A Wizard would need a 14 Intelligence to cast a 4th level spell, for example.

The saving throw against a spell (if any) has a DC of 10 + half the spellcaster's level + the ability modifier of their spellcasting stat.

A spellcaster gets bonus spell slots based on their spellcasting score (as shown on the Ability Modifiers and Bonus Spells table). Since you can always put a lower level spell in a higher level slot, if a caster can't yet cast every level of spell they have bonus slots for (which is very common in the early levels) they can still put lower level spells they do have access to into their higher level bonus spell slots. \textit{Example:} A 1st level Wizard with 16 Int has one bonus spell slot for 1st, 2nd, and 3rd level spells, but can only cast 1st level spells. At the start of the day she can prepare a total of four 1st level spells: one from her class level, and three using all her bonus spell slots.

For ability scores above those shown on the table, simply continue the progression shown for all spell levels. Regardless of ability score, a character never gains bonus 0th level spell slots.

%%%%%%%%%%%%%%%%%%%%%%%%%%%%%%%%%%%%%%%%%%%%%%%%%%
\section{Character Creation}
%%%%%%%%%%%%%%%%%%%%%%%%%%%%%%%%%%%%%%%%%%%%%%%%%%

Character creation isn't too many steps long, but each step can involve a lot of choices to pick from.

First you must select your race and class. It's best to pick both of these at the same time, because most races are better or worse at different classes. Usually you want to pick a race with stat bonuses in the stats that your class uses a lot. Any race \textit{can} work as any class, but if your race and class clash you'll have a harder time of it than if they go well together. It's at this point that you should think about your character's history and personality as it relates to their class, race, and decisions you're going to be making during the rest of character creation. In addition to picking a race and class, you can determine a background and alignment. If you don't pick a background it's not a big deal, you can expand it later on. If you don't care enough about alignment to bother to pick an alignment your alignment is almost for sure just Neutral.

Once you've picked what race and class you want to play as, you need to determine your ability scores. This varies from group to group, and the exact method isn't too important, but it should be fair among the players. Starting stats (before racial modifiers) are assumed to be in the 3 to 18 range, with most stats being in the 8 to 16 range. A favorite method to determine stats is for each player to roll 4d6 and pick the highest three dice, doing this six times to get a number for each stat. Then each player can use the stats of any other player if they want, and they can re-arrange the order of the numbers however they like. Alternately you could roll 1d10+8 for each stat, or have a number of points to assign to each stat according to some system, or everyone uses an array chosen by the MC (eg: 15, 14, 13, 12, 10, 8, arranged to taste, the "elite array"). Again, the exact method is not as important as fairness among the players is important. If there's rolling involved, players should be able to use the rolls of someone else instead of being stuck with bad stat rolls. If there's points, everyone should get the same number of points, and so on.

Once you've picked your race, class (including class options, such as spells or stances known), and stats, you need to assign your skill points. Each class has a list of class skills, and it's suggested (but not required) that you stick to just your class skills when creating your character. At 1st level you get 4 times as many skill points as the normal skill points per level of your class, and the maximum ranks in a skill is your level + 3. At each level above first you get the listed skill points and your maximum goes up by one. This means that you can normally pick a number of skills equal to your skill points per level and have them with maxed out ranks, regardless of your starting level. Having a few skills maxed out is usually better than having a lot of low skills, so this is a perfectly fine way to assign all your skill points quickly.

After skills are selected, you should pick your feats. You get one feat at first level, and one feat at every level divisible by three. Humans get an extra feat, and some classes give extra feats as well.

Then select your equipment. You probably want to have both a ranged weapon and a melee weapon, even if you usually use just one or the other (or none at all, as with many magical classes). You also probably also want some armor, and maybe a shield too. Of course you'll want food, rope, torches, bags to hold it all and to carry off treasure with, things like that. At first level you start with 100gp (a reasonable amount of basic equipment), at 2nd you get about 900gp (mostly masterwork equipment plus some healing potions and horses or whatever), at 3rd it's around 2,700gp (generally one or two magical items, or a magical item and a somewhat expensive armor), at 4th you get probably 5,500gp (an advanced armor and also some magic items), and at 5th you might get 9,000gp (even more bling than 4th level). Beyond that it gets increasingly campaign dependent and it's not really worth listing out.

At this point, you should be ready to adventure.
