\section{Introduction}

The prestige class system is as old as Advanced Dungeons \& Dragons, and it has never really lived up to peoples' high expectations of it. From the original Bard on, there has always been an expectation that getting into a Prestige Class somehow \emph{should be} an ordeal where you get less power now and more power later. Others think that you should get more power now and pay for it by getting less power later on. That's crap.

Gaining levels isn't like purchasing a car. You shouldn't be allowed to save up character power in interest drawing accounts. You shouldn't be allowed to borrow power from the future at heinous interest rates. The fact is that every game of D\&D is a \emph{game}. And that doesn't just mean that each campaign is a game, it means that each individual session of D\&D is a game. And games that aren't fair aren't very fun. In the greater scheme of things it is theoretically possible to arrange a situation where one session might be unfair to one player and another session is unfair to the same player in the \emph{opposite fashion}, but the fact is that in practice this is just very mean to all the players all the time. People feel it when they're being screwed far more than when the chips are stacked with them, so this sort of thing is just highly oppressive to everyone involved.

Worse, while you can create some sort of abstract proof about potential long-term balance in either the ``buy now, pay later" or ``save up for the awesome" models, in actual D\&D games this simply does not work. It can't. While D\&D is inherently open ended, each actual game has a beginning, a middle, and an end. And while it would be convenient if every game began at 1st level and ended at 20th, we know that isn't what really happens. Campaigns begin at later levels (after the pay-offs have kicked in or setups have become obsolete), and they end before Epic (before pay-offs or interest payments kick in). And this is \emph{normal}. Any set up in which a character is supposed to have less power at one part of his career and more in another is unenforceable, there's no possible guarantee that both the low power and the high power period will ever actually happen in-game. In fact, in almost all cases it's a pretty good bet that they \emph{won't}.

A character's level determines what they should be able to do. That's their \emph{character level}, not their Class level. When a character is 7th level they should go 50/50 with a Medusa, a Hill Giant, a Spectre, and a Succubus. We know this, because that's what being a 7th level character \emph{means} according to the CR system. If a character lacks the abilities or the numerics to compete evenly against those monsters, then he's underpowered. If a character has the mad skills to consistently crush that kind of opposition, then he's overpowered. And that's where Prestige Classes can come in to patch things up -- because PrCs have a tendency to be available at about 7th level. So if the party Fighter isn't doing well against monsters of his level (and unless he's a pretty min/maxed build, he probably won't be), feel free to throw in PrCs for that character that are much more powerful. And if the party Druid is smacking those opponents down like a line of shots in a red light bar -- then you should consider cutting him off.

What follows are some examples of prestige classes you can introduce into your game to do what Prestige Classes do well -- give characters flavor abilities that they can be proud of and keep underperforming characters on track with the rest of the party.

\subsection{Fiendish PrCs}

For something that has received so much ink, the world of fiend related prestige classes is remarkably non-functional. Fiendish Cultists, Dark Summoners, and Fiend-blooded Sorcerers are classic D\&D fodder. But unfortunately, the previously published classes for these archetypes are generally� not good. And that makes us sad. Here are some classes designed to fill those perceived holes.


\subsection{Necromatic PrCs}

The degree to which the published classes of Necromancy aren't good causes people physical pain. The degree of malarkey that people are willing to attempt in order to use these classes in a half-way level appropriate way causes us physical pain. While the flavor of published necromantic classes is frequently adequate or even engaging � the mechanics just aren't there. You shouldn't have to cheat just to make a concept character nearly the equal of a standard Cleric or Wizard. As a solution, we propose having actually mechanically viable prestige classes for the necromantically inclined to use:
