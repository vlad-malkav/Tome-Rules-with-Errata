\documentclass[10pt,twoside,onecolumn,openany,final]{memoir}
\setstocksize{11in}{8.5in}

\usepackage[toc,lot,lof]{multitoc}
\usepackage[top=.5in, bottom=.5in, left=.75in, right=.75in]{geometry}
\usepackage{graphicx} \graphicspath{{./images/}}
\usepackage{mdwlist}
\usepackage{microtype} \DisableLigatures{encoding = *, family = *}
\usepackage{multicol}
\usepackage{textcomp}
\usepackage[normalem]{ulem}
\usepackage{wrapfig}
\usepackage{xtab}
\usepackage{enumerate}
\usepackage{phonetic}
\usepackage{bbding}
\usepackage{linearb}

\usepackage{cypriot}

\usepackage{tipa}
\usepackage{xfrac}
\usepackage{appendix}
\usepackage{xparse}
\usepackage{letltxmacro}
\usepackage{makeidx} \makeindex
\usepackage[table,dvipsnames]{xcolor}
\definecolor{offyellow}{RGB}{255,255,128}
\definecolor{links}{RGB}{200,0,50}
\usepackage{placeins}
\usepackage{floatflt}
\usepackage{anyfontsize}
\usepackage{mdframed}
\usepackage{colortbl}
\usepackage{tabularx}
\usepackage{tabu}
\usepackage{longtable}
\usepackage{afterpage}
\usepackage{caption}

%\usepackage[bf, big, raggedright]{titlesec}

\usepackage{amsmath}

%% Font
\usepackage[T1]{fontenc}
\usepackage[bitstream-charter]{mathdesign}
\usepackage{aurical}

\usepackage[colorlinks=true,linkcolor=blue,urlcolor=links,pdfstartview={XYZ null null 1.00},bookmarksdepth=2]{hyperref}


%%%%%%%%%%%%%%%%%%%%%%%%%
%%%% End of Import %%%%%%
%%%%%%%%%%%%%%%%%%%%%%%%%



%%%%%%%%%%%%%%%%%%%%%%%%%%%%%%%%%%%%%%%%%%%%%%%%%%
%%%%%%%%%%%%%%%%%%%%%%%%%%%%%%%%%%%%%%%%%%%%%%%%%%
%%% General Font Display
%%%%%%%%%%%%%%%%%%%%%%%%%%%%%%%%%%%%%%%%%%%%%%%%%%
%%%%%%%%%%%%%%%%%%%%%%%%%%%%%%%%%%%%%%%%%%%%%%%%%%

\renewcommand*{\familydefault}{\sfdefault}
%% sets default text to sans-serif, so text doesn't flip back to serif in some environments.



%%%%%%%%%%%%%%%%%%%%%%%%%%%%%%%%%%%%%%%%%%%%%%%%%%
%%%%%%%%%%%%%%%%%%%%%%%%%%%%%%%%%%%%%%%%%%%%%%%%%%
%%% Sectioning Display
%%%%%%%%%%%%%%%%%%%%%%%%%%%%%%%%%%%%%%%%%%%%%%%%%%
%%%%%%%%%%%%%%%%%%%%%%%%%%%%%%%%%%%%%%%%%%%%%%%%%%






%%%%%%%%%%%%%%%%%%%%%%%%%%%%%%%%%%%%%%%%%%%%%%%%%%
%%%%%%%%%%%%%%%%%%%%%%%%%%%%%%%%%%%%%%%%%%%%%%%%%%
%%% Revised Commands
%%%%%%%%%%%%%%%%%%%%%%%%%%%%%%%%%%%%%%%%%%%%%%%%%%
%%%%%%%%%%%%%%%%%%%%%%%%%%%%%%%%%%%%%%%%%%%%%%%%%%
\makeatletter

%fiddles with how chapter titles are displayed
\renewcommand{\@makechapterhead}[1]{%
\vspace*{0 pt}{%
\raggedright \fontsize{32}{32} \selectfont \bfseries%
\ifnum \value{secnumdepth}>-1%
  \if@mainmatter \vspace{-8pt} {\fontsize{20}{20} \selectfont Chapter \thechapter:}\\[8pt]%
  \fi%
\fi
\hspace{0.65cm} #1\par\nobreak\vspace{20 pt}%
}}

%makes paragraphs show up closer together
\renewcommand{\paragraph}{%
\@startsection{paragraph}{4}%
{\z@}{1.0ex \@plus 1ex \@minus 0.2ex}{-1em} % wtf is an 'ex' anyways?
{\normalfont\normalsize\bfseries}%
}

%lets multicolumn have the proper background colors as defined by rowcolors
\let\oldmc\multicolumn
\newcommand{\mcinherit}{% Activate \multicolumn inheritance
  \renewcommand{\multicolumn}[3]{%
    \oldmc{##1}{##2}{\ifodd\rownum \@oddrowcolor\else\@evenrowcolor\fi ##3}%
  }}

\makeatother

%add labels within sections, subsections, and subsubsections
\LetLtxMacro{\oldsection}{\section}
\renewcommand{\section}[1]{\oldsection{#1}\label{sec:#1}}

\LetLtxMacro{\oldsubsection}{\subsection}
\renewcommand{\subsection}[1]{\oldsubsection{#1}\label{sec:#1}}

\LetLtxMacro{\oldsubsubsection}{\subsubsection}
\renewcommand{\subsubsection}[1]{\oldsubsubsection{#1}\label{sec:#1}}

%only put chapters and sections into the TOC
\setcounter{secnumdepth}{1}

%makes a subsubsection start off indented.
\setlength{\beforesubsubsecskip}{-\beforesubsubsecskip}



%%%%%%%%%%%%%%%%%%%%%%%%%%%%%%%%%%%%%%%%%%%%%%%%%%
%%%%%%%%%%%%%%%%%%%%%%%%%%%%%%%%%%%%%%%%%%%%%%%%%%
%%% Table Formatting
%%%%%%%%%%%%%%%%%%%%%%%%%%%%%%%%%%%%%%%%%%%%%%%%%%
%%%%%%%%%%%%%%%%%%%%%%%%%%%%%%%%%%%%%%%%%%%%%%%%%%
\newcolumntype{L}[1]{>{\raggedright\let\newline\\\arraybackslash\hspace{0pt}}m{#1}} %New type of column 'L' that is ragged-right, behaves like a paragraph, and allows manual definition of width like a 'p' column.
\newcolumntype{C}[1]{>{\centering\let\newline\\\arraybackslash\hspace{0pt}}m{#1}}  %New type of column 'C' that is centered, behaves like a paragraph, and allows manual definition of width like a 'p' column.
\newcolumntype{R}[1]{>{\raggedleft\let\newline\\\arraybackslash\hspace{0pt}}m{#1}}  %New type of column 'R' that is ragged-left, behaves like a paragraph, and allows manual definition of width like a 'p' column.
\newcommand{\header}{\rowcolor{headercolor}}
%when inserted in a row, makes that row the color headercolor

\global\tabulinesep=1mm


%%%%%%%%%%%%%%%%%%%%%%%%%%%%%%%%%%%%%%%%%%%%%%%%%%
%%%%%%%%%%%%%%%%%%%%%%%%%%%%%%%%%%%%%%%%%%%%%%%%%%
%%% List Formatting
%%%%%%%%%%%%%%%%%%%%%%%%%%%%%%%%%%%%%%%%%%%%%%%%%%
%%%%%%%%%%%%%%%%%%%%%%%%%%%%%%%%%%%%%%%%%%%%%%%%%%

\let\olditemize\itemize
\renewcommand{\itemize}{
  \olditemize
  \setlength{\itemsep}{1pt}
  \setlength{\parskip}{0pt}
  \setlength{\parsep}{0pt}
}
%fixes itemize spacing

\let\olddescription\description
\renewcommand{\description}{
  \olddescription
  \setlength{\itemsep}{1pt}
  \setlength{\parskip}{0pt}
  \setlength{\parsep}{0pt}
}
%fixes description spacing

\let\oldenumerate\enumerate
\renewcommand{\enumerate}{
  \oldenumerate
  \setlength{\itemsep}{1pt}
  \setlength{\parskip}{0pt}
  \setlength{\parsep}{0pt}
}
%fixes enumerate spacing

\newcommand{\descability}[2]{\item[#1:] #2}


%%%%%%%%%%%%%%%%%%%%%%%%%%%%%%%%%%%%%%%%%%%%%%%%%%
%%%%%%%%%%%%%%%%%%%%%%%%%%%%%%%%%%%%%%%%%%%%%%%%%%
%%% New Commands
%%%%%%%%%%%%%%%%%%%%%%%%%%%%%%%%%%%%%%%%%%%%%%%%%%
%%%%%%%%%%%%%%%%%%%%%%%%%%%%%%%%%%%%%%%%%%%%%%%%%%


%%%%%%%%%%%%%%%%%%%%%%%%
%%Basic Formatting
%%%%%%%%%%%%%%%%%%%%%%%%

\newcommand{\originallineskip}{\baselineskip}
 %A command that is equal to the original \baselineskip of the doc, in case we change it for a section and want to change it back later

\newcommand{\ability}[2]{\textbf{#1:} #2} 
%The \ability{#1}{#2} command from legacy-source. Should rarely be directly used, changes to this will cascade into other new commands that use its functionality

\newcommand{\shortability}[2]{\noindent\textbf{#1} #2\\}
%A specialized version of the \ability command

\newcommand{\itemspace}{\setlength{\itemsep}{-1mm}\setlength{\topsep}{-1mm} }
%A command from legacy-source for compatabilty

\newenvironment{awesomelist}{\begin{list}{$\bullet$}{\itemspace}}{\end{list}\vspace{8pt}}

\newcommand{\listone}{\begin{list}{$\bullet$}{\itemspace}}

\newcommand{\listtwo}{\begin{list}{$\triangleright$}{\itemspace}}
%A type of list from legacy sorce

\newcommand{\listnum}{\begin{list}{\textbf{\arabic{counter}}:}{\usecounter{counter}}}

\newcommand{\spell}[1]{\emph{\MakeLowercase{#1}}}
%makes spell name lowercase italics.

\setlength{\parindent}{12pt}
%sets the indentation of all paragraphs in the work

\newcommand{\quot}[1]{
	\vspace{-8pt}
	\noindent\emph{#1}\medskip}
%Displays a flavor quote.}

\newcommand{\half}[0]{\ensuremath{\sfrac{1}{2}} }

\newcommand{\third}[0]{\ensuremath{\sfrac{1}{3}} }

\newcommand{\fourth}[0]{\ensuremath{\sfrac{1}{4}} }

\newcommand{\ex}{(Ex)}
\newcommand{\sla}{(Sp)}
\newcommand{\su}{(Su)}

\newcommand{\condition}[1]{\emph{#1}}

%%%%%%%%%%%%%%%%%%%%%%%%
%%Logic
%%%%%%%%%%%%%%%%%%%%%%%%
\newcommand{\testempty}{\empty}
\newcommand{\isempty}{\empty}
%Two commands that can be compared to one another for \ifx logic tests. \isempty should never be changed. If \testempty holds a value of anything but empty, the test should return false.

\newcounter{counter}




%%%%%%%%%%%%%%%%%%%%%%%%
%%Colors
%%%%%%%%%%%%%%%%%%%%%%%%
\colorlet{colorone}{white}
\colorlet{colortwo}{gray!15}
\colorlet{headercolor}{gray!50}
\colorlet{tablecolorone}{gray!40}
\colorlet{tablecolortwo}{gray!20}


%%%%%%%%%%%%%%%%%%%%%%%%
%%Sectioning
%%%%%%%%%%%%%%%%%%%%%%%%
\newcommand{\classentry}[1]{\section{#1} \label{class:#1} \renewcommand{\class}{#1} \index{#1 (class)} \renewcommand{\testempty}{\isempty}}
%\newcommand{\classentry}[1]{\newpage \section{#1} \label{class:#1} \renewcommand{\class}{#1} \index{#1 (class)} \renewcommand{\testempty}{\isempty}}
%Starts a <new page>, creates a section with the name of the class (#1), sets \class to be the name of the class, indexes the class.

\newcommand{\raceentry}[2]{\newpage\renewcommand{\race}{#1}\section{#1} \label{race:#1}\vspace{-1em}\textit{#2}\newline}
%\newcommand{\raceentry}[1]{\oldsection{#1}\index{#1 (race)}\label{race:#1}}

\newcommand{\Requirements}{\oldsubsubsection*{Requirements}}

\newcommand{\Basics}{\oldsubsubsection*{Basics}}

\newcommand{\ClassFeatures}{\oldsubsubsection*{Class Features}}

\newcommand{\skillentry}[2]{\oldsubsection[#1]{#1 #2}\index{#1 (skill)}\label{skill:#1}}





%%%%%%%%%%%%%%%%%%%%%%%%
%%Unsorted Commands
%%%%%%%%%%%%%%%%%%%%%%%%
\newcommand{\tagline}[1]{\vspace{-6pt} \textit{#1} \medskip}

\newcommand{\gameterm}[1]{#1\index{#1}}

\NewDocumentCommand\featentry{m+g}{%
  \IfNoValueTF{#2}
    {\oldsubsubsection[#1]{#1 [General]}\label{feat:#1}}%no second arg, general feat
    {\oldsubsubsection[#1]{#1 [#2]}\label{feat:#1}}%second arg, special type of feat
}

\newcommand{\spellentry}[1]{\oldsubsubsection{#1}\label{spell:#1}}

\NewDocumentCommand\linkrace{m+g}{%
  \IfNoValueTF{#2}
    {\hyperref[race:#1]{#1}}%no second arg, display is same as link
    {\hyperref[race:#1]{#2}}%second arg, link to first and display second
}
\NewDocumentCommand\linkclass{m+g}{%
  \IfNoValueTF{#2}
    {\hyperref[class:#1]{#1}}%no second arg, display is same as link
    {\hyperref[class:#1]{#2}}%second arg, link to first and display second
}
\NewDocumentCommand\linkskill{m+g}{%
  \IfNoValueTF{#2}
    {\hyperref[skill:#1]{#1}}%no second arg, display is same as link
    {\hyperref[skill:#1]{#2}}%second arg, link to first and display second
}
\NewDocumentCommand\linkfeat{m+g}{%
  \IfNoValueTF{#2}
    {\hyperref[feat:#1]{#1}}%no second arg, display is same as link
    {\hyperref[feat:#1]{#2}}%second arg, link to first and display second
}
\NewDocumentCommand\linkspell{m+g}{%
  \IfNoValueTF{#2}
    {\hyperref[spell:#1]{#1}}%no second arg, display is same as link
    {\hyperref[spell:#1]{#2}}%second arg, link to first and display second
}
\NewDocumentCommand\linkcondition{m+g}{%
  \IfNoValueTF{#2}
    {\hyperref[condition:#1]{#1}}%no second arg, display is same as link
    {\hyperref[condition:#1]{#2}}%second arg, link to first and display second
}
\NewDocumentCommand\linkability{m+g}{%
  \IfNoValueTF{#2}
    {\hyperref[ability:#1]{#1}}%no second arg, display is same as link
    {\hyperref[ability:#1]{#2}}%second arg, link to first and display second
}
\NewDocumentCommand\linksec{m+g}{%
  \IfNoValueTF{#2}
    {\hyperref[sec:#1]{#1}}%no second arg, display is same as link
    {\hyperref[sec:#1]{#2}}%second arg, link to first and display second
}

\begin{document}

%%%%%%%%%%%%%%%%%%%%%%%%%%%%%%%%%%%%%%%%%%%%%%%%%%
%%%%%%%%%%%%%%%%%%%%%%%%%%%%%%%%%%%%%%%%%%%%%%%%%%
%%% Title Page
%%%%%%%%%%%%%%%%%%%%%%%%%%%%%%%%%%%%%%%%%%%%%%%%%%
%%%%%%%%%%%%%%%%%%%%%%%%%%%%%%%%%%%%%%%%%%%%%%%%%%
\thispagestyle{empty}
\begin{center}
\textsc{\Large}\\[0.25cm]
\rule{\linewidth}{0.5mm} \\[0.70cm]
\fontsize{30}{30} \selectfont Tome Reference Document\\[.30cm]
\fontsize{16}{18} \selectfont \guillemotleft{} For that game we all known and love \guillemotright{}\\
\rule{\linewidth}{0.5mm} \\[0.6cm]
%\includegraphics[clip,trim=5cm 2cm 9cm 1cm,width=\linewidth]{OldBookArt--MapImages-173.jpg}
\vfill
{\large \textit{This material is Open Game Content, and is licensed for public use under the terms of the Open Game License v1.0a.}\\
\today}
\end{center}

\pagebreak
\sffamily
\pagestyle{plain}
\raggedbottom

%%%%%%%%%%%%%%%%%%%%%%%%%%%%%%%%%%%%%%%%%%%%%%%%%%
%%%%%%%%%%%%%%%%%%%%%%%%%%%%%%%%%%%%%%%%%%%%%%%%%%
%%% Table of Contents
%%%%%%%%%%%%%%%%%%%%%%%%%%%%%%%%%%%%%%%%%%%%%%%%%%
%%%%%%%%%%%%%%%%%%%%%%%%%%%%%%%%%%%%%%%%%%%%%%%%%%
\renewcommand{\contentsname}{Table of Contents}
\setcounter{tocdepth}{1}
\tableofcontents

%%%%%%%%%%%%%%%%%%%%%%%%%%%%%%%%%%%%%%%%%%%%%%%%%%
%%%%%%%%%%%%%%%%%%%%%%%%%%%%%%%%%%%%%%%%%%%%%%%%%%
%%% Main Content %%%
%%%%%%%%%%%%%%%%%%%%%%%%%%%%%%%%%%%%%%%%%%%%%%%%%%
%%%%%%%%%%%%%%%%%%%%%%%%%%%%%%%%%%%%%%%%%%%%%%%%%%

%% Primary Chapters Here

\clearpage

\chapter{Introduction}
\section{What is a Role-playing Game?}
foo
\section{What You Need To Play}
foo
\section{The Core Mechanic}
foo
\section{Creating a Character}
foo
%%%%%%%%%%%%%%%%%%%%%%%%
%%Race Chapter Formatting
%%%%%%%%%%%%%%%%%%%%%%%%
\newcommand{\race}{placeholder}

\newcommand{\racedescription}[1]{\indent\ability{Physical Description}{#1}}
\newcommand{\racepersonality}[1]{\indent\ability{Personality}{#1}}
\newcommand{\racesociety}[1]{\indent\ability{Society}{#1}}
\newcommand{\racealignment}[1]{\indent\ability{Alignment}{#1}}

\newcommand{\type}[1]{\ability{Type}{#1}\\ }
\newcommand{\size}[1]{\ability{Size}{#1}\\ }
\newcommand{\speed}[1]{\ability{Speed}{#1 feet}\\ }
\newcommand{\scores}[1]{\ability{Racial Ability Score Modifiers}{#1}\\ }
\newcommand{\racialtraits}[1]{~\\*\ability{\race ~Special Abilities}{#1}\\ }
\newcommand{\racetrait}[2]{\newline\indent\ability{#1}{#2} }
\newcommand{\senses}[1]{\ability{Senses}{#1}\\ }
\newcommand{\autolanguages}[1]{\ability{Automatic Languages}{#1}\\ }
\newcommand{\bonuslanguages}[1]{\ability{Bonus Languages}{#1}\\ }
\newcommand{\favoredclasses}[1]{\ability{Favored Classes}{#1}\\ }
\newcommand{\male}[4]{Male &#1 &#2 &#3 &#4\\ }
\newcommand{\female}[4]{Female &#1 &#2 &#3 &#4\\ }

\newcommand{\racedatastart}{
\noindent
\begin{minipage}[t]{\linewidth}
\vspace{-.5em}
\begin{multicols}{2}
}

\newcommand{\racedataend}{\
\end{multicols}
\end{minipage}
}

\newenvironment{racetable}
{
\tabulinesep=1mm
\renewcommand\arraystretch{1.4}
\noindent
\begin{tabu} to \linewidth {X}
\header\textbf{\race ~Racial Traits} \\ 
\hline
\end{tabu}
\rowcolors{1}{colortwo}{colorone}
\begin{tabu} to \linewidth {X [1, l]}
}{
\hline
\end{tabu}
}

\newcommand{\agetable}[4]{
\columnbreak
\renewcommand\arraystretch{1.4}
\tabulinesep=1mm
\noindent
\begin{tabu} to \linewidth {X}
\header\textbf{\race ~Starting Age} \\ \hline
\end{tabu}
\rowcolors{1}{colortwo}{colorone}
\begin{tabu} to \linewidth {X X X X}
\textbf{Adulthood:} &\textbf{Simple:} &\textbf{Moderate:} &\textbf{Complex:} \\
#1 Years &#2 &#3 &#4 \\ \hline
\end{tabu}
}

\newenvironment{heightweighttable}
{
\tabulinesep=1mm
\renewcommand\arraystretch{1.4}
\noindent
\begin{tabu} to \linewidth {X}
\header\textbf{\race ~Height and Weight} \\ \hline
\end{tabu}
\vspace{-1pt}
\rowcolors{1}{colortwo}{colorone}
\begin{tabu} to \linewidth {X X X X X}
\textbf{Gender} &\textbf{Base Height} &\textbf{Height Mod.} &\textbf{Base Weight} &\textbf{Weight Mod.} \\
}{
\hline
\end{tabu}
}

\newenvironment{raceleft}
{
\vspace{0pt}
\begin{minipage}[t]{0.5\linewidth}
}{
\end{minipage}
}

\newenvironment{raceright}
{\begin{minipage}[t]{0.5\linewidth}
\vspace{0pt}
}{
\end{minipage}
}

\newenvironment{racebox}
{
\vspace{0pt}
%\nointerlineskip
\begin{minipage}{\textwidth}
}{
\end{minipage}
}


\chapter{Races}

%\begin{racebox}
\raceentry{Aasimar}{``My ancestors were more beautiful than you can imagine."}

Aasimar are humans that have a beautiful outsider, usually but not always a celestial, somewhere in their ancestry.

\racedescription{Aasimar look like especially beautiful humans, though they sometimes bear vestiges of their ancestry that denote them as being different (strangely colored eyes, silver-blonder or white hair, slightly `off' facial features).}

\racepersonality{Though mostly human, an aasimar's immortal heritage influences their mental development. Aasimar tend toward more extreme personalities, being especially quiet and introspective or particularly loud and boisterous. Most aasimar are very opinionated, and have strongly held beliefs.}

\racesociety{Aasimar are typically born and raised in human societies, and gain the same customs of that culture}

\racealignment{Most aasimar are the descendants of celestials, and tend towards the good alignments. Rarely, an aasimar might instead have an infernal heritage, being the descendant of an erinyes or succubus. Such aasimar instead tend towards an evil alignment.}

\racedatastart
\begin{racetable}
\type{Outsider (Native and Human Subtype)}
\size{Medium}
\scores{+2 Wisdom, +2 Charisma}
\speed{30}
\senses{Standard}
\autolanguages{Common}
\bonuslanguages{Abyssal, Aquan, Auran, Celestial, Formian, Ignan, Slaad, Sylvan, Terran.}
\favoredclasses{Paladin and Sorcerer}
\end{racetable}

\vspace{\baselineskip}
\agetable{20}{+1d6}{+2d6}{+3d6}

\vspace{\baselineskip}
\begin{heightweighttable}
\male{4' 7"}{+2d8}{90 lb.}{x(2d4)}
\female{4' 5"}{+2d8}{80 lb.}{x(2d4)}
\end{heightweighttable}
\racedataend

\racialtraits{
\racetrait{Inner Light \sla}{An Aasimar with a Charisma of at least 10 may cast \spell{light} once per day, with a caster level equal to their character level.}
\racetrait{Keen Senses}{+2 bonus to Spot, and Listen checks.}
}
%\end{racebox}
%%%%%%%%%%%%%%%%%%%%%%%%%%%%%%%%%%%%%%%%%%%%%%%%%%
\raceentry{Drow}
%%%%%%%%%%%%%%%%%%%%%%%%%%%%%%%%%%%%%%%%%%%%%%%%%%
\tagline{"Time to die for the Spider Queen."}

The Drow are perhaps the most overused bunch of villains ever. Their entire ability set is one that is supposed to neutralize the advantages of player characters so that characters can have mirror matches against NPC parties without doubling their treasure. With magic items that \textit{turn off} once they are brought out of Drow controlled regions, spell-resistance, and spell-like abilities designed to specifically negate common player-character tactical advantages, they can easily compete with Player Characters with massively more permanent magical equipment. And that means that they can be fought and killed several times without supercharging party treasure.

But if you want to \textit{play} a Drow character, you don't want any of that crap. In fact, if you want a Drow character, probably the maxim you are looking for is "WWDD?" and the answer is probably "Fight with two scimitars." But more than that, there are a number of abilities that Drow characters in stories exhibit that people want. And then there are the game mechanical abilities in the rulebook that the characters in stories obviously don't have (like \textit{Touch of Fatigue}, what's up with that?) So here it is, the LA +0 Drow that people actually want to play:


%%%%%%%%%%%%%%%%%%%%%%%%%%%%%%%%%%%%%%%%%%%%%%%%%%
\raceentry{Dwarf}
%%%%%%%%%%%%%%%%%%%%%%%%%%%%%%%%%%%%%%%%%%%%%%%%%%

foo?

%%%%%%%%%%%%%%%%%%%%%%%%%
\subsection{Mechanics}
%%%%%%%%%%%%%%%%%%%%%%%%%

\begin{itemize*}
\item \racesize{Medium}
\item \racetype{Humanoid (Dwarf)}
\item \racemovement{20ft}
\item \racevision{Darkvision 60ft}
\item +2 Constitution, -2 Charisma.
\item \textbf{Slow and Steady (Ex)} A dwarf's movement speed is only 20ft, but it is not further reduced by armor or encumbrance.
\item \textbf{Stonecunning (Ex):} This ability grants a dwarf a +2 racial bonus on \linkskill{Search} checks to notice unusual stonework, such as sliding walls, stonework traps, new construction (even when built to match the old), unsafe stone surfaces, shaky stone ceilings, and the like. Something that isn't stone but that is disguised as stone also counts as unusual stonework. A dwarf who merely comes within 10 feet of unusual stonework can make a Search check as if he were actively searching, and a dwarf counts as having Trapfinding when dealing with stonework. A dwarf can also intuit depth, sensing his approximate depth underground as naturally as a human can sense which way is up.
\item \textbf{Earth Stability (Ex):} A dwarf gains a +4 bonus on ability checks made to resist being bull rushed or tripped when standing on the ground (but not when climbing, flying, riding, or otherwise not already standing firmly on the ground).
\item Weapon Familiarity: Dwarves may treat dwarven waraxes and dwarven urgroshes as martial weapons, rather than exotic weapons.
\item +2 racial bonus on saving throws against poison, spells, and spell-like effects.
\item +1 racial bonus on attack rolls against orcs and goblinoids.
\item +4 dodge bonus to Armor Class against Giants.
\item +2 racial bonus on \linkskill{Appraise} checks that are related to stone or metal items.
\item +2 racial bonus on \linkskill{Craft} checks that are related to stone or metal.
\item \racefavoredclass{Any class with full BAB}
\item \racelang{Common and Dwarven}
\item \raceextralang{Giant, Gnome, Goblin, Orc, Terran, and Undercommon}
\end{itemize*}

%%%%%%%%%%%%%%%%%%%%%%%%%%%%%%%%%%%%%%%%%%%%%%%%%%
\raceentry{Elf}
%%%%%%%%%%%%%%%%%%%%%%%%%%%%%%%%%%%%%%%%%%%%%%%%%%
\raceentry{Elf}{``You shall never harm my beautiful trees!"}


\begin{itemize}
\item \racesize{Medium}
\item \racetype{Humanoid (Elf)}
\item \racemovement{30ft}
\item \racevision{Low-light vision}
\item -2 Constitution and +2 to any other stat. The other stat that gets a bonus is different based on the elf's subtype, but there is a subtype for every stat and different worlds have different names for each subtype.
\item Immunity to Sleep effects, and a +2 racial saving throw bonus against other Enchantment spells or effects. Elves do not sleep, but they must enter a trance-like state for 4 hours each day to maintain their wellbeing. While in a trance an elf does not take a penalty to \linkskill{Listen} checks like a sleeping character does.
\item Weapon Proficiency: Elves receive the Martial Weapon Proficiency feats for the longsword, rapier, longbow (including composite longbow), and shortbow (including composite shortbow) as bonus feats.
\item +2 racial bonus on \linkskill{Listen}, \linkskill{Search}, and \linkskill{Spot} checks. An elf who merely passes within 5 feet of a secret or concealed door is entitled to a Search check to notice it as if she were actively looking for it.
\item \racefavoredclass{Any class with Arcane spellcasting, or that has longsword proficiency.}
\item \racelang{Common and Elven}
\item \raceextralang{Draconic, Gnoll, Gnome, Goblin, Orc, and Sylvan}
\end{itemize}

%%%%%%%%%%%%%%%%%%%%%%%%%%%%%%%%%%%%%%%%%%%%%%%%%%
\raceentry{Feytouched}
%%%%%%%%%%%%%%%%%%%%%%%%%%%%%%%%%%%%%%%%%%%%%%%%%%

foo?

%%%%%%%%%%%%%%%%%%%%%%%%%
\subsection{Mechanics}
%%%%%%%%%%%%%%%%%%%%%%%%%

\begin{itemize*}
\item \racesize{Medium}
\item \racetype{Fey}
\item \racemovement{30ft}
\item \racevision{Low-light vision}
\item +2 Dexterity, +2 Charisma, -2 Constitution. Feytouched are graceful and those which are not beautiful are terrifying, but they are fragile like flowers.
\item Immunity to [Compulsion] Effects.
\item \textbf{Magic Affinity (Sp):} Every Feytouched is different, and marked by the signature magics of the fey in a different manner. Every Feytouched has one spell that can be used once per day as a spell-like ability. This spell is chosen at 1st level and cannot be changed. Any 1st level Illusion or Enchantment spell from the Sorcerer/Wizard list is fair game, and the save DC is Charisma-based.
\item \racefavoredclass{Any class with 3/4ths BAB}
\item \racelang{Common and Sylvan}
\item \raceextralang{Aquan, Auran, Elvish, Draconic, Dwarvish, Druidic, Goblin, Gnoll, Gnome, Halfling}
\end{itemize*}

%%%%%%%%%%%%%%%%%%%%%%%%%%%%%%%%%%%%%%%%%%%%%%%%%%
\raceentry{Gnome}
%%%%%%%%%%%%%%%%%%%%%%%%%%%%%%%%%%%%%%%%%%%%%%%%%%

foo?

%%%%%%%%%%%%%%%%%%%%%%%%%
\subsection{Mechanics}
%%%%%%%%%%%%%%%%%%%%%%%%%

\begin{itemize*}
\item \racesize{Small}
\item \racetype{Humanoid (Gnome)}
\item \racemovement{20ft}
\item \racevision{Low-light}
\item +2 Constitution, -2 Strength.
\item Weapon Familiarity: Gnomes may treat gnome hooked hammers as martial weapons rather than exotic weapons.
\item +2 racial bonus on saving throws against Illusions.
\item Add +1 to the Difficulty Class for all saving throws against illusion spells cast by gnomes. This adjustment stacks with those from similar effects.
\item +1 racial bonus on attack rolls against kobolds and goblinoids.
\item +4 dodge bonus to Armor Class against monsters of the giant type.
\item +2 racial bonus on \linkskill{Listen} checks.
\item +2 racial bonus on \linkskill{Craft} (alchemy) checks.
\item Spell-Like Abilities: At-will -- \linkspell{Speak With Animals} (burrowing mammals only). A gnome with a Charisma score of at least 10 also has the following spell-like abilities: 1/day -- \linkspell{Dancing Lights}, \linkspell{Ghost Sound}, \linkspell{Prestidigitation}. The save DC is Charisma based.
\item \racefavoredclass{Any Arcane spellcasting class that can cast Illusion spells}
\item \racelang{Common and Gnome}
\item \raceextralang{Draconic, Dwarven, Elven, Giant, Goblin, and Orc}
\end{itemize*}

%%%%%%%%%%%%%%%%%%%%%%%%%%%%%%%%%%%%%%%%%%%%%%%%%%
\raceentry{Goblin}
%%%%%%%%%%%%%%%%%%%%%%%%%%%%%%%%%%%%%%%%%%%%%%%%%%
\vspace*{-8pt}
\quot{``You weren't hired to think. You were hired because you have opposable thumbs."}

Goblins are the weakest and smallest of the Goblinoid races, and that means that in society in general they get a really crap deal. But that's not really important for a Player Character, because player characters get access to classes like Rogue, Knight, and Wizard for whom being small is not a huge problem. Indeed, Goblins have a number of saving graces that in the wild barely keep them alive that when used by a player character can make them very effective. Naturally adept at stealth, Goblins are virtually made to be a Rogue or Wizard, and indeed most Goblins who have class levels are one or the other.

But the Goblins are also extremely gifted mounted combatants. And why is that? Because they are the smallest and weakest of the Goblinoids, the Worgs long ago enslaved the Goblin people. That's right, the Worgs came in and imposed their dominion upon Goblins, not the other way around. But time does funny things\ldots\ Worgs are pretty stupid, and they don't have thumbs. So while they are individually powerful, eventually they were forced to have the Goblins do all the important stuff -- like keep records and make decisions.

So now, the Worgs have gone many generations doing pretty much whatever it is that their ``servants" tell them to do. Which means that really the Goblins are totally in control. And because of this, Goblin children are practically born into the saddle. Those rich enough to afford a wolf to ride (like well, player characters) can be devastatingly effective lancers.

%%%%%%%%%%%%%%%%%%%%%%%%%
\subsection{Mechanics}
%%%%%%%%%%%%%%%%%%%%%%%%%

\begin{itemize}
\item \racesize{Small}
\item \racetype{Humanoid (Goblin)}
\item \racemovement{30ft}
\item \racevision{Darkvision 60ft}
\item +2 Dexterity, -2 Strength, -2 Charisma
\item +4 bonus to \linkskill{Move Silently} and \linkskill{Ride} checks.
\item Bonus Feat: \linkfeat{Mounted Combat}.
\item Goblins benefit from an ancient pact with the Worgs, and every Goblin receives a +2 bonus to any \linkskill{Bluff}, \linkskill{Diplomacy}, \linkskill{Handle Animal}, \linkskill{Sense Motive}, or \linkskill{Survival} check made with respect to a Worg.
\item \racefavoredclass{Any class with Move Silently or Ride as a class skill}
\item \racelang{Common, Goblin}
\item \raceextralang{Draconic, Elvish, Dwarvish, Giant, Gnoll, Infernal, Orcish, and Undercommon}
\end{itemize}

\raceentry{Half-Elf}{``I don't fit in anywhere, please, listen to me cry.''}

\listone
		\item Medium Size
		\item 30' Movement
		\item Humanoid Type
		\item Low-Light Vision: Half-Elves can see twice as humans in poor lighting.
		\item Immunity to sleep spells and similar magical effects, and a +2 racial bonus on saving throws against enchantment spells or effects.
		\item +1 racial bonus on Listen, Search, and Spot checks.
		\item +2 racial bonus on Diplomacy and Gather Information checks.
		\item Elven Blood: For all effects related to race, a half-elf is considered an elf.
		\item Favored Class: Any
		\item Automatic Languages: Common and Elven.
		\item Bonus Languages: Any (other than secret languages, such as Druidic).
\end{list}
%%%%%%%%%%%%%%%%%%%%%%%%%%%%%%%%%%%%%%%%%%%%%%%%%%
\raceentry{Halfling}
%%%%%%%%%%%%%%%%%%%%%%%%%%%%%%%%%%%%%%%%%%%%%%%%%%

foo?

%%%%%%%%%%%%%%%%%%%%%%%%%
\subsection{Mechanics}
%%%%%%%%%%%%%%%%%%%%%%%%%

\begin{itemize*}
\item \racesize{Small}
\item \racetype{Humanoid (Halfling)}
\item \racemovement{20ft}
\item \racevision{Standard}
\item +2 racial bonus on \linkskill{Climb}, \linkskill{Jump}, and \linkskill{Move Silently} checks.
\item +1 racial bonus on all saving throws.
\item +2 morale bonus on saving throws against Fear effects.
\item +1 racial bonus on attack rolls with thrown weapons and slings.
\item +2 racial bonus on \linkskill{Listen} checks.
\item \racefavoredclass{Any class with Hide as a class skill}
\item \racelang{Common and Halfling}
\item \raceextralang{Dwarven, Elven, Gnome, Goblin, and Orc}
\end{itemize*}

%%%%%%%%%%%%%%%%%%%%%%%%%%%%%%%%%%%%%%%%%%%%%%%%%%
\raceentry{Half-Orc}
%%%%%%%%%%%%%%%%%%%%%%%%%%%%%%%%%%%%%%%%%%%%%%%%%%
\vspace*{-8pt}
\quot{``I don't fit in anywhere, but you may be surprised to know that this dagger fits all kinds of places."}

Ah, the Half-Orc. Has any race ever gotten quite as dusty a drumstick as they? The reason that we have half-orcs at all is because they were around in Tolkien. But they didn't really do much in those books, they were just easily deluded villains who were borderline racist stereotypes and made us want to forget them altogether. But time moves on, and where once the Half-Orcs were debased and pathetic pawns of The Dark One, now we have them as a legitimate playable race. And yet, their game mechanics have never really been compatible with that.

Here's what they're supposed to be: Half-Orcs have the smarts of a human and the strength of an Orc. If people didn't hate them so much, they'd rule everything. But people do hate them so much. And here's why: Human women are, compared to Orcs, weak; Orcish women are, compared to Humans, gullible. Making Half-Orcs is easy, and since the modern Orc looks like an Orc from World of Warcraft more than a pig-man, perfectly understandable.

With all the wars that Orcs and Humans have, even periods of relative peace are rarely considered periods of friendship. So any time a Half-Orc happens, both races tend to consider it an abomination. It doesn't matter that a Half-Orc is a better leader than any of the other Orcs. It doesn't matter that the Half-Orc is tougher than any of the other Humans -- he's hated for his talents. And that makes him perversely really good at finding out things he wants to know from people. He's dealt with prejudice all his life, and knows pretty much everything you'd want to know about working around it.

\begin{itemize}
\item \racesize{Medium}
\item \racetype{Humanoid (Human, Orc)}
\item \racemovement{30ft}
\item \racevision{Darkvision 60ft}
\item +2 Strength
\item +2 to \linkskill{Intimidate}, \linkskill{Gather Information}, and \linkskill{Survival}.
\item \racefavoredclass{Any class with full BAB}
\item \racelang{Common, Orc}
\item \raceextralang{Any}
\end{itemize}

\raceentry{Hobgoblin}
\quot{``That's some tough talk from a man who wears a basket on his head."}

\listone
    \item Medium Size
    \item 30' movement.
    \item Humanoid Type (Goblinoid subtype)
    \item Darkvision: Hobgoblins can see up to 60 feet in the dark.
    \item +2 Dexterity, +2 Constitution
    \item +4 bonus to Move Silently checks.
    \item Favored Classes: Fighter and Samurai
    \item Automatic Languages: Common, Goblin
    \item Bonus Languages: Draconic, Elvish, Dwarvish, Giant, Gnoll, Ignan, Infernal, Orcish.
\end{list}
\raceentry{Human}
\quot{``Yeah, I'm pretty normal.''}

\listone
	\item Medium Size
	\item 30' movement.
	\item Humanoid Type (Human subtype)
	\item 1 extra feat at 1st level.
	\item 4 extra skill points at 1st level and 1 extra skill point at each additional level.
	\item Favored Class: Any. When determining whether a multiclass human takes an experience point penalty, his or her highest-level class does not count.
	\item Automatic Language: Common. 
	\item Bonus Languages: Any (other than secret languages, such as Druidic). See the Speak Language skill.
\end{list}
%%%%%%%%%%%%%%%%%%%%%%%%%%%%%%%%%%%%%%%%%%%%%%%%%%
\raceentry{Kobold}
%%%%%%%%%%%%%%%%%%%%%%%%%%%%%%%%%%%%%%%%%%%%%%%%%%

foo?

%%%%%%%%%%%%%%%%%%%%%%%%%
\subsection{Mechanics}
%%%%%%%%%%%%%%%%%%%%%%%%%

\begin{itemize*}
\item \racesize{Small}
\item \racetype{Humanoid (Kobold)}
\item \racemovement{30ft}
\item \racevision{Darkvision 60ft}
\item -4 Strength, +2 Dexterity, -2 Constitution
\item \textbf{Light Sensitivity (Ex):} Kobolds are \linkcondition{Dazzled} in bright sunlight or within the radius of a \linkspell{Daylight} spell.
\item +1 Natural Armor bonus.
\item +2 racial bonus on \linkskill{Craft} (Trapmaking), \linkskill{Profession} (Miner), and \linkskill{Search} checks.
\item \racefavoredclass{Any class with 3/4ths BAB, as well as the Dragon bloodline Sorcerer}
\item \racelang{Common, Draconic}
\item \raceextralang{Dwarven, Formian, Gnome, Goblin, Infernal, Orc, Terran, and Undercommon}
\end{itemize*}

%%%%%%%%%%%%%%%%%%%%%%%%%%%%%%%%%%%%%%%%%%%%%%%%%%
\raceentry{Orc}
%%%%%%%%%%%%%%%%%%%%%%%%%%%%%%%%%%%%%%%%%%%%%%%%%%

foo?

%%%%%%%%%%%%%%%%%%%%%%%%%
\subsection{Mechanics}
%%%%%%%%%%%%%%%%%%%%%%%%%

\begin{itemize*}
\item \racesize{Medium}
\item \racetype{Humanoid (Orc)}
\item \racemovement{30ft}
\item \racevision{Darkvision 60ft}
\item +4 Strength, -2 Intelligence, -2 Charisma, -2 Wisdom
\item \textbf{Daylight Sensitivity (Ex):} Orcs are \linkcondition{Dazzled} in bright sunlight or within the radius of a \linkspell{Daylight} spell.
\item +2 racial bonus to saving throws vs. Poison and Disease.
\item Immunity to ingested poisons of all kinds. Orcs can also eat moldy and rotten items without fear, though it still tastes as bad as you'd imagine.
\item +2 to \linkskill{Jump} and \linkskill{Survival} checks.
\item \racefavoredclass{Any class with full BAB}
\item \racelang{Orc, Common}
\item \raceextralang{Dwarvish, Elvish, Giant, Gnoll, Goblin, Sylvan, Undercommon}
\end{itemize*}


%%%%%%%%%%%%%%%%%%%%%%%%%%%%%%%%%%%%%%%%%%%%%%%%%%
\raceentry{Tiefling}
%%%%%%%%%%%%%%%%%%%%%%%%%%%%%%%%%%%%%%%%%%%%%%%%%%

Tieflings are the most popular of the bad touched races, and for good reason. They are \textit{awesome}. Not mechanically, they're kind of unimpressive. But they have pizzazz as characters. They have fiendish ancestry, and that makes them great villains and great tortured heroes. What it doesn't make them is particularly \textit{powerful}. Tieflings aren't actually that great. \textit{Darkness} appears on some class lists as a cantrip, and that's not an accident. Fundamentally, \textit{darkness} just isn't a good effect. 

Tieflings are honestly somewhat less powerful than Aasimar are (having as they do, some reasonably annoying penalties), but they are descended from hideous monsters from all over the planes, and they are generally speaking more fun to play.

%%%%%%%%%%%%%%%%%%%%%%%%
%%Class Chapter Formatting
%%%%%%%%%%%%%%%%%%%%%%%%

\newcommand{\class}{placeholder}
%Holds the class's name, as defined by \classentry

\newenvironment{classpreamble}{
%\centering
%\rowcolors{1}{colorone}{colortwo}
%\begin{tabu} to \textwidth {X}
}{
%\end{tabu}
}

\newcommand{\desc}[1]{\noindent#1}

\newcommand{\playingaclass}[1]{\indent\ability{Playing a \class}{#1}}

\newcommand{\alignment}[1]{\indent\ability{Alignment}{#1}}

\newcommand{\races}[1]{\indent\ability{Races}{#1}}

\newcommand{\startinggold}[1]{\indent\ability{Starting Gold}{#1}}

\newcommand{\startingage}[1]{\indent\ability{Starting Age}{#1}}

\newcommand{\hitdie}[1]{\indent\ability{Hit Die}{#1}}

\newcommand{\classskills}[1]{\indent\ability{Class Skills}{The {\class}'s class skills (and the key ability for each skill) are #1}}

\newcommand{\skillpoints}[1]{\indent\ability{Skill Points per Level}{#1 + Intelligence Bonus}}


%\newcommand{\desc}[1]{ #1 \\}
%\newcommand{\playingaclass}[1]{\selectfont\ability{Playing a \class : }{#1}\\}
%\newcommand{\hitdie}[1]{\ability{Hit Die: }{#1}\\}
%\newcommand{\alignment}[1]{\ability{Alignment: }{#1}\\}
%\newcommand{\races}[1]{\ability{Races: }{#1}\\}
%\newcommand{\startinggold}[1]{\ability{Starting Gold: }{#1}\\}
%\newcommand{\startingage}[1]{\ability{Starting Age: }{#1}\\}  
%\newcommand{\skillpoints}[1]{\ability{Skill Points per Level: }{#1 + Intelligence Bonus}\\}
%\newcommand{\classskills}[1]{\ability{Class Skills: }{The {\class}'s class skills (and the key ability for each skill) are #1}\\}


\newcommand{\startclassfeatures}{
 \vspace{0.5em}\smallskip\noindent All of the following are class features of the \class ~class.}
%place before actual class features entries.

\newcommand{\proficiencies}[1]{
 \ability{Weapon and Armor Proficiencies}{The \class ~is proficient with #1}}
%Displays proficiencies with minimal input, implimentation looks like \proficiencies{the proficiencies}

\newcommand{\classfeature}[2]{
  \ability{#1}{#2}}
%No functional difference from \ability currently

%%%%Class Table Commands

\newcommand{\gbab}{\empty}
\newcommand{\mbab}{\empty}
\newcommand{\fort}{\empty}
\newcommand{\refl}{\empty}
\newcommand{\will}{\empty}
%Creates new commands for use in \ifx statements for formatting purposes.

\newcommand{\goodbab}{\renewcommand{\gbab}{\empty}\renewcommand{\mbab}{a}}
\newcommand{\modebab}{\renewcommand{\gbab}{a}\renewcommand{\mbab}{\empty}}
\newcommand{\poorbab}{\renewcommand{\gbab}{a}\renewcommand{\mbab}{a}}
%A set of commands to tell LaTeX what BAB progression the class has. Only one should be called per class.

\newcommand{\goodfor}{\renewcommand{\fort}{\empty}}
\newcommand{\poorfor}{\renewcommand{\fort}{a}}
%A set of commands to tell LaTeX what Fortitude progression the class has. Only one should be called per class.

\newcommand{\goodref}{\renewcommand{\refl}{\empty}}
\newcommand{\poorref}{\renewcommand{\refl}{a}}
%A set of commands to tell LaTeX what Reflex progression the class has. Only one should be called per class.

\newcommand{\goodwil}{\renewcommand{\will}{\empty}}
\newcommand{\poorwil}{\renewcommand{\will}{a}}
%A set of commands to tell LaTeX what Will progression the class has. Only one should be called per class.

\newenvironment{classtable}[1]
{
\begin{table}[tpb]
\centering
\rowcolors{1}{colorone}{colortwo}
\begin{tabu} to \textwidth {p{.275in} l p{0.275in} p{0.275in} p{0.275in} X l l l l} 
\rowcolor{headercolor} Level & Base Attack & Fort. & Ref. & Will & Special #1 \\
}{
\hline
\end{tabu}
\end{table}
}
%A a new environment that sets up the class tables. Include the \level commands between \begin{classtable}.

\newenvironment{minorcastingclasstable}
{
%\table[htb]
%\center
\centering
\rowcolors{1}{colorone}{colortwo}
\begin{tabu}to \textwidth{p{.275in}lp{0.275in}p{0.275in}p{0.275in}Xccccccc}
\rowcolor{headercolor} & & & & & &\multicolumn{7}{c}{Spells Per Day (By Level)} \\
\rowcolor{headercolor} Level & Base Attack & Fort. & Ref. & Will & Special &0&1&2&3&4&5&6\\
}{
\hline
\end{tabu}
%\endcenter
%\endtable
}
%A a new environment similar to classtable, but with columns for a minor (zero through six) spell slot progression.

\newenvironment{fullcastingclasstable}
{
\table[htb]
\center
\rowcolors{1}{colorone}{colortwo}
\begin{tabu}to \textwidth{p{.275in}lp{0.275in}p{0.275in}p{0.275in}Xcccccccccc}
\rowcolor{headercolor} & & & & & &\multicolumn{10}{c}{Spells Per Day (By Level)} \\
\rowcolor{headercolor} Level & Base Attack & Fort. & Ref. & Will & Special &0&1&2&3&4&5&6&7&8&9\\
}{
\hline
\end{tabu}
\endcenter
\endtable
}
%Another environment for class tables, this one for full (0 through 9) spell slot progression.

\newcommand{\levelone}[1]{
\hline
1st  & \ifx\gbab\isempty +1 \else\ifx\mbab\isempty +0 \else +0 \fi \fi
	 & \ifx\fort\isempty +2 \else +0 \fi
	 & \ifx\refl\isempty +2 \else +0 \fi
	 & \ifx\will\isempty +2 \else +0 \fi
	 & #1 \\}
%A command that declares a table row within the class feature table.

\newcommand{\leveltwo}[1]{
2nd  & \ifx\gbab\isempty +2 \else\ifx\mbab\isempty +1 \else +1 \fi \fi
	 & \ifx\fort\isempty +3 \else +0 \fi
	 & \ifx\refl\isempty +3 \else +0 \fi
	 & \ifx\will\isempty +3 \else +0 \fi
	 & #1 \\}
%A command that declares a table row within the class feature table.

\newcommand{\levelthree}[1]{
3rd  & \ifx\gbab\isempty +3 \else\ifx\mbab\isempty +2 \else +1 \fi \fi
	 & \ifx\fort\isempty +3 \else +1 \fi
	 & \ifx\refl\isempty +3 \else +1 \fi
	 & \ifx\will\isempty +3 \else +1 \fi
	 & #1 \\}
%A command that declares a table row within the class feature table.

\newcommand{\levelfour}[1]{
4th  & \ifx\gbab\isempty +4 \else\ifx\mbab\isempty +3 \else +2 \fi \fi
	 & \ifx\fort\isempty +4 \else +1 \fi
	 & \ifx\refl\isempty +4 \else +1 \fi
	 & \ifx\will\isempty +4 \else +1 \fi
	 & #1 \\}
%A command that declares a table row within the class feature table.

\newcommand{\levelfive}[1]{
5th  & \ifx\gbab\isempty +5 \else\ifx\mbab\isempty +3 \else +2 \fi \fi
	 & \ifx\fort\isempty +4 \else +1 \fi
	 & \ifx\refl\isempty +4 \else +1 \fi
	 & \ifx\will\isempty +4 \else +1 \fi
	 & #1 \\}
%A command that declares a table row within the class feature table.
	 
\newcommand{\levelsix}[1]{
6th  & \ifx\gbab\isempty +6/+1 \else\ifx\mbab\isempty +4 \else +3 \fi \fi
	 & \ifx\fort\isempty +5 \else +2 \fi
	 & \ifx\refl\isempty +5 \else +2 \fi
	 & \ifx\will\isempty +5 \else +2 \fi
	 & #1 \\}
%A command that declares a table row within the class feature table.

\newcommand{\levelseven}[1]{
7th  & \ifx\gbab\isempty +7/+2 \else\ifx\mbab\isempty +5 \else +3 \fi \fi
	 & \ifx\fort\isempty +5 \else +2 \fi
	 & \ifx\refl\isempty +5 \else +2 \fi
	 & \ifx\will\isempty +5 \else +2 \fi
	 & #1 \\}
%A command that declares a table row within the class feature table.
	 
\newcommand{\leveleight}[1]{
8th  & \ifx\gbab\isempty +8/+3 \else\ifx\mbab\isempty +6/+1 \else +4 \fi \fi
	 & \ifx\fort\isempty +6 \else +2 \fi
	 & \ifx\refl\isempty +6 \else +2 \fi
	 & \ifx\will\isempty +6 \else +2 \fi
	 & #1 \\}
%A command that declares a table row within the class feature table.
	 
\newcommand{\levelnine}[1]{
9th  & \ifx\gbab\isempty +9/+4 \else\ifx\mbab\isempty +6/+1 \else +4 \fi \fi
	 & \ifx\fort\isempty +6 \else +3 \fi
	 & \ifx\refl\isempty +6 \else +3 \fi
	 & \ifx\will\isempty +6 \else +3 \fi
	 & #1 \\}
%A command that declares a table row within the class feature table.
	 
\newcommand{\levelten}[1]{
10th & \ifx\gbab\isempty +10/+5 \else\ifx\mbab\isempty +7/+2 \else +5 \fi \fi
	 & \ifx\fort\isempty +7 \else +3 \fi
	 & \ifx\refl\isempty +7 \else +3 \fi
	 & \ifx\will\isempty +7 \else +3 \fi
	 & #1 \\}
%A command that declares a table row within the class feature table.
	 
\newcommand{\leveleleven}[1]{
11th & \ifx\gbab\isempty +11/+6/+6 \else\ifx\mbab\isempty +8/+3 \else +5 \fi \fi
	 & \ifx\fort\isempty +7 \else +3 \fi
	 & \ifx\refl\isempty +7 \else +3 \fi
	 & \ifx\will\isempty +7 \else +3 \fi
	 & #1 \\}
%A command that declares a table row within the class feature table.
	 
\newcommand{\leveltwelve}[1]{
12th & \ifx\gbab\isempty +12/+7/+7 \else\ifx\mbab\isempty +9/+4 \else +6/+1 \fi \fi
	 & \ifx\fort\isempty +8 \else +4 \fi
	 & \ifx\refl\isempty +8 \else +4 \fi
	 & \ifx\will\isempty +8 \else +4 \fi
	 & #1 \\}
%A command that declares a table row within the class feature table.
	 
\newcommand{\levelthirteen}[1]{
13th & \ifx\gbab\isempty +13/+8/+8 \else\ifx\mbab\isempty +9/+4 \else +6/+1 \fi \fi
	 & \ifx\fort\isempty +8 \else +4 \fi
	 & \ifx\refl\isempty +8 \else +4 \fi
	 & \ifx\will\isempty +8 \else +4 \fi
	 & #1 \\}
%A command that declares a table row within the class feature table.
	 
\newcommand{\levelfourteen}[1]{
14th & \ifx\gbab\isempty +14/+9/+9 \else\ifx\mbab\isempty +10/+5 \else +7/+2 \fi \fi
	 & \ifx\fort\isempty +9 \else +4 \fi
	 & \ifx\refl\isempty +9 \else +4 \fi
	 & \ifx\will\isempty +9 \else +4 \fi
	 & #1 \\}
%A command that declares a table row within the class feature table.
	 
\newcommand{\levelfifteen}[1]{
15th & \ifx\gbab\isempty +15/+10/+10 \else\ifx\mbab\isempty +11/+6/+6 \else +7/+2 \fi \fi
	 & \ifx\fort\isempty +9 \else +5 \fi
	 & \ifx\refl\isempty +9 \else +5 \fi
	 & \ifx\will\isempty +9 \else +5 \fi
	 & #1 \\}
%A command that declares a table row within the class feature table.
	 
\newcommand{\levelsixteen}[1]{
16th & \ifx\gbab\isempty +16/+11/+11/+11 \else\ifx\mbab\isempty +12/+7/+7 \else +8/+3 \fi \fi
	 & \ifx\fort\isempty +10 \else +5 \fi
	 & \ifx\refl\isempty +10 \else +5 \fi
	 & \ifx\will\isempty +10 \else +5 \fi
	 & #1 \\}
%A command that declares a table row within the class feature table.
	 
\newcommand{\levelseventeen}[1]{
17th & \ifx\gbab\isempty +17/+12/+12/+12 \else\ifx\mbab\isempty +12/+7/+7 \else +8/+3 \fi \fi
	 & \ifx\fort\isempty +10 \else +5 \fi
	 & \ifx\refl\isempty +10 \else +5 \fi
	 & \ifx\will\isempty +10 \else +5 \fi
	 & #1 \\}
%A command that declares a table row within the class feature table.
	 
\newcommand{\leveleighteen}[1]{
18th & \ifx\gbab\isempty +18/+13/+13/+13 \else\ifx\mbab\isempty +13/+8/+8 \else +9/+4 \fi \fi
	 & \ifx\fort\isempty +11 \else +6 \fi
	 & \ifx\refl\isempty +11 \else +6 \fi
	 & \ifx\will\isempty +11 \else +6 \fi
	 & #1 \\}
%A command that declares a table row within the class feature table.
	 
\newcommand{\levelnineteen}[1]{
19th & \ifx\gbab\isempty +19/+14/+14/+14 \else\ifx\mbab\isempty +14/+9/+9 \else +9/+4 \fi \fi
	 & \ifx\fort\isempty +11 \else +6 \fi
	 & \ifx\refl\isempty +11 \else +6 \fi
	 & \ifx\will\isempty +11 \else +6 \fi
	 & #1 \\}
%A command that declares a table row within the class feature table.
	 
\newcommand{\leveltwenty}[1]{
20th & \ifx\gbab\isempty +20/+15/+15/+15 \else\ifx\mbab\isempty +15/+10/+10 \else +10/+5 \fi \fi
	 & \ifx\fort\isempty +12 \else +6 \fi
	 & \ifx\refl\isempty +12 \else +6 \fi
	 & \ifx\will\isempty +12 \else +6 \fi
	 & #1 \\}
%A command that declares a table row within the class feature table.

\newmdenv[hidealllines=true,backgroundcolor=gray!20]{optionbox}

\newcommand{\option}[1]{
  \renewmdenv[hidealllines=true,backgroundcolor=colorone]{optionbox}
   \begin{optionbox}\noindent{#1}\end{optionbox}
   ~\\*
}

\newenvironment{optional}{
\colorlet{colortwo}{white}
\colorlet{colorone}{gray!15}
}

%%%%%%%%%%%%%%%%%%%%%%%%

\chapter{Classes}
\section{Class Basics}
foo
\section{Core Classes}
%%%%%%%%%%%%%%%%%%%%%%%%%%%%%%%%%%%%%%%%%%%%%%%%%%
\classentry{Assassin}
%%%%%%%%%%%%%%%%%%%%%%%%%%%%%%%%%%%%%%%%%%%%%%%%%%
\tagline{"I kill people. Individually, you are a person. Collectively, I think you count as people."}

An assassin is a master of the art of killing, a vicious weapon honed by experience and inclination to learn the myriad ways to end a life. Unlike common warriors or rogues, an Assassin does not study various fighting arts or muddle his training with martial dirty tricks, he instead studies the anatomy of the various creatures of wildly different anatomies and forms of existence, and he uses this knowledge to place his blows in areas vital for biological or mystical reasons. Stealth and sudden violence are his hallmarks, and various exotic tools and killing methods become his tools.

While most societies consider assassination to be a vile art, or at best a dishonorable or unvalorous one, the reasons that drive these killers vary. Cold-hearted mercenaries share a skill set with dedicated demon-hunters, differing only in the application of their skills. Only the most naïve student of ethics believes that all killing is evil, or that nobility cannot be found in a mercifully quick death.

\textbf{Alignment:} An Assassin may be of any alignment.

\textbf{Races:} any

\textbf{Starting Gold:} 6d4x10 gp (150 gold)

\textbf{Starting Age:} As Rogue.

\textbf{Hit Die:} d6

\textbf{Class Skills:} The Assassin's skills (and the key ability for each skill) are \linkskill{Balance} (Dex), \linkskill{Bluff} (Cha), \linkskill{Climb} (Str), \linkskill{Concentration} (Con), \linkskill{Craft} (Int), \linkskill{Diplomacy} (Cha), \linkskill{Disable Device} (Int), \linkskill{Disguise} (Cha), \linkskill{Gather Information} (Cha), \linkskill{Hide} (Dex), \linkskill{Intimidate} (Cha), \linkskill{Jump} (Str), \linkskill{Knowledge} (any) (Int), \linkskill{Listen} (Wis), \linkskill{Move Silently} (Dex), \linkskill{Perform} (Cha), \linkskill{Profession} (Wis), \linkskill{Search} (Int), \linkskill{Sense Motive} (Wis), \linkskill{Sleight of Hand} (Dex), \linkskill{Spellcraft} (Int), \linkskill{Spot} (Wis), \linkskill{Swim} (Str), \linkskill{Tumble} (Dex), and \linkskill{Use Magic Device} (Cha).

\textbf{Skills/Level:} 6 + Intelligence Bonus

\modebab{}
\goodfor{}
\goodref{}
\poorwil{}

\begin{minorcastingclasstable}
\levelone{Death Attack, Poison Use, Personal Immunity & -- & -- & -- & -- & -- & --}
\leveltwo{Uncanny Dodge & 0 & -- & -- & -- & -- & --}
\levelthree{Hide in Plain Sight & 1 & -- & -- & -- & -- & --}
\levelfour{Cloak of Discretion & 2 & 0 & -- & -- & -- & --}
\levelfive{Trapfinding, Trapmaking & 3 & 1 & -- & -- & -- & --}
\levelsix{Palm Weapon & 3 & 2 & -- & -- & -- & --}
\levelseven{Full Death Attack & 3 & 2 & 0 & -- & -- & --}
\leveleight{Nerve of the Assassin & 3 & 3 & 1 & -- & -- & --}
\levelnine{Improved Uncanny Dodge & 3 & 3 & 2 & -- & -- & --}
\levelten{Skill Mastery & 3 & 3 & 2 & 0 & -- & --}
\leveleleven{Poisonmaster & 3 & 3 & 3 & 1 & -- & --}
\leveltwelve{Personal Immunity & 3 & 3 & 3 & 2 & -- & --}
\levelthirteen{Exotic Method & 3 & 3 & 3 & 2 & 0 & --}
\levelfourteen{Personal Immunity & 3 & 3 & 3 & 3 & 1 & --}
\levelfifteen{Killer's Proof & 3 & 3 & 3 & 3 & 2 & --}
\levelsixteen{Exotic Method & 3 & 3 & 3 & 3 & 2 & 0}
\levelseventeen{Death by a Thousand Cuts & 3 & 3 & 3 & 3 & 3 & 1}
\leveleighteen{Mind Blank & 3 & 3 & 3 & 3 & 3 & 2}
\levelnineteen{Exotic Method & 3 & 3 & 3 & 3 & 3 & 3}
\leveltwenty{Killing Strike & 3 & 3 & 3 & 3 & 3 & 3}
\end{minorcastingclasstable}

\begin{basictable}{Assassin Spells Known}{l*{7}{c}}
\textbf{Level} & \textbf{0th} & \textbf{1st} & \textbf{2nd} & \textbf{3rd} & \textbf{4th} & \textbf{5th} & \textbf{6th} \\
1st & 4 & -- & -- & -- & -- & -- & --\\
2nd & 5 & 2\textsuperscript{1} & -- & -- & -- & -- & --\\
3rd & 6 & 3 & -- & -- & -- & -- & --\\
4th & 6 & 3 & 2\textsuperscript{1} & -- & -- & -- & --\\
5th & 6 & 4 & 3 & -- & -- & -- & --\\
6th & 6 & 4 & 3 & -- & -- & -- & --\\
7th & 6 & 4 & 4 & 2\textsuperscript{1} & -- & -- & --\\
8th & 6 & 4 & 4 & 3 & -- & -- & --\\
9th & 6 & 4 & 4 & 3 & -- & -- & --\\
10th & 6 & 4 & 4 & 4 & 2\textsuperscript{1} & -- & --\\
11th & 6 & 4 & 4 & 4 & 3 & -- & --\\
12th & 6 & 4 & 4 & 4 & 3 & -- & --\\
13th & 6 & 4 & 4 & 4 & 4 & 2\textsuperscript{1} & --\\
14th & 6 & 4 & 4 & 4 & 4 & 3 & --\\
15th & 6 & 4 & 4 & 4 & 4 & 3 & --\\
16th & 6 & 5 & 4 & 4 & 4 & 4 & 2\textsuperscript{1}\\
17th & 6 & 5 & 5 & 4 & 4 & 4 & 3\\
18th & 6 & 5 & 5 & 5 & 4 & 4 & 3\\
19th & 6 & 5 & 5 & 5 & 5 & 4 & 4\\
20th & 6 & 5 & 5 & 5 & 5 & 5 & 4\\
\multicolumn{8}{p{8.5cm}}{\textsuperscript{1} Provided the Assassin has a high enough Intelligence score to have a bonus spell of this level.}\\
\end{basictable}

\classfeatures

\textbf{Weapon and Armor Proficiency:} Assassins are proficient with all Simple Weapons, as well as any weapon that counts as a Light Weapon, repeating crossbows, and hand crossbows. At first level, an Assassin gains proficiency with one Exotic Weapon of her choice. Assassins are proficient with Light Armor but not with Shields.

\textbf{Spellcasting:} The Assassin is a Spontaneous Arcane Spellcaster (like a Bard or Sorcerer). An Assassin's spells known may be chosen from the Wizard list, and must be from the schools of Divination, Illusion, or Necromancy. To cast an Assassin spell, she must have an Intelligence at least equal to 10 + the Spell level. The DC of the Assassin's spells is Intelligence based and the bonus spells are Intelligence based. As with a Bard, the Assassin can cast spells in Light armor without any chance of Arcane Spell Failure.

\textbf{Cantrips:} In addition to her normal spell slots per day, an Assassin has a number of 0th level "cantrip" spells that can be cast an unlimited number of times per day.

\textbf{Poison Use (Ex):} An Assassin may prepare, apply, and use poison without any chance of poisoning herself.

\textbf{Death Attack (Ex):} An Assassin may spend a full-round action to study an opponent who would be denied their Dexterity bonus if she instead attacked that target. If she does so, her next attack is a Death Attack if she makes it within 1 round. A Death Attack inflicts a number of extra dice of damage equal to her Assassin level plus two dice, but only if the target is denied its Dexterity Bonus to AC against that attack. Special attacks such as a coup de grace may be a Death Attack. Assassins are well trained in eliminating magical or distant opponents, and do not have to meet the stringent requirements of a sneak attack, though if a character has both sneak attack and death attack, they stack if the character meets the requirements of both. As long as the victim is denied their dexterity against attacks from the assassin during the study action and the attack itself, it counts as a death attack. An Assassin may load a crossbow simultaneously with his action to study his target if he has a Base Attack Bonus of +1 or more.

\textbf{Personal Immunity (Ex):} Choose four poisons, an Assassin is immune to all four of those poisons, even if they are made available in a stronger strength. At levels 5, 7, and 12 the Assassin may choose one more type of poison to become immune to. At level 14, an Assassin becomes immune to all poisons.

\textbf{Uncanny Dodge (Ex):} Starting at 2nd level, an Assassin can react to danger before his senses would normally allow him to do so. He retains her Dexterity bonus to AC (if any) even if she is caught flat-footed or struck by an invisible attacker. However, he still loses her Dexterity bonus to AC if immobilized. If an Assassin already has uncanny dodge from a different class he automatically gains improved uncanny dodge (see below) instead.

\textbf{Hide in Plain Sight (Ex):} A 3rd level Assassin can hide in unusual locations, and may hide in areas without cover or concealment without penalty. An Assassin may even hide while being observed. This ability does not remove the -10 penalty for moving at full speed, or the -20 penalty for running or fighting.

\textbf{Cloak of Discretion (Su):} At 4th level, an Assassin is protected by a constant \linkspell{Nondetection} effect, with a caster level equal to his character level.

\textbf{Trapfinding (Ex):} At 5th level, Assassins can use the Search skill to locate traps when the task has a Difficulty Class higher than 20. Finding a nonmagical trap has a DC of at least 20, or higher if it is well hidden. Finding a magic trap has a DC of 25 + the level of the spell used to create it. Assasins can use the \linkskill{Disable Device} skill to disarm magic traps. A magic trap generally has a DC of 25 + the level of the spell used to create it. An Assassin who beats a trap's DC by 10 or more with a Disable Device check can study a trap, figure out how it works, and bypass it (with her party) without disarming it.

\textbf{Trapmaking (Ex):} At 5th level, the Assassin learns to build simple mechanical traps in out of common materials. As long as has access to ropes, flexible material like green wood, and weapon-grade materials like sharpened wooden sticks or steel weapons, he can build an improvised trap in 10 minutes. He can build any non-magical trap on the "CR 1'' trap list that doesn't involve a pit. These traps have a Search DC equal to 20 + the Assassin's level, have a BAB equal to his own, and are always single-use traps. He may add poison to these traps, if he has access to it, but it will dry out in an hour.

\textbf{Full Death Attack (Ex):} At 7th level, if the Assassin studies an opponent to perform a Death Attack, she can make a full attack during the next round where every attack inflicts Death Attack damage as long as the target was denied their Dexterity bonus to AC against the first attack in the full attack action.

\textbf{Skill Mastery (Ex):} At 10th level, an Assassin becomes so certain in the use of certain skills that she can use them reliably even under adverse conditions. When making a skill check with \linkskill{Climb}, \linkskill{Disable Device}, \linkskill{Hide}, \linkskill{Move Silently}, \linkskill{Search}, \linkskill{Spellcraft}, \linkskill{Use Magic Device}, \linkskill{Use Rope}, or \linkskill{Swim}, she may take 10 even if stress and distractions would normally prevent her from doing so.

\textbf{Palm Weapon (Su):} At 6th level, the Assassin learns to conceal weapons with supernatural skill. Any weapon successfully concealed with \linkskill{Sleight of Hand} cannot be found with divination magic.

\textbf{Nerve of the Killer (Ex):} At 8th level, an Assassin gains a limited immunity to compulsion and charm effects. While studying a target for a Death Attack, and for one round afterward, he counts as if he were within a \linkspell{Protection From Evil} effect. This does not confer a deflection bonus to AC.

\textbf{Improved Uncanny Dodge (Ex):} An Assassin of 9th level or higher can no longer be flanked. This defense denies another character the ability to sneak attack the character by flanking him, unless the attacker has at least four more levels in a class that provides sneak attack than the target.

\textbf{Poisonmaster (Ex):} At 11th level, the Assassin learns alchemic secrets for creating short-term poisons. By expending an entire healer's kit worth of materials and an hour of time, he can synthesize one dose of any poison in the DMG. This poison degrades to uselessness in one week.

\textbf{Exotic Method:} At 13th, 16th, and 19th level the Assassin learns an exotic form of killing from the list below. Once chosen, this ability does not change:

\begin{description*}
\item[Carrier (Sp):] Three times per day, the Assassin can cast \linkspell{Contagion} as a swift action spell-like ability.
\item[Poison of the Cockatrice (Sp):] Twice per day, the Assassin can cast \linkspell{Flesh to Stone} as a swift action spell-like ability.
\item[Killer Faerie Arts (Sp):] Twice per day, the Assassin can cast \linkspell{Polymorph Other} as a swift action spell-like ability.
\item[Proxy Assassin (Sp):] Twice per day, the Assassin can cast \linkspell{summon monster VII} as a spell-like ability. This effect lasts 10 minutes.
\item[Death By Plane (Sp):] Once per day, the Assassin can cast \linkspell{Plane Shift} as a spell-like ability.
\item[Dimensional Rip (Sp):] Once per day, the Assassin can cast \linkspell{Implosion} as a spell-like ability. The duration of this effect is three rounds.
\item[New School (Ex):] The Assassin may now choose spells known from a new school.
\end{description*}

\textbf{Killer's Proof (Su)}: At 15th level, the Assassin learns to steal the souls of those he kills. If he is holding an onyx worth at least 100 GP when he kills an enemy, he may place their soul within the gem as if he has cast \linkspell{Soul Bind} on them at the moment of their death.

\textbf{Death by a Thousand Cuts}: At 17th level, the assassin has learned to kill even the hardiest of foes by reducing their physical form to shambles. Every successful Death attack inflicts a cumulative -2 Dexterity penalty to the Assassin's victim. These penalties last one day.

\textbf{Mind Blank (Su)}: At 18th level, the Assassin is protected by a constant \linkspell{Mind Blank} effect.


\textbf{Killing Strike (Su)}: At 20th level, the Assassin's Death Attacks bypass his victim's DR and hardness.

\classentry{Barbarian}
\quot{``My name is Sharptooth of the Wolf Tribe. Your women, lands, and riches are mine.''}

\desc{  }

\playingaclass{Playing a Barbarian is actually very easy. In general, you hit things, and they fall down. A Barbarian's action in almost any circumstance can plausibly be ``I hit it with my great axe!" As such, a Barbarian character can be a good method to introduce a new player to the game or kill some orcs when you've had a few glasses of brew.}

\alignment{Every alignment has its share of Barbarians, however more Barbarians are of Chaotic alignment than of Lawful Alignment.}

\races{Anybody can become a barbarian, and in areas with little in the way of civilization, a lot of people do.}

\startinggold{4d6x10 gp (140 gold)}

\startingage{ }

\hitdie{d12}

\classskills{Balance (Dex), Climb (Str), Hide (Dex), Intimidate (Cha), Jump (Str), Knowledge: Nature (Int), Listen (Wis), Move Silently (Dex), Sense Motive (Wis), Spot (Wis), Survival (Wis), and Swim (Str).}

\skillpoints{4}

\goodbab\goodfor\poorref\poorwil
\begin{classtable}{}
\levelone{Rage, Fast Healing 1}
\leveltwo{Rage Dice +1d6, Combat Movement +5'}
\levelthree{Battle Hardened}
\levelfour{Rage Dice +2d6, Combat Movement +10'}
\levelfive{Sidestep Hazards , Fast Healing 5}
\levelsix{Rage Dice +3d6, Combat Movement +15'}
\levelseven{Great Blows}
\leveleight{Rage Dice +4d6, Combat Movement +20'}
\levelnine{Great Life}
\levelten{Rage Dice +5d6, Combat Movement +25', Fast Healing 10}
\leveleleven{Call the Horde}
\leveltwelve{Rage Dice +6d6, Combat Movement +30'}
\levelthirteen{Watched by Totems}
\levelfourteen{Rage Dice +7d6, Combat Movement +35'}
\levelfifteen{Primal Assault, Fast Healing 15}
\levelsixteen{Rage Dice +8d6, Combat Movement +40'}
\levelseventeen{Savagery}
\leveleighteen{Rage Dice +9d6, Combat Movement +45'}
\levelnineteen{One With The Beast}
\leveltwenty{Rage Dice +10d6, Combat Movement +50', Fast Healing 20}
\end{classtable}

\startclassfeatures

\proficiencies{simple weapons, martial weapons, light armor, medium armor and with shields.}

\classfeature{Rage (Ex)}{When doing melee damage to a foe or being struck by a foe, a Barbarian may choose to enter a Rage as an immediate action. While Raging, a Barbarian gains a +2 morale bonus to hit and damage in melee combat and may apply any Rage Dice he has to his melee damage rolls. He also gains a +2 to saves, a -2 to AC, and he gains DR X/-- with ``X" being equal to half his Barbarian level +2 (rounded down). For example, a 1st level Barbarian has DR 3/-- while Raging and a 10th level Barbarian has DR 7/-- while Raging. While Raging, a Barbarian may not cast spells, activate magic items, use spell-like abilities, or drop his weapons or shield. Rage lasts until he has neither struck an enemy for three consecutive rounds nor suffered damage from an enemy for three consecutive rounds. He may voluntarily end a Rage as a full-round action.}

\classfeature{Fast Healing}{Barbarians shrug off wounds that would cripple a lesser man, and have learned to draw upon deep reserves of energy and stamina. At 1st level, they gain Fast Healing 1. At 5th level this becomes Fast Healing 5, Fast Healing 10 at 10th level, Fast Healing 15 at 15th level, and Fast Healing 20 at 20th level. This healing only applies while he is not raging. \smallskip

If a Barbarian ever multiclasses, he permanently loses this ability. A multiclass character does not gain this ability.  A character with 4 or more levels of Barbarian gains this ability even if multiclassed.}

\classfeature{Rage Dice}{While Raging, a Barbarian may add these dice of damage to each of his melee attacks. These dice are not multiplied by damage multipliers, and are not applied to any bonus attacks beyond those granted by Base Attack Bonus. These dice are not sneak attack dice, and do not count as sneak attack dice for the prerequisites of prestige classes or feats.}

\classfeature{Combat Movement}{While Raging, a Barbarian moves faster in combat, and may add his Combat Movement to his speed when he takes a move action to move.}

\classfeature{Battle Hardened}{At 3th level, a Raging Barbarian's mind has been closed off from distractions by the depths of his bloodlust and battle fury. While Raging, he may use his Fortitude Save in place of his Will Save. If he is under the effects of a compulsion or fear effect, he may act normally while Raging as if he was inside a \spell{protection from evil} effect.}

\classfeature{Sidestep Hazards (Ex)}{At 5th level, a Raging Barbarian learns to sidestep hazards with an intuitive and primal danger sense. While Raging, he may use his Fortitude Save in place of his Reflex Save.}

\classfeature{Great Blows (Ex)}{At 7th level, a Raging Barbarian's melee attacks are Great Blows. Any enemy struck by the Barbarian's melee or thrown weapon attacks must make a Fort Save or be stunned for one round. No enemy can be targeted by this ability more than once a round, and the save DC for this ability is 10 + half the Barbarian's HD + his Constitution modifier.}

\classfeature{Great Life (Ex)}{While Raging, a 9th level Barbarian is immune to nonlethal damage, death effects, stunning, critical hits, negative levels, and ability damage (but not ability drain).}

\classfeature{Call the Horde (Ex)}{An 11th level Barbarian becomes a hero of his people. He gains the Command feat as a bonus feat, but his followers must be Barbarians. In campaigns that do not use Leadership feats, he instead gains a +2 unnamed bonus to all saves.}

\classfeature{Watched by Totems (Ex)}{At 13th level, a Barbarian may immediately reroll any failed save. He may do this no more than once per failed save.}

\classfeature{Primal Assault (Ex)}{At 15th level, a Raging Barbarian may choose to radiate an effect similar to an \spell{antimagic field} when he enters a Rage, with a caster level equal to his HD. Unlike a normal antimagic field, this effect does not suppress magic effects on him or the effects of magic items he is wearing or holding.}

\classfeature{Savagery (Ex)}{At 17th level, a Raging Barbarian may take a full round action to make a normal melee attack that has an additional effect similar to a \spell{mordenkainen's disjunction}. Unlike a normal \spell{mordenkainen's disjunction}, this effect only targets a single item or creature struck.}

\classfeature{One With The Beast}{At 19th level, a Barbarian no longer needs to be in a Rage to use any Barbarian ability.}

%\input{Bard}
%%%%%%%%%%%%%%%%%%%%%%%%%%%%%%%%%%%%%%%%%%%%%%%%%%
\classentry{Cleric}
%%%%%%%%%%%%%%%%%%%%%%%%%%%%%%%%%%%%%%%%%%%%%%%%%%

foo?

%\input{Druid}
\classentry{Fighter}
\goodbab
\goodfor
\goodref
\goodwil
\quot{``I've seen this kind of fire-breathing chicken-demon before. We're going to need more rope. Also a bigger cart.''}

\desc{The Fighter is a versatile combatant who is able to actively disrupt the activities of his enemies. Fighters represent plucky heroes and grizzled veterans, but they always appear to surmount impossible odds. Which means in retrospect that the odds weren't all that impossible. At least, not for someone with a Fighter's talents.}

\playingaclass{Fighters are often handed to beginning players in order to help them learn the ropes. This is a cruel practice that dates back to when the Fighter was explicitly a weak class that players were forced to play to the (quit proximate) death if for whatever reason they didn't roll well enough on their stats to play a real character. The Fighter described here is not the hazing ritual of old, but it is a more complicated character than many others, being the non-magical equivalent to the Wizard. Beginning characters should probably be given a Barbarian, Conduit, or Rogue character to introduce them to the game mechanics of D\&D.

A Fighter has an answer for virtually any circumstance and a great deal of adaptability and flexibility, and benefits greatly from being played by a player who actually knows how far a Roper's strands or a Beholder's rays reach. The Fighter character is archetypically a character who uses her opponent's limitations against them, and it really slows down play if the player needs to have those limitations explained during combat. As such, a full classed Fighter is recommended for experienced players of the game.

That being said, a level or two of Fighter can give some breadth and resilience to almost any martial build, and makes a good multiclassing dip even (sometimes especially) for inexperienced players.}

\alignment{Every alignment has its share of Fighters, however more Fighters are of Lawful alignment than of Chaotic Alignment.}

\races{Every humanoid race has warriors, but actual Fighters are rarer in societies that don't value logistics and planning. So while there are many Fighters among the Hobgoblins, Dwarves, and Fire Giants, a Fighter is rarely seen among the ranks of the Orcs, Gnomes, or Ogres.}

\startinggold{6d6x10 gp (210 gold)}

\startingage{ <-starting age, often written as a class reference like "As Rogue."-> }

\hitdie{d10}

\classskills{Balance (Dex), Bluff (Cha), Climb (Str), Craft (Int), Diplomacy (Cha), Escape Artist (Dex), Handle Animal (Cha), Intimidate (Cha), Jump (Str), Knowledge (all skills individually) (Int), Listen (Wis), Move Silently (Dex), Profession (Wis), Ride (Dex), Sense Motive (Wis), Spot (Wis), Survival (Wis), Swim (Str), Tumble (Dex), and Use Rope (Dex).}

\skillpoints{6}

\begin{classtable}{}
\levelone{Weapons Training, Combat Focus}
\leveltwo{Bonus Feat}
\levelthree{Problem Solver, Pack Mule}
\levelfour{Bonus Feat}
\levelfive{Logistics Mastery, Active Assualt}
\levelsix{Bonus Feat}
\levelseven{Forge Lore, Improved Delay}
\leveleight{Bonus Feat}
\levelnine{Foil Action}
\levelten{Bonus Feat}
\leveleleven{Lunging Attacks}
\leveltwelve{Bonus Feat}
\levelthirteen{Array of Stunts}
\levelfourteen{Bonus Feat}
\levelfifteen{Greater Combat Focus}
\levelsixteen{Bonus Feat}
\levelseventeen{Improved Foil Action}
\leveleighteen{Bonus Feat}
\levelnineteen{Intense Focus, Supreme Combat Focus}
\leveltwenty{Bonus Feat}
\end{classtable}

\startclassfeatures

\proficiencies{all Simple and Martial Weapons. Fighters are proficient with Light, Medium, and Heavy Armor and with Shields and Great Shields.}

\classfeature{Weapons Training (Ex)}{Fighters train obsessively with armor and weapons of all kinds, and using a new weapon is easy and fun. By practicing with a weapon he is not proficient with for a day, a Fighter may permanently gain proficiency with that weapon by succeeding at an Intelligence check DC 10 (you may not take 10 on this check).}

\classfeature{Combat Focus (Ex)}{A Fighter is at his best when the chips are down and everything is going to Baator in a handbasket. When the world is on fire, a Fighter keeps his head better than anyone. If the Fighter is in a situation that is stressful and/or dangerous enough that he would normally be unable to ``take 10" on skill checks, he may spend a Swift Action to gain Combat Focus. A Fighter may end his Combat Focus at any time to reroll any die roll he makes, and if not used it ends on its own after a number of rounds equal to his Base Attack Bonus.}

\classfeature{Problem Solver (Ex)}{A Fighter of 3rd level can draw upon his intense and diverse training to respond to almost any situation. As a Swift action, he may choose any [Combat] feat he meets the prerequisites for and use it for a number of rounds equal to his base attack bonus. This ability may be used once per hour.}

\classfeature{Pack Mule (Ex)}{Fighters are used to long journeys with a heavy pack and the use of a wide variety of weaponry and equipment. A 3rd level Fighter suffers no penalties for carrying a medium load, and may retrieve stored items from his person without provoking an attack of opportunity.}

\classfeature{Logistics Mastery (Ex)}{Fighters are excellent and efficient logisticians. When a Fighter reaches 5th level, he gains a bonus to his Command Rating equal to one third his Fighter Level.}

\classfeature{Active Assault (Ex)}{A 5th level Fighter can flawlessly place himself where he is most needed in combat. He may take a 5 foot step as an immediate action. This is in addition to any other movement he takes during his turn, even another 5 foot step.}

\classfeature{Forge Lore}{A 7th level Fighter can produce magical weapons and equipment as if he had a Caster Level equal to his ranks in Craft.}

\classfeature{Improved Delay (Ex)}{A Fighter of 7th level may delay his action in one round without compromising his Initiative in the next round. In addition, a Fighter may interrupt another action with his delayed action like it was a readied action (though he does not have to announce his intentions before hand).}

\classfeature{Foil Action (Ex)}{A 9th level Fighter may attempt to monkeywrench any action an opponent is taking. The Fighter may throw sand into a beholder's eye, bat aside a key spell component, or strike a weapon hand with a thrown object, but the result is the same: the opponent's action is wasted, and any spell slots, limited ability uses, or the like used to power it are expended. A Fighter must be within 30 feet of his opponent to use this ability, and must hit with a touch attack or ranged touch attack. Using Foil Action is an Immediate action. At 17th level, Foil Action may be used at up to 60 feet.}

\classfeature{Lunging Attacks (Ex)}{The battlefield is an extremely dangerous place, and 11th level Fighters are expected to hold off Elder Elementals, Hezrous, and Hamatulas. Fighters of this level may add 5 feet to the reach of any of their weapons.}

\classfeature{Array of Stunts (Ex)}{A 13th level Fighter may take one extra Immediate Action between his turns without sacrificing a Swift action during his next turn.}

\classfeature{Greater Combat Focus (Ex)}{At 15th level, a Fighter may voluntarily expend his Combat Focus as a non-action to suppress any status effect or ongoing spell effect on himself for his Base Attack Bonus in rounds.}

\classfeature{Intense Focus (Ex)}{A 19th level Fighter may take an extra Swift Action each round (in addition to the extra Immediate Action he can take from Array of Stunts).}

\classfeature{Supreme Combat Focus (Ex)}{A 19th level Fighter may expend his Combat Focus as a non-action to take 20 on any die roll. He must elect to use Supreme Combat Focus before rolling the die.}

%%%%%%%%%%%%%%%%%%%%%%%%%%%%%%%%%%%%%%%%%%%%%%%%%%
\classentry{Knight}
%%%%%%%%%%%%%%%%%%%%%%%%%%%%%%%%%%%%%%%%%%%%%%%%%%
\tagline{"Do you hear me you big lizard? You unhand that young man this instant!"}

Knights are more than a social position, in fact many knights don't have any social standing at all. These knight errants uphold the values of honor, and make a name for themselves adventuring.

\textbf{Playing a Knight:} A Knight has the potential to dish out tremendous damage to a single opponent, and it is tempting to think of them as monster killers. However, it is best to realize in advance that the Knight \textit{does not} often realize their tremendous damage output. The threat of the Knight's Designate Opponent ability is just that -- a threat. A Knight excels at \textit{defensive} tasks, and attacking a Knight is often one of the least effective options an opponent might exercise.

So by making it be a logical combat action for your opponents to attack your party's defensive expert, you've really contributed a lot to the party. A Knight can take a lot of the heat off the rest of the party. So don't get frustrated if enemies constantly interrupt your Designate Opponent action -- that's the whole point. A Knight's role is to protect others, and the best way you can do that is to provide a legitimate threat to your opponents.

\textbf{Alignment:} Many Knights are Lawful. But not all of them. You have to maintain your code of conduct, but plenty of Chaotic creatures can do that too.

\textbf{Races:} Knights require a fairly social background to receive their training. After all, a solitary creature generally has little use for honor. As such, while Knights often spend tremendous amounts of time far from civilization, they are almost exclusively recruited from the ranks of races that are highly urban in nature.

\textbf{Starting Gold:} 6d6x10 gp (210 gold)

\textbf{Starting Age:} As Fighter.

\textbf{Hit Die:} d12

\textbf{Class Skills:} The Knight's class skills (and the key ability for each skill) are \linkskill{Climb} (Str), \linkskill{Craft} (Int), \linkskill{Diplomacy} (Cha), \linkskill{Handle Animal} (Cha), \linkskill{Intimidate} (Cha), \linkskill{Jump} (Str), \linkskill{Knowledge} (History, Nobility, and Geography) (Int), \linkskill{Listen} (Wis), \linkskill{Perform} (Cha), \linkskill{Ride} (Dex), \linkskill{Sense Motive} (Wis), \linkskill{Spot} (Wis), and \linkskill{Swim} (Str).

\textbf{Skills/Level:} 4 + Intelligence Bonus

\goodbab{}
\poorfor{}
\poorref{}
\goodwil{}

\begin{classtable}
\levelone{Designate Opponent, Mounted Combat, Code of Conduct}
\leveltwo{Damage Reduction}
\levelthree{Energy Resistance, Speak to Animals}
\levelfour{Immunity to Fear, Knightly Spirit}
\levelfive{Command}
\levelsix{Defend Others, Quick Recovery}
\levelseven{Bastion of Defense, Draw Fire}
\leveleight{Mettle, Spell Shield}
\levelnine{Sacrifice}
\levelten{Knightly Order}
\end{classtable}

\classfeatures

\textbf{Weapon and Armor Proficiency:} Knights are proficient with all simple weapons and Martial Weapons. Knights are proficient with Light, Medium, and Heavy Armor, Shields, and Great Shields.

\textbf{Designate Opponent (Ex):} As a Swift Action, a Knight may mark an opponent as their primary foe. This foe must be within medium range and be able to hear the Knight's challenge. If the target creature inflicts ay damage on the Knight before the Knight's next turn, the attempt fails. Otherwise, any attacks the Knight uses against the opponent \textit{during} her next turn inflict an extra d6 of damage for each Knight level. This effect ends at the end of her next turn, or when she has struck her opponent a number of times equal to the number of attacks normally allotted her by her Base Attack Bonus.

\textit{Example:} Vayn is a 6th level Knight presently benefiting from a \linkspell{Haste} spell, granting her an extra attack during a Full Attack action. On her turn she designates an Ettin as her primary opponent, and the Ettin declines to attack her during the ensuing turn. When her next turn comes up, she uses a Full Attack and attacks 3 times. The first two hits inflict an extra 6d6 of damage, and then she designates the Ettin as her opponent again. It won't soon ignore her!

\textbf{Mounted Combat:} A Knight gains Mounted Combat as a bonus feat at 1st level. If she already has Mounted Combat, she may gain any Combat feat she meets the prerequisites for instead.

\textbf{Code of Conduct:} A Knight must fight with honor even when her opponents do not. Indeed, a Knight subscribes to honor to a degree far more than that which is strictly considered necessary by other honorable characters. Actions which even hint at the appearance of impropriety are anathema to the Knight:

\begin{itemize*}
\item A Knight must not accept undo assistance from allies even in combat. A Knight must refuse bonuses from Aid Another actions.
\item A Knight must refrain from the use poisons of any kind, even normally acceptable poisons such as blade toxins.
\item A Knight may not voluntarily change shape, whether she is impersonating a specific creature or not.
\item A Knight may not sell Magic Items.
\end{itemize*}

A Knight who fails to abide by her code of conduct loses the ability to use any of her Knightly abilities which require actions until she atones.

\textbf{Damage Reduction (Ex):} A Knight trains to suffer the unbearable with chivalry and grace. At 2nd level, she gains Damage Reduction of X/-, where X is half her Knight level, rounded down.

\textbf{Energy Resistance (Ex):} A Knight may protect herself from energy types that she expects. As a Swift Action, a 3rd level Knight may grant herself Energy Resistance against any energy type she chooses equal to her Knight Level plus her Shield Bonus. This energy resistance lasts until she spends a Swift Action to choose another Energy type or her Shield bonus is reduced.

\textbf{Speak to Animals (Ex):} A Knight can make herself understood by beasts. Her steed always seems to be able to catch the thrust of anything she says. A 3rd level Knight gains a bonus to any of her Ride and Handle Animal checks equal to half her Knight Level. In addition, there is no limit to how many tricks she can teach a creature, and her Handle Animal checks are not penalized for attempting to get a creature to perform a trick it does not know.

\textbf{Immunity to Fear (Ex):} At 4th level, a Knight becomes immune to [Fear] effects.

\textbf{Knightly Spirit (Ex):} As a Move Equivalent Action, a 4th level Knight may restore any amount of attribute damage or drain that she has suffered.

\textbf{Command:} A Knight gains \linkfeat{Command} as a bonus feat at level 5.

\textbf{Defend Others (Ex):} A 6th level Knight may use her own body to defend others. Any ally adjacent to the Knight gains Evasion, though she does not.

\textbf{Quick Recovery (Ex):} If a 6th level Knight is \textit{stunned} or \textit{dazed} during her turn, that condition automatically ends at the end of that turn, even if the duration would normally be longer.

\textbf{Bastion of Defense (Ex):} A 7th level Knight can defend others with great facility. All adjacent allies except the Knight gain a +2 Dodge bonus to their Armor Class and Reflex Saves.

\textbf{Draw Fire (Ex):} A 7th level Knight can exploit the weaknesses of unintelligent opponents. With a Swift Action, she may pique the interest of any mindless opponent within medium range. That creature must make a Willpower Save (DC 10 + ½ Hit Dice + Constitution Modifier) or spend all of its actions moving towards or attacking the Knight. This effect ends after a number of rounds equal to the Knight's class level.

\textbf{Mettle (Ex):} An 8th level Knight who succeeds at a Fortitude Partial or Willpower Partial save is not affected at all (basically like Evasion, but for Fortitude and Willpower saves).

\textbf{Spell Shield (Ex):} An 8th level Knight gains Spell Resistance of 5 + her character level. This Spell Resistance is increased by her shield bonus to AC if she has one.

\textbf{Sacrifice (Ex):} As an immediate action, a 9th level Knight may make herself the target of an attack or targeted effect that targets any creature within her reach.

\textbf{Knightly Order:} What is a powerful Knight without a descriptive adjective? Upon reaching 10th level, a Knight \textit{must} join or found a Knightly order. From this point on, she may ignore one of the prerequisites for joining a Knightly Order prestige class. In addition, becoming a member of an order has special meaning for a 10th level Knight, and she gains an ability related to the order she joins. Some sample orders are listed below:

\begin{itemize*}
\item \textbf{Angelic Knight} The Angelic Knights are a transformational order that attempts to live by the precepts of the upper planes. An Angelic Knight gains wings that allow her to Fly 60ft with perfect maneuverability. Also an Angelic Knight benefits from \linkspell{Protection From Evil} at all times.
\item \textbf{Bane Knight} The Bane Knights stand for running around burning the countryside with extreme burning. Bane Knights are immune to fire and do not have to breathe. In addition, a Bane Knight may set any unattended object on fire with a Swift Action at up to Medium Range.
\item \textbf{Chaos Knight} Chaos Knights stand for madness and Giant Frog. With the powers of Giant Frog, they can Giant Frog. Also their natural armor bonus increases by +5 and they are immune to \textit{Sleep} effects.
\item \textbf{Dragon Knight} Dedicated to the Platinum Dragon, the Dragon Knights serve love and justice in equal measure as dishes to those who need them. A Dragon Knight gains a +5 bonus to Sense Motive and any armor she wears has an enhancement bonus of 2 higher than normal (it also gains a platinum sheen in the process, and as a side effect a Dragon Knight is never dirty for more than a few seconds).
\item \textbf{Elemental Knight} The Elemental Knights may be dedicated to a particular element, or somehow dedicated to all of them. An Elemental Knight can \linkspell{Planeshift} at will to any Inner plane or the Prime Material plane. Also, she is immune to \textit{stunning} and ignores the harmful planar effects of the Inner Planes.
\item \textbf{Fey Knight} Using the powers of the Sprites, the Fey Knight has many fairy strengths. Firstly, she gains DR 10/Iron. Also, any of her attacks may do non-lethal damage at any time if this is desired. Also she never ages and does not need to drink.
\item \textbf{Great Knight} Clad in opulent armor, the Great Knight cares only for her own power. The Great Knight gains a +4 bonus on Disarm or Sunder tests, and gains a +4 Profane bonus to her Strength.
\item \textbf{Hell Knight} Forged in the sulphurous clouds of Hell, the Hell Knight is bathed in an evil radiance. In addition to being granted a ceremonial weapon made of green steel, a Hell Knight gains the coveted See In Darkness ability of the Devils. Also, she has an inherent ability to know what every creature within 60' her of finds most repugnant.
\item \textbf{Imperial Knight} The great Empire needs champions able to unswervingly support its interests, and the Imperial Knight is one of the best. She may impose a \linkspell{Zone of Truth} at will as a Supernatural ability, and all of her attacks are Lawfully aligned. Also, she continuously benefits from \linkspell{Magic Circle Against Chaos}.
\end{itemize*}

%%%%%%%%%%%%%%%%%%%%%%%%%%%%%%%%%%%%%%%%%%%%%%%%%%
\classentry{Monk}
%%%%%%%%%%%%%%%%%%%%%%%%%%%%%%%%%%%%%%%%%%%%%%%%%%

foo?

%\input{Paladin}
%\input{Ranger}
%\input{Rogue}
%\input{Samurai}
%\input{Sorcerer}
%%%%%%%%%%%%%%%%%%%%%%%%%%%%%%%%%%%%%%%%%%%%%%%%%%
\classentry{Wizard}
%%%%%%%%%%%%%%%%%%%%%%%%%%%%%%%%%%%%%%%%%%%%%%%%%%

foo?

%\section{Additional Classes}
\classentry{Templar}

\newcommand{\vow}[6]{
\option{\textbf{Vow of #1}\\
\textit{``#2''}
\listone \item \ability{First:}{#3} \item \ability{Second:}{#4} \item \ability{Third:}{#5} \end{list}
\vspace{8pt}
\ability{Roleplaying Ideas:}{#6}
}}

\newcommand{\specialvow}[7]{
\option{\textbf{Vow of #1}\\
\textit{``#2''}
\listone \item \ability{First:}{#3} \item \ability{Second:}{#4} \item \ability{Third:}{#5} \end{list}
\vspace{8pt}
\ability{Special:}{#7}
\ability{Roleplaying Ideas:}{#6}
}}

\newcommand{\faith}[7]{
\option{\textbf{#1}\\
#2
\listnum
	\item #3
	\item #4
	\item #5
	\item #6
	\item #7
\end{list}
}}

\goodbab
\goodfor
\poorref
\goodwil
\quot{``Nobody is more dangerous than he who imagines himself pure in heart, for his purity, by definition, is unassailable.''}

\desc{Every religion has clerics, those tasked with performing the duties of the religion. Many also have faithful members who leave their homes to travel distant lands, spreading the word of their god or pantheon. Templars are ordained warriors tasked with spreading the faith and defending the faithful, while also beating down the foes of a deity.

Templars are the militant arm of their church and/or cause. They are often guards of sacred places, dispatched away from the temples as agents of higher powers, or simply wander to share the virtues of their philosophy and ideal with others. Initially able and zealous warriors combining martial abilities with the power of their deity, they eventually become an active sword or shield for their deity, with high levels of offensive prowess and devastating crowd control. Whether as a bodyguard or a support character, they often find themselves in the ranks of adventuring parties who can make use of the talents.

A templar generally exemplifies a particular ideology of life, and associated nomenclature may depend on the side with which he aligns himself. A good templar, for instance, might assume the title of paladin while those who embrace evil are often known as blackguards and those who serve neutrality are called gray wardens. What truly differentiates these characters are the vows that they swear to uphold.}

\playingaclass{Templars value Charisma greatly, as it allows them to better convince those they encounter of the importance of their deity and provides force to their spells. They also value Strength as it allows them to beat up those who steadfastly refuse to believe and get in the way of the templar's work. Constitution is often the third most important ability for a templar, as it allows them to stand longer in the fray.}

\alignment{Any, though a templar may only select a deity who allows worshipers of the templar's alignment. Conversely, a templar of a specific deity is limited to only those alignments which would be allowed by the deity for a follower. Templars without a patron deity may select any alignment they like.}

\races{Any. Every race that has deities has templars to spread their teachings.}

\startinggold{3d10x10 gp (165 gp).}

\startingage{Moderate}

\hitdie{d10}

\classskills{Appraise (Int), Climb (Str), Concentration (Con), Craft (Int), Heal (Wis), Intimidate (Cha), Jump (Str), Knowledge (nobility and royalty) (religion) (Int), Listen (Wis), Ride (Dex), Sense Motive (Wis), Speak Language (None), Spellcraft (Int), Swim (Str).}

\skillpoints{4}

\afterpage{
\begin{minorcastingclasstable}
\levelone{Divine Vow (Once Vowed), Vow of Piety (Once Vowed)& 						2&-&-&-&-&-&-}
\leveltwo{Avenger of the Faith (Primary)& 															3&-&-&-&-&-&-}
\levelthree{Divine Vow (Once Vowed)& 																3&2&-&-&-&-&-}
\levelfour{Avenger of the Faith (Secondary)& 														3&2&-&-&-&-&-}
\levelfive{Divine Vow (Once Vowed)& 																	3&3&2&-&-&-&-}
\levelsix{Avenger of the Faith (Primary), Arms of the Faithful& 							3&3&2&-&-&-&-}
\levelseven{Divine Vow (Twice Vowed), Vow of Piety (Twice Vowed)& 				3&3&3&2&-&-&-}
\leveleight{Avenger of the Faith (Secondary), Inquisitor& 									3&3&3&2&-&-&-}
\levelnine{Divine Vow (Twice Vowed)& 																3&3&3&2&-&-&-}
\levelten{Avenger of the Faith (Primary)& 															3&3&3&3&2&-&-}
\leveleleven{Divine Vow (Twice Vowed)& 															3&3&3&3&2&-&-}
\leveltwelve{Avenger of the Faith (Secondary), Sustained by Faith& 					3&3&3&3&2&-&-}
\levelthirteen{Divine Vow (Thrice Vowed)& 															3&3&3&3&3&2&-}
\levelfourteen{Avenger of the Faith (Primary)& 													4&3&3&3&3&2&-}
\levelfifteen{Divine Vow (Thrice Vowed), Undying Faith (as raise dead)& 			4&4&3&3&3&2&-}
\levelsixteen{Avenger of the Faith (Secondary)& 													4&4&4&3&3&3&2}
\levelseventeen{Divine Vow (Thrice Vowed)& 														4&4&4&4&3&3&2}
\leveleighteen{Avenger of the Faith (Primary), Undying Faith (as resurrection)&	4&4&4&4&4&3&3}
\levelnineteen{Divine Vow (Thrice Vowed)& 														4&4&4&4&4&4&3}
\leveltwenty{Avenger of the Faith (Secondary), All Things Are Possible& 			4&4&4&4&4&4&4}
\end{minorcastingclasstable}}

\startclassfeatures

\proficiencies{simple and martial weapons, all forms of armor, and all shields.}

\classfeature{Spells}{A templar cast divine spells, which are drawn from the list below and supplemented by their deity's domains (see Vow of Piety). His caster level for these spells is equal to his class level. The save DCs for these spells are equal to 10 + the spell's level + his Charisma modifier. A templar must have a charisma score of at least 10 + the spell's level in order to cast the spell.
\newline
A Templar know all of the spells on his class list, and may cast any of them without preparation so long as he has an appropriate spell slot available and an charisma score of at least 10 + the spell's level. His maximum available slots per day are determined by his class level (as seen on Table: The Templar), and he gains bonus slots from his charisma score.
\newline
In order to receive their spell slots, the templar must pray for 1 hour without interruption in a place free from distractions or noise. At the end of this time, he receives his spell slots. After praying, the templar cannot pray again until one whole day (24 hours) has passed.
A templar’s spells are more for utility than combat efficacy, either allowing him to better solve problems through non-violent means or enhancing his combat abilities past even their already formidable limits.}

\classfeature{Code of Conduct (Ex)}{Like any other character, a templar does what he must to uphold the duties given to him by an organization of which he is a part, even if that organization is as loose as his alignment group. But let’s face it; sometimes even the good and honorable knight may want to lie about his identity or consort with unscrupulous characters in order to root out the evil, demonic cult. And evil knights can be obsessed with battle, honor, and battling with honor. A templar is not specifically prohibited from acts that lie outside of their alignment or run counter to their deity's wishes. Many aspire to these things and most follow them, but not all do so and no templar is punished for being found slightly wanting. Templars who actively displease or betray their deity may still be stripped of their powers and dismissed, however.}

\classfeature{Divine Vow (Su)}{A templar’s code is somewhat variable; different deities and philosophies extol different virtues that a templar must try to uphold. But more than that, each templar is permitted to extol these virtues in slightly different ways. The vows a templar makes are a representation of his personal or religious code, and determine which aspects he attempts to uphold most strongly. These vows grant him extraordinary powers (the nature of which vary based on the vows he takes). These are detailed in the section on divine vows below.
\begin{awesomelist}
	\item At 1st level the templar gains the Vow of Piety and one other rank 1 vow of their choice. At every odd-numbered class level thereafter the templar may take a new vow, but he may not advance one of his existing vows beyond rank 1 at this time.
	\item At 7th level, he reaffirms his Vow of Piety and gains a second domain. He may also reaffirm any other vow which he already possesses to gain the rank 2 ability. A vow that has been reaffirmed in this way is known as "twice vowed." Instead of reaffirming a rank 1 vow, he may instead select two new vows at rank 1. He may not advance a vow beyond rank 2 at this time.
	\item At 13th level, he may reaffirm any other vow in which he already possesses the rank 2 ability to gain the rank 3 ability. A vow that has been reaffirmed in this way is known as "thrice vowed." Instead of reaffirming a rank 2 vow, he may instead select two rank 1 vows at advance to rank 2, or may select a new vow to gain both the rank 1 and rank 2 benefits.
\end{awesomelist}}

\classfeature{Avenger of the Faith}{A templar trains himself in multiple forms of combat, so as to serve as both the weapon and shield of their church or ideals. Starting at second level, he chooses a primary combat form (see Avenger of the Faith Styles) for which he gains the corresponding abilities at 2nd level and every four class levels thereafter. At 4th level, he chooses his secondary style, and gains the benefits thereof at each 4 class levels.}

\classfeature{Arms of the Faithful (Ex)}{At sixth level a templar gains Craft Magic Arms and Armor as a bonus feat. When crafting any magic items with this feat, they are treated as having access to the spells of the war domain in addition to those on their class list. If they already possess Craft Magic Arms and Armor, they may select another item creation feat for which they qualify.}

\classfeature{Inquisitor (Su)}{An eigth level templar can detect the alignments of any creature that he can see as a swift action. He instantly gains all information about their alignment as if he had spent three rounds concentrating on them with the appropriate spells. If the creature is warded, the templar may make a caster level check against the warding spell to gain the information if such a check is allowed by the ward. In addition, all the templar’s attacks are automatically considered aligned (good or evil, lawful or chaotic, etc. based on his alignment) for the purposes of overcoming damage reduction.}

\classfeature{Sustained by Faith (Ex)}{An eleventh level templar gains everything they need to live from their relationship with their deity. They no longer need to eat, drink, breathe, or sleep. They can still do these things if they want to of course.}

\classfeature{Undying Faith (Su)}{Fifteenth level templars are extremely difficult to kill. The templar may elect to gain the benefit of a raise dead spell at any time within 1 minute of being killed. If they do, their return is announced by a powerful flash of light (as a daylight spell) for 1 round. Instead of the normal level loss, they instead suffer 2 points of Charisma burn. Once used, they may not return from the dead in this way for 24 hours; a templar who dies twice in a day will need someone else to bring them back to continue their work. At eighteenth level, this ability improves to offer the benefit of a resurrection spell instead, though the templar only returns with half of their maximum hit points.}

\classfeature{All Things Are Possible (Sp)}{The prayers of a twentieth level templar are taken very seriously. Once per day they may cast miracle as a spell-like ability, though they must still spend experience points if the effect would require them from a spellcaster casting it.}

\classfeature{Ex-Templars}{A templar who wishes to pursue other classes is welcome to do so. There are no multiclssing restrictions against the templar.

A templar who willingly leaves his faith or who is cast out loses all spells, spell-like, and supernatural abilities, as well as any ability stemming from one of their vows. They may return to the faith if a ranking member casts an atonement for them. They may also pursue a new faith entirely. They must still find a member of the faith to atone them, however. When joining a new faith in this way, the templar loses all of their old vows. They may swear a new one each day until they have reached the level allotted them based on their level.}
\vspace{8pt}
\begin{optional}
\subsubsection{Vows}
\noindent\textit{``So many vows, they make you swear and swear. Defend the King, obey the King. Obey your father. Protect the innocent. Defend the weak. What if your father despises the King? What if the King massacres the innocent? It's too much. No matter what you do, you're forsaking one vow or another.''}

\vow{Charity}
{A bone to the dog is not charity. Charity is the bone shared with the dog, when you are just as hungry as the dog.}
{Once per round on your turn you may aid another as a free action.}
{Once per round when you are targeted by a spell with an effect beneficial to you, you may allow another creature within Close Range to also gain the benefits of that spell. The spell must also be beneficial to the creature you wish to share it with (interpreted at the DM's discretion), or the sharing fails.}
{An ally within Close range of you may use your spell slots to cast a spell of an equivalent or lower spell level, so long as you possess the minimum charisma score to use the slot yourself. Your ally may use this slot to cast any spell that they have prepared or that they know (in the case of spontaneous casters), using your slot instead of their own. Your ally may also cast spells from your spell list, even if they would not normally be capable of casting divine spells. Anyone casting a spell in this fashion uses their own attributes, feats, and character level to determine the effects and DC of the spell. They do not need to meet the minimum charisma score requirement for a particular spell level cast from your list, but they must be of a sufficient level that they would be able to use the spell slot were they a templar of the same level.}
{Perhaps your church decrees that its members must give aid to others, or maybe you give out of the goodness of your heart. You are the quintessential selfless knight, giving to others without necessarily thinking of your own gains. There are times when you may give up more important things than money; the truest sacrifice a templar can make is to offer their own life in the service of their cause.}

\vow{Clemency}
{An eye for an eye makes the whole world blind.}
{Whenever you deal lethal damage with a weapon or spell, you may freely opt to deal nonlethal damage instead without suffering a penalty to attack or damage rolls.}
{You may automatically stabilize any creature within Close range of yourself. Additionally, you may keep them from being killed outright through hit point or ability damage. If a creature within Close range would reach -10 hit points or 0 in an attribute, you may instead set them to -9 hit point or 1 in the attribute, whichever is more appropriate. Creatures who are saved from reaching 0 in an ability score are rendered unconscious for 24 hours, though you may rouse them as a standard action at any time before that. You must be aware of a creature to use this ability.}
{You may administer healing or other status restoration effects to creatures who have been dead for less than 1 hour as if they were still alive. If you would heal a creature in such a way that they would not be dead, they recover from that condition without penalty.}
{A good templar may see legitimacy in the concept of defeating enemies in a non-fatal fashion, but it's just as possible that you may simply need to capture them so as to transport them to a more grisly fate.}

\vow{Confrontation}
{In the name of the church, I declare your life forfeit.}
{When you deal damage to a creature with an alignment component opposed to your own, you add your templar level to the damage. A lawful good templar, for example, would add this damage to chaotic or evil creatures. Neutral creatures are considered opposed to creatures with no neutral portion of their alignment. You may suppress this bonus damage at-will.}
{Any weapon you wield gains the benefits of an alignment related weapon ability. Chaotic templars grant the anarchic property, lawful templars grant the axiomatic property, good templars grant then holy property, and evil templars grant the unholy weapon property. If you would qualify for multiple properties, you gain them both. If you qualify for only 1 property, you may gain that one or select one from your neutral alignment axis. If you do not qualify for any property, you may select one.}
{Any foe who suffers additional damage from your alignment related weapon properties must also succeed on a Fortitude save or die. You may suppress this effect at will, and may not combine this with any other strike that would inflict a status condition. A creature that makes their save suffers normal damage from the strike and is immune to this effect until the start of your next turn. If the creature would only suffer additional damage from one weapon property, they gain a +4 bonus to this save. If you are a Neutral templar, the target gains an additional +2 on their save. This is a [Death] effect.}
{You don't back down in the face of your enemy, don't stomach the foes of your faith, and do what you can to quickly remove them from the world. It doesn't really matter what the rest of the world thinks about the plan.}

\vow{Diligence}
{My path to success is simple. I worked hard and I didn't stop until I was finished.}
{You no longer need to sleep 8 hours or trance for 4 hours in a night, instead being sufficiently rested after a single hour. You have no sense of the outside world during this time and are treated as unconscious, though you can be roused in the same way as any sleeper would be. This does not affect the schedule on which you regain spells, and any other classes you possess must still meet any other rest requirement, but it does allow them to craft things twice as quickly (if they have sufficient spells available in the case of magical items) or perform other downtime tasks in half as much time. Further, you gain immunity to any natural or magical effect that would cause you to lose consciousness, aside from the dying condition.}
{You do not suffer the fatigued or exhausted conditions directly. An effect that would normally cause you to be fatigued is reduced to having no effect, while an effect that would normally cause you to become exhausted is instead reduced to fatigued. Should you be exhausted again while suffering fatigue from a previous exhaustion, that still stacks to exhausted as normal.}
{Once per round, you can elect to not be affected by an attack, ability, or other effect that would cause you to die or be transformed into an inanimate form. You may do so even if you have already failed a save against the ability or been successfully hit by the attack. You ignore all parts of it when you ignore it in this fashion.}
{A good templar may work tirelessly for the advancement of a city or group, while an evil one might work tirelessly for their own.  He can sleep when he wants to, or when he's dead.}

\vow{Greed}
{I did it for the hoard of dragon gold. Your village needing help was just a coincidence.}
{As a standard action, you may detect metals and minerals as a Rod of Metal and Mineral Detection for 1 round. After this time you must wait 5 minutes before acting in this fashion again.}
{You can steal the health from those you harm, and recover hit points equal to the damage you deal to a living creature in melee or your class level, whichever is less. This healing cannot restore you beyond your normal hit point total.}
{Once per round when a spell is targeted on another creature within Close range of you, you may also gain the effects of that spell. A spell leeched in this fashion has the same duration (if applicable) for you as it does for the other recipient, but if it ends prematurely for the recipient it does not end for you. This ability is only useable at the moment the spell is cast, but does not grant you any particular knowledge of what spell is being cast. You must be a valid target for the spell; if you are not this ability is not considered expended. You can even use this ability to teleport along with a caster; if you do so you appear in the space next to them instead of in the same space.}
{While many religions place Greed among their sins, being selfish and simply taking your due is seen as a virtue in many eyes. The church also loves money and various assorted shiny things, and has its knights seek to recover either wherever possible (by scrupulous methods as often as not). Or perhaps covetousness and greed is more specific to you, and the church merely puts up with it because they like having badasses who do good things for them.}

\vow{Loyalty}
{If by my life or death I can protect you, I will.}
{Once per round you may intercept an attack, spell, or supernatural effect that specifically targets a creature adjacent to you. When you do this, you become treated as the intended recipient of the attack or effect. You must declare this before an attack roll is made against the target and before the target has made any saves against the effect.}
{Once per round as a free action, you may teleport to a space adjacent to any ally within Close Range (25 feet + 5 ft./2 levels) of you. This provokes attacks of opportunity.}
{You may designate one creature adjacent to you as protected as a free action on your turn. So long as you remain adjacent to them and don't designate a different creature, you grant them full cover and block line of effect from anyone other than yoruself. You may still intercept attacks for other adjacent allies as normal, however, and if you use your ability to teleport to a nearby ally you bring the protected creature with you as well.}
{Whether it's guarding a cleric of the church or some other less individually capable VIP, you protect them with your body and your life.}

\specialvow{Piety}
{I can hear the lord's voice in my ear; such communion is the mark of the truly faithful.}
{You gain one of the domains of your patron deity. If your deity offers more than 5 domains, you must also select which 5 you will have access to. The domain spells are added to your list of spells know, and you use your templar level to determine the strength of the domain power. If you wish, you may select a different domain (subject to the same restrictions) when you pray to restore your spells, losing access to the old domain spells and powers in favor of the new ones.}
{You gain a second domain of your patron deity, and access to its domain power. This domain is subject to all of the use and selection limitations as your first domain.}
{You gain a third domain of your patron deity, and access to its domain power. This domain is subject to all of the use and selection limitations as your first domain.}
{You have a talent for spellcasting that has never measured up to the clerics in the faith, but one that you can pursue should you choose.}
{A templar may have, at most, a total of 5 domains to choose from for the purposes of this vow. If a deity offers more than 5 domains, you must select which 5 you will have access to while you are in their service. If a deity offers less than 5 domains, you may supplement your options with an alignment domain (Chaos, Evil, Good, or Law) provided it matches a component of your own alignment.}

\vow{Perfidity}
{You didn't take my advice. Didn't I tell you not to trust anyone?
}{You gain Bluff as a class skill, and any effect that would interfere with your ability to lie has a 50\% chance to not affect you at all. This is rolled before spell resistance and saving throws.}
{You are shielded by a constant ''[[SRD:Nondetection|nondetection]]'' effect with a caster level equal to your class level. If you successfully block a ''detect'' spell, you may provide instead provide a false reading for the caster of the divination if you wish.}
{You are able to mimic other templars, down to gaining the benefits of vows that they receive. You gain the once vowed and twice vowed ability of one vow that you do not already possess; by meditating without interruption for 8 hours, you may change which vow you possess the abilities of.}
{Deceitful churches employ deceitful templars, able to disguise themselves and assume the mantles of other churches and knightly orders. In order to protect the secrets of your faith, you have sworn to become such a templar.}

\vow{Perseverance}
{Yes, our comrades have fallen. But we still stand, and we shall remember them.}
{You become immune to the shaken and frightened conditions, and only suffer the penalties of the shaken condition if you happen to become panicked. Against [Fear] effects that do not result in one of the above conditions, you gain a +4 bonus on your saves.}
{You can prevent yourself from losing consciousness or dying as a result of hit point loss for 1 round, no matter how low your hit point total falls. You may gain this protection as a swift or immediate action, and it automatically activates in any round you use an ability from your Avenger of the Faith styles.}
{Once per round as a free action you may revive a dead ally within Medium range in order to allow them to keep fighting. This ability lasts until the ally takes damage again, suffers a condition that would kill them, or until the beginning of your next turn. You may revive an ally multiple times with this ability, but may not return them to life permanently without suitable magic.}
{Open to cliches galore. You are the sole survivor of a group of knights slaughtered by some great opponent. Your experience in the horrors of war has seen you lose many comrades, but hardened your body and soul in the face of imminent danger.}

\vow{Purity}
{Cleanliness is next to godliness.}
{You gain immunity to all poisons and diseases (even those of magical nature}
{You gain immunity to [Mind-Affecting] effects cast by those whose alignments are oppose yours. Neutral creatures are considered opposed to creatures with no neutral portion of their alignment.}
{As a move action useable at will, you may purge your system of any negative condition affecting you including: ability penalties (such as from ray of enfeeblement, touch of idiocy, etc.), ability burn, ability damage, ability drain, blindness, confusion, dazing, dazzling, deafness, entanglement, exhaustion, fatigue, fascination, fright, level drain, shaken, panicked, cowering, nausea, paralysis, sickness, stunning, uncenteredness and any other condition that this list does not include but the DM deems permissible.. If you are unable to take a move action but are still conscious, you may purge yourself of one negative effect as a 1-round interruptable action.}
{You keep a clean body, and a clean soul. And maybe you force everyone to try to live that way as well...}

\vow{Taint}
{Watch yourself. You might catch something.}
{If an ally within Close range is afflicted by a harmful condition listed below (death and dying do not count) that could also affect you, you may take a move action to take that condition from them and instead apply it to yourself: ability penalties (such as from ray of enfeeblement, touch of idiocy, etc.), ability burn, ability damage, ability drain, blindness, confusion, dazing, dazzling, deafness, entanglement, exhaustion, fatigue, fascination, fright, level drain, shaken, panicked, cowering, nausea, paralysis, sickness, stunning, uncenteredness and any other condition that this list does not include but the DM deems permissible.}
{You may suppress the effects of one of the above negative status effects currently imposed on yourself. While suppressing it in this fashion, you suffer no penalties for it. You may suppress an effect or select a new effect to suppress once per round as a free action on your turn.<br />Additionally, you may spread your suppressed condition to an enemy struck with a melee attack, forcing them to make saving throws as needed to avoid contracting the same ailment. If they make their save, they are immune to this effect until the start of your next turn. You may not apply this effect when your attack would deliver another status effect. If the ailment stacks, such as negative levels, you may apply it to a target additional times in later rounds.}
{Taint oozes off of you, even when you're otherwise clean. On a successful attack, you may force the target to make a save or become Nauseated for 1 round. If they make their save, you may not attempt to nauseate them again until the start of your next turn. You may not apply this effect when your attack would deliver another status effect, either with the above ability or another spell, feat, or similar feature.}
{Evil power can only be contained by the body and will of good's greatest servants, or harnessed by the most ambitious and ruthless of tyrants.}

\vow{Truth}
{Whoever is careless with the truth in small matters cannot be trusted with important matters.}
{You may radiate a Zone of Truth, as the spell, for 1 round by concentrating as a standard action. This is a supernatural ability useable at will.}
{You may not be compelled to lie or be untruthful to your faith. If a spell would cause you to act against a known adherent to your faith or philosophy (including alignment), break a vow, or lie you may instead state that you are unable to commit such an act and perform no actions for the round. If the effect would end following the completion of the compulsion, as in ''suggestion'', it is automatically discharged and ended at the start of your next turn. Otherwise you gain a new save against the effect, with a +4 bonus.}
{You are constantly under the effects of a True Seeing spell. This is a supernatural ability.}
{Dishonesty really sticks in your craw, and you like to rattle the saber against those who would use treachery and subterfuge. For without truth, how can anything ever be accomplished in the world?}

\vow{Valor}
{Cowards die a thousand times before their deaths, whilst the brave man dies but once.}
{You respond quickly to the threat of a charge. If a charge attack is ever declared against you, you may declare a charge against the opponent charging you as an immediate action. You gain all normal charge benefits on this action. You and the opponent charging meet at the midpoint of your charges, regardless of your respective speeds.}
{Your valor allows you to stand in the face of adversity when others can not. As a swift or immediate action (or as a free-action in any round you use an ability from your Avenger of the Faith styles) you can become rooted to a space unless you elect to move from it. If you are falling or sinking, you immediately cease at your current elevation. Should you allow yourself to fall in later rounds, you suffer falling damage from your new position. Your position can be changed, however, but it requires substantial effort. You gain a bonus equal to twice your templar level on any check or save to resist falling, losing your footing, or being forcibly moved to another space. This protection lasts until the beginning of your next turn. You may take move actions normally while this effect is active.}
{Your fearlessness is terrifying in its own right. On a successful attack, you may force the target to make a save or become Frightened for 1 round. If they make their save, they are immune to this effect until the start of your next turn. You may not apply this effect when your attack would deliver another status effect.}
{You are one of those hardcore zealots who throw themselves at the enemy, striking fear deep into their hearts. It's hard for enemies to fight someone who doesn't fear death.}
\end{optional}

\begin{optional}
\subsubsection{Avenger of the Faith Styles}
As there are many different vows that a templar can swear, so to are there different combat styles that they may practice. A templar selects one of these styles as their primary style and another as a secondary. They are both then advanced as the templar gains levels.

\faith{Charger
}{A charger is a very straightforward templar. They see their foes, and they run or ride out to meet them. This generally leads to the defeat of their foes.
}{\ability{Knight Errant (Ex)}{A charger needs to work around the limitations of the bulky armor that is so often part of his attire. You no longer suffer penalties to your base speed from wearing medium or heavy armor. You also gain additional benefits while charging. You may make 1 turn up to 90 degrees as part of your charge action, though you must still travel at least 10 feet in a straight line immediately before you attack a target. Additionally, you are not required to move to the closest space to your opponent during a charge, and may make your charge attack when your opponent is in any of your threatened spaces. This would allow you to take a charge attack while running past an opponent, but this movement would provoke attacks of opportunity as normal.}
}{\ability{Cataphract (Ex)}{When charging you gain a +4 bonus to your attack roll instead of the normal +2 and you may make a full attack on a charge. You also may charge up to three times your normal base speed when you make a charge as a full-round action. If you would only be limited to a partial charge, you may move twice your base speed as part of that action. You may not make a full-attack when you perform a partial charge, however. This benefit also applies while you are mounted.}
}{\ability{Charge of Necessity (Su)}{While charging or running, you gain the benefit or air walk for the round, until the start of your next turn. If you do not continue running or charging at the start of the next round, you instead fall to the ground under the effect of feather fall. If you begin a fall from other circumstances you do not benefit from this effect. This benefit also applies while you are mounted.}
}{\ability{Charge of Glory (Ex)}{You can trample over those who fall before your charge, continuing to seek more blood. If you destroy an effect in your path, render a charged opponent unconscious or dead, or otherwise clear the way forward while charging you may continue the charge along the same path (following all normal restrictions as they apply) up to your full allowed distance. You may make additional attacks against those in your way along this additional distance as if they were your intended charge target. This benefit also applies while you are mounted.}
}{\ability{Charge of Destruction (Su)}{When a foe is struck with your charge attack and killed, they are destroyed utterly as if they had been immolated or disintegrated. Further, while charging or running you may leave behind a blade barrier as you leaves each space. The wall need not be continuous, and may have as many or as few breaks in it as you desire. This wall deals 15d6 points of damage, has a save DC of 16 + the templar's Charisma modifer, and dissipates at the start of your next turn. This benefit also applies while you are mounted.}
}

\faith{Herald
}{A herald is a shining beacon of the strength of their patron or philosophy. While they generally do so with protective and restorative auras, they are eventually capable of showing the terrible might of their beliefs as well.
}{\ability{Aura of Vitality (Su)}{As a swift or move action, you may radiate a protective divine aura. All designated creatures within Close range (25 ft, +5 feet per 2 class levels) of you when you activate the aura gain its benefits until the start of your next turn. You must have line-of-effect to a creature to designate them, however. You may also exclude yourself from the effect if you prefer. There is no limit to the number of times per day that a herald may create a dine aura. <br />Creatures benefiting from your protective aura gain temporary hit points equal to your class level or your charisma modifier, whichever is higher. These temporary hit points last until used or 1 day has passed, and they do not stack with additional exposure to the aura or with any other source of temporary hit points.}
}{\ability{Aura of Sanctuary (Su)}{Creatures benefiting from your protective aura also gain the effects of the \spell{Sanctuary} spell. If a warded creature takes an offensive action, the sanctuary effect is only broken for them. The effect may be restored next round as long as they remain within range of you when your aura is renewed, however. If a creature attacks any warded creature and successfully saves against the sanctuary effect, they are considered to have saved against it for all creatures protected by your aura. Further, they need not make any additional saves against the sanctuary effect of your aura for 24 hours, and ignore it even if you continue to renew it during that time.}
}{\ability{Aura of Protection (Su)}{Creatures benefiting from your protective aura are protected by a ''protection from X'' spell, where X is any alignment descriptor opposed to your own. Characters with a Neutral alignment may select an opposed alignment. You may change the alignment protected against whenever your aura is renewed.}
}{\ability{Aura of Assistance (Su)}{You may add the benefits of one personal or touch range spell of level 2 or less that you currently benefit from to your protective aura. A creature who is not a legal target for the spell may not gain the benefit of it from your aura, however.}
}{\ability{Otherworldly Aura (Su)}{As a standard action, you can project an otherworldly aura of divine might. You may project this aura in addition to your protective aura, but you must spend both actions to do so. When you project this aura, every creature within close range must make a will save or cower for 1 round. Creatures that are immune to fear are instead dazed for 1 round on a failed save.}
}

\faith{Hoplite
}{Templars who follow the hoplite path are those who feel that a combination of offense and defense is often the most appropriate one to bring against your foes. These templars can wait behind the increased protection of their shield, and then strike with an unexpectedly strong blow.
}{\ability{Vanguard (Ex)}{When wielding a shield in your off hand, you may wield a spear in one hand. When you do so, you still deal damage with the spear as though it is wielded in two hands.}
}{\ability{Resolute Strike (Ex)}{You pour your passion and devotion into your strikes, and your foes can tell. While wielding a shield in your off hand, you may add your Charisma modifier to your damage rolls. If you wield a tower shield, you may also add your Charisma modifier to your attack rolls. Additionally, you may set your spear against a charge as an immediate action.}
}{\ability{Divine Thrust (Ex)}{Your spear thrusts are so strong that you need not even strike a foe with the tip to pierce them. This increases the area you threaten while wielding a spear by 5', and movement through this expanded area provokes attacks of opportunity as normal. If your wielded spear is a reach weapon, this ability does not allow you to attack adjacent targets though you can strike even farther away.}
}{\ability{Warding Strike (Su)}{You spear strikes hurl back the targets you hit. Targets are moved away from you a distance equal to the half the damage dealt (rounded down to the nearest 5 foot increment), with a minimum of 10 feet. If they can not move the full distance, they take 1d6 points of damage for every 2 squares they are unable to move. They land in their destination square prone. A successful reflex save negates the hurling effect.}
}{\ability{Resounding Strike (Su)}{When you strike a foe with your warding strike, you may expand the effect to include all others in a 30' cone which expands away from you with your struck target in the middle. The creature hit by your strike suffers damage normally and makes their save against the effect as above. Other targets within this cone are entitled to a reflex save against the same DC. On a successful save they suffer only half the damage of your strike and are not moved. On a failed save they suffer the full damage of your strike and are hurled as above.}
}

\faith{Protector
}{Protectors understand a simple truth about the world and faith: when faced with throngs of unbelievers or the enemies of your deity, it's important to stand your ground. Which they do quite admirably.
}{\ability{Hardline Stance (Ex)}{You may enter a hardline stance as a move action, and may maintain it additional rounds as either a move or swift action. While holding a hardline stance you are treated as if you had readied an attack against any foe's movement within the spaces you threaten. There is no limit to the number of attacks you can make against moving opponents in this way, and you may make an attack against a foe for each space moved. These attacks are not attacks of opportunity and occur in place of them. You may use an attack of opportunity instead of these bonus attacks if you wish. Your movement rate is reduced to 5' in all movement forms, however, regardless of bonuses or their values before you entered the stance. There is no limit to the number of times in a day when a protector can enter a hardline stance.}
}{\ability{Emenating Stance (Su)}{You visibly radiate a tangible divine energy that can be used to harm foes as well as deflect blows. This grants you reach as if they were one size larger (small and medium are considered to be the same size category for these purposes). Additionally, if you are carrying a shield, the emanations provide you with [[SRD:Cover|cover]]. Your threatened spaces may deflect attacks passing through them, if you wish it, granting cover to creatures targeted by any attack or spell that passes through a space you threaten.}
}{\ability{Hold the Line (Su)}{While holding a hardline stance you may, you may break line of effect across your threatened spaces as a free action on your turn. This break must be a straight line that passes from one side of your threatened area to the opposite side and passes through yourself. It can be maintained for as long as you maintains your hardline stance, but it may only be changed on your turn.}
}{\ability{Persistant Stance (Su)}{Rounds spent in a hardline stance do not count against the duration of any spell that you have cast on yourself.}
}{\ability{Sacred Space (Su)}{When you take on their hardline stance, you may also choose to radiate an effect similar to forbiddance in a 60' radius. You may activate or deactivate this effect as a free action on your turn, but it must remain activated or deactivated for 1 round before you may change it. This effect otherwise lasts as long as you maintains the stance, and it travels with you.\newline
Additionally, you may block any attempt to teleport by a creature that you can see if the shortest distance between their start and end points would pass through this effect. Your own travel powers, those granted by class feature and by spells, are not blocked by this effect. Any creatures who enter on their own suffer the appropriate damage, but creatures within the area of effect when the effect begins, or who find themselves in it because of your movement, do not suffer damage from the effect. Similarly, summoned creatures in the area when the effect is activated are not dispelled, nor are those who wind up in the area of effect as a result of your movement.}
}

\end{optional}

\chapter{Skills}
\section{How Skills Work}
foo
\section{Appraise}
foo
\section{Athletics}
foo
\section{Balance}
foo
\section{Bluff}
foo
\section{Concentration}
foo
\section{Craft}
foo
\section{Decipher Script}
foo
\section{Diplomacy}
foo
\section{Disable Device}
foo
\section{Disguise}
foo
\section{Escape Artist}
foo
\section{Forgery}
foo
\section{Gather Information}
foo
\section{Handle Animal}
foo
\section{Heal}
foo
\section{Intimidate}
foo
\section{Knowledge}
foo
\section{Perception}
foo
\section{Perform}
foo
\section{Profession}
foo
\section{Ride}
foo
\section{Search}
foo
\section{Sense Motive}
foo
\section{Sleight of Hand}
foo
\section{Speak Language}
foo
\section{Spellcraft}
foo
\section{Stealth}
foo
\section{Survival}
foo
\section{Tumble}
foo
\section{Use Magic Device}
foo
%%%%%%%%%%%%%%%%%%%%%%%%
%%Feats Chapter Formatting
%%%%%%%%%%%%%%%%%%%%%%%%
\newcommand{\combatfeat}[8]{
\belowpdfbookmark{#1}{feat:#1}\paragraph{\Large#1}\normalsize\textbf{#2}

\noindent\textit{#3} \\
%\indent\hypertarget{feat:#1}{}\textbf{#1 [Combat] #2} \\
\indent\ability{+0 BAB}{#4} \\
\indent\ability{+1 BAB}{#5} \\
\indent\ability{+6 BAB}{#6} \\
\indent\ability{+11 BAB}{#7} \\
\indent\ability{+16 BAB}{#8} \\
%\begin{description}
%\item[Benefit:] #4
%\item[BAB +1:] #5
%\item[BAB +6:] #6
%\item[BAB +11:] #7
%\item[BAB +16:] #8
%\end{description}
%~\\*
}

\newcommand{\skillfeat}[8]{
\belowpdfbookmark{#1}{feat:#1}\paragraph{\Large#1}\normalsize\textbf{#2}

\noindent\textit{#3} \\
%\indent\hypertarget{feat:#1}{}\textbf{#1 [Skill:#2] #3} \\
\indent\ability{0 Ranks}{#4} \\
\indent\ability{4 Ranks}{#5} \\
\indent\ability{9 Ranks}{#6} \\
\indent\ability{14 Ranks}{#7} \\
\indent\ability{19 Ranks}{#8} \\
%\begin{description}
%\item[Benefit:] #4
%\item[4 Ranks:] #5
%\item[9 Ranks:] #6
%\item[14 Ranks:] #7
%\item[19 Ranks:] #8
%\end{description}
%~\\*
}

\newcommand{\spellfeat}[8]{
\belowpdfbookmark{#1}{feat:#1}\paragraph{\Large#1}\normalsize\textbf{#2}

\noindent\textit{#3} \\
%\indent\hypertarget{feat:#1}{}\textbf{#1 [Spellcasting] #2} \\
\indent\ability{0th Level}{#4} \\
\indent\ability{1st Level}{#5} \\
\indent\ability{3rd Level}{#6} \\
\indent\ability{6th Level}{#7} \\
\indent\ability{9th Level}{#8} \\
%\begin{description}
%\item[Benefit:] #4
%\item[Level 1:] #5
%\item[Level 3:] #6
%\item[Level 6:] #7
%\item[Level 9:] #8
%\end{description}
%~\\*
}

\newcommand{\genfeat}[3]{
\belowpdfbookmark{#1}{feat:#1}\paragraph{\Large#1}\normalsize\textbf{#2}

\noindent\textit{#3} \\
}

\newcommand{\featprereq}[1]{
\indent\ability{Prerequisite}{#1} \\
}

\newcommand{\featbenefit}[1]{
\indent\ability{Benefit}{#1} \\
}

\newcommand{\featspecial}[1]{
\indent\ability{Special}{#1} \\
}

%%%%%%%%%%%%%%%%%%%%%%%%

\chapter{Feats}
\section{How Feats Work}
foo

\section{General Feats}

Foo
\section{Combat Feats}

%\begin{multicols}{2}

\combatfeat{Blind Fighting}{[Combat]}
{You don't have to see to kill.}
{You may reroll your miss chances caused by concealment.}
{While in darkness, you may move your normal speed without difficulty.}
{You have Blindsense out to 60', this allows you to know the location of all creatures within 60'.}
{You have Tremorsense out to 120', this allows you to ``see" anything within 120' that is touching the earth.}
{You cannot be caught flat footed.}
\combatfeat{Blitz}{[Combat]}
{You go all out and try to achieve goals in a proactive manner.}
{While charging, you may opt to lose your Dexterity Bonus to AC for one round, but inflicting an extra d6 of damage if you hit.}
{You may go all out when attacking, adding your Base Attack Bonus to your damage, but provoking an Attack of Opportunity.}
{Bonus attacks made in a Full Attack for having a high BAB are made with a -2 penalty instead of a -5 penalty.}
{Every time you inflict damage upon an opponent with your melee attacks, you may immediately use an Intimidate attempt against that opponent as a bonus action.}
{You may make a Full Attack action as a Standard Action.}
\combatfeat{Combat Looting}{[Combat]}
{You can put things into your pants in the middle of combat.}
{You may sheathe or store an object as a free action.}
{You get a +3 bonus to disarm attempts. Picking up objects off the ground does not provoke an attack of opportunity.}
{As a Swift action, you may take a ring, amulet/necklace, headband, bracer, or belt from an opponent you have successfully grappled. You may pick up an item off the ground in the middle of a move action.}
{If you are grappling with an opponent, you may activate or deactivate their magic items with a successful Use Magic Device check. You may make Appraise checks as a free action.}
{You can take 10 on Use Magic Device and Sleight of Hand checks.}
\combatfeat{Combat School}{[Combat]}
{You are a member of a completely arbitrary fighting school that has a number of recognizable signature fighting moves.}
{First, name your fighting style (such as ``Hammer and Anvil Technique'' or ``Crescent Moon Style'', or ``Way of the Lightning Mace''). This fighting style only works with a small list of melee weapons that you have to describe the connectedness to the DM in a half-way believable way. Now, whenever you are using that technique in melee combat, you gain a +2 bonus on attack rolls.}
{Your immersion in your technique gives you great martial prowess, you gain a +2 to damage rolls in melee combat.}
{When you strike your opponent with the signature moves of your fighting school in melee, they must make a Fortitude Save (DC 10 + 1/2 your level + your Strength bonus) or become dazed for one round. If they succeed on this save they are immune to further dazing attempts for one round.}
{You may take 10 on attack rolls while using your special techniques. The DC to disarm you of a school-appropriate weapon is increased by 4.}
{You may add +5 to-hit on any one attack you make after the first each turn. If you hit an opponent twice in one round, all further attacks this round against that opponent are made with The Edge.}
\combatfeat{Command}{[Combat] [Leadership]}
{You lead tiny men.}
{You have a Command Rating equal to your Base Attack Bonus divided by five (round up).}
{You can muster a group of followers. Your leadership score is your Base Attack Bonus plus your Charisma Modifier.}
{You are able to delegate command to a loyal cohort. A cohort is an intelligent and loyal creature with a CR at least 2 less than your character level. Cohorts gain levels when you do.}
{With a Swift Action you may rally troops, allowing all allies within medium range of yourself to reroll their saves vs. Fear and gain a +2 Morale Bonus to attack and damage rolls for 1 minute. This is a language-dependent ability that may be used an unlimited number of times.}
{Your allies gain a +2 morale bonus to all saving throws if they can see you and you are within medium range.}
\combatfeat{Danger Sense}{[Combat]}
{Maybe Spiders tell you what's up. You certainly react to danger with uncanny effectiveness.}
{You get a +3 bonus on Initiative checks.}
{For the purpose of Search, Spot, and Listen, you are always considered to be ``actively searching". You also get Uncanny Dodge.}
{You may take 10 on Listen, Spot, and Search checks.}
{You may make a Sense Motive check (opposed by your opponent's Bluff check) immediately whenever any creature approaches within 60' of you with harmful intent. If you succeed, you know the location of the creature even if you cannot see it.}
{You are never surprised and always act on the first round of any combat.}
\combatfeat{Elusive Target}{[Combat]}
{You are very hard to hit when you want to be.}
{You gain a +2 Dodge bonus to AC.}
{Your opponents do not gain flanking or higher ground bonuses against you.}
{Your opponents do not inflict extra damage from the power attack option.}
{Diverting Defense -- As an immediate action, you may redirect an attack against you to any creature in your threatened range, friend or foe. You may not redirect an attack to the creature making the attack.}
{As an immediate action, you may make an attack that would normally hit you miss instead.}
\combatfeat{Expert Tactician}{[Combat]}
{You benefit your allies so good they remember you long time.}
{You gain a +4 bonus when flanking instead of the normal +2 bonus. Your allies who flank with you gain the same advantage.}
{You may feint as an Immediate action.}
{As a move action, you may make any 5' square adjacent to yourself into difficult ground.}
{For determining flanking with your allies, you may count your location as being 5' in any direction from your real location.}
{You ignore Cover bonuses less than full cover.}
\combatfeat{Ghost Hunter}{[Combat]}
{You smack around those folks in the spirit world.}
{Your attacks have a 50\% chance of striking incorporeal opponents even if they are not magical.}
{You can hear incorporeal and ethereal creatures as if they lacked those traits (note that shadows and the like rarely bother to actively move silently).}
{You can see invisible and ethereal creatures as if they lacked those traits.}
{Your attacks count as if you had the Ghost Touch property on your weapons.}
{Any Armor or shield you use benefits from the Ghost Touch quality.}
\combatfeat{Giant Slayer}{[Combat]}
{Everyone has a specialty. Yours is miraculously finding ways to stab creatures in the face when it seems improbable that you would be able to reach that high.}
{When you perform a grab on grapple maneuver, you do not provoke an attack of opportunity.}
{You gain a +4 Dodge bonus to your AC and Reflex Saves against attacks from any creature with a longer natural reach than your own.}
{You have The Edge against any creature you attack that is larger than you. Also, an opponent using the Improved Grab ability on you provokes an attack of opportunity from you. You may take this attack even if you do not threaten a square occupied by your opponent.}
{When you attempt to trip an opponent, you may choose whether your opponent resists with Strength or Dexterity.}
{When involved in an opposed bull rush, grapple, or trip check as the attacker or defender, you may negate the size modifier of both participants. You may not choose to negate the size modifier of only one character.}
\combatfeat{Great Fortitude}{[Combat]}
{You are so tough. Your belly is like a prism.}
{You gain a +3 bonus to your Fortitude Saves.}
{You die at -20 instead of -10.}
{You gain 1 hit point per level.}
{You gain DR of 5/-.}
{You are immune to the fatigued and exhausted conditions. If you are already immune to these conditions, you gain 1 hit point per level for each condition you were already immune to.}
\combatfeat{Horde Breaker}{[Combat]}
{You kill really large numbers of people.}
{You gain a number of extra attacks of opportunity each round equal to your Dexterity Bonus (if positive).}
{Whenever you drop an opponent with a melee attack, you are entitled to a bonus ``cleave" attack against another opponent you threaten. You may not take a 5' step or otherwise move before taking this bonus attack. This cleave attack is considered an attack of opportunity.}
{You may take a bonus 5' step every time you are entitled to a cleave attack, which you may take either before or after the attack.}
{You may generate an aura of fear on any opponents within 10' of yourself whenever you drop an opponent in melee. The save DC is 10 + the Hit Dice of the dropped creature.}
{Opponents you have the Edge against provoke an attack of opportunity from you by moving into your threatened area or attacking you.}
\combatfeat{Hunter}{[Combat]}
{You can move around and shoot things with surprising effectiveness.}
{The penalties for using a ranged weapon from an unstable platform (such as a ship or a moving horse) are halved.}
{Shot on the Run -- you may take a standard action to attack with a ranged weapon in the middle of a move action, taking some of your movement before and some of your movement after your attack. That still counts as your standard and move action for the round.}
{You suffer no penalties for firing from unstable ground, a running steed, or any of that.}
{You may take a full round action to take a double move and make a single ranged attack from any point during your movement.}
{You may take a full round action to run a full four times your speed and make a single ranged attack from any point during your movement. You retain your Dexterity modifier to AC while running.}
\combatfeat{Insightful Strike}{[Combat]}
{You Hack people down with inherent awesomeness.}
{You may use your Wisdom Modifier in place of your Strength Modifier for your melee attack rolls.}
{Your attacks have The Edge against an opponent who has a lower Wisdom and Dexterity than your own Wisdom, regardless of relative BAB.}
{Your melee attacks have a doubled critical threat range.}
{You make horribly telling blows. The extra critical multiplier of your melee attacks is doubled (x2 becomes x3, x3 becomes x5, and x4 becomes x7).}
{Any Melee attack you make is considered to be made with a magic weapon that has an enhancement bonus equal to your Wisdom Modifier (if positive).}
\combatfeat{Iron Will}{[Combat]}
{You are able to grit your teeth and shake off mental influences.}
{You gain a +3 bonus to your Willpower saves.}
{You gain the slippery mind ability of a Rogue.}
{If you are stunned, you are dazed instead.}
{You do not suffer penalties from pain and fear.}
{You are immune to compulsion effects.}
\combatfeat{Juggernaut}{[Combat]}
{You are an unstoppable Juggernaut.}
{You may be considered one size category larger for the purposes of any size dependant roll you make (such as a Bull Rush, Overrun, or Lift action).}
{You do not provoke an attack of opportunity for entering an opponent's square.}
{You gain a +4 bonus to attack and damage rolls to destroy objects. You may shatter a Force Effect by inflicting 30 damage on it.}
{When you successfully bullrush or overrun an opponent, you automatically Trample them, inflicting damage equal to a natural slam attack for a creature of your size.}
{You gain the Rock Throwing ability of any standard Giant with a strength equal to or less than yourself.}
\combatfeat{Lightning Reflexes}{[Combat]}
{You are fasty McFastFast. It helps keep you alive.}
{You gain a +3 bonus to your Reflex saves.}
{You gain Evasion, if you already have Evasion, that stacks to Improved Evasion.}
{You may make a Balance Check in place of your Reflex save.}
{You gain a +3 bonus to your Initiative.}
{When you take the Full Defense Action, add your level to your AC.}
\combatfeat{Mage Slayer}{[Combat]}
{You have trained long and hard to kill magic users. Maybe you hate them, maybe you just noticed that most of the really dangerous creatures in the world use magic.}
{You gain Spell Resistance of 5 + Character Level.}
{Damage you inflict is considered ``ongoing damage" for the purposes of concentration checks made before the beginning of your next round. All your attacks in a round are considered the same source of continuing damage.}
{Creatures cannot cast defensively within your threat range.}
{Your attacks ignore Deflection bonuses to AC.}
{When a creature uses a [Teleportation] effect within medium range of yourself, you may choose to be transported as well. This is not an action.}
\combatfeat{Murderous Intent}{[Combat]}
{You stab people in the face.}
{You may make a Coup de Grace as a standard action.}
{When you kill an opponent, you gain a +2 Morale Bonus to your attack and damage rolls for 1 minute.}
{Once per round, you may take an attack of opportunity against an opponent who is denied their Dexterity bonus to AC.}
{You may take a Coup de Grace action against opponents who are stunned.}
{You may take a Coup de Grace action against opponents who are dazed.}
\combatfeat{Phalanx Fighter}{[Combat]}
{You fight well in a group.}
{You may take attacks of opportunity even while flat footed.}
{Any Dodge bonus to AC you gain is also granted to any adjacent allies for as long as you benefit from the bonus and your ally remains adjacent.}
{Charging is an action that provokes an attack of opportunity from you. This attack is considered to be a ``readied attack" if it matters for purposes like setting against a charge.}
{You may attack with a reach weapon as if it was not a reach weapon. Thus, a medium creature would normally threaten at 5' and 10' with a reach weapon.}
{You may take an Aid Another action once per round as a free action. You provide double normal bonuses from this effect.}
\combatfeat{Point Blank Shot}{[Combat]}
{You are crazy good using a ranged weapon in close quarters.}
{When you are within 30' of your target, your attacks with a ranged weapon gain a +3 bonus to-hit.}
{You add your base attack bonus to damage with any ranged attack within the first range increment.}
{You do not provoke an attack of opportunity when you make a ranged attack.}
{When armed with a Ranged Weapon, you may make attacks of opportunity against opponents who provoke them within 30' of you. Movement within this area does not provoke an attack of opportunity.}
{With a Full Attack action, you may fire a ranged weapon once at every single opponent within the first range increment of your weapon. You gain no additional attacks for having a high BAB. Make a single attack roll for the entire round, and compare to the armor class of each opponent within range.}
\combatfeat{Sniper}{[Combat]}
{Your shooting is precise and dangerous.}
{Your range increments are 50\% longer than they would ordinarily be. Any benefit of being within 30' of an opponent is retained out to 60'.}
{You do not suffer a -4 penalty when firing a ranged weapon into melee and never hit an unintended target in close combats or grapples.}
{Your ranged attacks ignore Cover Bonuses (total cover still bones you).}
{Opponents struck by your ranged attacks do not automatically know what square your attack came from, and must attempt to find you normally.}
{Any time you hit an opponent with a ranged weapon, it is counted as a critical threat. If your weapon already had a 19-20 threat range, increase its critical multiplier by 1.}
\combatfeat{Subtle Cut}{[Combat]}
{You cut people so bad they have to ask you about it later.}
{Any time you damage an opponent, that damage is increased by 1.}
{As a standard action, you can make a weapon attack that also reduces a creature's movement rate. For every 5 points of damage this attack does, reduce the creature's movement by 5'. This penalties lasts until the damage is healed.}
{As a standard action, you may make a weapon attack that also does 2d4 points of Dexterity damage.}
{Any weapon attack that you make at this level acts as if the weapon had the wounding property.}
{As a standard action, you may make an attack that dazes your opponent. This effect lasts one round, and has a DC of 10 + half your level + your Intelligence bonus.}
\combatfeat{Two Weapon Fighting}{[Combat]}
{When armed with two weapons, you fight with two weapons rather than picking and choosing and fighting with only one. Kind of obvious in retrospect.}
{You suffer no penalty for doing things with your off-hand. When you make an attack or full-attack action, you may make a number of attacks with your off-hand weapon equal to the number of attacks you are afforded with your primary weapon.}
{While armed with two weapons, you gain an extra Attack of Opportunity each round for each attack you would be allowed for your BAB, these extra attacks of opportunity must be made with your off-hand.}
{You gain a +2 Shield Bonus to your armor class when fighting with two weapons and not flat footed.}
{You may feint as a Swift action.}
{While fighting with two weapons and not flat footed you may add the enhancement bonus of either your primary or your off-hand weapon to your Shield Bonus to AC.}
\combatfeat{Weapon Finesse}{[Combat]}
{You are incredibly deft with a sword.}
{You may use your Dexterity Modifier instead of your Strength modifier for calculating your melee attack bonus.}
{Your special attacks are considered to have the Edge when you attack an opponent with a Dexterity modifier smaller than yours, even if your Base Attack Bonus is not larger.}
{You may use your Dexterity modifier in place of your Strength modifier when attempting to trip an opponent.}
{You may use your Dexterity modifier in place of your Strength modifier for calculating your melee damage.}
{Once per turn, when an opponent is struck, you may take an attack of opportunity on that opponent.}
%\combatfeat{Weapon of Righteous Destruction}{[Combat]}
{Your hands make whatever is being held by them holy and on fire. For some reason this doesn't make them melt or burn up.}
{Whatever weapon you are wielding is considered Magical (+\sfrac{1}{3} bonus/level) in addition to any other properties that it has. Your unarmed attacks, even if not proficient, count for this effect.}
{The above, plus Flaming.}
{The above, Holy instead of Flaming.}
{The above, plus Sun weapon, Fort save. (BoG)}
{The above, plus Vorpal weapon (BoG).}
\combatfeat{Whirlwind}{[Combat]}
{You are just as dangerous to everyone around you as to anyone around you.}
{As a full round action, you may make a whirlwind attack - you may make a single attack against each opponent you can reach. Roll one attack roll and compare to each available opponent's AC individually.}
{You gain a +3 bonus to Balance checks.}
{When you make a whirlwind attack, you may also take a regular move action. You may make a single attack against each opponent you can reach at any point during your movement. Roll one attack roll and compare to each available opponent's AC individually, as normal.}
{Until your next round after making a whirlwind attack, you may take an attack of opportunity against any opponent that enters your threatened area.}
{When you make a whirlwind attack, you may also take a double move action as if you had charged. You overrun any creature in your path and may make a single attack against each opponent you can reach at any point during your movement. Roll one attack roll and compare to each available opponent's AC individually, as normal.}
\combatfeat{Zen Archery}{[Combat]}
{You are very calm about shooting people in the face. That's a good place to be.}
{You may use your Wisdom Modifier in place of your Dexterity Modifier on ranged attack rolls.}
{Any opponent you can hear is considered an opponent you can see for purposes of targeting them with ranged attacks.}
{If you cast a Touch Spell, you can deliver it with a ranged weapon (though you must hit with a normal attack to deliver the spell).}
{As a Full Round Action, you may make one ranged attack with a +20 Insight bonus to hit.}
{As a Full Round Action, you may make one ranged attack with a +20 Insight bonus to hit. If this attack hits, your attack is automatically upgraded to a critical threat. If the threat range of your weapon is 19-20, your critical multiplier is increased by one.}

%\end{multicols}
\section{Skill Feats}

%\begin{multicols}{2}

\skillfeat{Acquirer's Eye}{[Skill:Appraise]}
{You know what you want, even if other people have it right now.}
{You gain +3 to your Appraise checks.}
{You automatically know if something is ordinary, masterwork, or magic when looking at it.}
{You can discover the properties of a magic item, including how to activate it (if appropriate) and how many charges are left (if it has them), with a successful Appraise check (DC item's caster level + 10) and 10 minutes of work.}
{Once per round as a free action, you can examine a magic item and attempt an Appraise check (DC item's caster level + 20) to determine its properties, including its functions, how to activate those functions (if necessary), and how many charges it has left (if it has charges).}
{You know what the most valuable piece of treasure is in any collection, such as the most valuable magic item an enemy is wearing or the most valuable object in a dragon's horde, just by looking at the collection. You automatically recognize an artifact when looking at it.}
\skillfeat{Acrobatic}{[Skill:Tumble]}
{You can totally flip out and kill someone with your gymnastic prowess.}
{You gain a +3 bonus to Tumble checks.}
{When using the Combat Expertise option, your dodge bonus to AC increases by +1. This further increases by +1 for every ten ranks of Tumble you have (+2 at 14, +3 at 24, and so on).}
{If an opponent attempts to bull-rush, overrun, or trample you, if you succeed on Tumble check of DC 25 + their base attack bonus, their movement continues in a straight line to the maximum allowed by their speed, you remain where you were, and you don't suffer from the effects of their bull-rush, overrun, or trample. If you fail, you provoke an attack of opportunity from that enemy.}
{If you succeed on a DC 40 Tumble check, you can move 10 feet when taking a 5-foot step.}
{If you succeed on a Tumble check against a DC of 30 + an opponent's base attack bonus, an action that would normally provoke an attack of opportunity doesn't.}
\skillfeat{Alertness}{[Skill:Listen]}
{Your ears are so sharp you probably wouldn't miss your eyes.}
{You gain a +3 bonus to Listen checks.}
{You can make a Listen check once a round as a free action. You don't take penalties for distractions on your Listen checks.}
{You gain Blindsense to 60 feet. You don't take penalties for ambient noise, such as loud winds. Divide any distance penalties you take on Listen checks by two.}
{You gain Blindsight to 120 feet.}
{You can hear through magical silence and similar effects, but you take a -20 penalty on your check. Divide any distance penalties you take on Listen checks by five.}
\skillfeat{Animal Affinity}{[Skill:Handle Animal]}
{You're one of those people animals just won't leave alone for no apparent reason.}
{You gain the wild empathy ability, with your check equal to your character level plus your Charisma modifier plus any other applicable bonuses. If you already have wild empathy, or later gain it from another source, you gain a +3 bonus on Handle Animal checks.}
{You can handle an animal as a free action, and push it as a move action.}
{You gain the benefits of \spell{speak with animals} permanently as an extraordinary ability. The DCs for you to rear and train creatures are halved.}
{With a DC 30 Handle Animal check, you can use a mass version of \spell{charm animal} as a spell-like ability, with save DC equal to 10 + \half your character level + your Cha modifier and effective caster level equal to your bonus on Handle Animal checks.}
{You can summon animals to your aid. Choose an animal with a CR equal to or less than your character level, and make a Handle Animal check at a DC of 25 + your character level. If you succeed, you summon a number of animals depending on how much the animal's CR is less than your character level for an hour. You can't use this ability again until any animals you've summoned with it have unsummoned or you've dismissed them.

\vspace{2pt}
\begin{tabular}{l l|l l}
	\textbf{CR} &\textbf{Appearing} &\textbf{CR} &\textbf{Appearing}\\
	Level - 1 &1   &Level - 11 &15+3d10\\
	Level - 2 &1d3 &Level - 12 &40\\
	Level - 3 &1d4 &Level - 13 &50\\
	Level - 4 &1d6 &Level - 14 &60\\
	Level - 5 &1d8 &Level - 15 &80\\
	Level - 6 &1d10 &Level - 16 &100\\
	Level - 7 &2d6 &Level - 17 &150\\
	Level - 8 &3d6 &Level - 18 &200\\
	Level - 9 &3d10 &Level - 19 &300 \\
	Level - 10 &10+3d6 &Level - 20 &450
\end{tabular}}
%\listone
%	\bolditem{CR;}{Number Appearing}
%	\bolditem{Level - 1;}{1}
%	\bolditem{Level - 2;}{1d3}
%	\bolditem{Level - 3;}{1d4}
%	\bolditem{Level - 4;}{1d6}
%	\bolditem{Level - 5;}{1d8}
%	\bolditem{Level - 6;}{1d10}
%	\bolditem{Level - 7;}{2d6}
%	\bolditem{Level - 8;}{3d6}
%	\bolditem{Level - 9;}{3d10}
%	\bolditem{Level - 10;}{10+3d6}
%	\bolditem{Level - 11;}{15+3d10}
%	\bolditem{Level - 12;}{40}
%	\bolditem{Level - 13;}{50}
%	\bolditem{Level - 14;}{60}
%	\bolditem{Level - 15;}{80}
%	\bolditem{Level - 16;}{100}
%	\bolditem{Level - 17;}{150}
%	\bolditem{Level - 18;}{200}
%	\bolditem{Level - 19;}{300}
%\end{list}
\skillfeat{Army of Demons}{[Celestial] [Fiend] [Leadership] [Skill:Knowledge (The Planes)]}
{You have an army of planar crazy crap.}
{You have a Command Rating equal to your Knowledge (The Planes) ranks divided by five (round up).}
{You can muster a group of followers. Your leadership score is your ranks in Knowledge: Planes plus your Charisma mod. These followers can and must be outsiders.}
{Your followers swell in number to that of an army.}
{You own a planar stronghold.}
{Your allies gain a +2 morale bonus to all saving throws if they can see you and you are within medium range.}
\skillfeat{Battlefield Surgeon}{[Skill:Heal]}
{You like to cut people open with a saw. But it's good for them. Seriously.}
{You gain +3 to your Heal checks.}
{You can make first aid, treat poison, and treat wound checks as move actions.}
{For every 5 points your Heal check exceeds the DC for long term care, your patients recover another +100\% faster. For instance, if your Heal check result is 23, your patients would heal at thrice the normal rate.}
{If you operate on a patient for a minute, they regain hit points equal to your Heal check result. You also may, instead of healing hit point damage, cure any condition that heal could, reattach severed limbs, or repair ruined organs, if you succeed on a DC 30 check. Patients under your long-term care heal permanent ability drain as if it was ability damage.}
{With one hour of work, 25,000 gp worth of materials (which are consumed in the process), and a DC 40 Heal check, you can restore a creature that died within the last twenty-four hours to life. The subject's soul must be free and willing to return for the effect to work.}
\skillfeat{Bureaucrat}{[Leadership] [Skill:Appraise]}
{You have a functioning guild that makes stuff for you and gives you money.}
{You draw an income for working as an administrator, getting 1 GP/week per rank in Appraise.}
{You can muster a group of followers. Your leadership score is your ranks in Appraise plus your Intelligence modifier. These followers all have profession and craft skills.}
{You get your own Stronghold.}
{You get a +2 bonus to profit checks.}
{Your guild goes planar, your number of followers swell to the size of an army and their ranks start filling up with producers and managers from other planes of existence.}

\skillfeat{Combat Casting}{[Skill:Concentration]}
{Having a sword sticking out of your chest doesn't noticeably impede your ability to do\ldots well, just about anything.}
{You gain +3 to your Concentration checks.}
{You can take 10 on Concentration checks and caster level checks.}
{You may maintain concentration on a spell as a move action (DC 25 + spell level). If you beat the DC by 10 or more, you can maintain concentration as a swift action. If you fail your check, you lose concentration.}
{If you would be nauseated, you're sickened instead.}
{All Concentration DCs are halved for you.}
\skillfeat{Con Artist}{[Skill:Bluff]}
{You can fool some of the people, all of the time.}
{You gain a +3 bonus to Bluff checks.}
{Magic effects that would detect your lies or force you to speak the truth must succeed on a caster level check with DC equal to 10 plus your ranks in Bluff or fail.}
{Divination magic used on you detects a false alignment of your choice. You can present false surface thoughts to \spell{detect thoughts} and similar effects, changing your apparent Intelligence score (and thus your apparent mental strength) by as much as 10 points and can place any thought in your ``surface thoughts" to be read by such spells or effects.}
{If you beat someone's Sense Motive check by 25, you can instill a \spell{suggestion} in them, as the spell. This suggestion lasts for one hour for each of your character levels.}
{You are protected from all spells and effects that detect or read emotions or thoughts, as by \spell{mind blank}.}
\skillfeat{Cryptographer}{[Skill:Decipher Script]}
{You're good at reading things no one intended you to.}
{You gain +3 to your Decipher Script checks.}
{You can decipher a written spell (like a scroll) without using \spell{read magic}, if you succeed on a Decipher Script check of DC 20 + the spell's level. You can try once per day on any particular written spell.}
{You don't trigger written magic traps (like \spell{explosive runes} or \spell{symbols}) by reading them. You can disable them with Decipher Script as if you were using Disable Device. You can read the material hidden by a \spell{secret page} with a DC 25 Decipher Script check.}
{When you cast a spell from a scroll, the spell's save DC is equal to 10 + the spell's level + your Intelligence modifier + any other applicable bonuses, and its caster level is equal to your character level, plus other applicable bonuses.}
{Reading text using Decipher Script is a free action for you. You may disable written magical traps as a swift action, and you can cast 5th-level or lower spells from scrolls as a swift action.}
\skillfeat{Deft Fingers}{[Skill:Sleight of Hand]}
{Your amazing manual dexterity is the talk of princes and princesses.}
{You gain a +3 bonus on your Sleight of Hand checks.}
{If you draw a hidden weapon and attack with it in the same round, your opponent loses their Dexterity bonus to AC against your first attack with that weapon that round. This ability can only be used once per round.}
{You can make an adjacent creature or object your size or smaller ``disappear" with your legerdemain. If you succeed on a DC 30 Sleight of Hand check as a standard action, your target can make a Hide check, or you can make the Hide check for them or it. As usual, you can hide larger creatures or objects by taking a -20 cumulative penalty for each size category larger they are than you.}
{With a DC 30 Sleight of Hand check, you can use \spell{shrink item} as a spell-like ability.}
{With a DC 40 Sleight of Hand check, you can use \spell{teleport object} as a spell-like ability. You can also retrieve items placed in the Ethereal Plane using \spell{teleport object}. With a DC 40 Sleight of Hand check, you can use \spell{instant summons} as a spell-like ability without requiring \spell{arcane mark}, but you may only designate one item at a time.}
\skillfeat{Detective}{[Skill:Gather Information]}
{You're good at finding things out just by conversing with townsfolk.}
{You gain a +3 bonus on your Gather Information checks.}
{Your ability to pick up on the social context aids you in establishing rapport. After succeeding on a Gather Information check, you gain a +2 bonus to Knowledge checks, Sense Motive checks, and checks for Cha-based skills in the same milieu.}
{With 2d6 hours of research, you can study a specific topic, such as a particular location or a well-known local monster, and substitute a Gather Information check for any Knowledge checks pertaining to the topic. You need access to local informants, a library, scholars, or other appropriate sources to use this ability.}
{You can gain the benefits of \spell{legend lore} with a DC 30 Gather Information check. If you have the person or thing at hand, or are in the place, this takes a day; otherwise, it consumes the time as normal for \spell{legend lore}. You need access to individuals or resources with relevant knowledge to use this ability.}
{With a DC 40 Gather Information check and 1d4+1 days of talking to people, you can either find an answer to any question you can pose in ten words or less, or find out where you need to go to get the answer. You need access to individuals or resources with relevant knowledge to use this ability.}
\skillfeat{Dreadful Demeanor}{[Skill:Intimidate]}
{People know you're a badass motherfvcker the instant you enter the room.}
{You gain +3 to your Intimidate checks.}
{You can demoralize an opponent as a move action.}
{Opponents you've demoralized remain \condition{shaken} until they lose sight of you.}
{Opponents who would be \condition{panicked} because of your fear effects are \condition{cowered} instead for the duration of the effect.}
{Any time you confirm a critical hit in melee, your target is \condition{cowered} until they lose sight of you. This is a fear effect.}
\skillfeat{Expert Counterfeiter}{[Skill:Forgery]}
{You aren't a common forger, you're an \textit{}artiste.}
{You gain a +3 bonus to Forgery checks.}
{When creating a forgery, you roll twice and take the better result.}
{In situations where you can present a legal document of some sort, you can substitute a Forgery check for a Bluff, Diplomacy, or Intimidate check.}
{You can purchase items with counterfeit bills of exchange, falsified credit vouchers, and the like. You can acquire any item available through the gold economy in this method. Normally, your counterfeits are so good they don't provoke suspicion, but if someone examines them, they must still beat you in an opposed Forgery check to recognize they're not the real thing.}
{You can duplicate a scroll with eight hours of work and a Forgery check against DC 35 + the spell's level. The duplicate functions in all manners like the original scroll. You must have appropriate materials on hand for scribing the scroll, and if the spell requires XP or expensive material components, you must provide the requisite components or make up the XP cost in materials.}
\skillfeat{Ghost Step}{[Skill:Move Silently]}
{You might as well be incorporeal for all the noise you make.}
{You gain +3 to your Move Silently checks.}
{Anyone attempting to use Survival to track you must beat you in an opposed check against Move Silently.}
{Creatures with blindsense, blindsight, tremorsense, or similar abilities do not automatically detect your presence, but must succeed on a Listen check, opposed by your Move Silently check, to notice you.}
{With success on a DC 30 Move Silently check as a standard action, you can control ambient sounds within 30 feet of yourself for a round. You can specifically duplicate any effect from \spell{control sound} (XPH), \spell{silence}, or \spell{ventriloquism}, and in general can make sound you've heard come from any part of the area, displace sounds in the area, or suppress any sounds or sounds. Also, if you take a -10 DC penalty on your Move Silently check, anyone within 30 feet of you can substitute your check result for their own.}
{You're so quiet that people don't even remember you when you're standing right next to them. Your opponents count as flat-footed whenever you attack them.}
\skillfeat{Investigator}{[Skill:Search]}
{You have an eye for detail and so much patience that going through a 100' by 100' room inch-by-inch doesn't even try it.}
{You can use Search to find traps like a character with trapfinding. If you already have that ability, you gain +3 to your Search checks. Search is always a class skill for you.}
{You can Search a 10' by 10' area with a full-round action.}
{You automatically sense any active magic effects in an area you search. If you succeed on a DC 20 Search check, you can determine their number, strength, and school, as if using \spell{detect magic}.}
{You can Search objects or areas within 30 feet of yourself. You can make a Search check as a swift action.}
{You have an intuitive sense for hidden things. Anytime something that someone has hidden is within 60 feet of you, you know it; if there are multiple things, you know how many. However, you must still make Search checks as normal to locate them.}
\skillfeat{Item Master}{[Skill:Use Magic Device]}
{You make magic items do things you want.}
{You gain a +3 bonus to Use Magic Device checks.}
{You don't suffer mishaps with magic items.}
{When rolling Use Magic Device checks or random effects from magic items, you may roll twice and take the better result.}
{With a swift action and a successful Use Magic Device check against a DC of 30 + the item's caster level, you can gain the benefits of a slotted magic item without needing to have a slot available (for instance, a third ring on your finger) for one round.}
{When you activate a wand or staff, you can substitute a spell slot instead of using a charge. The spell slot must be one you have not used for the day, though you may lose a prepared spell to emulate a wand charge (you may not lose prepared spells from your school of specialty, if any). The spell slot lost must be equal to or higher in level than the spell stored in the wand, including any level-increasing metamagic enhancements. When using spell trigger, spell completion, or other consumable magic items, if you succeed on a Use Magic Device check of 40 + the caster level of the item as a swift action, the item or charges thereof are not consumed.}
\skillfeat{Leadership}{[Leadership] [Skill:Diplomacy]}
{You convince people that obeying you is a good career move.}
{You can awe even strangers and enemies into following your orders. With a DC 20 Diplomacy check, you can use \spell{command} as a spell-like ability, with save DC equal to 10 + \half your character level + your Cha modifier.}
{Your natural talent for leaderships attracts followers. Your leadership score is equal to your ranks in Diplomacy plus your Charisma modifier.}
{You persuade someone that you are so awesome that they should follow you around all the time, acquiring a cohort. A cohort is an intelligent and loyal creature with a CR at least 2 less than your character level. Cohorts gain levels when you do.}
{Your natural majesty stirs guilt in those who refuse your demands. With a DC 30 Diplomacy check, you can use \spell{geas} as a spell-like ability, but it offers a Will save at DC 10 + \half your character level + your Cha modifier.}
{You command the loyalty of armies\ldots even opposing ones. With a DC 40 Diplomacy check, you can use \spell{greater command} as a spell-like ability, with save DC equal to 10 + \half your character level + your Cha modifier and effective caster level equal to your bonus on Diplomacy checks.}
\skillfeat{Legendary Wrangler}{[Skill:Use Rope]}
{No one can tell where you end and your ropes begin.}
{You gain a +3 bonus to Use Rope checks and proficiency with the bolas, net, and whip.}
{You can use a rope as if it was a bolas or whip, and you can substitute your ranks in Use Rope for your Base Attack Bonus for combat maneuvers made with it. You can also use it as a net, replacing the normal DC 20 Escape Artist check for someone entangled with it with your Use Rope check. You can throw a grappling hook, tie a knot, tie a special knot, or tie a rope around yourself one-handed as a move action. You don't provoke attacks of opportunity for using Use Rope.}
{You can use a rope, whip, grappling hook, or similar item to manipulate any item within 30 feet of yourself as easily as if it was in your hands; you can also make disarm, entangling (as if with a net), and trip attempts with it. You can move around on ropes and similar structures, like webs, as easily as you can on the ground.}
{With a DC 30 Use Rope check, you can use \spell{animate rope} as a spell-like ability; you can use any ability you can with an ordinary rope with an animated rope.}
{You can manipulate items out to 60 feet with ropes and similar items. You can use ropes for the grab on and hold down grapple maneuvers. When using combat maneuvers with ropes, you can replace the relevant check (disarm, grapple, trip, etc.) with a Use Rope check.}
\skillfeat{Lord of Death}{[Leadership] [Necromantic] [Skill:Knowledge (Religion)]}
{A whole bunch of skeletons and crap show up to fight under your tattered banner.}
{You have a Command Rating equal to your ranks in Knowledge (Religion divided by five (round up). You are a Necromantic leader (see Heroes of Battle).}
{You can muster a group of followers. Your leadership score is your ranks in Knowledge Religion plus your Wisdom modifier. Your followers are all mindless Undead. You don't make them or anything, they just show up.}
{You are able to delegate command to a loyal cohort. Your cohort is an intelligent and loyal Undead creature with a CR at least 2 less than your character level. Cohorts gain levels when you do.}
{Your followers swell in number to that of an army.}
{Your allies gain energy resistance to Positive Energy equal to your level while they are within line of sight of you.}
\skillfeat{Magical Aptitude}{[Skill:Spellcasting]}
{You're crazy good at manipulating magic.}
{You gain a +3 bonus on Spellcraft checks.}
{When counterspelling, you may use a spell of the same school that is one or more spell levels higher than the target spell.}
{You can dismiss a spell as a free action. You can redirect a spell as a move action, if it normally requires a standard action, or a swift action, if it normally takes a move action. You gain a +3 bonus on dispel checks.}
{You can counter a spell as an immediate action.}
{You automatically know which spells or magic effects are active on upon any individual object you see, as if you had \spell{greater arcane sight} active on yourself.}
\skillfeat{Many-Faced}{[Skill:Disguise]}
{You change identities so often even you don't remember what you look like anymore.}
{You gain +3 to your Disguise checks.}
{When creating a disguise, you roll twice and take the better result.}
{You can use \spell{Nystul's magic aura} as a spell-like ability at will, with a caster level equal to your character level and a save DC of 10 + \half your character level + your Cha modifier.}
{You can create a disguise as a full-round action, but you take a -10 penalty to your Disguise check. You can't be under direct observation while doing this, but you can use Bluff to create a diversion to allow you to change guises, as for the Hide skill.}
{You can choose an appearance that anyone viewing you with scrying or other divination magic sees instead of your ``real" appearance. Even someone who benefits from \spell{true seeing} must succeed on a caster level check (DC 11 + your ranks in Disguise) to penetrate the illusion.}
\skillfeat{Master of Terror}{[Leadership] [Skill:Intimidate]}
{You scare people so bad they follow you around hoping you won't hurt them.}
{Whenever you use Intimidate in combat, it affects everyone within 30 feet of you.}{
You gain followers. Your leadership score is equal to your ranks in Intimidate plus your Charisma modifier.}
{You gain a cohort who enjoys frightening your underlings almost as much as you do. A cohort is an intelligent and loyal creature with a CR at least 2 less than your character level. Cohorts gain levels when you do.}
{You gain the frightful presence ability. When you speak or attack, enemies within 30 feet of you must succeed on a Will save (DC 10 + \half your character level + your Cha modifier) or become shaken for 5d6 rounds. An opponent that succeeds on its saving throw is immune to your frightful presence for 24 hours.}
{Your opponents take a -2 morale penalty to saving throws if they can see you and you are within medium range (based on your character level).}
\skillfeat{Monster Rancher}{[Leadership] [Skill:Handle Animal]}
{You can breed and train a large number of crazy beasts.}
{You can use Handle Animal as if it were Diplomacy when dealing with Magical Beasts and Dragons. You can do similarly with Aberrations and Plants with an Intelligence Score that is less than 9.}
{You can muster a group of followers. Your leadership score is your ranks in Handle Animal plus any synergy bonuses you gt to that skill. Your followers can, and must be monsters.}
{You have a loyal cohort that is a monster of some kind. A cohort is an intelligent and loyal creature with a CR at least 2 less than your character level. Cohorts gain levels when you do.}
{You know what any monster is unless it is disguised by illusion, and you can look up its stat line in the appropriate monster book when devising your strategies.}
{Once per day, you can reroll a saving throw allowed by a Supernatural Ability.}
\skillfeat{Mounted Combat}{[Skill:Ride]}
{You are at your best when fighting with an ally that you are sitting on.}
{Once per turn, you may attempt to negate an attack that hits your mount by making a Ride skill check with a DC equal to the AC that the attack hit. Attacks that do not require an attack roll cannot be negated in this way.}
{While Mounted, you may take a charge attack at any point along your mount's movement, so long as your mount is moving in a straight line up to the point of your attack.}
{You suffer no penalty to your ride or handle animal skill checks when training or riding unusual mounts such as magical beasts or dragons.}
{You may use your Ride Check in place of your mount's Balance, Jump, Climb, or Reflex Saving Throws.}
{Any time a spell effect would target your mount, you may elect to have it target you instead. Any time a spell effect would target you, you may elect to have it affect your Mount instead.}
\skillfeat{Natural Empath}{[Skill:Sense Motive]}
{You read people like books.}
{You gain a +3 bonus to Sense Motive checks.}
{You can quickly size up potential opponents. If you succeed on a Sense Motive check as a free action, opposed by their Bluff, you can tell if they're an even match (their CR equals your character level), an easy challenge (their CR is 1-3 less than your level), irrelevant (their CR is 4 or more less than your level), stronger (their CR is 1-3 higher than your level), or overwhelmingly powerful (their CR is 4 or more higher than your level). You can use this ability once on a particular creature every 24 hours.}
{If you succeed on a Sense Motive check, opposed by Bluff, you know your opponent's alignment. If you beat their Bluff by 20 or more, you can read their surface thoughts, as if during the third round of \spell{detect thoughts}.}
{You have an uncanny intuition for when people are interested in you. Any time someone uses a remote spell or effect, like \spell{scrying}, to examine you, you know you're under observation and if you make a Sense Motive check that beats their Bluff check, you know some details about them: if you've met them before, you recognize them, but if not, you get a basic idea of their reasons for their interest in you. Similarly, if you use Sense Motive on someone influenced by an enchantment effect, you can find out who created the effect with a Sense Motive check opposed by the controller's Bluff, getting the same information.}
{You know what people are going to do before they do. Any time someone you're aware of attacks you, make a Sense Motive check opposed by their Bluff: if you succeed, you get a free surprise round.}
\skillfeat{Persuasive}{[Skill:Diplomacy]}
{When you tell you people something that contradicts the evidence of their own eyes, they believe you.}
{You gain a +3 bonus to Diplomacy checks.}
{Your words can stop fights before they start. Any creature that can hear you speak must make a Will save (DC 10 + \half your character level + your Cha modifier) or it can't attack you directly; however, you aren't protected from its area or effect spells, or similar abilities. Any creature that succeeds on its save is immune to this ability for 24 hours. You may use nonattack spells or otherwise act, but if you attack the creature or its allies, it may attack you. This is a mind-affecting, language-dependent charm effect.}
{You can fascinate creatures with your silver tongue. You can affect as many HD of creatures as your bonus on Diplomacy checks; any creature that fails a Will save (DC 10 + \half your character level + your Cha modifier) becomes fascinated. If you use this ability in combat, each target gains a +2 bonus on its saving throw. If the spell affects only a single creature not in combat at the time, the saving throw has a penalty of -2. While a subject is fascinated by this spell, it reacts as though it were two steps more friendly in attitude, allowing you to make a single request of an affected creature. The request must be brief and reasonable. Even after the spell ends, the creature retains its new attitude toward you, but only with respect to that particular request. A creature that fails its saving throw does not remember that you enspelled it.}
{You can influence even hostile creatures into talking things over with you. With a DC 30 Diplomacy check, you can use a language-dependent version of \spell{charm monster} as a spell-like ability, with save DC equal to 10 + \half your character level + your Cha modifier; this is a mind-affecting charm effect.}
{You can convince an entire group of enemies to listen to you. If you succeed on a DC 40 Diplomacy check, your \spell{charm monster} ability improves to \spell{mass charm monster}, with a caster level equal to your bonus on Diplomacy checks.}
\skillfeat{Professional Luddite}{[Skill:Disable Device]}
{You've learned to break machines because you're an antitechnology fanatic -- or maybe you just work for the local protection racket.}
{You can use Disable Device on magic traps like a character with trapfinding. If you already have that ability, you gain +3 to your Disable Device checks. Disable Device is always a class skill for you.}
{You can use your Dexterity modifier instead of your Intelligence modifier for Disable Device checks. Darkness and blindness do not hinder your ability to disable devices.}
{You can reduce the amount of time required to disable a device. For each multiple of 10 you beat the required DC, you can decrease the time required from 2d4 rounds to 1d4 rounds to 1 round to a standard action to a move-equivalent action to a free action.}
{You can use Disable Device to end any persistent effect or area spell effect as if it was a magic trap, but the DC is 25 + twice the spell's level.}
{As an attack action, you can disable magic items. You must succeed on a melee touch attack roll for attended objects. Make a Disable Device check against a DC of 15 + the item's caster level: if your check succeeds, the item must make a Will save against a DC of 10 + \half your character level or be turned into a normal item, and even if it saves, its magical properties are suppressed for 1d4 rounds.}
\skillfeat{Sharp-Eyed}{[Skill:Spot]}
{Nothing escapes your sight.}
{You gain a +3 bonus to Spot checks.}
{You can make a Spot check once a round as a free action. You don't take penalties for distractions on your Spot checks.}
{As a move action, you can make a Spot check against a DC of an opponent's Armor Class: if you succeed, you can ignore their Armor and Natural Armor bonus to AC for the next attack you make against them. If you accept a -20 penalty to your check, you can attempt this check as a swift action. Divide any distance penalties you take on Spot checks by two.}
{If you beat an opponent's Hide check with a Spot check at a -10 penalty, you can ignore concealment. If you beat their Hide check at a -30 penalty, you can ignore total concealment.}
{You can see through solid objects, but you take a -20 penalty on your Spot check for each 5'. Divide any distance penalties you take on Spot checks by five.}
\skillfeat{Slippery Contortionist}{[Skill:Escape Artist]}
{Your childhood nickname was ``Greasy the Pig," but now people call you ``The Great Hamster."}
{You gain +3 to your Escape Artist checks.}
{While squeezing into a space at least half as wide as your normal space, you may move your normal speed and you take no penalty to your attack rolls or AC for squeezing.}
{You can squeeze through a tight space or an extremely tight space as a full-round action, but you take a -10 penalty to your Escape Artist check. Opponents grappling you don't get positive size modifiers added to their grapple bonus when you use Escape Artist to try to break their hold.}
{If you succeed on a DC 30 Escape Artist check, you can ignore magical effects that impede movement as if you were under the effects of \spell{freedom of movement} for one round; this is not an action. You can also slip through a \spell{wall of force} or similar barrier with a DC 40 check.}
{You can make an Escape Artist check instead of a saving throw for any effect that would keep you from taking actions. (This does not help against effects that don't allow a saving throw.)}
\skillfeat{Steady Stance}{[Skill:Balance]}
{You can fight just about anywhere.}
{You gain a +3 bonus to your Balance checks.}
{If an effect would knock you prone, if you succeed on a DC 20 Balance check, you remain standing.}
{If your opponent is balancing, you gain a +3 dodge bonus to AC against their attacks unless they succeed at beating you in an opposed Balance check.}
{All Balance DCs are halved for you.}
{You never suffer any impairment or damage from anything you're standing on, whether it's molten lava, a cloud, or even another creature. Ambient conditions, such as lighting or weather, can still impair you.}
\skillfeat{Stealthy}{[Skill:Hide]}
{If someone sees you, you have to kill them.}
{You gain a +3 bonus to your Hide checks.}
{You can Hide as a free action after attacking, and snipe with melee attacks (or ranged attacks from closer than 10').}
{A constant \spell{nondetection effect} protects you and your equipment, with an effective caster level equal to your ranks in Hide.}
{You can attempt to Hide even when under direct observation, but you take the usual -20 penalty to your check.}
{Even opponents who can see you have trouble locating you. If they succeed at beating your Hide check with Spot (and thus can see you), they have a 50\% concealment miss chance when attacking you, which decreases by 5\% for each point they beat your Hide DC.}
\skillfeat{Swim Like a Fish}{[Skill:Swim]}
{You're at least as home in the water as you are on land.}
{You gain +3 to your Swim checks.}
{You gain a swim speed equal to your base land speed, with the attendant benefits. You don't take armor check penalties to your Swim checks.}
{You can breathe water, and you can attack through water as if under the effects of \spell{freedom of movement}.}
{While under water, you can substitute Swim checks for Reflex saves, and you gain a +4 bonus to attack and damage rolls.}
{As a swift action, you can add your ranks in Swim as a dodge bonus to your Armor Class while under water.}
\skillfeat{Track}{[Skill:Survival]}
{You feel at home no matter where you are.}
{You can follow tracks using Survival, as the Track and Legendary Tracker feats.}
{You can identify the race/kind of creatures from their tracks.}
{You can move through or over difficult natural terrain without being slowed, taking nonlethal damage, or suffering other impairment. You take no penalties for moving your speed when tracking, and only -10 when moving double your speed. You can track subjects protected by \spell{pass without trace} or similar spells at a -20 penalty.}
{You can track through the Astral Plane with a DC 35 Survival check. You can determine the destination of a teleportation spell when standing at the point of departure with a DC 40 Survival check; if you have \spell{teleport} or a similar spell, you can follow as if you had seen the destination once.}
{You're immune to natural planar effects as if you had \spell{planar tolerance} always active.}
\skillfeat{Tyrant}{[Leadership] [Skill:Intimidate]}
{You push people around and get larger and larger groups trapped in the iron gauntlet of your brutal rule.}
{You inspire such terror that creatures you intimidate continue to act intimidated after you leave, too afraid to raise their voice in defiance even after you have apparently left them far behind.}
{You can muster a group of followers. Your leadership score is your ranks in Intimidate plus your Strength modifier.}
{Your followers swell in number to that of an army.}
{Your mere presence inspires fear and can break a battle. Enemies with more than 5 hit dice less than you do must make a Will save (DC 10 + ½ Level + Strength Modifier) of flee in panic. This is a [Fear] effect.}
{Your presence causes despair in even brave opponents. All enemies within 30' of your suffer a -2 Morale penalty to Willpower saves.}

%\end{multicols}
\section{Spellcasting Feats}

Foo

%%%%%%%%%%%%%%%%%%%%%%%%
%%Goods and Services chapter formatting
%%%%%%%%%%%%%%%%%%%%%%%%

\newcommand{\armorentry}[4]{\ability{#1}{\textit{#3}}  #4}

%%%%%%%%%%%%%%%%%%%%%%%%

\chapter{Goods and Services}
\section{The Three Economies}
foo
\section{Armor}
%%%%%%%%%%%%%%%%%%%%%%%%%%%%%%%%%%%%%%%%%%%%%%%%%%
\section{Armor}
%%%%%%%%%%%%%%%%%%%%%%%%%%%%%%%%%%%%%%%%%%%%%%%%%%

foo?

%%%%%%%%%%%%%%%%%%%%%%%%%
\subsection{Armor and Shield Traits}
%%%%%%%%%%%%%%%%%%%%%%%%%

%%%
\subsubsection{Armor Traits}
%%%

foo?

%%%
\subsubsection{Shield Traits}
%%%

foo?

%%%
\subsubsection{Non-proficiency}
%%%

foo?

%%%
\subsubsection{Effects of High BAB}
%%%

foo?

%%%
\subsubsection{Armor Check Penalty and Movement}
%%%

foo?

%%%
\subsubsection{Arcane Spell Failure}
%%%

foo?

%%%
\subsubsection{Donning and Removing Armor}
%%%

foo?

%%%
\subsubsection{Non-Standard Armors}
%%%

foo?

%%%
\subsubsection{Special Armor Materials}
%%%
\tagline{"I know it's stupid looking, but I get the best possible protection from having this duck sit on my head, so I'm going to let it do that."}

foo?

%%%%%%%%%%%%%%%%%%%%%%%%%
\subsection{Non-Armors}
%%%%%%%%%%%%%%%%%%%%%%%%%

foo?

%%%%%%%%%%%%%%%%%%%%%%%%%
\subsection{Light Armors}
%%%%%%%%%%%%%%%%%%%%%%%%%

foo?

%%%%%%%%%%%%%%%%%%%%%%%%%
\subsection{Medium Armors}
%%%%%%%%%%%%%%%%%%%%%%%%%

foo?

%%%%%%%%%%%%%%%%%%%%%%%%%
\subsection{Heavy Armors}
%%%%%%%%%%%%%%%%%%%%%%%%%

foo?

%%%%%%%%%%%%%%%%%%%%%%%%%
\subsection{Shields}
%%%%%%%%%%%%%%%%%%%%%%%%%

foo?

%%%%%%%%%%%%%%%%%%%%%%%%%
\subsection{Great Shields}
%%%%%%%%%%%%%%%%%%%%%%%%%

foo?

\section{Weapons}
\section{Weapons}
\vspace*{-10pt}
\quot{``No. This is a knife."}


The weapon system of D\&D, in general, makes us feel pretty good. There are ample reasons to use weapons as diverse as a flail, a warhammer, and a morningstar. There are, however, some glaring problems that do need to be addressed. The most obvious of those is Weapon Size, which works very badly on every level. The 3rd edition rules were not good, and the 3.5 changes to them made them worse in every single way. So here's the big deal: Weapons don't have special size rules anymore. In 3rd edition a Shortsword was a small weapon, and in 3.5 it's supposed to be a Medium Light Weapon, but that's all stupid. The fact is, a Shortsword is a Tiny Object, and that's all we need to know.

Here's how weapon sizes ought to work:

\listone
    \item You may not use a weapon that is a larger than yourself. A Large character can use a Large (or smaller) object as a weapon, but may not use a Huge (or larger) object as a weapon.
    \item You may not use an object that is too heavy for you to lift as a light load as a weapon.
    \item An object of your own size must be used in two hands.
    \item An object of a size smaller than your size may be used in one hand or two hands.
    \item An object that is at least two sizes smaller than yourself counts as a Light Weapon.
\end{list}


\subsection{Bows}

The bow is a very expensive proposition in the normal D\&D rules. Especially for Orcs. That's really dumb. So here are the new rules:

Every bow has a strength minimum. And it doesn't cost any more if it has a Strength Minimum of 34 than it does if it has a Strength minimum of 6. In any case, your bow can't be used if your strength is less than the strength minimum of the bow. But, your bow does damage based on your actual strength -- or 4 more than the strength minimum of the bow, whichever is less.

Now, certain groups are not going to have bows available with a strength bonus applicable to yourself. If you have a strength of 8, the Bugbears probably won't have any bows off the shelf to sell to you. If you have a strength of 18, the Kobolds won't have anything for you. If you're in an area that doesn't normally make bows for you, you're going to have to get a masterwork bow made for you -- and that costs extra moneys.

Now, the range of a bow is based on its object size. A Medium object (the kind of bow you are most likely to use) has a range increment of 100 feet. Every size it is smaller than that decreases the range increment by 30 feet (yes, that means that Fine creatures don't even have bows, and we're OK with that). Every size that a bow is larger than medium increases the range increment by 30 feet. A composite bow has an extra 10 feet of range increment. A character may only use a composite bow or a bow that is smaller than herself while mounted. And yes, a bow is two handed even if it is an object two sizes smaller than yourself.

\subsection{Ammunition}
\vspace*{-8pt}
\quot{``The Black Arrow was forged by Thror the Dwarf, who was ``King Under The (Lonely) Mountain", and ultimately was destroyed when Bard used it for target practice against a swallow, thereby dooming most of Middle Earth."}

The ammunition rules are in need of adjustment. And that's not just because having a shuriken get destroyed permanently every time it hits is really dumb. It's almost balanced to have magic arrows cost about 1/50th of what a real magic weapon does and then explode when used like they were bullets or something. Almost. But it is also dumb, so we're putting our foot down.

Magic Arrows are supposed to be awesome. Some of them even have names. I cannot recall any story where an insipid adventurer went to War with 137 magic arrows and then called it a day when every one of them had been fired once. So here's the new rubric: the cost of enchanting a magical arrow is a mere 1/10th that of enchanting a weapon (move the decimal place over one place), and magical arrows are always recoverable. That's part of what makes them magic. Of course, just because it's recoverable, doesn't mean that you will actually recover it. If you shoot three arrows into a guy and then you run away, chances are good that he has your arrow.

Heck, even regular ammunition is way too fragile in D\&D. Shuriken are fairly reusable even after you pull them out of the eye of a fallen foe. And we're fine with that. A good rule of thumb is that an item of ammunition is no longer usable if it inflicts more damage that it has hardness. And precision damage, such as Sneak Attack, Death Attack, and Sudden Strike, does not count. So yeah, Shuriken aren't going to break on impact with small children, happy birthday.

Naturally enough, there are still one-use arrows in the world. Alchemical arrows, such as fire arrows or poison arrows, are generally not as useful after they've been shot into an appropriate target. Those don't require magical forging however, and don't really count as magic weapons. One use ranged weapons should be marked as such (such as the vial of acid, hard to reuse that one).

\subsection{Necromatic Weapons}

\listone
\itemability{Boneblades:} Boneblades are alchemically and necromantically hardened blades made from the bones of intelligent creatures, and the material can only be created by craftsmen with the Boneblade Master feat. For an unknown reason, they only retain their special properties if they are made into light slashing or piercing weapons.

Boneblades used in melee combat ignore the damage reduction of any undead creature and can hit incorporeal creatures as if they were magic weapons with the ghost touch property.

Boneblades made from dragon bones can be combined with the Dragoncrafter feat to produce items with both properties.

Cost: 1,000 gp per lb.

\itemability{Blood Steel:} Blood steel is steel that has been mixed with the blood of certain powerful creatures, making it redder than normal steel and with unusual properties.

Weapons made of blood steel do 2 additional points of damage on a successful hit.

Cost: 2,000 gp for a weapon

\itemability{Black Steel:} Black steel is steel that has been mixed with necromantically charged obsidian, making it as sharp as adamantine and as dangerous as obsidian. Weapons made of black steel count as adamantine for all effects, but perform as if enhanced with the Ghost Touch and Wounding properties (without additional cost).

Characters using items made of black steel suffer one point of Wisdom drain for every day they are held, worn or carried.

Cost: 15,000 gp for a weapon  %, 24,000 gp for light armor, 28,000 gp for medium armor, 32,000 gp for heavy armor.\\

\end{list} 

\section{Adventuring Gear}
\begin{table}
\rowcolors{1}{colorone}{colortwo}
\caption{Adventuring Gear}
{\tabulinesep=1mm
\begin{tabu}to \linewidth{X c c | X c c}
\header\textbf{Item} & \textbf{Cost} & \textbf{Weight} & \textbf{Item} & \textbf{Cost} & \textbf{Weight}\\ \hline
10ft Ladder & 5 cp & 20 lb. & Lantern (bullseye) & 12 gp & 3 lb. \\
10ft Pole & 2 sp & 8 lb. & Lantern (hooded) & 7 gp & 2 lb. \\
Backpack (empty) & 2 gp & 2 lb.\textsuperscript{1} & Lock & -- & 1 lb. \\
Barrel (empty) & 2 gp & 30 lb. & \hspace{.25cm}Very simple & 20 gp & 1 lb. \\
Basket (empty) & 4 sp & 1 lb. & \hspace{.25cm}Average & 40 gp & 1 lb. \\
Bedroll & 1 sp & 5 lb.\textsuperscript{1} & \hspace{.25cm}Good & 80 gp & 1 lb. \\
Bell & 1 gp & -- & \hspace{.25cm}Amazing & 150 gp & 1 lb. \\
Belt Pouch (empty) & 1 gp & \sfrac{1}{2} lb.\textsuperscript{1} & Manacles (common) & 15 gp & 2 lb. \\
Block and tackle & 5 gp & 5 lb. & Manacles (masterwork) & 50 gp & 2 lb. \\
Bucket (empty) & 5 sp & 2 lb. & Miner's Pick & 3 gp & 10 lb. \\
Caltrops & 1 gp & 2 lb. & Oil (1-pint flask) & 1 sp & 1 lb. \\
Candle & 1 cp & -- & Paper (sheet) & 4 sp & -- \\
Canvas (sq. yd.) & 1 sp & 1 lb. & Parchment (sheet) & 2 sp & -- \\
Case, map or scroll & 1 gp & \sfrac{1}{2} lb. & Piton & 1 sp & \sfrac{1}{2} lb. \\
Chain (10 ft.) & 30 gp & 2 lb. & Portable Ram & 10 gp & 20 lb. \\
Chalk, 1 piece & 1 cp & -- & Rope (hempen, 50 ft.) & 1 gp & 10 lb. \\
Chest (empty) & 2 gp & 25 lb. & Rope (silk, 50 ft.) & 10 gp & 5 lb. \\
Clay Jug & 3 cp & 9 lb. & Sack (empty) & 1 sp & \sfrac{1}{2} lb.\textsuperscript{1} \\
Clay Mug/Tankard & 2 cp & 1 lb. & Sealing wax & 1 gp & 1 lb. \\
Clay Pitcher & 2 cp & 5 lb. & Sewing needle & 5 sp & -- \\
Common Lamp & 1 sp & 1 lb. & Signal whistle & 8 sp & -- \\
Crowbar & 2 gp & 5 lb. & Signet ring & 5 gp & -- \\
Firewood (per day) & 1 cp & 20 lb. & Sledge & 1 gp & 10 lb. \\
Fishhook & 1 sp & -- & Small Steel Mirror & 10 gp & \sfrac{1}{2} lb. \\
Fishing net, 25 sq. ft. & 4 gp & 5 lb. & Soap (per lb.) & 5 sp & 1 lb. \\
Flask (empty) & 3 cp & 1.5 lb. & Spade or shovel & 2 gp & 8 lb. \\
Flint and steel & 1 gp & -- & Spyglass & 1,000 gp & 1 lb. \\
Glass Wine Bottle & 2 gp & -- & Tent & 10 gp & 20 lb.\textsuperscript{1} \\
Grappling hook & 1 gp & 4 lb. & Torch & 1 cp & 1 lb. \\
Hammer & 5 sp & 2 lb. & Trail Rations (per day) & 5 sp & 1 lb.\textsuperscript{1} \\
Ink (1 oz. vial) & 8 gp & -- & Vial, ink or potion & 1 gp & \sfrac{1}{10} lb. \\
Inkpen & 1 sp & -- & Waterskin & 1 gp & 4 lb.\textsuperscript{1} \\
Iron Pot & 5 sp & 10 lb. & Whetstone & 2 cp & 1 lb. \\
&&&Winter Blanket & 5 sp & 3 lb.\textsuperscript{1} \\ \hline
\multicolumn{6}{p{\linewidth}}{\textsuperscript{1} These items weigh one-quarter this amount when made for Small characters. Containers for Small characters also carry one-quarter the normal amount.}\\ \hline
\end{tabu}}
\end{table}

A few of the pieces of adventuring gear found on Table: Adventuring Gear are described 
below, along with any special benefits they confer on the user ("you").

\textbf{Caltrops:} A caltrop is a four-pronged iron spike crafted so that one prong 
faces up no matter how the caltrop comes to rest. You scatter caltrops on the ground 
in the hope that your enemies step on them or are at least forced to slow down 
to avoid them. One 2- pound bag of caltrops covers an area 5 feet square.

Each time a creature moves into an area covered by caltrops (or spends a round 
fighting while standing in such an area), it might step on one. The caltrops make 
an attack roll (base attack bonus +0) against the creature. For this attack, the 
creature's shield, armor, and deflection bonuses do not count. If the creature 
is wearing shoes or other footwear, it gets a +2 armor bonus to AC. If the caltrops 
succeed on the attack, the creature has stepped on one. The caltrop deals 1 point 
of damage, and the creature's speed is reduced by one-half because its foot is 
wounded. This movement penalty lasts for 24 hours, or until the creature is successfully 
treated with a DC 15 Heal check, or until it receives at least 1 point of magical 
curing. A charging or running creature must immediately stop if it steps on a caltrop. 
Any creature moving at half speed or slower can pick its way through a bed of caltrops 
with no trouble.

Caltrops may not be effective against unusual opponents.

\textbf{Candle:} A candle dimly illuminates a 5-foot radius and burns for 1 hour.

\textbf{Chain:} Chain has hardness 10 and 5 hit points. It can be burst with a 
DC 26 Strength check.

\textbf{Clay Jug:} This basic ceramic jug is fitted with a stopper and holds 1 gallon of liquid.

\textbf{Common Lamp:} A lamp clearly illuminates a 15-foot radius, provides shadowy 
illumination out to a 30-foot radius, and burns for 6 hours on a pint of oil. You 
can carry a lamp in one hand. 

\textbf{Crowbar:} A crowbar it grants a +2 circumstance bonus on Strength checks 
made for such purposes. If used in combat, treat a crowbar as a one-handed improvised 
weapon that deals bludgeoning damage equal to that of a club of its size.

\textbf{Flint and Steel:} Lighting a torch with flint and steel is a full-round 
action, and lighting any other fire with them takes at least that long.

\textbf{Grappling Hook:} Throwing a grappling hook successfully requires a Use 
Rope check (DC 10, +2 per 10 feet of distance thrown).

\textbf{Hammer:} If a hammer is used in combat, treat it as a one-handed improvised 
weapon that deals bludgeoning damage equal to that of a spiked gauntlet of its 
size.

\textbf{Ink:} This is black ink. You can buy ink in other colors, but it costs 
twice as much.

\textbf{Lantern, Bullseye:} A bullseye lantern provides clear illumination in a 
60-foot cone and shadowy illumination in a 120-foot cone. It burns for 6 hours 
on a pint of oil. You can carry a bullseye lantern in one hand.

\textbf{Lantern, Hooded:} A hooded lantern clearly illuminates a 30-foot radius 
and provides shadowy illumination in a 60-foot radius. It burns for 6 hours on 
a pint of oil. You can carry a hooded lantern in one hand.

\textbf{Lock:} The DC to open a lock with the Open Lock skill depends on the lock's 
quality: simple (DC 20), average (DC 25), good (DC 30), or superior (DC 40).

\textbf{Manacles and Manacles, Masterwork:} Manacles can bind a Medium creature. 
A manacled creature can use the Escape Artist skill to slip free (DC 30, or DC 
35 for masterwork manacles). Breaking the manacles requires a Strength check (DC 
26, or DC 28 for masterwork manacles). Manacles have hardness 10 and 10 hit points.

Most manacles have locks; add the cost of the lock you want to the cost of the 
manacles.

For the same cost, you can buy manacles for a Small creature.

For a Large creature, manacles cost ten times the indicated amount, and for a Huge 
creature, one hundred times this amount. Gargantuan, Colossal, Tiny, Diminutive, 
and Fine creatures can be held only by specially made manacles.

\textbf{Oil:} A pint of oil burns for 6 hours in a lantern. You can use a flask 
of oil as a splash weapon. Use the rules for alchemist's fire, except that it takes 
a full round action to prepare a flask with a fuse. Once it is thrown, there is 
a 50\% chance of the flask igniting successfully.

You can pour a pint of oil on the ground to cover an area 5 feet square, provided 
that the surface is smooth. If lit, the oil burns for 2 rounds and deals 1d3 points 
of fire damage to each creature in the area.

\textbf{Portable Ram:} This iron-shod wooden beam gives you a +2 circumstance 
bonus on Strength checks made to break open a door and it allows a second person 
to help you without having to roll, increasing your bonus by 2.

\textbf{Rope, Hempen:} This rope has 2 hit points and can be burst with a DC 23 
Strength check.

\textbf{Rope, Silk:} This rope has 4 hit points and can be burst with a DC 24 Strength 
check. It is so supple that it provides a +2 circumstance bonus on Use Rope checks.

\textbf{Spyglass:} Objects viewed through a spyglass are magnified to twice their 
size.

\textbf{Torch:} A torch burns for 1 hour, clearly illuminating a 20-foot radius 
and providing shadowy illumination out to a 40- foot radius. If a torch is used 
in combat, treat it as a one-handed improvised weapon that deals bludgeoning damage 
equal to that of a gauntlet of its size, plus 1 point of fire damage.

\textbf{Vial:} A vial holds 1 ounce of liquid. The stoppered container usually 
is no more than 1 inch wide and 3 inches high.

%%%%%%%%%%%%%%%%%%%%%%%%%
\subsection{Special Substances and Items}
%%%%%%%%%%%%%%%%%%%%%%%%%

Any of these substances except for the everburning torch and holy water can be 
made by a character with the Craft (alchemy) skill.

\textbf{Acid:} You can throw a flask of acid as a splash weapon. Treat this attack 
as a ranged touch attack with a range increment of 10 feet. A direct hit deals 
1d6 points of acid damage. Every creature within 5 feet of the point where the 
acid hits takes 1 point of acid damage from the splash.

\textbf{Alchemist's Fire:} You can throw a flask of alchemist's fire as a splash 
weapon. Treat this attack as a ranged touch attack with a range increment of 10 
feet.

A direct hit deals 1d6 points of fire damage. Every creature within 5 feet of the 
point where the flask hits takes 1 point of fire damage from the splash. On the 
round following a direct hit, the target takes an additional 1d6 points of damage. 
If desired, the target can use a full-round action to attempt to extinguish the 
flames before taking this additional damage. Extinguishing the flames requires 
a DC 15 Reflex save. Rolling on the ground provides the target a +2 bonus on the 
save. Leaping into a lake or magically extinguishing the flames automatically smothers 
the fire.

\textbf{Antitoxin:} If you drink antitoxin, you get a +5 alchemical bonus on Fortitude 
saving throws against poison for 1 hour.

\textbf{Everburning Torch:} This otherwise normal torch has a \textit{continual 
flame }spell cast upon it. An everburning torch clearly illuminates a 20-foot radius 
and provides shadowy illumination out to a 40-foot radius.

\textbf{Holy Water:} Holy water damages undead creatures and evil outsiders almost 
as if it were acid. A flask of holy water can be thrown as a splash weapon.

Treat this attack as a ranged touch attack with a range increment of 10 feet. A 
flask breaks if thrown against the body of a corporeal creature, but to use it 
against an incorporeal creature, you must open the flask and pour the holy water 
out onto the target. Thus, you can douse an incorporeal creature with holy water 
only if you are adjacent to it. Doing so is a ranged touch attack that does not 
provoke attacks of opportunity.

\begin{wraptable}{r}{.5\linewidth}
\rowcolors{1}{colorone}{colortwo}
\caption{Special Substances and Items}
{\tabulinesep=1mm
\begin{tabu}to \linewidth{X c c}
\header\textbf{Item} & \textbf{Cost} & \textbf{Weight}\\ \hline
Acid (flask) & 10 gp & 1 lb.\\
Alchemist's fire (flask) & 20 gp & 1 lb.\\
Antitoxin (vial) & 50 gp & --\\
Everburning torch & 110 gp & 1 lb.\\
Holy water (flask) & 25 gp & 1 lb.\\
Smokestick & 20 gp & \sfrac{1}{2} lb.\\
Sunrod & 2 gp & 1 lb.\\
Tanglefoot bag & 50 gp & 4 lb.\\
Thunderstone & 30 gp & 1 lb.\\
Tindertwig & 1 gp & --\\ \hline
\end{tabu}}
\end{wraptable}

A direct hit by a flask of holy water deals 2d4 points of damage to an undead creature 
or an evil outsider. Each such creature within 5 feet of the point where the flask 
hits takes 1 point of damage from the splash.

Temples to good deities sell holy water at cost (making no profit).

\textbf{Smokestick:} This alchemically treated wooden stick instantly creates thick, 
opaque smoke when ignited. The smoke fills a 10- foot cube (treat the effect as 
a \textit{fog cloud }spell, except that a moderate or stronger wind dissipates 
the smoke in 1 round). The stick is consumed after 1 round, and the smoke dissipates 
naturally.

\textbf{Sunrod:} This 1-foot-long, gold-tipped, iron rod glows brightly when struck. 
It clearly illuminates a 30-foot radius and provides shadowy illumination in a 
60-foot radius. It glows for 6 hours, after which the gold tip is burned out and 
worthless.

\textbf{Tanglefoot Bag:} When you throw a tanglefoot bag at a creature (as a ranged 
touch attack with a range increment of 10 feet), the bag comes apart and the goo 
bursts out, entangling the target and then becoming tough and resilient upon exposure 
to air. An entangled creature takes a -2 penalty on attack rolls and a -4 penalty 
to Dexterity and must make a DC 15 Reflex save or be glued to the floor, unable 
to move. Even on a successful save, it can move only at half speed. Huge or larger 
creatures are unaffected by a tanglefoot bag. A flying creature is not stuck to 
the floor, but it must make a DC 15 Reflex save or be unable to fly (assuming it 
uses its wings to fly) and fall to the ground. A tanglefoot bag does not function 
underwater.

A creature that is glued to the floor (or unable to fly) can break free by making 
a DC 17 Strength check or by dealing 15 points of damage to the goo with a slashing 
weapon. A creature trying to scrape goo off itself, or another creature assisting, 
does not need to make an attack roll; hitting the goo is automatic, after which 
the creature that hit makes a damage roll to see how much of the goo was scraped 
off. Once free, the creature can move (including flying) at half speed. A character 
capable of spellcasting who is bound by the goo must make a DC 15 Concentration 
check to cast a spell. The goo becomes brittle and fragile after 2d4 rounds, cracking 
apart and losing its effectiveness. An application of \textit{universal solvent} 
to a stuck creature dissolves the alchemical goo immediately.

\textbf{Thunderstone:} You can throw this stone as a ranged attack with a range 
increment of 20 feet. When it strikes a hard surface (or is struck hard), it creates 
a deafening bang that is treated as a sonic attack. Each creature within a 10-foot-radius 
spread must make a DC 15 Fortitude save or be deafened for 1 hour. A deafened creature, 
in addition to the obvious effects, takes a -4 penalty on initiative and has a 
20\% chance to miscast and lose any spell with a verbal component that it tries 
to cast.

Since you don't need to hit a specific target, you can simply aim at a particular 
5-foot square. Treat the target square as AC 5.

\textbf{Tindertwig:} The alchemical substance on the end of this small, wooden 
stick ignites when struck against a rough surface. Creating a flame with a tindertwig 
is much faster than creating a flame with flint and steel (or a magnifying glass) 
and tinder. Lighting a torch with a tindertwig is a standard action (rather than 
a full-round action), and lighting any other fire with one is at least a standard 
action.

%%%%%%%%%%%%%%%%%%%%%%%%%
\subsection{Tools and Skill Kits}
%%%%%%%%%%%%%%%%%%%%%%%%%

\begin{wraptable}{r}{.5\linewidth}
\rowcolors{1}{colorone}{colortwo}
\caption{Tools and Skill Kits}
{\tabulinesep=1mm
\begin{tabu}to \linewidth{X c c}
\header\textbf{Item} & \textbf{Cost} & \textbf{Weight}\\ \hline
Alchemist's lab & 500 gp & 40 lb.\\
Artisan's tools (common) & 5 gp & 5 lb.\\
Artisan's tools (masterwork) & 55 gp & 5 lb.\\
Climber's kit & 80 gp & 5 lb.\textsuperscript{1}\\
Disguise kit & 50 gp & 8 lb.\textsuperscript{1}\\
Healer's kit & 50 gp & 1 lb.\\
Holly and mistletoe & -- & --\\
Holy symbol (silver) & 25 gp & 1 lb.\\
Holy symbol (wooden) & 1 gp & --\\
Hourglass & 25 gp & 1 lb.\\
Magnifying glass & 100 gp & --\\
Masterwork  Tool & 50 gp & 1 lb.\\
Merchant's Scale & 2 gp & 1 lb.\\
Musical instrument (common) & 5 gp & 3 lb.\textsuperscript{1}\\
Musical instrument (masterwork) & 100 gp & 3 lb.\textsuperscript{1}\\
Spell component pouch & 5 gp & 2 lb.\\
Thieves' tools (common) & 30 gp & 1 lb.\\
Thieves' tools (masterwork) & 100 gp & 2 lb.\\
Water clock & 1,000 gp & 200 lb.\\
Wizard's Spellbook (blank) & 15 gp & 3 lb.\\ \hline
\multicolumn{3}{p{\linewidth}}{\textsuperscript{1} These items weigh one-quarter this amount when made for Small characters.}\\
\hline
\end{tabu}}
\end{wraptable}

\indent\textbf{Alchemist's Lab:} An alchemist's lab always has the perfect tool for making 
alchemical items, so it provides a +2 circumstance bonus on Craft (alchemy) checks. 
It has no bearing on the costs related to the Craft (alchemy) skill. Without this 
lab, a character with the Craft (alchemy) skill is assumed to have enough tools 
to use the skill but not enough to get the +2 bonus that the lab provides.

\textbf{Artisan's Tools (common):} These special tools include the items needed to pursue 
any craft. Without them, you have to use improvised tools (-2 penalty on Craft 
checks), if you can do the job at all.

\textbf{Artisan's Tools (masterwork):} These tools serve the same purpose as artisan's 
tools (above), but masterwork artisan's tools are the perfect tools for the job, 
so you get a +2 circumstance bonus on Craft checks made with them.

\textbf{Climber's Kit:} This is the perfect tool for climbing and gives you a +2 
circumstance bonus on Climb checks.

\textbf{Disguise Kit:} The kit is the perfect tool for disguise and provides a 
+2 circumstance bonus on Disguise checks. A disguise kit is exhausted after ten 
uses.

\textbf{Healer's Kit:} It is the perfect tool for healing and provides a +2 circumstance 
bonus on Heal checks. A healer's kit is exhausted after ten uses.

\textbf{Holy Symbol, Silver or Wooden:} A holy symbol focuses positive energy. 
A cleric or paladin uses it as the focus for his spells and as a tool for turning 
undead. Each religion has its own holy symbol.

\textbf{Magnifying Glass:} This simple lens allows a closer look at small objects. 
It is also useful as a substitute for flint and steel when starting fires. Lighting 
a fire with a magnifying glass requires light as bright as sunlight to focus, tinder 
to ignite, and at least a full-round action. A magnifying glass grants a +2 circumstance 
bonus on Appraise checks involving any item that is small or highly detailed.

\textbf{Masterwork Tool:} This well-made item is the perfect tool for the job. 
It grants a +2 circumstance bonus on a related skill check (if any). Bonuses provided 
by multiple masterwork items used toward the same skill check do not stack.

\textbf{Merchant's Scale:} A scale grants a +2 circumstance bonus on Appraise 
checks involving items that are valued by weight, including anything made of precious 
metals.

\textbf{Musical Instrument, Common or Masterwork:} A masterwork instrument grants 
a +2 circumstance bonus on Perform checks involving its use.

\textbf{Spell Component Pouch:} A spellcaster with a spell component pouch is assumed 
to have all the material components and focuses needed for spellcasting, except 
for those components that have a specific cost, divine focuses, and focuses that 
wouldn't fit in a pouch.

\textbf{Thieves' Tools (common):} This kit contains the tools you need to use the Disable 
Device and Open Lock skills. Without these tools, you must improvise tools, and 
you take a -2 circumstance penalty on Disable Device and Open Locks checks.

\textbf{Thieves' Tools (masterwork):} This kit contains extra tools and tools of 
better make, which grant a +2 circumstance bonus on Disable Device and Open Lock 
checks.

\textit{Unholy Symbols:} An unholy symbol is like a holy symbol except that it 
focuses negative energy and is used by evil clerics (or by neutral clerics who 
want to cast evil spells or command undead).

\textbf{Water Clock:} This large, bulky contrivance gives the time accurate to 
within half an hour per day since it was last set. It requires a source of water, 
and it must be kept still because it marks time by the regulated flow of droplets 
of water.

\textbf{Wizard's Spellbook (blank):} A spellbook has 100 pages of parchment, and 
each spell takes up one page per spell level (one page each for 0-level spells).

%%%%%%%%%%%%%%%%%%%%%%%%%
\subsection{Clothing}
%%%%%%%%%%%%%%%%%%%%%%%%%

\textbf{Artisan's Outfit:} This outfit includes a shirt with buttons, a skirt or 
pants with a drawstring, shoes, and perhaps a cap or hat. It may also include a 
belt or a leather or cloth apron for carrying tools.

\textbf{Cleric's Vestments:} These ecclesiastical clothes are for performing priestly 
functions, not for adventuring.

\textbf{Cold Weather Outfit:} A cold weather outfit includes a wool coat, linen 
shirt, wool cap, heavy cloak, thick pants or skirt, and

boots. This outfit grants a +5 circumstance bonus on Fortitude saving throws against 
exposure to cold weather.

\begin{wraptable}{r}{.5\linewidth}
\rowcolors{1}{colorone}{colortwo}
\caption{Clothing}
{\tabulinesep=1mm
\begin{tabu}to \linewidth{X c c}
\header\textbf{Item} & \textbf{Cost} & \textbf{Weight}\\ \hline
Artisan's outfit & 1 gp & 4 lb\textsuperscript{1}\\
Cleric's vestments & 5 gp & 6 lb\textsuperscript{1}\\
Cold weather outfit & 8 gp & 7 lb\textsuperscript{1}\\
Courtier's outfit & 30 gp & 6 lb\textsuperscript{1}\\
Entertainer's outfit & 3 gp & 4 lb\textsuperscript{1}\\
Explorer's outfit & 10 gp & 8 lb\textsuperscript{1}\\
Monk's outfit & 5 gp & 2 lb\textsuperscript{1}\\
Noble's outfit & 75 gp & 10 lb\textsuperscript{1}\\
Peasant's outfit & 1 sp & 2 lb\textsuperscript{1}\\
Royal outfit & 200 gp & 15 lb\textsuperscript{1}\\
Scholar's outfit & 5 gp & 6 lb\textsuperscript{1}\\
Traveler's outfit & 1 gp & 5 lb\textsuperscript{1}\\ \hline
\multicolumn{3}{p{\linewidth}}{\textsuperscript{1} These items weigh one-quarter this amount when made for Small characters.}\\
\hline
\end{tabu}}
\end{wraptable}

\textbf{Courtier's Outfit:} This outfit includes fancy, tailored clothes in whatever 
fashion happens to be the current style in the courts of the nobles. Anyone trying 
to influence nobles or courtiers while wearing street dress will have a hard time 
of it (-2 penalty on Charisma-based skill checks to influence such individuals). 
If you wear this outfit without jewelry (costing an additional 50 gp), you look 
like an out-of-place commoner.

\textbf{Entertainer's Outfit:} This set of flashy, perhaps even gaudy, clothes 
is for entertaining. While the outfit looks whimsical, its practical design lets 
you tumble, dance, walk a tightrope, or just run (if the audience turns ugly).

\textbf{Explorer's Outfit:} This is a full set of clothes for someone who never 
knows what to expect. It includes sturdy boots, leather breeches or a skirt, a 
belt, a shirt (perhaps with a vest or jacket), gloves, and a cloak. Rather than 
a leather skirt, a leather overtunic may be worn over a cloth skirt. The clothes 
have plenty of pockets (especially the cloak). The outfit also includes any extra 
items you might need, such as a scarf or a wide-brimmed hat.

\textbf{Monk's Outfit:} This simple outfit includes sandals, loose breeches, and 
a loose shirt, and is all bound together with sashes. The outfit is designed to 
give you maximum mobility, and it's made of high-quality fabric. You can hide small 
weapons in pockets hidden in the folds, and the sashes are strong enough to serve 
as short ropes.

\textbf{Noble's Outfit:} This set of clothes is designed specifically to be expensive 
and to show it. Precious metals and gems are worked into the clothing. To fit into 
the noble crowd, every would-be noble also needs a signet ring (see Adventuring 
Gear, above) and jewelry (worth at least 100 gp).

\textbf{Peasant's Outfit:} This set of clothes consists of a loose shirt and baggy 
breeches, or a loose shirt and skirt or overdress. Cloth wrappings are used for 
shoes.

\textbf{Royal Outfit:} This is just the clothing, not the royal scepter, crown, 
ring, and other accoutrements. Royal clothes are ostentatious, with gems, gold, 
silk, and fur in abundance.

\textbf{Scholar's Outfit:} Perfect for a scholar, this outfit includes a robe, 
a belt, a cap, soft shoes, and possibly a cloak.

\textbf{Traveler's Outfit:} This set of clothes consists of boots, a wool skirt 
or breeches, a sturdy belt, a shirt (perhaps with a vest or jacket), and an ample 
cloak with a hood.
\section{Animals and Related Gear}

\begin{wraptable}{r}{.45\textwidth}
\rowcolors{1}{colorone}{colortwo}
\caption{Mounts and Related Gear}
{\tabulinesep=1mm
\begin{tabu}to \linewidth{X r r}
\header\textbf{Item} & \textbf{Cost} & \textbf{Weight}\\ \hline
Barding&&\\
\hspace{.5cm}Medium creature & x2 & x1\\
\hspace{.5cm}Large creature & x4 & x2\\
Bit and bridle & 2 gp & 1 lb.\\
Dog&&\\
\hspace{.5cm}Guard Dog & 25 gp & --\\
\hspace{.5cm}Riding Dog & 150 gp & --\\
Donkey or mule & 8 gp & --\\
Feed (per day) & 5 cp & 10 lb.\\
Horse&&\\
\hspace{.5cm}Heavy Horse & 200 gp & --\\
\hspace{.5cm}Heavy Warhorse & 400 gp & --\\
\hspace{.5cm}Light Horse & 75 gp & --\\
\hspace{.5cm}Light Warhorse & 150 gp & --\\
\hspace{.5cm}Pony & 30 gp & --\\
\hspace{.5cm}Warpony & 100 gp & --\\
Saddle (common)&&\\
\hspace{.5cm}Military & 20 gp & 30 lb.\\
\hspace{.5cm}Pack & 5 gp & 15 lb.\\
\hspace{.5cm}Riding & 10 gp & 25 lb.\\
Saddle (exotic)&&\\
\hspace{.5cm}Military & 60 gp & 40 lb.\\
\hspace{.5cm}Pack & 15 gp & 20 lb.\\
\hspace{.5cm}Riding & 30 gp & 30 lb.\\
Saddlebags & 4 gp & 8 lb.\\
Stabling (per day) & 5 sp & --\\
\hline
\end{tabu}}

\rowcolors{1}{colorone}{colortwo}
\caption{Mount Speed In Armor}
{\tabulinesep=1mm
\begin{tabu}to \linewidth{X c c c}
\header\textbf{Barding} & \textbf{40ft} & \textbf{50ft} & \textbf{60ft}\\ \hline
Medium & 30ft & 35ft & 40ft\\
Heavy & 30ft\textsuperscript{1} & 35ft\textsuperscript{1} & 40ft\textsuperscript{1}\\ \hline
\multicolumn{4}{p{\linewidth}}{\textsuperscript{1} A mount wearing heavy armor moves at only triple its normal speed when running, instead of quadruple.}\\ \hline
\end{tabu}}
\end{wraptable}

\textbf{Barding, Medium Creature and Large Creature:} Barding is a type of armor 
that covers the head, neck, chest, body, and possibly legs of a horse or other 
mount. Barding made of medium or heavy armor provides better protection than light 
barding, but at the expense of speed. Barding can be made of any of the armor types 
found on Table: Armor and Shields.

Armor for a horse (a Large nonhumanoid creature) costs four times as much as armor 
for a human (a Medium humanoid creature) and also weighs twice as much as the armor 
found on Table: Armor and Shields (see Armor for Unusual Creatures). If the barding 
is for a pony or other Medium mount, the cost is only double, and the weight is 
the same as for Medium armor worn by a humanoid. Medium or heavy barding slows 
a mount that wears it, as shown on the table below.

Flying mounts can't fly in medium or heavy barding.

Removing and fitting barding takes five times as long as the figures given on Table: 
Donning Armor. A barded animal cannot be used to carry any load other than the 
rider and normal saddlebags.

\textbf{Dog, Riding:} This Medium dog is specially trained to carry a Small humanoid 
rider. It is brave in combat like a warhorse. You take no damage when you fall 
from a riding dog.

\textbf{Donkey or Mule:} Donkeys and mules are stolid in the face of danger, hardy, 
surefooted, and capable of carrying heavy loads over vast distances. Unlike a horse, 
a donkey or a mule is willing (though not eager) to enter dungeons and other strange 
or threatening places.

\textbf{Feed:} Horses, donkeys, mules, and ponies can graze to sustain themselves, 
but providing feed for them is much better. If you have a riding dog, you have 
to feed it at least some meat.

\textbf{Horse:} A horse (other than a pony) is suitable as a mount for a human, 
dwarf, elf, half-elf, or half-orc. A pony is smaller than a horse and is a suitable 
mount for a gnome or halfling.

Warhorses and warponies can be ridden easily into combat. Light horses, ponies, 
and heavy horses are hard to control in combat.

\textbf{Saddle, Exotic:} An exotic saddle is like a normal saddle of the same sort 
except that it is designed for an unusual mount. Exotic saddles come in military, 
pack, and riding styles.

\textbf{Saddle, Military:} A military saddle braces the rider, providing a +2 circumstance 
bonus on Ride checks related to staying in the saddle. If you're knocked unconscious 
while in a military saddle, you have a 75\% chance to stay in the saddle (compared 
to 50\% for a riding saddle).

\textbf{Saddle, Pack:} A pack saddle holds gear and supplies, but not a rider. 
It holds as much gear as the mount can carry.

\textbf{Saddle, Riding:} The standard riding saddle supports a rider
\section{Other Goods and Services}
%%%%%%%%%%%%%%%%%%%%%%%%%
\subsection{Food, Drink, and Lodging}
%%%%%%%%%%%%%%%%%%%%%%%%%

\textbf{Inn:} Poor accommodations at an inn amount to a place on the floor near 
the hearth. Common accommodations consist of a place on a raised, heated floor, 
the use of a blanket and a pillow. Good accommodations consist of a small, private 
room with one bed, some amenities, and a covered chamber pot in the corner.

\textbf{Meals:} Poor meals might be composed of bread, baked turnips, onions, and 
water. Common meals might consist of bread, chicken stew, carrots, and watered-down 
ale or wine. Good meals might be composed of bread and pastries, beef, peas, and 
ale or wine.

\begin{table}[h]
\rowcolors{1}{colorone}{colortwo}
\caption{Food, Drink, and Lodging}
\centering
{\tabulinesep=1mm
\begin{tabu}to \linewidth{X[2] X X | X[2] X X}
\header \textbf{Item} & \textbf{Cost} & \textbf{Weight} & \textbf{Item} & \textbf{Cost} & \textbf{Weight}\\ \hline
Banquet (per person) & 10 gp & --&Meals (per day)&&\\
Chunk of Meat & 3 sp & \sfrac{1}{2} lb.&\hspace{.5cm}Good & 5 sp & --\\
Hunk of Cheese & 1 sp & \sfrac{1}{2} lb.&\hspace{.5cm}Common & 3 sp & --\\
Loaf of Bread & 2 cp & \sfrac{1}{2} lb.&\hspace{.5cm}Poor & 1 sp & --\\
Ale&&&Wine&&\\
\hspace{.5cm}Gallon & 2 sp & 8 lb.&\hspace{.5cm}Common (pitcher) & 2 sp & 6 lb.\\
\hspace{.5cm}Mug & 4 cp & 1 lb. &\hspace{.5cm}Fine (bottle) & 10 gp & 1.5 lb.\\
Inn stay (per day)&&&&&\\
\hspace{.5cm}Good & 2 gp & --&&&\\
\hspace{.5cm}Common & 5 sp & --&&&\\
\hspace{.5cm}Poor & 2 sp & --&&&\\ \hline
\end{tabu}}
\end{table}

%%%%%%%%%%%%%%%%%%%%%%%%%
\subsection{Transport}
%%%%%%%%%%%%%%%%%%%%%%%%%

\textbf{Carriage}: This four-wheeled vehicle can transport as many as four people 
within an enclosed cab, plus two drivers. In general, two horses (or other beasts 
of burden) draw it. A carriage comes with the harness needed to pull it.

\textbf{Cart:} This two-wheeled vehicle can be drawn by a single horse (or other 
beast of burden). It comes with a harness.

\textbf{Galley:} This three-masted ship has seventy oars on either side and requires 
a total crew of 200. A galley is 130 feet long and 20 feet wide, and it can carry 
150 tons of cargo or 250 soldiers. For 8,000 gp more, it can be fitted with a ram 
and castles with firing platforms fore, aft, and amidships. This ship cannot make 
sea voyages and sticks to the coast. It moves about 4 miles per hour when being 
rowed or under sail.

\textbf{Keelboat:} This 50- to 75-foot-long ship is 15 to 20 feet wide and has 
a few oars to supplement its single mast with a square sail. It has a crew of eight 
to fifteen and can carry 40 to 50 tons of cargo or 100 soldiers. It can make sea 
voyages, as well as sail down rivers (thanks to its flat bottom). It moves about 
1 mile per hour.

\begin{wraptable}{r}{.4\linewidth}
\rowcolors{1}{colorone}{colortwo}
\caption{Transport}
\centering
{\tabulinesep=1mm
\begin{tabu}to \linewidth{X r r}
\header\textbf{Item} & \textbf{Cost} & \textbf{Weight}\\\hline
Carriage & 100 gp & 600 lb.\\
Cart & 15 gp & 200 lb.\\
Galley & 30,000 gp & --\\
Keelboat & 3,000 gp & --\\
Longship & 10,000 gp & --\\
Rowboat & 50 gp & 100 lb.\\
Oar & 2 gp & 10 lb.\\
Sailing ship & 10,000 gp & --\\
Sled & 20 gp & 300 lb.\\
Wagon & 35 gp & 400 lb.\\
Warship & 25,000 gp & --\\ \hline
\end{tabu}}
\end{wraptable}

\textbf{Longship:} This 75-foot-long ship with forty oars requires a total crew 
of 50. It has a single mast and a square sail, and it can carry 50 tons of cargo 
or 120 soldiers. A longship can make sea voyages. It moves about 3 miles per hour 
when being rowed or under sail.

\textbf{Rowboat:} This 8- to 12-foot-long boat holds two or three Medium passengers. 
It moves about 1.5 miles per hour.

\textbf{Sailing Ship:} This larger, seaworthy ship is 75 to 90 feet long and 20 
feet wide and has a crew of 20. It can carry 150 tons of cargo. It has square sails 
on its two masts and can make sea voyages. It moves about 2 miles per hour.

\textbf{Sled:} This is a wagon on runners for moving through snow and over ice. 
In general, two horses (or other beasts of burden) draw it. A sled comes with the 
harness needed to pull it.

\textbf{Wagon:} This is a four-wheeled, open vehicle for transporting heavy loads. 
In general, two horses (or other beasts of burden) draw it. A wagon comes with 
the harness needed to pull it.

\textbf{Warship:} This 100-foot-long ship has a single mast, although oars can 
also propel it. It has a crew of 60 to 80 rowers. This ship can carry 160 soldiers, 
but not for long distances, since there isn't room for supplies to support that 
many people. The warship cannot make sea voyages and sticks to the coast. It is 
not used for cargo. It moves about 2.5 miles per hour when being rowed or under 
sail.

%%%%%%%%%%%%%%%%%%%%%%%%%
\subsection{Spellcasting and Services}
%%%%%%%%%%%%%%%%%%%%%%%%%

\begin{table}[b]
\rowcolors{1}{colorone}{colortwo}
\caption{Spellcasting and Services}
{\tabulinesep=1mm
\begin{tabu}to \textwidth{X X}
\header\textbf{Service} & \textbf{Cost}\\ \hline
Coach cab & 3 cp per mile\\
Messenger & 2 cp per mile\\
Road or gate toll & 1 cp\\
Ship's passage & 1 sp per mile\\
Spell, 0th-level & Caster level x 5 gp\textsuperscript{1}\\
Spell, 1st-level & Caster level x 10 gp\textsuperscript{1}\\
Spell, 2nd-level & Caster level x 20 gp\textsuperscript{1}\\
Spell, 3rd-level & Caster level x 30 gp\textsuperscript{1}\\
Spell, 4th-level & Caster level x 40 gp\textsuperscript{1}\\
Spell, 5th-level & Caster level x 50 gp\textsuperscript{1}\\
Spell, 6th-level & Caster level x 60 gp\textsuperscript{1}\\
Spell, 7th-level & Caster level x 70 gp\textsuperscript{1}\\
Spell, 8th-level & Caster level x 80 gp\textsuperscript{1}\\
Spell, 9th-level & Caster level x 90 gp\textsuperscript{1}\\
Trained Hireling & 3 sp per day\\
Untrained Hireling & 1 sp per day\\\hline
\end{tabu}
\begin{tabu}to \linewidth{X}
\rowcolor{colortwo}\textsuperscript{1} See spell description for additional costs. If the additional costs put the spell's total cost above 3,000 gp, that spell is not generally available.\\ \hline
\end{tabu}}
\end{table}

Sometimes the best solution for a problem is to hire someone else to take care 
of it.

\textbf{Coach Cab:} The price given is for a ride in a coach that transports people 
(and light cargo) between towns. For a ride in a cab that transports passengers 
within a city, 1 copper piece usually takes you anywhere you need to go.

\textbf{Hireling, Trained:} The amount given is the typical daily wage for mercenary 
warriors, masons, craftsmen, scribes, teamsters, and other trained hirelings. This 
value represents a minimum wage; many such hirelings require significantly higher 
pay.

\textbf{Hireling, Untrained:} The amount shown is the typical daily wage for laborers, 
porters, cooks, maids, and other menial workers.

\textbf{Messenger:} This entry includes horse-riding messengers and runners. Those 
willing to carry a message to a place they were going anyway may ask for only half 
the indicated amount.

\textbf{Road or Gate Toll:} A toll is sometimes charged to cross a well-trodden, 
well-kept, and well-guarded road to pay for patrols on it and for its upkeep. Occasionally, 
a large walled city charges a toll to enter or exit (or sometimes just to enter).

\textbf{Ship's Passage:} Most ships do not specialize in passengers, but many have 
the capability to take a few along when transporting cargo. Double the given cost 
for creatures larger than Medium or creatures that are otherwise difficult to bring 
aboard a ship.

\textbf{Spell:} The indicated amount is how much it costs to get a spellcaster 
to cast a spell for you. This cost assumes that you can go to the spellcaster and 
have the spell cast at his or her convenience (generally at least 24 hours later, 
so that the spellcaster has time to prepare the spell in question). If you want 
to bring the spellcaster somewhere to cast a spell you need to negotiate with him 
or her, and the default answer is no.

The cost given is for a spell with no cost for a material component or focus component 
and no XP cost. If the spell includes a material component, add the cost of that 
component to the cost of the spell.

If the spell has a focus component (other than a divine focus), add 1/10 the cost 
of that focus to the cost of the spell. If the spell has an XP cost, add 5 gp per 
XP lost. 

Furthermore, if a spell has dangerous consequences, the spellcaster will certainly 
require proof that you can and will pay for dealing with any such consequences 
(that is, assuming that the spellcaster even agrees to cast such a spell, which 
isn't certain). In the case of spells that transport the caster and characters 
over a distance, you will likely have to pay for two castings of the spell, even 
if you aren't returning with the caster.

In addition, not every town or village has a spellcaster of sufficient level to 
cast any spell. In general, you must travel to a small town (or larger settlement) 
to be reasonably assured of finding a spellcaster capable of casting 1st-level 
spells, a large town for 2nd-level spells, a small city for 3rd- or 4th-level spells, 
a large city for 5th- or 6th-level spells, and a metropolis for 7th- or 8th-level 
spells. Even a metropolis isn't guaranteed to have a local spellcaster able to 
cast 9th-level spells.
\chapter{Description}
\section{Physical Appearance}
foo
\section{Personality}
foo
\section{Alignment}
foo
\section{Religion}
foo
\chapter{Adventuring}
\section{Overland Travel}
foo
\section{Exploration}
foo
\section{Traps}
foo
\section{Encounters}
foo
\chapter{Your Role in the Campaign}

%\input{phb/your-role/intro}
\section{Types of Leadership}
A character moving into a leadership position can be a natural way for a character to advance as a game progresses. There are several ways a character can do this.

\ability{Cohort}{Having a cohort is very similar to having a secondary character. You have direct control over your cohort, just as you have control over your own character. Your cohort, if you have one, always has a CR that is 2 less than your character's level. A cohort increases in power, gaining a level or hit dice, when you gain a level. You may only ever have one cohort at a time, if a source would grant you more than one cohort, you instead gain a companion. Cohorts are usually gained by taking [Leadership] feats.}

\ability{Companion}{Companions play supporting roles to your own character. You have direct control of your companions as long as they are with your character. If for some reason one of your companions is separated from your character, the companion continues to act in your best interest, typically following any orders you have given it. Your companions always have a CR equal to your level -4 or to \sfrac{1}{2} of your level (round down), whichever is greater. Companions are usually gained through class features.}

\ability{Followers}{Your followers are people that take orders from you, possibly as members of an organization you lead or have a leadership position within. Instead of keeping track of all of your followers individually, they are represented by your Leadership Score (see below). By spending points of your Leadership Score (and temporarily reducing it) you can have your followers accomplish tasks for you.}

\section{Leadership Score}

Your Leadership Score represents the amount of resources and manpower your followers can produce. Your Leadership Score has a maximum value equal to your level, and is replenished by an amount equal to your Charisma bonus each week (minimum 1). You only have one Leadership Score, if a source would grant you more than one Leadership Score, your score is instead increased by 4. Below are the tasks you can accomplish by spending your Leadership Score.

When you first gain a leadership score, you should select a location to serve as the base for your organization. The location of your base affects the DC of and the the time it takes perform certain actions.

%\begin{wraptable}{o}{.6\linewidth}
\begin{table}[h!]
\centering
\caption{Uses of Leadership}
\rowcolors{1}{colorone}{colortwo}
\begin{tabu}to \linewidth{X X}
\header Action & Leadership Cost \\ \hline
Gather Intelligence & 1\\
Labor & Varies, see text\\
Personal Retinue & Equal to EL of Followers\\
Provide Service & Varies\\
Skill Check Equivalent & 1 per +5 Bonus\\ \hline
\end{tabu}
\end{table}
%\end{wraptable}

\ability{Gather Intelligence}{You can use your followers to find out about a person place or thing. If successful you learn a single peice of information about the object of your investigation at the end of the week, and an additional peice of information for every 5 points by which your check exceeds the DC. On a failed check, you gain a piece of information that turns out to be false. The base DC for this check is 10, modified by the conditions shown on the table below. You have a bonus to the check equal to you level.
	\begin{awesomelist}
		\item \ability{Person}{A successful check indicates that you learn one thing of your choice about a person: their current location, where they plan to be during the next week, if there is currently a plot against that person, the truth about one rumor about the person.}
		\item \ability{Place}{A successful check allows you to learn one thing about the place: presence of people or monsters and a general indication of how dangerous the area is, the number of a specific group of people that you already know the presence of in the area, the presence or absence of ambush points.}
		\item \ability{Thing}{A successful check lets you learn one of the following about an object: the history of the object, the previous owner of the object, the function of the object, the activation method for the object (if magical).}
	\end{awesomelist}
	
\begin{table}[h!]
\centering
\caption{Gather Intelligence DC Modifiers}
\rowcolors{1}{colorone}{colortwo}
\begin{tabu}to \linewidth{X[3] X || X[3] X || X[3] X}
\header Person is... & DC & Place is... & DC & Thing is... & DC \\ \hline
Friendly  & -5 & Friendly Territory & -5 & In Your Posession & -5 \\
An Enemy  & +5 & Hostile Territory  & +5 & Not Specific      & +5 \\
Secretive & +5 & Remote             & +5 & Medium or Major   & +5 \\
In Hiding & +5 & Guarded            & +5 & An Artifact       & +10 \\ \hline
\end{tabu}
\end{table}
}

\ability{Labor}{You can have your followers work to build or craft something for one week. The amount of work done is based on how much of your leadership score you spend, and is equivalent to a given number of people working for one week. This cost recurs weekly if you have your followers continue to labor. If the task would normally require some sort of skill check to complete, you must arrange for that to be done separately.}

%\begin{wraptable}{o}{.4\linewidth}
\begin{table}[h!]
\centering
\caption{Cost of Labor}
\rowcolors{1}{colorone}{colortwo}
\begin{tabu}{l l}
\header Points Spent & Number of Laborers \\ \hline
1 & 1 \\
2 & 3 \\
3 & 5 \\
4 & 10 \\
5 & 25 \\
6 & 55 \\
7 & 110 \\
8 & 225 \\
9 & 450 \\
10 & 900 \\
11 & 1,800 \\
12 & 3,750 \\
13 & 7,500 \\
14 & 15,000 \\
15 & 30,000 \\
16 & 60,000 \\
17 & 125,000 \\
18 & 250,000 \\
19 & 500,000 \\
20 & 1,000,000 \\
+1 & x2 \\ \hline
\end{tabu}
\end{table}
%\end{wraptable}

\ability{Personal Retinue}{You can take followers with you on adventures. The cost to do this is equal to the encounter level (EL) of the followers you bring with you, up to a maximum of \sfrac{1}{2} of your level. The followers remain with you for one week, and the cost recurs for each week they remain with you. If some of them are killed or otherwise rendered unfit for service, a one-time penalty equal to half of the cost to bring them is incurred. If \emph{all} of them are killed or otherwise rendered unfit for service, the penalty is instead equal to the full cost to bring them. This can temporarily reduce your Leadership Score to a negative value.}

\ability{Skill Check Equivalent}{You can have your followers perform some task that is equivalent to a skill check. The skill may be any relevant skill, and the bonus is equal to your character level. For every additional point you spend, the check is made at an additional +5 bonus, up to a maximum of double your character level.}

\subsection{Followers and Location}

When you first gain your leadership score, you should select a place to serve as the base of operations for your organization. Tasks take additional time to start based on how distant the task is to take place from your base. This amount of time does not necessarily equate to the time it takes for your followers to physically reach the location, but could be the time it takes to get in touch with with local contacts or contractors.

\begin{table}[h!]
\centering
\caption{Distance and Time}
\rowcolors{1}{colorone}{colortwo}
\begin{tabu}{l l}
\header Distance to Task & Extra Time Required \\ \hline
Local & No Extra Time \\
Neighboring Province (100 mi.) & 1 Week \\
Distant Province (250 mi.) &  2 Weeks \\
Neighboring Country (500 mi.) & 1 Month \\
Distant Country (1000 mi.) & 2 Months \\
Another Continent (3000 mi) & 3 Months \\
Another Plane & 4 Months \\ \hline
\end{tabu}
\end{table}

\subsection{Rushing Your Followers}

By spending an extra 50\% of the cost of a use of your leadership (round up), you can reduce the time it takes to complete the task from a number of weeks, to an equivalent number of days.

\section{Replacing Cohorts and companions}

Sometimes followers, companions, and even cohorts might die. Or maybe the one you have just isn't cutting it anymore. Often the requirements involve some amount of in-game time passing, and this assumes that the character spends some minimal amount of time between adventures looking for suitable replacements. Unless there are extenuating circumstances, the DM should try to follow these guidelines or work out another suitable means of replacing the minion. The requirements to replace them are as follows.

\ability{Replacing a Cohort}{A cohort can be replaced any time you gain a level, or after some period of in-game time (typically a month). Your DM may also allow you to immediately replace your cohort with an appropriate NPC that is already in the campaign.}

\ability{Replacing a Companion}{Typically the steps necessary to replace a companion are given by the feat or class feature that granted it. If no method of replacement is specified, a companion can usually be replaced after one week of in-game time.}

\section{Converting Followers}

Sometimes you might want to convert a specific NPC into a follower. You can try to turn an NPC that is friendly to you with at least one minute of effort and a successful Charisma check with a DC of 10 + the NPC's CR. If the NPC is already the follower of another character the DC is increased by the Charisma bonus of the character they are already a follower of. A failure carries no penalty, but you cannot try to convert the same NPC again.
%\input{phb/your-role/owning-a-business}
%\input{phb/your-role/going-to-war}
\chapter{Combat}
\section{How Combat Works}
Combat is cyclical; everybody acts in turn in a regular cycle of rounds. Combat follows this sequence:
\vspace*{10pt}
\begin{enumerate}
	\item{Each combatant starts out flat-footed. Once a combatant acts, he or she is no longer flat-footed.}
	\item{Determine which characters are aware of their opponents at the start of the battle. If some but not all of the combatants are aware of their opponents, a surprise round happens before regular rounds of combat begin. The combatants who are aware of the opponents can act in the surprise round, so they roll for initiative. In initiative order (highest to lowest), combatants who started the battle aware of their opponents each take one action (either a standard action or a move action) during the surprise round. Combatants who were unaware do not get to act in the surprise round. If no one or everyone starts the battle aware, there is no surprise round.}
	\item{Combatants who have not yet rolled initiative do so. All combatants are now ready to begin their first regular round of combat.}
	\item{Combatants act in initiative order (highest to lowest).}
	\item{When everyone has had a turn, the combatant with the highest initiative acts again, and steps 4 and 5 repeat until combat ends.}
\end{enumerate}

\section{Combat Statistics}

This section summarizes the statistics that determine success in combat, and then details how to use them.

\subsection{Attack Rolls}
An attack roll represents your attempt to strike your opponent on your turn in a round. When you make an attack roll, you roll a d20 and add your attack bonus. (Other modifiers may also apply to this roll.) If your result equals or beats the target's Armor Class, you hit and deal damage.

\vspace*{10pt}

\ability{Automatic Misses and Hits: }A natural 1 (the d20 comes up 1) on an attack roll is always a miss. A natural 20 (the d20 comes up 20) is always a hit. A natural 20 is also a threat�a possible critical hit.

\subsection{Attack Bonus}

Your attack bonus with a melee weapon is your Base attack bonus + Strength modifier + size modifier. With a ranged weapon, your attack bonus is your Base attack bonus + Dexterity modifier + size modifier + range penalty. See the table below for size modifiers.

\vspace*{10pt}

\ability{Base Attack Bonus:}A base attack bonus is an attack roll bonus derived from character class and level or creature type and Hit Dice (or combinations thereof). Base attack bonuses increase at different rates for different character classes and creature types. A second attack is gained when a base attack bonus reaches +6, a third with a base attack bonus of +11 or higher, and a fourth with a base attack bonus of +16 or higher. Base attack bonuses gained from different sources, such as when a character is a multiclass character, stack. 

\subsection{Damage}
When your attack succeeds, you deal damage. The type of weapon used determines the amount of damage you deal. Effects that modify weapon damage apply to unarmed strikes and the natural physical attack forms of creatures. Damage reduces a target's current hit points.

\vspace*{10pt}

\ability{Minimum Damage:}If penalties reduce the damage result to less than 1, a hit still deals 1 point of damage.

\ability{Strength Bonus:}When you hit with a melee or thrown weapon, including a sling, add your Strength modifier to the damage result. A Strength penalty, but not a bonus, applies on attacks made with a bow that is not a composite bow.

\ability{Off-Hand Weapon:}When you deal damage with a weapon in your off hand, you add only 1/2 your Strength bonus.

\ability{Wielding a Weapon Two-Handed:}When you deal damage with a weapon that you are wielding two-handed, you add 1-1/2 times your Strength bonus. However, you don't get this higher Strength bonus when using a light weapon with two hands.

\ability{Multiplying Damage:}Sometimes you multiply damage by some factor, such as on a critical hit. Roll the damage (with all modifiers) multiple times and total the results. Note: When you multiply damage more than once, each multiplier works off the original, unmultiplied damage.

\ability{Exception:}Extra damage dice over and above a weapon's normal damage are never multiplied.

\ability{Ability Damage:}Certain creatures and magical effects can cause temporary ability damage (a reduction to an ability score).

\subsection{Armor}

Your Armor Class (AC) represents how hard it is for opponents to land a solid, damaging blow on you. It's the attack roll result that an opponent needs to achieve to hit you. Your AC is equal to 10 + armor bonus + shield bonus + Dexterity modifier + size modifier. Note that armor limits your Dexterity bonus, so if you're wearing armor, you might not be able to apply your whole Dexterity bonus to your AC. Sometimes you can't use your Dexterity bonus (if you have one). If you can't react to a blow, you can't use your Dexterity bonus to AC. (If you don't have a Dexterity bonus, nothing happens.)

\vspace*{10pt}

\noindent Many other factors may modify your AC.

\vspace*{10pt}

\begin{wraptable}{O}{3.5in}
\caption{Size Modifiers}
\begin{tabular}[h]{c|c||c|c}
Size & Size Modifier & Size & Size Modifier \\ \hline
Colossal & \textendash8 & Small & +1 \\
Gargantuan & \textendash4 & Tiny & +2 \\
Huge & \textendash2 & Diminutive & +4	\\
Large & \textendash1 & Fine & +8 \\
Medium & +0	& & \\
\end{tabular}
\end{wraptable}
\ability{Enhancement Bonuses:}Enhancement effects make your armor better.

\ability{Deflection Bonus:}Magical deflection effects ward off attacks and improve your AC.

\ability{Natural Armor:}Natural armor improves your AC, if you wear armor you only receive one half of your natural armor bonus or one half of your armor bonus (whichever is higher).

\ability{Dodge Bonuses:}Some other AC bonuses represent actively avoiding blows. These bonuses are called dodge bonuses. Any situation that denies you your Dexterity bonus also denies you dodge bonuses. (Wearing armor, however, does not limit these bonuses the way it limits a Dexterity bonus to AC.) Unlike most sorts of bonuses, dodge bonuses stack with each other.

\ability{Touch Attacks:}Some attacks disregard armor, including shields and natural armor. In these cases, the attacker makes a touch attack roll (either ranged or melee). When you are the target of a touch attack, your AC doesn't include any armor bonus, shield bonus, or natural armor bonus. All other modifiers, such as your size modifier, Dexterity modifier, and deflection bonus (if any) apply normally.

\subsection{Hit Points}

When your hit point total reaches 0, you're disabled. When it reaches -1, you're dying. When it gets to -10, you're dead.

\subsection{Speed}

Your speed tells you how far you can move in a round and still do something, such as attack or cast a spell. Your speed depends mostly on your race and what armor you're wearing. If you use two move actions in a round (sometimes called a ``double move'' action), you can move up to double your speed. If you spend the entire round to run all out, you can move up to quadruple your speed (or triple if you are in heavy armor).

\subsection{Saving Throws}

Generally, when you are subject to an unusual or magical attack, you get a saving throw to avoid or reduce the effect. Like an attack roll, a saving throw is a d20 roll plus a bonus based on your class, level, and an ability score. Your saving throw modifier is: Base save bonus + ability modifier

\vspace*{10pt}

\ability{Saving Throw Types:}The three different kinds of saving throws are Fortitude, Reflex, and Will.

\begin{description}
	\item[Fortitude:]These saves measure your ability to stand up to physical punishment or attacks against your vitality and health. Apply your Constitution modifier to your Fortitude saving throws.
	\item[Reflex:]These saves test your ability to dodge area attacks. Apply your Dexterity modifier to your Reflex saving throws.
	\item[Will:]These saves reflect your resistance to mental influence as well as many magical effects. Apply your Wisdom modifier to your Will saving throws.
\end{description}

\ability{Saving Throw Difficulty Class:}The DC for a save is determined by the attack itself.

\ability{Automatic Failures and Successes: }A natural 1 (the d20 comes up 1) on a saving throw is always a failure (and may cause damage to exposed items; see Items Surviving after a Saving Throw). A natural 20 (the d20 comes up 20) is always a success.

\subsection{Initiative}

\ability{Initiative Checks:}At the start of a battle, each combatant makes an initiative check. An initiative check is a Dexterity check. Each character applies his or her Dexterity modifier to the roll. Characters act in order, counting down from highest result to lowest. In every round that follows, the characters act in the same order (unless a character takes an action that results in his or her initiative changing; see Special Initiative Actions).

If two or more combatants have the same initiative check result, the combatants who are tied act in order of total initiative modifier (highest first). If there is still a tie, the tied characters should roll again to determine which one of them goes before the other.

\ability{Flat-Footed:}At the start of a battle, before you have had a chance to act (specifically, before your first regular turn in the initiative order), you are flat-footed. You can't use your Dexterity bonus to AC (if any) while flat-footed. Barbarians and rogues have the uncanny dodge extraordinary ability, which allows them to avoid losing their Dexterity bonus to AC due to being flat-footed. A flat-footed character can't make attacks of opportunity.

\ability{Inaction:}Even if you can't take actions, you retain your initiative score for the duration of the encounter.

\subsection{Surprise}

When a combat starts, if you are not aware of your opponents and they are aware of you, you're surprised.

\vspace*{10pt}

\ability{Determining Awareness}

Sometimes all the combatants on a side are aware of their opponents, sometimes none are, and sometimes only some of them are. Sometimes a few combatants on each side are aware and the other combatants on each side are unaware.
Determining awareness may call for Listen checks, Spot checks, or other checks.
The Surprise Round: If some but not all of the combatants are aware of their opponents, a surprise round happens before regular rounds begin. Any combatants aware of the opponents can act in the surprise round, so they roll for initiative. In initiative order (highest to lowest), combatants who started the battle aware of their opponents each take a standard action during the surprise round. You can also take free actions during the surprise round. If no one or everyone is surprised, no surprise round occurs.

\vspace*{10pt}

\ability{Unaware Combatants:}Combatants who are unaware at the start of battle don't get to act in the surprise round. Unaware combatants are flat-footed because they have not acted yet, so they lose any Dexterity bonus to AC.

\subsection{Attacks of Opportunity}

Sometimes a combatant in a melee lets her guard down. In this case, combatants near her can take advantage of her lapse in defense to attack her for free. These free attacks are called attacks of opportunity.

\vspace*{10pt}

\ability{Threatened Squares:}You threaten all squares into which you can make a melee attack, even when it is not your action. Generally, that means everything in all squares adjacent to your space (including diagonally). An enemy that takes certain actions while in a threatened square provokes an attack of opportunity from you. If you're unarmed, you don't normally threaten any squares and thus can't make attacks of opportunity.

\ability{Reach Weapons:}Most creatures of Medium or smaller size have a reach of only 5 feet. This means that they can make melee attacks only against creatures up to 5 feet (1 square) away. However, Small and Medium creatures wielding reach weapons threaten more squares than a typical creature. In addition, most creatures larger than Medium have a natural reach of 10 feet or more.

\ability{Provoking an Attack of Opportunity:}Two kinds of actions can provoke attacks of opportunity: moving out of a threatened square and performing an action within a threatened square.

\ability{Moving:}Moving out of a threatened square usually provokes an attack of opportunity from the threatening opponent. There are two common methods of avoiding such an attack�the 5-foot-step and the withdraw action (see below).

\ability{Performing a Distracting Act:}Some actions, when performed in a threatened square, provoke attacks of opportunity as you divert your attention from the battle. Table: Actions in Combat notes many of the actions that provoke attacks of opportunity. Remember that even actions that normally provoke attacks of opportunity may have exceptions to this rule.

\ability{Making an Attack of Opportunity:}An attack of opportunity is a single melee attack, and you can only make one per round. You don't have to make an attack of opportunity if you don't want to. An experienced character gets additional regular melee attacks (by using the full attack action), but at a lower attack bonus. You make your attack of opportunity, however, at your normal attack bonus�even if you've already attacked in the round. An attack of opportunity ``interrupts'' the normal flow of actions in the round. If an attack of opportunity is provoked, immediately resolve the attack of opportunity, then continue with the next character's turn (or complete the current turn, if the attack of opportunity was provoked in the midst of a character's turn).

\ability{Number of Attacks of Opportunity:}You can make a number of attacks of opportunity equal to the number of attacks granted by your Base Attack Bonus in a round. A character with less than a BAB of +6 can make 1 AoO each round, a character with a BAB of +6 can make 2 AoOs each round, a character with a BAB of +11 can make 3, and a character with +16 can make 4. This ability does not let you make more than one attack for a given opportunity, but if the same opponent provokes two attacks of opportunity from you, you could make two separate attacks of opportunity (since each one represents a different opportunity). Moving out of more than one square threatened by the same opponent in the same round doesn't count as more than one opportunity for that opponent. All these attacks are at your full normal attack bonus.

\section{Actions in Combat}

\subsection{The Combat Round}

Each round represents 6 seconds in the game world. A round presents an opportunity for each character involved in a combat situation to take an action. 

Each round's activity begins with the character with the highest initiative result and then proceeds, in order, from there. Each round of a combat uses the same initiative order. When a character's turn comes up in the initiative sequence, that character performs his entire round's worth of actions. (For exceptions, see Attacks of Opportunity and Special Initiative Actions.)

For almost all purposes, there is no relevance to the end of a round or the beginning of a round. A round can be a segment of game time starting with the first character to act and ending with the last, but it usually means a span of time from one round to the same initiative count in the next round. Effects that last a certain number of rounds end just before the same initiative count that they began on.

\subsection{Action Types}

An action's type essentially tells you how long the action takes to perform (within the framework of the 6-second combat round) and how movement is treated. There are four types of actions: standard actions, move actions, full-round actions, and free actions.
In a normal round, you can perform a standard action and a move action, or you can perform a full-round action. You can also perform one or more free actions. You can always take a move action in place of a standard action. In some situations (such as in a surprise round), you may be limited to taking only a single move action or standard action.

\vspace*{10pt}

\ability{Standard Action:}A standard action allows you to do something, most commonly make an attack or cast a spell. See Table: Actions in Combat for other standard actions.

\ability{Move Action:}A move action allows you to move your speed or perform an action that takes a similar amount of time. See Table: Actions in Combat. You can take a move action in place of a standard action. If you move no actual distance in a round (commonly because you have swapped your move for one or more equivalent actions), you can take one 5-foot step either before, during, or after the action.

\ability{Full-Round Action:}A full-round action consumes all your effort during a round. The only movement you can take during a full-round action is a 5-foot step before, during, or after the action. You can also perform free actions (see below).
Some full-round actions do not allow you to take a 5-foot step. Some full-round actions can be taken as standard actions, but only in situations when you are limited to performing only a standard action during your round. The descriptions of specific actions, below, detail which actions allow this option.

\ability{Swift Action:}A swift action consumes a very small amount of time, but represents a larger expenditure of effort and energy than a free action. You can perform only a single swift action per turn. 

\ability{Immediate Action:}An immediate action is very similar to a swift action, but can be performed at any time � even if it's not your turn. 

\ability{Free Action:}Free actions consume a very small amount of time and effort. You can perform one or more free actions while taking another action normally. However, there are reasonable limits on what you can really do for free.

\ability{Not an Action:}Some activities are so minor that they are not even considered free actions. They literally don't take any time at all to do and are considered an inherent part of doing something else.

\ability{Restricted Activity:}In some situations, you may be unable to take a full round's worth of actions. In such cases, you are restricted to taking only a single standard action or a single move action (plus free actions as normal). You can't take a full-round action (though you can start or complete a full-round action by using a standard action; see below).

%Table: Actions in Combat		   

\subsection{Standard Actions}

\subsubsection{Attack}

Making an attack is a standard action.

\ability{Melee Attacks:}With a normal melee weapon, you can strike any opponent within 5 feet. (Opponents within 5 feet are considered adjacent to you.) Some melee weapons have reach, as indicated in their descriptions. With a typical reach weapon, you can strike opponents 10 feet away, but you can't strike adjacent foes (those within 5 feet).

\ability{Unarmed Attacks:}Striking for damage with punches, kicks, and head butts is much like attacking with a melee weapon, except for the following:

\listtwo
		\item\ability{Attacks of Opportunity:}Attacking unarmed provokes an attack of opportunity from the character you attack, provided she is armed. The attack of opportunity comes before your attack. An unarmed attack does not provoke attacks of opportunity from other foes nor does it provoke an attack of opportunity from an unarmed foe. An unarmed character can't take attacks of opportunity (but see ``Armed'' Unarmed Attacks, below).

		\item\ability{``Armed'' Unarmed Attacks:}Sometimes a character's or creature's unarmed attack counts as an armed attack. A monk, a character with the Improved Unarmed Strike feat, a spellcaster delivering a touch attack spell, and a creature with natural physical weapons all count as being armed. Note that being armed counts for both offense and defense (the character can make attacks of opportunity)
		
		\item\ability{Unarmed Strike Damage:}An unarmed strike from a Medium character deals 1d3 points of damage (plus your Strength modifier, as normal). A Small character's unarmed strike deals 1d2 points of damage, while a Large character's unarmed strike deals 1d4 points of damage. All damage from unarmed strikes is nonlethal damage. Unarmed strikes count as light weapons (for purposes of two-weapon attack penalties and so on).

		\item\ability{Dealing Lethal Damage:}You can specify that your unarmed strike will deal lethal damage before you make your attack roll, but you take a �4 penalty on your attack roll. If you have the Improved Unarmed Strike feat, you can deal lethal damage with an unarmed strike without taking a penalty on the attack roll.
\end{list}

\ability{Ranged Attacks:}With a ranged weapon, you can shoot or throw at any target that is within the weapon's maximum range and in line of sight. The maximum range for a thrown weapon is five range increments. For projectile weapons, it is ten range increments. Some ranged weapons have shorter maximum ranges, as specified in their descriptions.

\ability{Attack Rolls:}An attack roll represents your attempts to strike your opponent.  Your attack roll is 1d20 + your attack bonus with the weapon you're using. If the result is at least as high as the target's AC, you hit and deal damage.

\ability{Automatic Misses and Hits:}A natural 1 (the d20 comes up 1) on the attack roll is always a miss. A natural 20 (the d20 comes up 20) is always a hit. A natural 20 is also a threat�a possible critical hit.

\ability{Damage Rolls:}If the attack roll result equals or exceeds the target's AC, the attack hits and you deal damage. Roll the appropriate damage for your weapon. Damage is deducted from the target's current hit points.

\ability{Multiple Attacks:}A character who can make more than one attack per round must use the full attack action (see Full-Round Actions, below) in order to get more than one attack.

\ability{Shooting or Throwing into a Melee:}If you shoot or throw a ranged weapon at a target engaged in melee with a friendly character, you take a �4 penalty on your attack roll. Two characters are engaged in melee if they are enemies of each other and either threatens the other. (An unconscious or otherwise immobilized character is not considered engaged unless he is actually being attacked.) If your target (or the part of your target you're aiming at, if it's a big target) is at least 10 feet away from the nearest friendly character, you can avoid the �4 penalty, even if the creature you're aiming at is engaged in melee with a friendly character.

\ability{Precise Shot:}If you have Precise Shot from the Sniper feat you don't take this penalty.

\ability{Fighting Defensively as a Standard Action:}You can choose to fight defensively when attacking. If you do so, you take a �4 penalty on all attacks in a round to gain a +2 dodge bonus to AC for the same round.

\ability{Critical Hits:}When you make an attack roll and get a natural 20 (the d20 shows 20), you hit regardless of your target's Armor Class, and you have scored a threat. The hit might be a critical hit (or �crit�). To find out if it's a critical hit, you immediately make a critical roll�another attack roll with all the same modifiers as the attack roll you just made. If the critical roll also results in a hit against the target's AC, your original hit is a critical hit. (The critical roll just needs to hit to give you a crit. It doesn't need to come up 20 again.) If the critical roll is a miss, then your hit is just a regular hit. A critical hit means that you roll your damage more than once, with all your usual bonuses, and add the rolls together. Unless otherwise specified, the threat range for a critical hit on an attack roll is 20, and the multiplier is x2.

\listtwo\item\ability{Exception:}Extra damage over and above a weapon's normal damage is not multiplied when you score a critical hit.
Increased Threat Range: Sometimes your threat range is greater than 20. That is, you can score a threat on a lower number. In such cases, a roll of lower than 20 is not an automatic hit. Any attack roll that doesn't result in a hit is not a threat.\end{list}

\ability{Increased Critical Multiplier:}Some weapons deal better than double damage on a critical hit.

\ability{Spells and Critical Hits:}A spell that requires an attack roll can score a critical hit. A spell attack that requires no attack roll cannot score a critical hit.

\subsubsection{Casting a Spell}

Most spells require 1 standard action to cast. You can cast such a spell either before or after you take a move action. Note: You retain your Dexterity bonus to AC while casting.

\ability{Spell Components:}To cast a spell with a verbal (V) component, your character must speak in a firm voice. If you're gagged or in the area of a silence spell, you can't cast such a spell. A spellcaster who has been deafened has a 20\% chance to spoil any spell he tries to cast if that spell has a verbal component.
		
\listone 
		\item To cast a spell with a somatic (S) component, you must gesture freely with at least one hand. You can't cast a spell of this type while bound, grappling, or with both your hands full or occupied.
			
		\item To cast a spell with a material (M), focus (F), or divine focus (DF) component, you have to have the proper materials, as described by the spell. Unless these materials are elaborate preparing these materials is a free action. For material components and focuses whose costs are not listed, you can assume that you have them if you have your spell component pouch.
				
		\item Some spells have an experience point (XP) component and entail an experience point cost to you. No spell can restore the lost XP. You cannot spend so much XP that you lose a level, so you cannot cast the spell unless you have enough XP to spare. However, you may, on gaining enough XP to achieve a new level, immediately spend the XP on casting the spell rather than keeping it to advance a level. The XP are expended when you cast the spell, whether or not the casting succeeds.
\end{list}
		
\ability{Concentration:}You must concentrate to cast a spell. If you can't concentrate you can't cast a spell. If you start casting a spell but something interferes with your concentration you must make a Concentration check or lose the spell. The check's DC depends on what is threatening your concentration (see the Concentration skill). If you fail, the spell fizzles with no effect. If you prepare spells, it is lost from preparation. If you cast at will, it counts against your daily limit of spells even though you did not cast it successfully.
		
\ability{Concentrating to Maintain a Spell:}Some spells require continued concentration to keep them going. Concentrating to maintain a spell is a standard action that doesn't provoke an attack of opportunity. Anything that could break your concentration when casting a spell can keep you from concentrating to maintain a spell. If your concentration breaks, the spell ends.

\ability{Casting Time:}Most spells have a casting time of 1 standard action. A spell cast in this manner immediately takes effect.
Attacks of Opportunity: Generally, if you cast a spell, you provoke attacks of opportunity from threatening enemies. If you take damage from an attack of opportunity, you must make a Concentration check (DC 10 + points of damage taken + spell level) or lose the spell. Spells that require only a free action to cast don't provoke attacks of opportunity.

\ability{Casting on the Defensive:}Casting a spell while on the defensive does not provoke an attack of opportunity. It does, however, require a Concentration check (DC 15 + spell level) to pull off. Failure means that you lose the spell.
	
\ability{Touch Spells in Combat:}Many spells have a range of touch. To use these spells, you cast the spell and then touch the subject, either in the same round or any time later. In the same round that you cast the spell, you may also touch (or attempt to touch) the target. You may take your move before casting the spell, after touching the target, or between casting the spell and touching the target. You can automatically touch one friend or use the spell on yourself, but to touch an opponent, you must succeed on an attack roll.
	
\ability{Touch Attacks:}Touching an opponent with a touch spell is considered to be an armed attack and therefore does not provoke attacks of opportunity. However, the act of casting a spell does provoke an attack of opportunity. Touch attacks come in two types: melee touch attacks and ranged touch attacks. You can score critical hits with either type of attack. Your opponent's AC against a touch attack does not include any armor bonus, shield bonus, or natural armor bonus. His size modifier, Dexterity modifier, and deflection bonus (if any) all apply normally.
	
\ability{Holding the Charge:}If you don't discharge the spell in the round when you cast the spell, you can hold the discharge of the spell (hold the charge) indefinitely. You can continue to make touch attacks round after round. You can touch one friend as a standard action or up to six friends as a full-round action. If you touch anything or anyone while holding a charge, even unintentionally, the spell discharges. If you cast another spell, the touch spell dissipates. Alternatively, you may make a normal unarmed attack (or an attack with a natural weapon) while holding a charge. In this case, you aren't considered armed and you provoke attacks of opportunity as normal for the attack. (If your unarmed attack or natural weapon attack doesn't provoke attacks of opportunity, neither does this attack.) If the attack hits, you deal normal damage for your unarmed attack or natural weapon and the spell discharges. If the attack misses, you are still holding the charge.

\ability{Dismiss a Spell:}Dismissing an active spell is a standard action that doesn't provoke attacks of opportunity.

\subsubsection{Activate Magic Item}

Many magic items don't need to be activated. However, certain magic items need to be activated, especially potions, scrolls, wands, rods, and staffs. Activating a magic item is a standard action (unless the item description indicates otherwise).

\ability{Spell Completion Items:}Activating a spell completion item is the equivalent of casting a spell. It requires concentration and provokes attacks of opportunity. You lose the spell if your concentration is broken, and you can attempt to activate the item while on the defensive, as with casting a spell.

\ability{Spell Trigger, Command Word, or Use-Activated Items:}Activating any of these kinds of items does not require concentration and does not provoke attacks of opportunity.

\subsubsection{Use Special Ability}

Using a special ability is usually a standard action, but whether it is a standard action, a full-round action, or not an action at all is defined by the ability.

\ability{Spell-Like Abilities:}Using a spell-like ability works like casting a spell in that it requires concentration and provokes attacks of opportunity. Spell-like abilities can be disrupted. If your concentration is broken, the attempt to use the ability fails, but the attempt counts as if you had used the ability. The casting time of a spell-like ability is 1 standard action, unless the ability description notes otherwise.

\ability{Using a Spell-Like Ability on the Defensive:}You may attempt to use a spell-like ability on the defensive, just as with casting a spell. If the Concentration check (DC 15 + spell level) fails, you can't use the ability, but the attempt counts as if you had used the ability.

\ability{Supernatural Abilities:}Using a supernatural ability is usually a standard action (unless defined otherwise by the ability's description). Its use cannot be disrupted, does not require concentration, and does not provoke attacks of opportunity.
Extraordinary Abilities: Using an extraordinary ability is usually not an action because most extraordinary abilities automatically happen in a reactive fashion. Those extraordinary abilities that are actions are usually standard actions that cannot be disrupted, do not require concentration, and do not provoke attacks of opportunity.

\subsubsection{Total Defense}

You can defend yourself as a standard action. You get a +4 dodge bonus to your AC for 1 round. Your AC improves at the start of this action. You can't combine total defense with fighting defensively or with the benefit of the Combat Expertise feat (since both of those require you to declare an attack or full attack). You can't make attacks of opportunity while using total defense.

\subsubsection{Start/Complete Full-Round Action}

The ``start full-round action'' standard action lets you start undertaking a full-round action, which you can complete in the following round by using another standard action. You can't use this action to start or complete a full attack, charge, run, or withdraw.

\subsection{Move Actions}

With the exception of specific movement-related skills, most move actions don't require a check.

\subsubsection{Move}

The simplest move action is moving your speed. If you take this kind of move action during your turn, you can't also take a 5-foot step.
Many nonstandard modes of movement are covered under this category, including climbing (up to one-quarter of your speed) and swimming (up to one-quarter of your speed).

\ability{Accelerated Climbing:}You can climb one-half your speed as a move action by accepting a �5 penalty on your Climb check.

\ability{Crawling:}You can crawl 5 feet as a move action. Crawling incurs attacks of opportunity from any attackers who threaten you at any point of your crawl.

\subsubsection{Draw or Sheathe a Weapon}

Drawing a weapon so that you can use it in combat, or putting it away so that you have a free hand, requires a move action. This action also applies to weapon-like objects carried in easy reach, such as wands. If your weapon or weapon-like object is stored in a pack or otherwise out of easy reach, treat this action as retrieving a stored item. If you have a base attack bonus of +1 or higher, you may draw a weapon as a free action combined with a regular move. If you have the Two-Weapon Fighting feat, you can draw two light or one-handed weapons in the time it would normally take you to draw one. Drawing ammunition for use with a ranged weapon (such as arrows, bolts, sling bullets, or shuriken) is a free action.

\subsubsection{Ready or Loose a Shield}

Strapping a shield to your arm to gain its shield bonus to your AC, or unstrapping and dropping a shield so you can use your shield hand for another purpose, requires a move action. If you have a base attack bonus of +1 or higher, you can ready or loose a shield as a free action combined with a regular move. Dropping a carried (but not worn) shield is a free action.

\subsubsection{Manipulate an Item}

In most cases, moving or manipulating an item is a move action. This includes retrieving or putting away a stored item, picking up an item, moving a heavy object, and opening a door. Examples of this kind of action, along with whether they incur an attack of opportunity, are given in Table: Actions in Combat.

\subsubsection{Direct or Redirect a Spell}

Some spells allow you to redirect the effect to new targets or areas after you cast the spell. Redirecting a spell requires a move action and does not provoke attacks of opportunity or require concentration.

\subsubsection{Stand Up}

Standing up from a prone position requires a move action and provokes attacks of opportunity.

\subsubsection{Mount/Dismount a Steed}

Mounting or dismounting from a steed requires a move action.

\ability{Fast Mount or Dismount:}You can mount or dismount as a free action with a DC 20 Ride check (your armor check penalty, if any, applies to this check). If you fail the check, mounting or dismounting is a move action instead. (You can't attempt a fast mount or fast dismount unless you can perform the mount or dismount as a move action in the current round.)

\subsection{Full-Round Actions}

A full-round action requires an entire round to complete. Thus, it can't be coupled with a standard or a move action, though if it does not involve moving any distance, you can take a 5-foot step.

\subsubsection{Full Attack}

If you get more than one attack per round because your base attack bonus is high enough, because you fight with two weapons or a double weapon or for some special reason you must use a full-round action to get your additional attacks. You do not need to specify the targets of your attacks ahead of time. You can see how the earlier attacks turn out before assigning the later ones. The only movement you can take during a full attack is a 5-foot step. You may take the step before, after, or between your attacks. If you get multiple attacks because your base attack bonus is high enough, you may make the attacks in any order you want. All extra attacks derived from base attack bonus are made at a \textendash5 penalty. If you are using two weapons, you can strike with either weapon first. If you are using a double weapon, you can strike with either part of the weapon first.

\vspace*{10pt}

\ability{Deciding between an Attack or a Full Attack:}After your first attack, you can decide to take a move action instead of making your remaining attacks, depending on how the first attack turns out. If you've already taken a 5-foot step, you can't use your move action to move any distance, but you could still use a different kind of move action.

\ability{Fighting Defensively as a Full-Round Action:}You can choose to fight defensively when taking a full attack action. If you do so, you take a �4 penalty on all attacks in a round to gain a +2 dodge bonus to AC for the same round.
Cleave: The extra attack granted by the Cleave feat or Great Cleave feat can be taken whenever they apply. This is an exception to the normal limit to the number of attacks you can take when not using a full attack action.

\ability{Natural Attacks:}{During a full attack a creature may attack once with each natural weapon it has.  Primary natural weapons take no penalty to hit or damage (1 times strength modifier, or 1 and a \half times strength modifier if it is the creatures only natural attack), but secondary natural weapons take a \textendash5 penalty to hit and only deal \half strength modifier damage on a successful hit.}

\subsubsection{Cast a Spell}

A spell that takes 1 round to cast is a full-round action. It comes into effect just before the beginning of your turn in the round after you began casting the spell. You then act normally after the spell is completed.  A spell that takes 1 minute to cast comes into effect just before your turn 1 minute later (and for each of those 10 rounds, you are casting a spell as a full-round action). These actions must be consecutive and uninterrupted, or the spell automatically fails.

When you begin a spell that takes 1 round or longer to cast, you must continue the invocations, gestures, and concentration from one round to just before your turn in the next round (at least). If you lose concentration after starting the spell and before it is complete, you lose the spell.

You only provoke attacks of opportunity when you begin casting a spell, even though you might continue casting for at least one full round. While casting a spell, you don't threaten any squares around you. This action is otherwise identical to the cast a spell action described under Standard Actions.

\vspace*{10pt}

\ability{Casting a Metamagic Spell:}Sorcerers and bards must take more time to cast a metamagic spell (one enhanced by a metamagic feat) than a regular spell. If a spell's normal casting time is 1 standard action, casting a metamagic version of the spell is a full-round action for a sorcerer or bard. Note that this isn't the same as a spell with a 1-round casting time�the spell takes effect in the same round that you begin casting, and you aren't required to continue the invocations, gestures, and concentration until your next turn. For spells with a longer casting time, it takes an extra full-round action to cast the metamagic spell.

Clerics must take more time to spontaneously cast a metamagic version of a cure or inflict spell. Spontaneously casting a metamagic version of a spell with a casting time of 1 standard action is a full-round action, and spells with longer casting times take an extra full-round action to cast.

\subsubsection{Use Special Ability}

Using a special ability is usually a standard action, but some may be full-round actions, as defined by the ability.

\subsubsection{Withdraw}

Withdrawing from melee combat is a full-round action. When you withdraw, you can move up to double your speed. The square you start out in is not considered threatened by any opponent you can see, and therefore visible enemies do not get attacks of opportunity against you when you move from that square. (Invisible enemies still get attacks of opportunity against you, and you can't withdraw from combat if you're blinded.) You can't take a 5-foot step during the same round in which you withdraw.

If, during the process of withdrawing, you move out of a threatened square (other than the one you started in), enemies get attacks of opportunity as normal. 

You may not withdraw using a form of movement for which you don't have a listed speed. 
Note that despite the name of this action, you don't actually have to leave combat entirely.

\vspace*{10pt}

\ability{Restricted Withdraw:}If you are limited to taking only a standard action each round you can withdraw as a standard action. In this case, you may move up to your speed (rather than up to double your speed).

\subsubsection{Run}

You can run as a full-round action. (If you do, you do not also get a 5-foot step.) When you run, you can move up to four times your speed in a straight line (or three times your speed if you're in heavy armor). You lose any Dexterity bonus to AC unless you have the Run feat 

You can run for a number of rounds equal to your Constitution score, but after that you must make a DC 10 Constitution check to continue running. You must check again each round in which you continue to run, and the DC of this check increases by 1 for each check you have made. When you fail this check, you must stop running. A character who has run to his limit must rest for 1 minute (10 rounds) before running again. During a rest period, a character can move no faster than a normal move action.
You can't run across difficult terrain or if you can't see where you're going.
A run represents a speed of about 12 miles per hour for an unencumbered human.

\subsubsection{Move 5 Feet through Difficult Terrain}

In some situations, your movement may be so hampered that you don't have sufficient speed even to move 5 feet (a single square). In such a case, you may spend a full-round action to move 5 feet (1 square) in any direction, even diagonally. Even though this looks like a 5-foot step, it's not, and thus it provokes attacks of opportunity normally.

\subsection{Swift Actions}

A swift action consumes a very small amount of time, but represents a larger expenditure of effort and energy than a free action. You can perform one swift action per turn without affecting your ability to perform other actions. In that regard, a swift action is like a free action. However, you can perform only a single swift action per turn, regardless of what other actions you take. You can take a swift action any time you would normally be allowed to take a free action. Swift actions usually involve spellcasting or the activation of magic items; many characters (especially those who don't cast spells) never have an opportunity to take a swift action.

Casting a quickened spell is a swift action. In addition, casting any spell with a casting time of 1 swift action is a swift action.

Casting a spell with a casting time of 1 swift action does not provoke attacks of opportunity.

\subsection{Immediate Actions}

Much like a swift action, an immediate action consumes a very small amount of time, but represents a larger expenditure of effort and energy than a free action. However, unlike a swift action, an immediate action can be performed at any time � even if it's not your turn. Casting feather fall is an immediate action, since the spell can be cast at any time.

Using an immediate action on your turn is the same as using a swift action, and counts as your swift action for that turn. You cannot use another immediate action or a swift action until after your next turn if you have used an immediate action when it is not currently your turn (effectively, using an immediate action before your turn is equivalent to using your swift action for the coming turn). You also cannot use an immediate action if you are flat-footed. 

\subsection{Free Actions}

Free actions don't take any time at all, though there may be limits to the number of free actions you can perform in a turn. Free actions rarely incur attacks of opportunity. Some common free actions are described below.

\subsubsection{Drop an Item}

Dropping an item in your space or into an adjacent square is a free action.

\subsubsection{Drop Prone}

Dropping to a prone position in your space is a free action.

\subsubsection{Speak}

In general, speaking is a free action that you can perform even when it isn't your turn. Speaking more than few sentences is generally beyond the limit of a free action.

\subsubsection{Cease Concentration on Spell}

You can stop concentrating on an active spell as a free action.

%Now a swift action
%Cast a Quickened Spell
%You can cast a quickened spell (see the Quicken Spell feat) or any spell whose casting time is designated as a free action as a free action. Only one such spell can be cast in any round, and such spells don't count toward your normal limit of one spell per round. Casting a spell with a casting time of a free action doesn't incur an attack of opportunity.

\section{Miscellaneous Actions}

\subsubsection{Take 5-Foot Step}

You can move 5 feet in any round when you don't perform any other kind of movement. Taking this 5-foot step never provokes an attack of opportunity. You can't take more than one 5-foot step in a round, and you can't take a 5-foot step in the same round when you move any distance. You can take a 5-foot step before, during, or after your other actions in the round.  You can only take a 5-foot-step if your movement isn't hampered by difficult terrain or darkness. Any creature with a speed of 5 feet or less can't take a 5-foot step, since moving even 5 feet requires a move action for such a slow creature. You may not take a 5-foot step using a form of movement for which you do not have a listed speed. 

\subsubsection{Use Feat}

Certain feats let you take special actions in combat. Other feats do not require actions themselves, but they give you a bonus when attempting something you can already do. Some feats are not meant to be used within the framework of combat. The individual feat descriptions tell you what you need to know about them.

\subsubsection{Use Skill}

Most skill uses are standard actions, but some might be move actions, full-round actions, free actions, or something else entirely.
The individual skill descriptions tell you what sorts of actions are required to perform skills.

\section{Injury and Death}

Your hit points measure how hard you are to kill. No matter how many hit points you lose, your character isn't hindered in any way until your hit points drop to 0 or lower.

\section{Loss of Hitpoints}

The most common way that your character gets hurt is to take lethal damage and lose hit points

\vspace*{10pt}

\ability{What Hit Points Represent:}Hit points mean two things in the game world: the ability to take physical punishment and keep going, and the ability to turn a serious blow into a less serious one.

\ability{Effects of Hit Point Damage:}Damage doesn't slow you down until your current hit points reach 0 or lower. At 0 hit points, you're disabled.
	\listtwo
		\item At from �1 to �9 hit points, you're dying.
		\item At �10 or lower, you're dead.
	\end{list}
	
\ability{Massive Damage:}If you ever sustain a single attack deals 50 points of damage or more and it doesn't kill you outright, you must make a DC 15 Fortitude save. If this saving throw fails, you die regardless of your current hit points. If you take 50 points of damage or more from multiple attacks, no one of which dealt 50 or more points of damage itself, the massive damage rule does not apply.

\section{Disabled (0 Hit Points)}

When your current hit points drop to exactly 0, you're disabled.

You can only take a single move or standard action each turn (but not both, nor can you take full-round actions). You can take move actions without further injuring yourself, but if you perform any standard action (or any other strenuous action) you take 1 point of damage after the completing the act. Unless your activity increased your hit points, you are now at �1 hit points, and you're dying.

Healing that raises your hit points above 0 makes you fully functional again, just as if you'd never been reduced to 0 or fewer hit points.

You can also become disabled when recovering from dying. In this case, it's a step toward recovery, and you can have fewer than 0 hit points (see Stable Characters and Recovery, below).

\section{Dying (\textendash1 to \textendash9 Hit Points)}

\listtwo
	\item When your character's current hit points drop to between �1 and �9 inclusive, he's dying.
	\item A dying character immediately falls unconscious and can take no actions.
	\item A dying character loses 1 hit point every round. This continues until the character dies or becomes stable (see below).
\end{list}

\section{Dead (\textendash10 Hit Points or Lower)}

When your character's current hit points drop to �10 or lower, or if he takes massive damage (see above), he's dead. A character can also die from taking ability damage or suffering an ability drain that reduces his Constitution to 0.

\section{Stable Characters and Recovery}

On the next turn after a character is reduced to between \textendash1 and \textendash9 hit points and on all subsequent turns, roll d\% to see whether the dying character becomes stable. He has a 10\% chance of becoming stable. If he doesn't, he loses 1 hit point. (A character who's unconscious or dying can't use any special action that changes the initiative count on which his action occurs.) If the character's hit points drop to �10 or lower, he's dead.

You can keep a dying character from losing any more hit points and make him stable with a DC 15 Heal check.
If any sort of healing cures the dying character of even 1 point of damage, he stops losing hit points and becomes stable.
Healing that raises the dying character's hit points to 0 makes him conscious and disabled. Healing that raises his hit points to 1 or more makes him fully functional again, just as if he'd never been reduced to 0 or lower. A spellcaster retains the spellcasting capability she had before dropping below 0 hit points.

A stable character who has been tended by a healer or who has been magically healed eventually regains consciousness and recovers hit points naturally. If the character has no one to tend him, however, his life is still in danger, and he may yet slip away.
Recovering with Help: One hour after a tended, dying character becomes stable, roll d\%. He has a 10\% chance of becoming conscious, at which point he is disabled (as if he had 0 hit points). If he remains unconscious, he has the same chance to revive and become disabled every hour. Even if unconscious, he recovers hit points naturally. He is back to normal when his hit points rise to 1 or higher.
Recovering without Help: A severely wounded character left alone usually dies. He has a small chance, however, of recovering on his own. 
A character who becomes stable on his own (by making the 10\% roll while dying) and who has no one to tend to him still loses hit points, just at a slower rate. He has a 10\% chance each hour of becoming conscious. Each time he misses his hourly roll to become conscious, he loses 1 hit point. He also does not recover hit points through natural healing.

Even once he becomes conscious and is disabled, an unaided character still does not recover hit points naturally. Instead, each day he has a 10\% chance to start recovering hit points naturally (starting with that day); otherwise, he loses 1 hit point.
Once an unaided character starts recovering hit points naturally, he is no longer in danger of naturally losing hit points (even if his current hit point total is negative).

\section{Healing}

After taking damage, you can recover hit points through natural healing or through magical healing. In any case, you can't regain hit points past your full normal hit point total.

\vspace*{10pt}

\ability{Natural Healing:}With a full night's rest (8 hours of sleep or more), you recover 1 hit point per character level. Any significant interruption during your rest prevents you from healing that night. If you undergo complete bed rest for an entire day and night, you recover twice your character level in hit points. 

\ability{Magical Healing:}Various abilities and spells can restore hit points.

\ability{Healing Limits:}You can never recover more hit points than you lost. Magical healing won't raise your current hit points higher than your full normal hit point total.

\ability{Healing Ability Damage:}Ability damage is temporary, just as hit point damage is. Ability damage returns at the rate of 1 point per night of rest (8 hours) for each affected ability score. Complete bed rest restores 2 points per day (24 hours) for each affected ability score.

\section{Temporary Hit Points}

Certain effects give a character temporary hit points. When a character gains temporary hit points, note his current hit point total. When the temporary hit points go away the character's hit points drop to his current hit point total. If the character's hit points are below his current hit point total at that time, all the temporary hit points have already been lost and the character's hit point total does not drop further.

When temporary hit points are lost, they cannot be restored as real hit points can be, even by magic.

\vspace*{10pt}

\ability{Increases in Constitution Score and Current Hit Points:}An increase in a character's Constitution score, even a temporary one, can give her more hit points (an effective hit point increase), but these are not temporary hit points. They can be restored and they are not lost first as temporary hit points are.

\section{Nonlethal Damage}

\ability{Dealing Nonlethal Damage:}Certain attacks deal nonlethal damage. Other effects, such as heat or being exhausted, also deal nonlethal damage. When you take nonlethal damage, keep a running total of how much you've accumulated. Do not deduct the nonlethal damage number from your current hit points. It is not �real� damage. Instead, when your nonlethal damage equals your current hit points, you're staggered, and when it exceeds your current hit points, you fall unconscious. It doesn't matter whether the nonlethal damage equals or exceeds your current hit points because the nonlethal damage has gone up or because your current hit points have gone down.

\ability{Nonlethal Damage with a Weapon that Deals Lethal Damage:}You can use a melee weapon that deals lethal damage to deal nonlethal damage instead, but you take a �4 penalty on your attack roll.

\ability{Lethal Damage with a Weapon that Deals Nonlethal Damage:}You can use a weapon that deals nonlethal damage, including an unarmed strike, to deal lethal damage instead, but you take a �4 penalty on your attack roll.

\ability{Staggered and Unconscious:}When your nonlethal damage equals your current hit points, you're staggered. You can only take a standard action or a move action in each round. You cease being staggered when your current hit points once again exceed your nonlethal damage. When your nonlethal damage exceeds your current hit points, you fall unconscious. While unconscious, you are helpless.
Spellcasters who fall unconscious retain any spellcasting ability they had before going unconscious.

\ability{Healing Nonlethal Damage:}You heal nonlethal damage at the rate of 1 hit point per hour per character level. When a spell or a magical power cures hit point damage, it also removes an equal amount of nonlethal damage.


\section{Movement, Position, and Distance}

Miniatures are on the 30mm scale - a miniature figure of a six\textendash foot tall human is approximately 30mm tall. A square on the battle grid is 1 inch across, representing a 5\textendash foot by 5\textendash foot area.

\subsection{Tactical Movement}

\subsubsection{How Far Can Your Character Move?}

%This is not the original table from the SRD, but instead a better one that encompases a wider range of speeds.
\begin{wraptable}{o}{2in}
\caption{Reduced Speed}
\rowcolors{1}{colorone}{colortwo}
\begin{tabular}[h]{cc}
\rowcolor{headercolor} Base Speed & Reduced Speed \\
20 ft. & 15 ft. \\
30 ft. & 20 ft. \\
40 ft. & 30 ft. \\
50 ft. & 35 ft. \\
60 ft. & 40 ft. \\
70 ft. & 50 ft. \\
80 ft. & 55 ft. \\
90 ft. & 60 ft. \\
100 ft. & 70 ft. \\
\end{tabular}
\end{wraptable}

Your speed is determined by your race and your armor (see Table: Tactical Speed). Your speed while unarmored is your base land speed.

\vspace*{10pt}

\ability{Encumbrance:}A character encumbered by carrying a large amount of gear, treasure, or fallen comrades may move slower than normal.  

\ability{Hampered Movement:}Difficult terrain, obstacles, or poor visibility can hamper movement.

\ability{Movement in Combat:}Generally, you can move your speed in a round and still do something (take a move action and a standard action).
If you do nothing but move (that is, if you use both of your actions in a round to move your speed), you can move double your speed.
If you spend the entire round running, you can move quadruple your speed. If you do something that requires a full round you can only take a 5\textendash foot step.

\ability{Bonuses to Speed:}Some class features or magical items may grant bonuses to a characters speed. Always apply any modifiers to a character's speed before adjusting the character's speed based on armor or encumbrance, and remember that multiple bonuses of the same type to a character's speed don't stack.

\ability{Reduced Speed:}Being encumbered up to a medium or heavy load or wearing medium or heavy armor reduces your speed.  The table below gives the reduced speeds for different base land speeds.


\subsubsection{Measuring Distance}

\ability{Diagonals:}When measuring distance, the first diagonal counts as 1 square, the second counts as 2 squares, the third counts as 1, the fourth as 2, and so on. You can't move diagonally past a corner (even by taking a 5\textendash foot step). You can move diagonally past a creature, even an opponent. You can also move diagonally past other impassable obstacles, such as pits.

\ability{Closest Creature:}When it's important to determine the closest square or creature to a location, if two squares or creatures are equally close, randomly determine which one counts as closest by rolling a die.

\subsubsection{Moving through a Square}

\ability{Friend:}You can move through a square occupied by a friendly character, unless you are charging. When you move through a square occupied by a friendly character, that character doesn't provide you with cover.

\ability{Opponent:}You can't move through a square occupied by an opponent, unless the opponent is helpless. You can move through a square occupied by a helpless opponent without penalty. (Some creatures, particularly very large ones, may present an obstacle even when helpless. In such cases, each square you move through counts as 2 squares.)

\ability{Ending Your Movement:}You can't end your movement in the same square as another creature unless it is helpless.

\ability{Overrun:}During your movement you can attempt to move through a square occupied by an opponent.

\ability{Tumbling:}A trained character can attempt to tumble through a square occupied by an opponent (see the Tumble skill).

\ability{Very Small Creature:}A Fine, Diminutive, or Tiny creature can move into or through an occupied square. The creature provokes attacks of opportunity when doing so.
Square Occupied by Creature Three Sizes Larger or Smaller: Any creature can move through a square occupied by a creature three size categories larger than it is. A big creature can move through a square occupied by a creature three size categories smaller than it is.

\ability{Designated Exceptions:}Some creatures break the above rules. A creature that completely fills the squares it occupies cannot be moved past, even with the Tumble skill or similar special abilities.

\subsubsection{Terrain and Obstacles}

\ability{Difficult Terrain:}Difficult terrain hampers movement. Each square of difficult terrain counts as 2 squares of movement. (Each diagonal move into a difficult terrain square counts as 3 squares.) You can't run or charge across difficult terrain. If you occupy squares with different kinds of terrain, you can move only as fast as the most difficult terrain you occupy will allow. Flying and incorporeal creatures are not hampered by difficult terrain.

\ability{Obstacles:}Like difficult terrain, obstacles can hamper movement. If an obstacle hampers movement but doesn't completely block it each obstructed square or obstacle between squares counts as 2 squares of movement. You must pay this cost to cross the barrier, in addition to the cost to move into the square on the other side. If you don't have sufficient movement to cross the barrier and move into the square on the other side, you can't cross the barrier. Some obstacles may also require a skill check to cross. On the other hand, some obstacles block movement entirely. A character can't move through a blocking obstacle. Flying and incorporeal creatures can avoid most obstacles

\ability{Squeezing:}In some cases, you may have to squeeze into or through an area that isn't as wide as the space you take up. You can squeeze through or into a space that is at least half as wide as your normal space. Each move into or through a narrow space counts as if it were 2 squares, and while squeezed in a narrow space you take a \textendash4 penalty on attack rolls and a \textendash4 penalty to AC. When a Large creature (which normally takes up four squares) squeezes into a space that's one square wide, the creature's miniature figure occupies two squares, centered on the line between the two squares. For a bigger creature, center the creature likewise in the area it squeezes into. A creature can squeeze past an opponent while moving but it can't end its movement in an occupied square. To squeeze through or into a space less than half your space's width, you must use the Escape Artist skill. You can't attack while using Escape Artist to squeeze through or into a narrow space, you take a \textendash4 penalty to AC, and you lose any Dexterity bonus to AC.

\subsubsection{Special Movement Rules}

These rules cover special movement situations.

\vspace*{10pt}

\ability{Accidentally Ending Movement in an Illegal Space:}Sometimes a character ends its movement while moving through a space where it's not allowed to stop. When that happens, put your miniature in the last legal position you occupied, or the closest legal position, if there's a legal position that's closer.

\ability{Double Movement Cost:}When your movement is hampered in some way, your movement usually costs double. For example, each square of movement through difficult terrain counts as 2 squares, and each diagonal move through such terrain counts as 3 squares (just as two diagonal moves normally do). If movement cost is doubled twice, then each square counts as 4 squares (or as 6 squares if moving diagonally). If movement cost is doubled three times, then each square counts as 8 squares (12 if diagonal) and so on. This is an exception to the general rule that two doublings are equivalent to a tripling.

\ability{Minimum Movement:}Despite penalties to movement, you can take a full\textendash round action to move 5 feet (1 square) in any direction, even diagonally. (This rule doesn't allow you to move through impassable terrain or to move when all movement is prohibited.) Such movement provokes attacks of opportunity as normal (despite the distance covered, this move isn't a 5\textendash foot step).

\subsection{Big and Little Creatures in Combat}

Creatures smaller than Small or larger than Medium have special rules relating to position. 

\vspace*{10pt}

\begin{wraptable}{O}{3in}
\caption{Creature Size and Scale}
\rowcolors{1}{colorone}{colortwo}
\begin{tabular}[h]{ccc}
\rowcolor{headercolor} Creature Size & Space\textsuperscript{1} &	Natural Reach\textsuperscript{1} \\
Fine & 1/2 ft. & 0 \\
Diminutive & 1 ft. & 0 \\
Tiny & 2\textendash1/2 ft. & 0 \\
Small & 5 ft. & 5 ft.	\\
Medium & 5 ft. & 5 ft. \\
Large (tall) & 10 ft. & 10 ft. \\
Large (long) & 10 ft. & 5 ft. \\
Huge (tall) & 15 ft. & 15 ft. \\
Huge (long) & 15 ft. & 10 ft. \\
Gargantuan (tall) & 20 ft. & 20 ft. \\
Gargantuan (long) & 20 ft. & 15 ft.	\\
Colossal (tall) & 30 ft. & 30 ft. \\
Colossal (long) & 30 ft. & 20 ft. \\ 
\multicolumn{3}{p{3in}}{\textsuperscript{1} These values are typical for creatures of the indicated size.}
\end{tabular}
\end{wraptable}

\ability{Tiny, Diminutive, and Fine Creatures:} Very small creatures take up less than 1 square of space. This means that more than one such creature can fit into a single square. A Tiny creature typically occupies a space only 2 \half feet across, so four can fit into a single square. Twenty\textendash five Diminutive creatures or 100 Fine creatures can fit into a single square. Creatures that take up less than 1 square of space typically have a natural reach of 0 feet, meaning they can't reach into adjacent squares. They must enter an opponent's square to attack in melee. This provokes an attack of opportunity from the opponent. You can attack into your own square if you need to, so you can attack such creatures normally. Since they have no natural reach, they do not threaten the squares around them. You can move past them without provoking attacks of opportunity. They also can't flank an enemy.

\ability{Large, Huge, Gargantuan, and Colossal Creatures:} Very large creatures take up more than 1 square. Creatures that take up more than 1 square typically have a natural reach of 10 feet or more, meaning that they can reach targets even if they aren't in adjacent squares. Unlike when someone uses a reach weapon, a creature with greater than normal natural reach (more than 5 feet) still threatens squares adjacent to it. A creature with greater than normal natural reach usually gets an attack of opportunity against you if you approach it, because you must enter and move within the range of its reach before you can attack it. (This attack of opportunity is not provoked if you take a 5\textendash foot step.) Large or larger creatures using reach weapons can strike up to double their natural reach but can't strike at their natural reach or less.

\pagebreak

\section{Combat Modifiers}

\subsection{Favorable and Unfavorable Conditions}

\begin{minipage}[t]{0.5\textwidth}
\noindent\captionof{table}{Armor Class Modifiers}
\rowcolors{1}{colorone}{colortwo}
\noindent\begin{tabular}[h]{lcc}
\rowcolor{headercolor} Defender is \ldots & Melee & Ranged	\\
Behind cover & +4 & +4 \\	   
Blinded & \textendash2\textsuperscript{1} & \textendash2\textsuperscript{1} \\
Concealed or invisible & \multicolumn{2}{c}{See Concealment} \\
Cowering & \textendash2\textsuperscript{1} & \textendash2\textsuperscript{1} \\
Entangled & +0\textsuperscript{2} & +0\textsuperscript{2} \\
Flat\textendash footed & +0\textsuperscript{1} & +0\textsuperscript{1} \\
Grappling (attacker is not) & +0\textsuperscript{1} & +0\textsuperscript{1, 3} \\
Helpless & \textendash4\textsuperscript{4} & +0\textsuperscript{4} \\
Kneeling or sitting & \textendash2 & +2 \\
Pinned & \textendash4\textsuperscript{4} & +0\textsuperscript{4} \\
Prone  & \textendash4 & +4 \\
Squeezing through a space & \textendash4 & \textendash4 \\
Stunned & \textendash2\textsuperscript{1} & \textendash2\textsuperscript{1} \\
\multicolumn{3}{p{3in}}{\textsuperscript{1} The defender loses any Dexterity bonus to AC.} \\
\multicolumn{3}{p{3in}}{\textsuperscript{2} An entangled character takes a \textendash4 penalty to Dexterity.} \\
\multicolumn{3}{p{3in}}{\textsuperscript{3} Roll randomly to see which grappling combatant you strike. That defender loses any Dexterity bonus to AC.} \\
\multicolumn{3}{p{3in}}{\textsuperscript{4} Treat the defender's Dexterity as 0 (\textendash5 modifier). Rogues can sneak attack helpless or pinned defenders.} \\
\end{tabular}
\end{minipage}
\begin{minipage}[t]{0.5\textwidth}
\noindent\captionof{table}{Attack Roll Modifiers}
\rowcolors{1}{colorone}{colortwo}
\begin{tabular}[h]{lcc} 
\rowcolor{headercolor} Attacker is \ldots & Melee & Ranged \\
Dazzled & \textendash1 & \textendash1 \\	   
Entangled & \textendash2\textsuperscript{1} &	\textendash2\textsuperscript{1} \\	   
Flanking defender & +2 & \textendash \\	   
Invisible & +2\textsuperscript{2} & +2\textsuperscript{2} \\  
On higher ground & +1 & +0 \\	   
Prone & \textendash4 & \textendash3 \\   
Shaken or frightened & \textendash2 & \textendash2 \\	   
Squeezing through a space & \textendash4 & \textendash4 \\
\multicolumn{3}{p{3in}}{\textsuperscript{1} An entangled character also takes a \textendash4 penalty to Dexterity, which may affect his attack roll.} \\
\multicolumn{3}{p{3in}}{\textsuperscript{2} The defender loses any Dexterity bonus to AC. This bonus doesn't apply if the target is blinded.} \\
\multicolumn{3}{p{3in}}{\textsuperscript{3} Most ranged weapons can't be used while the attacker is prone, but you can use a crossbow or shuriken while prone at no penalty.} \\
\end{tabular}
\end{minipage}

\subsection{Cover}

To determine whether your target has cover from your ranged attack, choose a corner of your square. If any line from this corner to any corner of the target's square passes through a square or border that blocks line of effect or provides cover, or through a square occupied by a creature, the target has cover (+4 to AC).
When making a melee attack against an adjacent target, your target has cover if any line from your square to the target's square goes through a wall (including a low wall). When making a melee attack against a target that isn't adjacent to you (such as with a reach weapon), use the rules for determining cover from ranged attacks.
Low Obstacles and Cover: A low obstacle (such as a wall no higher than half your height) provides cover, but only to creatures within 30 feet (6 squares) of it. The attacker can ignore the cover if he's closer to the obstacle than his target.

\vspace*{10pt}

\ability{Cover and Attacks of Opportunity:}You can't execute an attack of opportunity against an opponent with cover relative to you.

\ability{Cover and Reflex Saves:}Cover grants you a +2 bonus on Reflex saves against attacks that originate or burst out from a point on the other side of the cover from you. Note that spread effects can extend around corners and thus negate this cover bonus.

\ability{Cover and Hide Checks:}You can use cover to make a Hide check. Without cover, you usually need concealment (see below) to make a Hide check.

\ability{Soft Cover:}Creatures, even your enemies, can provide you with cover against ranged attacks, giving you a +4 bonus to AC. However, such soft cover provides no bonus on Reflex saves, nor does soft cover allow you to make a Hide check.

\ability{Big Creatures and Cover:}Any creature with a space larger than 5 feet (1 square) determines cover against melee attacks slightly differently than smaller creatures do. Such a creature can choose any square that it occupies to determine if an opponent has cover against its melee attacks. Similarly, when making a melee attack against such a creature, you can pick any of the squares it occupies to determine if it has cover against you.

\ability{Total Cover:}If you don't have line of effect to your target he is considered to have total cover from you. You can't make an attack against a target that has total cover.

\ability{Varying Degrees of Cover:}In some cases, cover may provide a greater bonus to AC and Reflex saves. In such situations the normal cover bonuses to AC and Reflex saves can be doubled (to +8 and +4, respectively). A creature with this improved cover effectively gains improved evasion against any attack to which the Reflex save bonus applies. Furthermore, improved cover provides a +10 bonus on Hide checks.

\subsection{Concealment}

To determine whether your target has concealment from your ranged attack, choose a corner of your square. If any line from this corner to any corner of the target's square passes through a square or border that provides concealment, the target has concealment. When making a melee attack against an adjacent target, your target has concealment if his space is entirely within an effect that grants concealment. When making a melee attack against a target that isn't adjacent to you use the rules for determining concealment from ranged attacks. In addition, some magical effects provide concealment against all attacks, regardless of whether any intervening concealment exists.

\vspace*{10pt}

\ability{Concealment Miss Chance:}Concealment gives the subject of a successful attack a 20\% chance that the attacker missed because of the concealment. If the attacker hits, the defender must make a miss chance percentile roll to avoid being struck. Multiple concealment conditions do not stack.

\ability{Concealment and Hide Checks:}You can use concealment to make a Hide check. Without concealment, you usually need cover to make a Hide check.

\ability{Total Concealment:}If you have line of effect to a target but not line of sight he is considered to have total concealment from you. You can�t attack an opponent that has total concealment, though you can attack into a square that you think he occupies. A successful attack into a square occupied by an enemy with total concealment has a 50\% miss chance (instead of the normal 20\% miss chance for an opponent with concealment). You can�t execute an attack of opportunity against an opponent with total concealment, even if you know what square or squares the opponent occupies.

\ability{Ignoring Concealment:}Concealment isn't always effective. A shadowy area or darkness doesn't provide any concealment against an opponent with darkvision. Characters with low\textendash light vision can see clearly for a greater distance with the same light source than other characters. Although invisibility provides total concealment, sighted opponents may still make Spot checks to notice the location of an invisible character. An invisible character gains a +20 bonus on Hide checks if moving, or a +40 bonus on Hide checks when not moving (even though opponents can't see you, they might be able to figure out where you are from other visual clues).

\ability{Varying Degrees of Concealment:}Certain situations may provide more or less than typical concealment, and modify the miss chance accordingly.

\subsection{Flanking}

When making a melee attack, you get a +2 flanking bonus if your opponent is threatened by a character or creature friendly to you on the opponent's opposite border or opposite corner. When in doubt about whether two friendly characters flank an opponent in the middle, trace an imaginary line between the two friendly characters' centers. If the line passes through opposite borders of the opponent's space (including corners of those borders), then the opponent is flanked. 
\textbf{Exception:} If a flanker takes up more than 1 square, it gets the flanking bonus if any square it occupies counts for flanking. Only a creature or character that threatens the defender can help an attacker get a flanking bonus. Creatures with a reach of 0 feet can't flank an opponent.

\subsection{Helpless Defenders}

A helpless opponent is someone who is bound, sleeping, paralyzed, unconscious, or otherwise at your mercy.

\vspace*{10pt}

\ability{Regular Attack:}A helpless character takes a \textendash4 penalty to AC against melee attacks, but no penalty to AC against ranged attacks.
A helpless defender can't use any Dexterity bonus to AC. In fact, his Dexterity score is treated as if it were 0 and his Dexterity modifier to AC as if it were \textendash5 (and a rogue can sneak attack him).

\ability{Coup de Grace:}As a full\textendash round action, you can use a melee weapon to deliver a coup de grace to a helpless opponent. You can also use a bow or crossbow, provided you are adjacent to the target. You automatically hit and score a critical hit. If the defender survives the damage, he must make a Fortitude save (DC 10 + damage dealt) or die. A rogue also gets her extra sneak attack damage against a helpless opponent when delivering a coup de grace. Delivering a coup de grace may provoke attacks of opportunity from threatening opponents. You can't deliver a coup de grace against a creature that is immune to critical hits. You can deliver a coup de grace against a creature with total concealment, but doing this requires two consecutive full\textendash round actions (one to ``find'' the creature once you've determined what square it's in, and one to deliver the coup de grace).

\subsection{Attack Options}

Characters have a number of options when they attack their opponents. Expertise and Power Attack can be used on any attacks.

\listone\hypertarget{combat:expertise}{}
\item \textbf{{Expertise}}\\
\emph{You leverage your combat skill into defense rather than offense.}\\
\shortability{Requirement:}{You must make an attack action and have a BAB of at least +1. You need not specifically attack an enemy.}
\shortability{Effect:}{Before making an attack roll, you may take an attack penalty of up to your BAB on this attack and all further attacks until your next turn, and gain an equal Dodge Bonus to AC. You may only use this option once per turn.}\\

\hypertarget{combat:powerattack}{}
\item\textbf{{Power Attack}}\\
\emph{You leverage your combat skill into devastating attacks at the expense of accuracy.}\\
\shortability{Requirement:}{You must make an attack action and have a BAB of at least +1.}
\shortability{Effect:}{Before making an attack roll, you may voluntarily take an attack penalty of up to your BAB, and inflict two times that amount in extra damage with that attack. You may take this option on any or all of your attacks if you wish.}
\end{list}

\section{Special Combat Actions}

The following are special actions that can be performed in combat.\\

\subsection{Combat Maneuvers}
 
%Table: Special Attacks	   
%Special Attack 	Brief Description	   
%Aid another 	Grant an ally a +2 bonus on attacks or AC	   
%Bull rush 	Push an opponent back 5 feet or more	   
%Charge 	Move up to twice your speed and attack with +2 bonus	   
%Disarm 	Knock a weapon from your opponent's hands	   
%Feint 	Negate your opponent's Dex bonus to AC	   
%Grapple 	Wrestle with an opponent	   
%Overrun 	Plow past or over an opponent as you move	   
%Sunder 	Strike an opponent's weapon or shield	   
%Throw splash weapon 	Throw container of dangerous liquid at target	   
%Trip 	Trip an opponent	   
%Turn (rebuke) undead 	Channel positive (or negative) energy to turn away (or awe) undead	   
%Two\textendash weapon fighting 	Fight with a weapon in each hand	 

\ability{Having the Edge:}If you have more BAB than the target of your attacks, you are considered to ``Have the Edge'' on that attack. Some combat maneuvers will perform better when used by someone with the Edge.

\listone

\hypertarget{combat:aidanother}{}
\normalsize\item\textbf{{Aid Another}}\\ \small
In melee combat, you can help a friend attack or defend by distracting or interfering with an opponent. If you're in position to make a melee attack on an opponent that is engaging a friend in melee combat, you can attempt to aid your friend as a standard action. You make an attack roll against AC 10. If you succeed, your friend gains either a +2 bonus on his next attack roll against that opponent or a +2 bonus to AC against that opponent's next attack (your choice), as long as that attack comes before the beginning of your next turn. Multiple characters can aid the same friend, and similar bonuses stack. You can also use this standard action to help a friend in other ways, such as when he is affected by a spell, or to assist another character's skill check.\\

\hypertarget{combat:bullrush}{}
\normalsize\item\textbf{{Bullrush}}\\ \small
If you have not moved your entire allotted distance this turn, you may attempt to push your opponent back as a melee attack. First, you move into your opponent's square (which probably provokes an attack of opportunity, see movement). Then you make an opposed size-modified strength check against a DC of 10 + the target's Strength modifier + the target's size modifier (you do not have to roll to hit). If you succeed, you push your opponent back 5 feet. If you succeed by more than 1, you may move your opponent back a single 5' square for every 2 points your check exceeds the DC.

\ability{Modifiers:} The Size Modifier to both the Bullrush check and the DC is +4 for every size larger than medium and -4 for every size smaller than medium.

\ability{Special:} The movement used during a Bullrush counts against your movement this turn. If you do not take a move or charge action this turn, you will normally be limited to five feet of movement. This movement does not provoke an attack of opportunity from you or the target, but is quite likely to provoke an attack of opportunity from any other creature standing nearby. During a bullrush, both characters provide cover for each other.

\smallskip\emph{\underline{Edge Option:} If you have the edge on your target, you do not provide cover for your opponent even if they are the same size as you. Further, you may move your opponent in a direction up to 45 degrees off from your initial approach, altering your own course to push them more than 5 feet if necessary. If you fail the initial strength check, you may choose which adjacent square you are pushed into.}\\

\hypertarget{combat:charge}{}
\normalsize\item\textbf{{Charge}}\\ \small
Charging is a special full\textendash round action that allows you to move up to twice your speed and attack during the action. However, it carries tight restrictions on how you can move.

\ability{Movement During a Charge}{You must move before your attack, not after. You must move at least 10 feet (2 squares) and may move up to double your speed directly toward the designated opponent. You must have a clear path toward the opponent, and nothing can hinder your movement (such as difficult terrain or obstacles). Here's what it means to have a clear path. First, you must move to the closest space from which you can attack the opponent. (If this space is occupied or otherwise blocked, you can't charge.) Second, if any line from your starting space to the ending space passes through a square that blocks movement, slows movement, or contains a creature (even an ally), you can't charge. (Helpless creatures don't stop a charge.) If you don't have line of sight to the opponent at the start of your turn, you can't charge that opponent. You can't take a 5\textendash foot step in the same round as a charge.

If you are able to take only a standard action or a move action on your turn, you can still charge, but you are only allowed to move up to your speed (instead of up to double your speed). You can't use this option unless you are restricted to taking only a standard action or move action on your turn.}

\ability{Attacking on a Charge}{After moving, you may make a single melee attack. You get a +2 bonus on the attack roll. and take a \textendash2 penalty to your AC until the start of your next turn.A charging character gets a +2 bonus on the Strength check made to bull rush an opponent (see Bull Rush, above). Even if you have extra attacks, such as from having a high enough base attack bonus or from using multiple weapons, you only get to make one attack during a charge.}

\ability{Lances and Charge Attacks:}{	A lance deals double damage if employed by a mounted character in a charge.}
	
\ability{Weapons Readied against a Charge :}{Spears, tridents, and certain other piercing weapons deal double damage when readied (set) and used against a charging character.}\\

\hypertarget{combat:coupdegrace}{}
\normalsize\item\textbf{{Coup de Grace}}\\\small
You may attempt to slay an opponent outright if they are helpless. As a full-round action, you may automatically hit a helpless opponent in melee range. This attack is automatically a critical hit. This action provokes an attack of opportunity.

\ability{Interrupting a Coup de Grace:}{A character who suffers damage during the Coup de Grace must make a Concentration Check (DC 10 + Damage Inflicted) or the action is resolved as a normal attack.}

\smallskip\emph{\underline{Edge Option:} If you have the Edge on an opponent who threatens you during a Coup de Grace, you do not provoke an attack of opportunity from them.}\\

\hypertarget{combat:coveringfire}{}
\normalsize\item\textbf{{Covering Fire}}\\\small
You may use your ranged attacks to provide cover for your allies. Take an attack with your ranged weapon and roll a normal attack roll. Until the beginning of your next turn one of your allies may use the result of your attack roll as their Armor Class against one attack of opportunity.

\smallskip\emph{\underline{Edge Option:} If you have The Edge against an opponent whose attack of opportunity was negated by Covering Fire, your ranged weapon may hit them. Simply compare the attack roll to their armor class as if it was also a normal attack.}\\

\hypertarget{combat:disarm}{}
\normalsize\item\textbf{{Disarm}}\\\small
You may attempt to disarm your opponent with a melee attack. Disarm is a special attack action. Make an attack roll against an ``armor class" of 10 + the target's melee attack bonuses with the item in question. If you succeed, one weapon or held item is snatched out of your opponent's grasp. Failing a Disarm attempt provokes an attack of opportunity from the target. A disarmed item lands in a randomly determined square adjacent to the target.

\ability{Grabbing Items:}{You can use a disarm action to snatch an item worn by the target. If you want to have the item in your hand, the disarm must be made as an unarmed attack. If the item is poorly secured or otherwise easy to snatch or cut away the attacker gets a +4 bonus. Unlike on a normal disarm attempt, failing the attempt doesn't allow the defender to attempt to disarm you. This otherwise functions identically to a disarm attempt, as noted above. You can't snatch an item that is well secured unless you have pinned the wearer (see Grapple). Even then, the defender gains a +4 bonus on his roll to resist the attempt.}

\ability{Defending against a Disarm:}{An item held in two hands is harder to disarm, increasing the DC by +4. An item tied to one's body with a sword-wrap or locked gauntlet is much harder to disarm, increasing the DC by +8.}

\ability{Special:}{A Disarm may be used to attempt to remove a weapon that is presently being used in an attack against the disarmer even if the creature using the weapon is out of range or otherwise not threatened by the character. A Disarm (or any attack) is normally only usable during an attack against such creatures as an Attack of Opportunity or a Readied Action.}

\smallskip\emph{\underline{Edge Option:} If you have the Edge on your target, your Disarm attempt does not provoke an attack of opportunity, and you may choose which adjacent square your opponent's weapon or held item lands in. If you have a free hand, the item may end up in your possession instead.}\\


\hypertarget{combat:feint}{}
\normalsize\item\textbf{{Feint}}\\\small
By performing a distracting maneuver or fencing your opponent into a poor position, you may make an attack against them at their worst. You take an attack action to make a Bluff check with a DC of 10 + your opponent's Wisdom modifier + the higher of your opponent's BAB or ranks in Sense Motive. If you succeed, your opponent does not get their Dexterity Bonus to AC against the next attack you make against them (if it is within the next round).

\smallskip\emph{\underline{Edge Option:} If you have the Edge on your target and you successfully Feint, you may make an attack against that opponent this round as a Swift action.}\\

\hypertarget{combat:grapple}{}
\normalsize\item\textbf{{Grapple}}\\\small
Grapple is collectively 3 separate maneuvers that all fall under the super-heading of ``grappling". Any grapple attempt provokes an attack of opportunity unless your attack has the edge.

\listtwo\hypertarget{combat:grabon}{}
      \normalsize\item\textbf{{Grab On}}\\\small
      Sometimes, you want to attach yourself to a larger creature, getting inside their reach and then repeatedly stabbing them or simply weighing them down.  As an attack action you may attempt to grab on to an opponent.

      Grabbing on to an opponent provokes an attack of opportunity and requires a check with the same bonuses as a melee attack. The DC to grab on to an opponent is their Touch AC plus their BAB. If you have 5 ranks of Climb or Ride, you get a +2 synergy bonus on this maneuver for each skill.

      \ability{Holding on:}{Once you've attached yourself to your opponent, you go wherever they go. Move in to their space, and move where they do automatically (this movement does not provoke attacks of opportunity or count against your movement in any way). You may attack with any light or one handed weapon, and your opponent is denied his Dexterity bonus against you.}

      \ability{Being Held on to:}{If another creature has grabbed on to your character, their weight counts against your carrying capacity. If you're overloaded, you may be unable to move or even collapse until you shake your opponent off. You can attempt to attack a creature holding on to you, but your strength modifier is halved for such attacks and your attacks are at -4. You may attempt to shake your opponent off as an attack action by making a check with a bonus equal to your melee attack or Escape Artist and a DC of 10 + the greatest of your opponent's BAB, Climb Ranks, or Ride Ranks.}

      \smallskip\emph{\underline{Edge Options:} If you have the edge on an opponent when you grab them, they may not attack you at all once you have grabbed on to them. Further, grabbing on to an opponent does not provoke an attack of opportunity.}\\

     \hypertarget{combat:holddown}{}\hypertarget{combat:pin}{}
      \normalsize\item\textbf{{Hold Down}}\\\small
      Sometimes you want to pin an opponent to the ground. First, make a touch attack. Then, make a Grapple Check (BAB + Strength Modifier + Special Size Modifier) with a DC of 10 + Defender's Grapple Check Modifier. If you succeed, your opponent is pinned for one round. They can't move, and you may put ropes or manacles on them if you wish with an attack action. At the end of any turn you are pinning your opponent, you may inflict unarmed or constriction damage. With subsequent attack actions, you may attack with natural weapons or light weapons with no penalty.

      \ability{Escaping a Pin:}{If you're pinned you can attempt to fight back, but you're prone and you suffer an additional -4 penalty to attack the creature pinning you (generally a -8 total penalty to attack your attacker). You can get out with an attack action by making a Grapple or Escape Artist check with a DC of 10 + your opponent's Grapple Modifier.}

      \smallskip\emph{\underline{Edge Options:} If you're pinning an opponent and your attacks have the edge, your opponent cannot attack you or anyone else until they get free. Furthermore, if anyone else attacks them, they are considered helpless.}\\

     \hypertarget{combat:lift}{}
      \normalsize\item\textbf{{Lift}}\\\small
      Sometimes you want to put an opponent in your mouth or carry away a struggling princess. Make a touch attack and then make a Grapple Check with a DC equal to 10 + your opponent's Grapple modifier. If you succeed, your opponent is hefted into the air. You may move around freely while carrying your opponent (their weight counts against your limits of course). You may perform a coup de grace or swallow whole action on a character you have lifted, but doing so ends the lift whether it succeeds or fails.

      \ability{Escaping a Lift:}{When you've been lifted, you cannot move under your own power, but you can continue to attack. Attacks against the creature which has lifted you are at a -4 penalty. You can also attempt to escape with an attack action by making a Grapple or Escape Artist check with a DC of 10 + your opponent's Grapple Modifier.}

      \smallskip\emph{\underline{Edge Options:} If you have the edge on an opponent you have lifted, they may not attack you or anyone else until they escape.}\\
\end{list}

\hypertarget{combat:mountedcombat}{}
\normalsize\item\textbf{{Mounted Combat}}\\\small
\ability{Horses in Combat:}{Warhorses and warponies can serve readily as combat steeds. Light horses, ponies, and heavy horses, however, are frightened by combat. If you don't dismount, you must make a DC 20 Ride check each round as a move action to control such a horse. If you succeed, you can perform a standard action after the move action. If you fail, the move action becomes a full round action and you can't do anything else until your next turn.

Your mount acts on your initiative count as you direct it. You move at its speed, but the mount uses its action to move.

A horse (not a pony) is a Large creature and thus takes up a space 10 feet (2 squares) across. For simplicity, assume that you share your mount's space during combat.}

\ability{Combat while Mounted:}{With a DC 5 Ride check, you can guide your mount with your knees so as to use both hands to attack or defend yourself. This is a free action.

When you attack a creature smaller than your mount that is on foot, you get the +1 bonus on melee attacks for being on higher ground. If your mount moves more than 5 feet, you can only make a single melee attack. Essentially, you have to wait until the mount gets to your enemy before attacking, so you can't make a full attack. Even at your mount's full speed, you don't take any penalty on melee attacks while mounted.

If your mount charges, you also take the AC penalty associated with a charge. If you make an attack at the end of the charge, you receive the bonus gained from the charge. When charging on horseback, you deal double damage with a lance (see Charge).

You can use ranged weapons while your mount is taking a double move, but at a \textendash4 penalty on the attack roll. You can use ranged weapons while your mount is running (quadruple speed), at a \textendash8 penalty. In either case, you make the attack roll when your mount has completed half its movement. You can make a full attack with a ranged weapon while your mount is moving. Likewise, you can take move actions normally.}

\ability{Casting Spells while Mounted:}{You can cast a spell normally if your mount moves up to a normal move (its speed) either before or after you cast. If you have your mount move both before and after you cast a spell, then you're casting the spell while the mount is moving, and you have to make a Concentration check due to the vigorous motion (DC 10 + spell level) or lose the spell. If the mount is running (quadruple speed), you can cast a spell when your mount has moved up to twice its speed, but your Concentration check is more difficult due to the violent motion (DC 15 + spell level).}

\ability{If Your Mount Falls in Battle:}{If your mount falls, you have to succeed on a DC 15 Ride check to make a soft fall and take no damage. If the check fails, you take 1d6 points of damage.}

\ability{If You Are Dropped:}{If you are knocked unconscious, you have a 50\% chance to stay in the saddle (or 75\% if you're in a military saddle). Otherwise you fall and take 1d6 points of damage.

Without you to guide it, your mount avoids combat.}\\

\hypertarget{combat:overrun}{}
\normalsize\item\textbf{{Overrun}}\\\small
You can attempt an overrun as a standard action taken during your move. (In general, you cannot take a standard action during a move; this is an exception.) With an overrun, you attempt to plow past or over your opponent (and move through his square) as you move. You can only overrun an opponent who is one size category larger than you, the same size, or smaller. You can make only one overrun attempt per round.

If you're attempting to overrun an opponent, follow these steps.
\ability{Step 1:}{Attack of Opportunity. Since you begin the overrun by moving into the defender's space, you provoke an attack of opportunity from the defender.}

\ability{Step 2:}{Opponent Avoids? The defender has the option to simply avoid you. If he avoids you, he doesn't suffer any ill effect and you may keep moving (You can always move through a square occupied by someone who lets you by.) The overrun attempt doesn't count against your actions this round (except for any movement required to enter the opponent's square). If your opponent doesn't avoid you, move to Step 3.}

\ability{Step 3:}{ Opponent Blocks? If your opponent blocks you, make a Strength check opposed by the defender's Dexterity or Strength check (whichever ability score has the higher modifier). A combatant gets a +4 bonus on the check for every size category he is larger than Medium or a \textendash4 penalty for every size category he is smaller than Medium. The defender gets a +4 bonus on his check if he has more than two legs or is otherwise more stable than a normal humanoid. If you win, you knock the defender prone. If you lose, the defender may immediately react and make a Strength check opposed by your Dexterity or Strength check (including the size modifiers noted above, but no other modifiers) to try to knock you prone.}

\ability{Step 4:}Consequences. If you succeed in knocking your opponent prone, you can continue your movement as normal. If you fail and are knocked prone in turn, you have to move 5 feet back the way you came and fall prone, ending your movement there. If you fail but are not knocked prone, you have to move 5 feet back the way you came, ending your movement there. If that square is occupied, you fall prone in that square.

%\ability{Improved Overrun:}{If you have the Improved Overrun feat, your target may not choose to avoid you.}

\ability{Mounted Overrun (Trample):}{If you attempt an overrun while mounted, your mount makes the Strength check to determine the success or failure of the overrun attack (and applies its size modifier, rather than yours). If you have the Trample feat and attempt an overrun while mounted, your target may not choose to avoid you, and if you knock your opponent prone with the overrun, your mount may make one hoof attack against your opponent.}\\

\hypertarget{combat:sunder}{}
\normalsize\item\textbf{{Sunder}}\\\small
You can use a melee attack with a slashing or bludgeoning weapon to strike a weapon or shield that your opponent is holding. If you're attempting to sunder a weapon or shield, follow the steps outlined here. (Attacking held objects other than weapons or shields is covered below.)

\ability{Step 1:}{Attack of Opportunity. You provoke an attack of opportunity from the target whose weapon or shield you are trying to sunder. (If you have the Improved Sunder feat, you don't incur an attack of opportunity for making the attempt.)}

\ability{Step 2:}{Opposed Rolls. You and the defender make opposed attack rolls with your respective weapons. The wielder of a two\textendash handed weapon on a sunder attempt gets a +4 bonus on this roll, and the wielder of a light weapon takes a \textendash4 penalty. If the combatants are of different sizes, the larger combatant gets a bonus on the attack roll of +4 per difference in size category.}

\ability{Step 3:}{Consequences. If you beat the defender, roll damage and deal it to the weapon or shield. See Table: Common Armor, Weapon, and Shield Hardness and Hit Points to determine how much damage you must deal to destroy the weapon or shield. If you fail the sunder attempt, you don't deal any damage.}

\ability{Sundering a Carried or Worn Object:}{You don't use an opposed attack roll to damage a carried or worn object. Instead, just make an attack roll against the object's AC. A carried or worn object's AC is equal to 10 + its size modifier + the Dexterity modifier of the carrying or wearing character. Attacking a carried or worn object provokes an attack of opportunity just as attacking a held object does. To attempt to snatch away an item worn by a defender rather than damage it, see Disarm. You can't sunder armor worn by another character.}\\

\begin{table}
\centering
\caption{Armor, weapon, and Shield Hardness and Hit Points}
\rowcolors{1}{colorone}{colortwo}
\begin{tabularx}{\linewidth}{lcc}
\rowcolor{headercolor}Weapon or Shield & Hardness & HP\textsuperscript{1} \\
Light blade & 10 & 2 \\ 
One\textendash handed blade & 10 & 5 \\
Two\textendash handed blade & 10 & 10	\\   
Light metal\textendash hafted weapon & 10 & 10 \\	   
One\textendash handed metal\textendash hafted weapon & 10 & 20 \\	   
Light hafted weapon & 5 & 2 \\	   
One\textendash handed hafted weapon & 5 & 5 \\	   
Two\textendash handed hafted weapon & 5 & 10 \\   
Projectile weapon & 5 & 5	\\   
Armor &	special\textsuperscript{2} & armor bonus x 5 \\
Buckler & 10 & 5 \\  
Light wooden shield & 5 & 7 \\	   
Heavy wooden shield & 5 & 15 \\   
Light steel shield & 10 & 10 \\	   
Heavy steel shield & 10 & 20 \\	   
Tower shield & 5 & 20 \\
\multicolumn{3}{p{\linewidth}}{\textsuperscript{1} The hp value given is for Medium armor, weapons, and shields. Divide by 2 for each size category of the item smaller than Medium, or multiply it by 2 for each size category larger than Medium.} \\
\multicolumn{3}{p{\linewidth}}{\textsuperscript{2} Varies by material.} \\
\end{tabularx}
\end{table}

\normalsize\item\textbf{{Throw Splash Weapon}}\\\small
A splash weapon is a ranged weapon that breaks on impact, splashing or scattering its contents over its target and nearby creatures or objects. To attack with a splash weapon, make a ranged touch attack against the target. Thrown weapons require no weapon proficiency, so you don't take the \textendash4 nonproficiency penalty. A hit deals direct hit damage to the target, and splash damage to all creatures within 5 feet of the target.

You can instead target a specific grid intersection. Treat this as a ranged attack against AC 5. However, if you target a grid intersection, creatures in all adjacent squares are dealt the splash damage, and the direct hit damage is not dealt to any creature. (You can't target a grid intersection occupied by a creature, such as a Large or larger creature; in this case, you're aiming at the creature.)

If you miss the target (whether aiming at a creature or a grid intersection), roll 1d8. This determines the misdirection of the throw, with 1 being straight back at you and 2 through 8 counting clockwise around the grid intersection or target creature. Then, count a number of squares in the indicated direction equal to the range increment of the throw.

After you determine where the weapon landed, it deals splash damage to all creatures in adjacent squares.\\

\hypertarget{combat:trip}{}
\normalsize\item\textbf{{Trip}}\\\small
As an attack action, you may attempt to knock an opponent prone. Make a touch attack, and if you succeed make a Strength + BAB check against a DC of 10 + your opponent's Strength + BAB or Balance modifier (whichever is greater). Success leaves your opponent prone. Failure provokes an attack of opportunity.

\ability{Modifiers:}{The DC to trip an opponent who has four legs or is otherwise inherently stabile is increased by 4. Radially symmetrical creatures like Oozes cannot be tripped at all.}

\smallskip\emph{\underline{Edge Option:} If you have the edge on your target, you do not provoke an attack of opportunity if your trip attempt fails, but your target provokes an attack of opportunity from you if your trip succeeds.}\\

\hypertarget{combat:turning}{}
\normalsize\item\textbf{{Turn or Rebuke Undead}}\\\small

\subsubsection{Turning Checks}
Good clerics and paladins and some neutral clerics can channel positive energy, which can halt, drive off (rout), or destroy undead.
Evil clerics and some neutral clerics can channel negative energy, which can halt, awe (rebuke), control (command), or bolster undead.
Regardless of the effect, the general term for the activity is ``turning.'' When attempting to exercise their divine control over these creatures, characters make turning checks.

Turning undead is a supernatural ability that a character can perform as a standard action. It does not provoke attacks of opportunity.

\begin{table}[b]
\caption{Turning Undead}
\rowcolors{1}{colorone}{colortwo}
\begin{tabu}to \linewidth{X[2,l,m]X[3,l,m]}
\rowcolor{headercolor}Turning Check Result & Most Powerful Undead Affected (Maximum Hit Dice) \\
0 or lower & Cleric's level \textendash 4	 \\   
1\textendash3 & Cleric's level \textendash 3 \\
4\textendash6 & Cleric's level \textendash 2 \\
7\textendash9 & Cleric's level \textendash 1 \\
10\textendash12 & Cleric's level \\
13\textendash15 & Cleric's level + 1 \\
16\textendash18 & Cleric's level + 2 \\
19\textendash21 & Cleric's level + 3 \\
22 or higher & Cleric's level + 4 \\
\end{tabu}
\end{table}

You must present your holy symbol to turn undead. Turning is considered an attack.

\ability{Times per Day:}{You may attempt to turn undead a number of times per day equal to 3 + your Charisma modifier. You can increase this number by taking the Extra Turning feat.}

\ability{Range:}{You turn the closest turnable undead first, and you can't turn undead that are more than 60 feet away or that have total cover relative to you. You don't need line of sight to a target, but you do need line of effect.}

\ability{Turning Check:}{The first thing you do is roll a turning check to see how powerful an undead creature you can turn. This is a Charisma check (1d20 + your Charisma modifier). Table: Turning Undead gives you the Hit Dice of the most powerful undead you can affect, relative to your level. On a given turning attempt, you can turn no undead creature whose Hit Dice exceed the result on this table.}

\ability{Turning Damage:}{If your roll on Table: Turning Undead is high enough to let you turn at least some of the undead within 60 feet, roll 2d6 + your cleric level + your Charisma modifier for turning damage. That's how many total Hit Dice of undead you can turn.
If your Charisma score is average or low, it's possible to roll fewer Hit Dice of undead turned than indicated on Table: Turning Undead.

You may skip over already turned undead that are still within range, so that you do not waste your turning capacity on them.}

\ability{Effect and Duration of Turning:}{Turned undead flee from you by the best and fastest means available to them. They flee for 10 rounds (1 minute). If they cannot flee, they cower (giving any attack rolls against them a +2 bonus). If you approach within 10 feet of them, however, they overcome being turned and act normally. (You can stand within 10 feet without breaking the turning effect�you just can't approach them.) You can attack them with ranged attacks (from at least 10 feet away), and others can attack them in any fashion, without breaking the turning effect.}

\ability{Destroying Undead:}{If you have twice as many levels (or more) as the undead have Hit Dice, you destroy any that you would normally turn.}

\ability{Effect and Duration of Turning:}{Turned undead flee from you by the best and fastest means available to them. They flee for 10 rounds (1 minute). If they cannot flee, they cower (giving any attack rolls against them a +2 bonus). If you approach within 10 feet of them, however, they overcome being turned and act normally. (You can stand within 10 feet without breaking the turning effect�you just can't approach them.) You can attack them with ranged attacks (from at least 10 feet away), and others can attack them in any fashion, without breaking the turning effect.}

\ability{Destroying Undead:}{If you have twice as many levels (or more) as the undead have Hit Dice, you destroy any that you would normally turn.}

\subsubsection{Evil Clerics and Undead}

Evil clerics channel negative energy to rebuke (awe) or command (control) undead rather than channeling positive energy to turn or destroy them. An evil cleric makes the equivalent of a turning check. Undead that would be turned are rebuked instead, and those that would be destroyed are commanded.

\ability{Rebuked:}{A rebuked undead creature cowers as if in awe (attack rolls against the creature get a +2 bonus). The effect lasts 10 rounds.}

\ability{Commanded:}{A commanded undead creature is under the mental control of the evil cleric. The cleric must take a standard action to give mental orders to a commanded undead. At any one time, the cleric may command any number of undead whose total Hit Dice do not exceed his level. He may voluntarily relinquish command on any commanded undead creature or creatures in order to command new ones.}

\ability{Dispelling Turning:}{An evil cleric may channel negative energy to dispel a good cleric's turning effect. The evil cleric makes a turning check as if attempting to rebuke the undead. If the turning check result is equal to or greater than the turning check result that the good cleric scored when turning the undead, then the undead are no longer turned. The evil cleric rolls turning damage of 2d6 + cleric level + Charisma modifier to see how many Hit Dice worth of undead he can affect in this way (as if he were rebuking them).}

\ability{Bolstering Undead:}{An evil cleric may also bolster undead creatures against turning in advance. He makes a turning check as if attempting to rebuke the undead, but the Hit Dice result on Table: Turning Undead becomes the undead creatures' effective Hit Dice as far as turning is concerned (provided the result is higher than the creatures' actual Hit Dice). The bolstering lasts 10 rounds. An evil undead cleric can bolster himself in this manner.}

\subsubsection{Neutral Clerics and Undead}

A cleric of neutral alignment can either turn undead but not rebuke them, or rebuke undead but not turn them. See Turn or Rebuke Undead for more information.

Even if a cleric is neutral, channeling positive energy is a good act and channeling negative energy is evil.

\subsubsection{Paladins and Undead}

Beginning at 4th level, paladins can turn undead as if they were clerics of three levels lower than they actually are.

\subsubsection{Turning Other Creatures}

Some clerics have the ability to turn creatures other than undead.

The turning check result is determined as normal.

\hypertarget{combat:twoweapon}{}
\normalsize\item\textbf{{Two\textendash Weapon Fighting}}\\\small

If you wield a second weapon in your off hand, you can get one extra attack per round with that weapon. You suffer a \textendash6 penalty with your regular attack or attacks with your primary hand and a \textendash10 penalty to the attack with your off hand when you fight this way. If your off\textendash hand weapon is light, the penalties are reduced by 2 each. (An unarmed strike is always considered light.) The Two\textendash Weapon Fighting feat eleminates the penalties entirely.
 
%Now that it's one feat to reduce penalties completely, a table is kinda silly
%Table: Two\textendash Weapon Fighting Penalties	   
%Circumstances	Primary Hand	Off Hand	   
%Normal penalties	\textendash6	\textendash10	   
%Off\textendash hand weapon is light	\textendash4	\textendash8	   
%Two\textendash Weapon Fighting feat	\textendash4	\textendash4	   
%Off\textendash hand weapon is light and Two\textendash Weapon Fighting feat	\textendash2	\textendash2	 

\ability{Double Weapons:}{You can use a double weapon to make an extra attack with the off\textendash hand end of the weapon as if you were fighting with two weapons. The penalties apply as if the off\textendash hand end of the weapon were a light weapon.}

\ability{Thrown Weapons:}{The same rules apply when you throw a weapon from each hand. Treat a dart or shuriken as a light weapon when used in this manner, and treat a bolas, javelin, net, or sling as a one\textendash handed weapon.}

\end{list}
\subsection{Special Initiative Actions}

Here are ways to change when you act during combat by altering your place in the initiative order.

\listone
\hypertarget{combat:delay}{}
\normalsize\item\textbf{{Delay}}\\\small
By choosing to delay, you take no action and then act normally on whatever initiative count you decide to act. When you delay, you voluntarily reduce your own initiative result for the rest of the combat. When your new, lower initiative count comes up later in the same round, you can act normally. You can specify this new initiative result or just wait until some time later in the round and act then, thus fixing your new initiative count at that point.

You never get back the time you spend waiting to see what's going to happen. You can't, however, interrupt anyone else's action (as you can with a readied action).

\ability{Initiative Consequences of Delaying:}{Your initiative result becomes the count on which you took the delayed action. If you come to your next action and have not yet performed an action, you don't get to take a delayed action (though you can delay again).
If you take a delayed action in the next round, before your regular turn comes up, your initiative count rises to that new point in the order of battle, and you do not get your regular action that round.}\\

\hypertarget{combat:ready}{}
\normalsize\item\textbf{{Ready}}\\\small
The ready action lets you prepare to take an action later, after your turn is over but before your next one has begun. Readying is a standard action. It does not provoke an attack of opportunity (though the action that you ready might do so).

\ability{Readying an Action:}{You can ready a standard action, a move action, or a free action. To do so, specify the action you will take and the conditions under which you will take it. Then, any time before your next action, you may take the readied action in response to that condition. The action occurs just before the action that triggers it. If the triggered action is part of another character's activities, you interrupt the other character. Assuming he is still capable of doing so, he continues his actions once you complete your readied action. Your initiative result changes. For the rest of the encounter, your initiative result is the count on which you took the readied action, and you act immediately ahead of the character whose action triggered your readied action.
You can take a 5\textendash foot step as part of your readied action, but only if you don't otherwise move any distance during the round.}

\ability{Initiative Consequences of Readying:}{Your initiative result becomes the count on which you took the readied action. If you come to your next action and have not yet performed your readied action, you don't get to take the readied action (though you can ready the same action again). If you take your readied action in the next round, before your regular turn comes up, your initiative count rises to that new point in the order of battle, and you do not get your regular action that round.}

\ability{Distracting Spellcasters:}{You can ready an attack against a spellcaster with the trigger ``if she starts casting a spell.'' If you damage the spellcaster, she may lose the spell she was trying to cast (as determined by her Concentration check result).}

\ability{Readying to Counterspell:}{You may ready a counterspell against a spellcaster (often with the trigger ``if she starts casting a spell''). In this case, when the spellcaster starts a spell, you get a chance to identify it with a Spellcraft check (DC 15 + spell level). If you do, and if you can cast that same spell (are able to cast it and have it prepared, if you prepare spells), you can cast the spell as a counterspell and automatically ruin the other spellcaster's spell. Counterspelling works even if one spell is divine and the other arcane.

A spellcaster can use dispel magic to counterspell another spellcaster, but it doesn't always work.}

\ability{Readying a Weapon against a Charge:}{You can ready certain piercing weapons, setting them to receive charges. A readied weapon of this type deals double damage if you score a hit with it against a charging character.}

\end{list}
\section{Conditions}

If more than one condition affects a character, apply them all. If certain effects can't combine, apply the most severe effect.

\vspace*{10pt}

\ability{Ability Damaged:}{The character has temporarily lost 1 or more ability score points. Lost points return at a rate of 1 per day unless noted otherwise by the condition dealing the damage. A character with Strength 0 falls to the ground and is helpless. A character with Dexterity 0 is paralyzed. A character with Constitution 0 is dead. A character with Intelligence, Wisdom, or Charisma 0 is unconscious. Ability damage is different from penalties to ability scores, which go away when the conditions causing them go away.}

\ability{Ability Drained:}{The character has permanently lost 1 or more ability score points. The character can regain these points only through magical means. A character with Strength 0 falls to the ground and is helpless. A character with Dexterity 0 is paralyzed. A character with Constitution 0 is dead. A character with Intelligence, Wisdom, or Charisma 0 is unconscious.}

\ability{Blinded:}{The character cannot see. He takes a \- 2 penalty to Armor Class, loses his Dexterity bonus to AC (if any), moves at half speed, and takes a \- 4 penalty on Search checks and on most Strength\-  and Dexterity\- based skill checks. All checks and activities that rely on vision (such as reading and Spot checks) automatically fail. All opponents are considered to have total concealment (50\% miss chance) to the blinded character. Characters who remain blinded for a long time grow accustomed to these drawbacks and can overcome some of them.}

\ability{Blown Away:}{Depending on its size, a creature can be blown away by winds of high velocity. A creature on the ground that is blown away is knocked down and rolls 1d4 x 10 feet, taking 1d4 points of nonlethal damage per 10 feet. A flying creature that is blown away is blown back 2d6 x 10 feet and takes 2d6 points of nonlethal damage due to battering and buffering. }

\ability{Checked:}{Prevented from achieving forward motion by an applied force, such as wind. Checked creatures on the ground merely stop. Checked flying creatures move back a distance specified in the description of the effect.}

\ability{Confused:}{A confused character's actions are determined by rolling d\% at the beginning of his turn: 01\- 10, attack caster with melee or ranged weapons (or close with caster if attacking is not possible); 11\- 20, act normally; 21\- 50, do nothing but babble incoherently; 51\- 70, flee away from caster at top possible speed; 71\- 100, attack nearest creature (for this purpose, a familiar counts as part of the subject's self ). A confused character who can't carry out the indicated action does nothing but babble incoherently. Attackers are not at any special advantage when attacking a confused character. Any confused character who is attacked automatically attacks its attackers on its next turn, as long as it is still confused when its turn comes. A confused character does not make attacks of opportunity against any creature that it is not already devoted to attacking (either because of its most recent action or because it has just been attacked).}

\ability{Cowering:}{The character is frozen in fear and can take no actions. A cowering character takes a \- 2 penalty to Armor Class and loses her Dexterity bonus (if any).}

\ability{Dazed:}{The creature is unable to act normally. A dazed creature can take no actions, but has no penalty to AC.
A dazed condition typically lasts 1 round.}

\ability{Dazzled:}{The creature is unable to see well because of overstimulation of the eyes. A dazzled creature takes a \- 1 penalty on attack rolls, Search checks, and Spot checks.}

\ability{Dead:}{The character's hit points are reduced to  -10, his Constitution drops to 0, or he is killed outright by a spell or effect. The character's soul leaves his body. Dead characters cannot benefit from normal or magical healing, but they can be restored to life via magic. A dead body decays normally unless magically preserved, but magic that restores a dead character to life also restores the body either to full health or to its condition at the time of death (depending on the spell or device). Either way, resurrected characters need not worry about rigor mortis, decomposition, and other conditions that affect dead bodies.}

\ability{Deafened:}{A deafened character cannot hear. She takes a  -4 penalty on initiative checks, automatically fails Listen checks, and has a 20\% chance of spell failure when casting spells with verbal components. Characters who remain deafened for a long time grow accustomed to these drawbacks and can overcome some of them.}

\ability{Disabled:}{A character with 0 hit points, or one who has negative hit points but has become stable and conscious, is disabled. A disabled character may take a single move action or standard action each round (but not both, nor can she take full-round actions). She moves at half speed. Taking move actions doesn't risk further injury, but performing any standard action (or any other action the DM deems strenuous, including some free actions such as casting a quickened spell) deals 1 point of damage after the completion of the act. Unless the action increased the disabled character's hit points, she is now in negative hit points and dying.
A disabled character with negative hit points recovers hit points naturally if she is being helped. Otherwise, each day she has a 10\% chance to start recovering hit points naturally (starting with that day); otherwise, she loses 1 hit point. Once an unaided character starts recovering hit points naturally, she is no longer in danger of losing hit points (even if her current hit points are negative).}

\ability{Dying:}{A dying character is unconscious and near death. She has -1 to -9 current hit points. A dying character can take no actions and is unconscious. At the end of each round (starting with the round in which the character dropped below 0 hit points), the character rolls d\% to see whether she becomes stable. She has a 10\% chance to become stable. If she does not, she loses 1 hit point. If a dying character reaches -10 hit points, she is dead.}

\ability{Energy Drained:}{The character gains one or more negative levels, which might permanently drain the character's levels. If the subject has at least as many negative levels as Hit Dice, he dies. Each negative level gives a creature the following penalties: -1 penalty on attack rolls, saving throws, skill checks, ability checks; loss of 5 hit points; and -1 to effective level (for determining the power, duration, DC, and other details of spells or special abilities). In addition, a spellcaster loses one spell or spell slot from the highest spell level castable.}

\ability{Entangled:}{The character is ensnared. Being entangled impedes movement, but does not entirely prevent it unless the bonds are anchored to an immobile object or tethered by an opposing force. An entangled creature moves at half speed, cannot run or charge, and takes a -2 penalty on all attack rolls and a -4 penalty to Dexterity. An entangled character who attempts to cast a spell must make a Concentration check (DC 15 + the spell's level) or lose the spell.}

\ability{Exhausted:}{An exhausted character moves at half speed and takes a -6 penalty to Strength and Dexterity. After 1 hour of complete rest, an exhausted character becomes fatigued. A fatigued character becomes exhausted by doing something else that would normally cause fatigue.}

\ability{Fascinated:}{A fascinated creature is entranced by a supernatural or spell effect. The creature stands or sits quietly, taking no actions other than to pay attention to the fascinating effect, for as long as the effect lasts. It takes a -4 penalty on skill checks made as reactions, such as Listen and Spot checks. Any potential threat, such as a hostile creature approaching, allows the fascinated creature a new saving throw against the fascinating effect. Any obvious threat, such as someone drawing a weapon, casting a spell, or aiming a ranged weapon at the fascinated creature, automatically breaks the effect. A fascinated creature's ally may shake it free of the spell as a standard action.}

\ability{Fatigued:}{A fatigued character can neither run nor charge and takes a \- 2 penalty to Strength and Dexterity. Doing anything that would normally cause fatigue causes the fatigued character to become exhausted. After 8 hours of complete rest, fatigued characters are no longer fatigued.}

\ability{Flat\- Footed:}{A character who has not yet acted during a combat is flat\- footed, not yet reacting normally to the situation. A flat\- footed character loses his Dexterity bonus to AC (if any) and cannot make attacks of opportunity.}

\ability{Frightened:}{A frightened creature flees from the source of its fear as best it can. If unable to flee, it may fight. A frightened creature takes a \- 2 penalty on all attack rolls, saving throws, skill checks, and ability checks. A frightened creature can use special abilities, including spells, to flee; indeed, the creature must use such means if they are the only way to escape. Frightened is like shaken, except that the creature must flee if possible. Panicked is a more extreme state of fear.}

\ability{Grappling:}{Engaged in wrestling or some other form of hand\- to\- hand struggle with one or more attackers. A grappling character can undertake only a limited number of actions. He does not threaten any squares, and loses his Dexterity bonus to AC (if any) against opponents he isn't grappling.}

\ability{Helpless:}{A helpless character is paralyzed, held, bound, sleeping, unconscious, or otherwise completely at an opponent's mercy. A helpless target is treated as having a Dexterity of 0 (\- 5 modifier). Melee attacks against a helpless target get a +4 bonus (equivalent to attacking a prone target). Ranged attacks gets no special bonus against helpless targets. Rogues can sneak attack helpless targets.}

% Right, this is part of combat now
 
%As a full\- round action, an enemy can use a melee weapon to deliver a coup de grace to a helpless foe. An enemy can also use a bow or crossbow, provided he is adjacent to the target. The attacker automatically hits and scores a critical hit. (A rogue also gets her sneak attack damage bonus against a helpless foe when delivering a coup de grace.) If the defender survives, he must make a Fortitude save (DC 10 + damage dealt) or die. 
%Delivering a coup de grace provokes attacks of opportunity. 
%Creatures that are immune to critical hits do not take critical damage, nor do they need to make Fortitude saves to avoid being killed by a coup de grace.

\ability{Incorporeal:}{Having no physical body. Incorporeal creatures are immune to all nonmagical attack forms. They can be harmed only by other incorporeal creatures, +1 or better magic weapons, spells, spell\- like effects, or supernatural effects.}

\ability{Invisible:}{Visually undetectable. An invisible creature gains a +2 bonus on attack rolls against sighted opponents, and ignores its opponents' Dexterity bonuses to AC (if any). (See Invisibility, under Special Abilities.)}

\ability{Knocked Down:}{Depending on their size, creatures can be knocked down by winds of high velocity. Creatures on the ground are knocked prone by the force of the wind. Flying creatures are instead blown back 1d6 x 10 feet.}

\ability{Nauseated:}{Experiencing stomach distress. Nauseated creatures are unable to attack, cast spells, concentrate on spells, or do anything else requiring attention. The only action such a character can take is a single move action per turn.}

\ability{Panicked:}{A panicked creature must drop anything it holds and flee at top speed from the source of its fear, as well as any other dangers it encounters, along a random path. It can't take any other actions. In addition, the creature takes a \- 2 penalty on all saving throws, skill checks, and ability checks. If cornered, a panicked creature cowers and does not attack, typically using the total defense action in combat. A panicked creature can use special abilities, including spells, to flee; indeed, the creature must use such means if they are the only way to escape. Panicked is a more extreme state of fear than shaken or frightened.}

\ability{Paralyzed:}{A paralyzed character is frozen in place and unable to move or act. A paralyzed character has effective Dexterity and Strength scores of 0 and is helpless, but can take purely mental actions. A winged creature flying in the air at the time that it becomes paralyzed cannot flap its wings and falls. A paralyzed swimmer can't swim and may drown. A creature can move through a space occupied by a paralyzed creature\- ally or not. Each square occupied by a paralyzed creature, however, counts as 2 squares.}

\ability{Petrified:}{A petrified character has been turned to stone and is considered unconscious. If a petrified character cracks or breaks, but the broken pieces are joined with the body as he returns to flesh, he is unharmed. If the character's petrified body is incomplete when it returns to flesh, the body is likewise incomplete and there is some amount of permanent hit point loss and/or debilitation.}

\ability{Pinned:}{Held immobile (but not helpless) in a grapple.}

\ability{Prone:}{The character is on the ground. An attacker who is prone has a \- 4 penalty on melee attack rolls and cannot use a ranged weapon (except for a crossbow). A defender who is prone gains a +4 bonus to Armor Class against ranged attacks, but takes a \- 4 penalty to AC against melee attacks.

Standing up is a move\- equivalent action that provokes an attack of opportunity.}

\ability{Shaken:}{A shaken character takes a -2 penalty on attack rolls, saving throws, skill checks, and ability checks. Shaken is a less severe state of fear than frightened or panicked.}

\ability{Sickened:}{The character takes a \- 2 penalty on all attack rolls, weapon damage rolls, saving throws, skill checks, and ability checks.}

\ability{Stable:}{A character who was dying but who has stopped losing hit points and still has negative hit points is stable. The character is no longer dying, but is still unconscious. If the character has become stable because of aid from another character (such as a Heal check or magical healing), then the character no longer loses hit points. He has a 10\% chance each hour of becoming conscious and disabled (even though his hit points are still negative).

If the character became stable on his own and hasn't had help, he is still at risk of losing hit points. Each hour, he has a 10\% chance of becoming conscious and disabled. Otherwise he loses 1 hit point.}

\ability{Staggered:}{A character whose nonlethal damage exactly equals his current hit points is staggered. A staggered character may take a single move action or standard action each round (but not both, nor can she take full\- round actions).
A character whose current hit points exceed his nonlethal damage is no longer staggered; a character whose nonlethal damage exceeds his hit points becomes unconscious.}

\ability{Stunned:}{A stunned creature drops everything held, can't take actions, takes a \- 2 penalty to AC, and loses his Dexterity bonus to AC (if any).}

\ability{Turned:}{Affected by a turn undead attempt. Turned undead flee for 10 rounds (1 minute) by the best and fastest means available to them. If they cannot flee, they cower.}

\ability{Unconscious:}{Knocked out and helpless. Unconsciousness can result from having current hit points between \- 1 and \- 9, or from nonlethal damage in excess of current hit points.}

\chapter{Magic}
\section{Casting Spells}
foo
\section{How To Read A Spell Description}
foo
\section{Arcane Spells}
foo
\section{Divine Spells}
foo
\section{Special Abilities and Spells}
foo
\section{Spell Lists}
foo
\chapter{Magic Items}
\section{Magic Item Basics}
Scaling, 8 Item Limit, etc
\section{Minor Magical Items}
foo
\section{Moderate Magical Items}
foo
\section{Major Magical Items}
foo

%%%%%%%%%%%%%%%%%%%%%%%%%%%%%%%%%%%%%%%%%%%%%%%%%%
%%%%%%%%%%%%%%%%%%%%%%%%%%%%%%%%%%%%%%%%%%%%%%%%%%
\appendix
%%%%%%%%%%%%%%%%%%%%%%%%%%%%%%%%%%%%%%%%%%%%%%%%%%
%%%%%%%%%%%%%%%%%%%%%%%%%%%%%%%%%%%%%%%%%%%%%%%%%%
\appendixpage

\makeatletter
\renewcommand{\@makechapterhead}[1]{%
\vspace*{0 pt}{
\raggedright \normalfont \fontsize{32}{32} \selectfont \bfseries
\ifnum \value{secnumdepth}>-1
  \if@mainmatter \vspace{-8pt} {\fontsize{20}{20} \selectfont Appendix \thechapter:}\\[8pt]
  \fi%
\fi
\hspace{0.65cm} #1\par\nobreak\vspace{20 pt}
}}
\makeatother

\clearpage

%% Appendix Chapters Here

%%%%%%%%%%%%%%%%%%%%%%%%
%%Spell Formatting
%%%%%%%%%%%%%%%%%%%%%%%%

\newenvironment{spellwriteup}{
\begin{minipage}[h]{\linewidth}
}{
\end{minipage}
}

\newcommand{\spellhead}[4]{\belowpdfbookmark{#1}{spell:#1}\paragraph{\Large#1}\normalsize\textbf{#2 (#3) [#4]} \\}

\newcommand{\spellfoot}{
\vspace{1em}
}

\newcommand{\spelldatastart}{
\begin{wraptable}{R!}{.45\linewidth}
\vspace{-1.1em}
\tabulinesep=1mm
\begin{tabular}{|r p{.55\linewidth}|}
\hline
}

\newcommand{\spelldataend}{
\hline
\end{tabular}
\vspace{-1em}
\end{wraptable}
\vspace{-1em}
}

%\newcommand{\spellschool}[3]{\multicolumn{2}{l}{\textbf{#1 (#2) [#3]}}\\}
% unused

\newcommand{\spelllevel}[1]{\vspace{-.9em} & \vspace{-.9em} \\
\textbf{Level:} & #1\\}
\newcommand{\spellcomp}[1]{\textbf{Components:} & #1 \\}
\newcommand{\spellcast}[1]{\textbf{Casting Time:} & #1 \\}
\newcommand{\spellrange}[1]{\textbf{Range:} & #1 \\}
\newcommand{\spelleffect}[1]{\textbf{Effect:} & #1 \\}
\newcommand{\spellarea}[1]{\textbf{Area:} & #1 \\}
\newcommand{\spelltarget}[1]{\textbf{Target:} & #1 \\}
\newcommand{\spelltargets}[1]{\textbf{Targets:} & #1 \\}
\newcommand{\spellduration}[1]{\textbf{Duration:} & #1 \\}
\newcommand{\spellsave}[1]{\textbf{Saving Throw:} & #1 \\}
\newcommand{\spellsr}[1]{\textbf{Spell Resistance:} & #1 \\}

\newcommand{\closerange}{Close}
\newcommand{\mediumrange}{Medium}
\newcommand{\longrange}{Long}

\newcommand{\component}[1]{\textit{Material Component:} #1}
\newcommand{\focus}[1]{\textit{Focus:} #1}

%%%%%%%%%%%%%%%%%%%%%%%%

\chapter{Spells}
\section{Spells, A through Z}
\belowpdfbookmark{A}{Spells:A}
%\begin{spellwriteup}
\spellhead{Acid Arrow}{Conjuration}{Creation}{Acid}
\spelldatastart
\spelllevel{Sor/Wiz 2}
\spellcomp{V, S, M, F}
\spellcast{1 standard action}
\spellrange{\longrange}
\spelleffect{One arrow of acid}
\spellduration{1 round + 1 round / three levels}
\spellsave{None}
\spellsr{No}
\spelldataend

A magical arrow of acid springs from your hand and speeds to its target. You must succeed on a ranged touch attack to hit your target. The arrow deals 2d4 points of acid damage with no splash damage. For every three caster levels (to a maximum of 18th), the acid, unless somehow neutralized, lasts for another round, dealing another 2d4 points of damage in that round.

\component{Powdered rhubarb leaf and an adder's stomach.}

\focus{A dart.}
\spellfoot
%\end{spellwriteup}
%\begin{spellwriteup}
\spellhead{Acid Fog}{Conjuration}{Creation}{Acid}
\spelldatastart
\spelllevel{Sor/Wiz 6, Water 7}
\spellcomp{V, S, M/DF}
\spellcast{1 standard action}
\spellrange{\mediumrange}
\spelleffect{Fog spreads in 20-ft. radius, 20 ft. high}
\spellduration{1 round / level}
\spellsave{None}
\spellsr{No}
\spelldataend

Acid Fog creates a billowing mass of misty vapors similar to that produced by a \linkspell{Solid Fog} spell. In addition to slowing creatures down and obscuring sight, this spell's vapors are highly acidic. Each round on your turn, starting when you cast the spell, the fog deals 2d6 points of acid damage to each creature and object within it.

\component{A pinch of dried, powdered peas combined with powdered animal hoof.}
\newline
\newline
\spellfoot
%\end{spellwriteup}
\spellhead{Acid Splash}{Conjuration}{Creation}{Acid}
\spelldatastart
\spelllevel{Sor/Wiz 0}
\spellcomp{V, S}
\spellcast{1 standard action}
\spellrange{\closerange}
\spelleffect{One missile of acid}
\spellduration{Instantaneous}
\spellsave{None}
\spellsr{No}
\spelldataend

You fire a small orb of acid at the target. You must succeed on a ranged touch attack to hit your target. The orb deals 1d3 points of acid damage.
\newline
\newline
\newline
\newline
\newline
\spellfoot
\spellhead{Aid}{Enchantment}{Compulsion}{Mind-Affecting}
\spelldatastart
\spelllevel{Clr 2, Good 2, Luck 2}
\spellcomp{V, S, DF}
\spellcast{1 standard action}
\spellrange{Touch}
\spelltarget{Living creature touched}
\spellduration{1 min. / level}
\spellsave{None}
\spellsr{Yes (harmless)}
\spelldataend

Aid grants the target a +1 morale bonus on attack rolls and saves against fear effects, plus temporary hit points equal to 1d8 + caster level (to a maximum of 1d8+10 temporary hit points at caster level 10th).
\newline
\newline
\newline
\newline
\newline
\newline
\spellfoot



\spellhead{Air Walk}{Transmutation}{}{Air}
\spelldatastart
\spelllevel{Air 4, Clr 4, Drd 4}
\spellcomp{V, S, DF}
\spellcast{1 standard action}
\spellrange{Touch}
\spelltarget{Creature (Gargantuan or smaller) touched}
\spellduration{10 min. / level}
\spellsave{None}
\spellsr{Yes (harmless)}
\spelldataend

The subject can tread on air as if walking on solid ground. Moving upward is similar to walking up a hill. The maximum upward or downward angle possible is 45 degrees, at a rate equal to one-half the air walker's normal speed.

A strong wind (21+ mph) can push the subject along or hold it back. At the end of its turn each round, the wind blows the air walker 5 feet for each 5 miles per hour of wind speed. The creature may be subject to additional penalties in exceptionally strong or turbulent winds, such as loss of control over movement or physical damage from being buffeted about.

Should the spell duration expire while the subject is still aloft, the magic fails slowly. The subject floats downward 60 feet per round for 1d6 rounds. If it reaches the ground in that amount of time, it lands safely. If not, it falls the rest of the distance, taking 1d6 points of damage per 10 feet of fall. Since dispelling a spell effectively ends it, the subject also descends in this way if the Air Walk spell is dispelled, but not if it is negated by an \linkspell{Antimagic Field}.

You can cast Air Walk on a specially trained mount so it can be ridden through the air. You can train a mount to move with the aid of Air Walk (counts as a trick; see \linkskill{Handle Animal} skill) with one week of work and a DC 25 Handle Animal check.
\spellfoot




\chapter{Prestige Classes}
\section{Prestige Class Basics}
\section{WhatClasses}

\chapter{Monsters}
\section{Reading a Monster Entry}
\section{Monsters, A though Z}

\chapter{NPC Classes}
\section{Adept}
foo
\section{Aristocrat}
foo
\section{Commoner}
foo
\section{Expert}
foo
\section{Warrior}
foo

%%%%%%%%%%%%%%%%%%%%%%%%%%%%%%%%%%%%%%%%%%%%%%%%%%
%%%%%%%%%%%%%%%%%%%%%%%%%%%%%%%%%%%%%%%%%%%%%%%%%%
\chapter{The Open Game License}
%%%%%%%%%%%%%%%%%%%%%%%%%%%%%%%%%%%%%%%%%%%%%%%%%%
%%%%%%%%%%%%%%%%%%%%%%%%%%%%%%%%%%%%%%%%%%%%%%%%%%
\label{Open Game License}

The text and tables of this document are Open Game Content as defined in the Open Game License below. The images of this document are Product Identity. You can compare this work to any other work that you like in any way.

%Images within this document are not released under the OGL. Instead they are used under a \href{http://creativecommons.org/licenses/by-sa/3.0/}{Creative Commons Attribution-ShareAlike} license, and come from the following sources:
%\begin{description*}
%\item[Cover Image] \href{http://www.gallery.oldbookart.com/main.php?g2_itemId=10682}{Courtesy of OldBookArt.com}
%\end{description*}

\oldsection*{OPEN GAME LICENSE Version 1.0a}

\begin{small}

The following text is the property of Wizards of the Coast, Inc. and is Copyright 2000 Wizards of the Coast, Inc ("Wizards"). All Rights Reserved.

\begin{oldenumerate}\raggedright
\item \textbf{Definitions}:
	\begin{oldenumerate}[(a)]\raggedright
	\item "Contributors" means the copyright and/or trademark owners who have contributed Open Game Content;
	\item "Derivative Material" means copyrighted material including derivative works and translations (including into other computer languages), potation, modification, correction, addition, extension, upgrade, improvement, compilation, abridgment or other form in which an existing work may be recast, transformed or adapted;
	\item "Distribute" means to reproduce, license, rent, lease, sell, broadcast, publicly display, transmit or otherwise distribute;
	\item "Open Game Content" means the game mechanic and includes the methods, procedures, processes and routines to the extent such content does not embody the Product Identity and is an enhancement over the prior art and any additional content clearly identified as Open Game Content by the Contributor, and means any work covered by this License, including translations and derivative works under copyright law, but specifically excludes Product Identity.
	\item "Product Identity" means product and product line names, logos and identifying marks including trade dress; artifacts; creatures characters; stories, storylines, plots, thematic elements, dialogue, incidents, language, artwork, symbols, designs, depictions, likenesses, formats, poses, concepts, themes and graphic, photographic and other visual or audio representations; names and descriptions of characters, spells, enchantments, personalities, teams, personas, likenesses and special abilities; places, locations, environments, creatures, equipment, magical or supernatural abilities or effects, logos, symbols, or graphic designs; and any other trademark or registered trademark clearly identified as Product identity by the owner of the Product Identity, and which specifically excludes the Open Game Content;
	\item "Trademark" means the logos, names, mark, sign, motto, designs that are used by a Contributor to identify itself or its products or the associated products contributed to the Open Game License by the Contributor
	\item "Use", "Used" or "Using" means to use, Distribute, copy, edit, format, modify, translate and otherwise create Derivative Material of Open Game Content.
	\item "You" or "Your" means the licensee in terms of this agreement.
	\end{oldenumerate}

\item \textbf{The License}: This License applies to any Open Game Content that contains a notice indicating that the Open Game Content may only be Used under and in terms of this License. You must affix such a notice to any Open Game Content that you Use. No terms may be added to or subtracted from this License except as described by the License itself. No other terms or conditions may be applied to any Open Game Content distributed using this License. 

\item \textbf{Offer and Acceptance}: By Using the Open Game Content You indicate Your acceptance of the terms of this License. 

\item \textbf{Grant and Consideration}: In consideration for agreeing to use this License, the Contributors grant You a perpetual, worldwide, royalty-free, non-exclusive license with the exact terms of this License to Use, the Open Game Content. 

\item \textbf{Representation of Authority to Contribute}: If You are contributing original material as Open Game Content, You represent that Your Contributions are Your original creation and/or You have sufficient rights to grant the rights conveyed by this License. 

\item \textbf{Notice of License Copyright}: You must update the COPYRIGHT NOTICE portion of this License to include the exact text of the COPYRIGHT NOTICE of any Open Game Content You are copying, modifying or distributing, and You must add the title, the copyright date, and the copyright holder's name to the COPYRIGHT NOTICE of any original Open Game Content you Distribute. 

\item \textbf{Use of Product Identity}: You agree not to Use any Product Identity, including as an indication as to compatibility, except as expressly licensed in another, independent Agreement with the owner of each element of that Product Identity. You agree not to indicate compatibility or co-adaptability with any Trademark or Registered Trademark in conjunction with a work containing Open Game Content except as expressly licensed in another, independent Agreement with the owner of such Trademark or Registered Trademark. The use of any Product Identity in Open Game Content does not constitute a challenge to the ownership of that Product Identity. The owner of any Product Identity used in Open Game Content shall retain all rights, title and interest in and to that Product Identity. 

\item \textbf{Identification}: If you distribute Open Game Content You must clearly indicate which portions of the work that you are distributing are Open Game Content. 

\item \textbf{Updating the License}: Wizards or its designated Agents may publish updated versions of this License. You may use any authorized version of this License to copy, modify and distribute any Open Game Content originally distributed under any version of this License. 

\item \textbf{Copy of this License}: You MUST include a copy of this License with every copy of the Open Game Content You Distribute. 

\item \textbf{Use of Contributor Credits}: You may not market or advertise the Open Game Content using the name of any Contributor unless You have written permission from the Contributor to do so. 

\item \textbf{Inability to Comply}: If it is impossible for You to comply with any of the terms of this License with respect to some or all of the Open Game Content due to statute, judicial order, or governmental regulation then You may not Use any Open Game Material so affected. 

\item \textbf{Termination}: This License will terminate automatically if You fail to comply with all terms herein and fail to cure such breach within 30 days of becoming aware of the breach. All sublicenses shall survive the termination of this License. 

\item \textbf{Reformation}: If any provision of this License is held to be unenforceable, such provision shall be reformed only to the extent necessary to make it enforceable. 

\item \textbf{COPYRIGHT NOTICE}

\href{http://www.wizards.com/default.asp?x=d20/oglfaq/20040123f}{\textit{Open Game License v 1.0a}} Copyright 2000, Wizards of the Coast, Inc. 

\href{http://www.wizards.com/default.asp?x=d20/article/srd35}{\textit{System Reference Document}} Copyright 2000-2003, Wizards of the Coast, Inc.; Authors Jonathan Tweet, Monte Cook, Skip Williams, Rich Baker, Andy Collins, David Noonan, Rich Redman, Bruce R. Cordell, John D. Rateliff, Thomas Reid, James Wyatt, based on original material by E. Gary Gygax and Dave Arneson.

\href{http://www.tgdmb.com/viewtopic.php?t=34248}{\textit{Tome of Necromancy}} Copyright 2006, Frank Trollman and K

\href{http://www.tgdmb.com/viewtopic.php?t=28828}{\textit{Tome of Fiends}} Copyright 2006, Frank Trollman and K

\href{http://www.tgdmb.com/viewtopic.php?t=28547}{\textit{Dungeonomicon}} Copyright 2006, Frank Trollman and K

\href{http://www.tgdmb.com/viewtopic.php?t=33294}{\textit{Races of War}} Copyright 2006, Frank Trollman and K

\href{http://www.tgdmb.com/viewtopic.php?t=35813}{\textit{Book of Gears}} Copyright 2007, Frank Trollman and K

\href{https://github.com/SqueeG/awesomeTome}{\textit{Tome Reference Document}} Copyright 2013, Daniel Gee, ExplosiveRunes, SqueeG, and Tarkisflux

\end{oldenumerate}

END OF LICENSE

\end{small}

\clearpage
\phantomsection
\listoftables

\clearpage
\phantomsection
\printindex

\end{document}