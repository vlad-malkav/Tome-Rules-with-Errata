
\begin{wraptable}{r}{.45\textwidth}
\rowcolors{1}{colorone}{colortwo}
\caption{Mounts and Related Gear}
{\tabulinesep=1mm
\begin{tabu}to \linewidth{X r r}
\header\textbf{Item} & \textbf{Cost} & \textbf{Weight}\\ \hline
Barding&&\\
\hspace{.5cm}Medium creature & x2 & x1\\
\hspace{.5cm}Large creature & x4 & x2\\
Bit and bridle & 2 gp & 1 lb.\\
Dog&&\\
\hspace{.5cm}Guard Dog & 25 gp & --\\
\hspace{.5cm}Riding Dog & 150 gp & --\\
Donkey or mule & 8 gp & --\\
Feed (per day) & 5 cp & 10 lb.\\
Horse&&\\
\hspace{.5cm}Heavy Horse & 200 gp & --\\
\hspace{.5cm}Heavy Warhorse & 400 gp & --\\
\hspace{.5cm}Light Horse & 75 gp & --\\
\hspace{.5cm}Light Warhorse & 150 gp & --\\
\hspace{.5cm}Pony & 30 gp & --\\
\hspace{.5cm}Warpony & 100 gp & --\\
Saddle (common)&&\\
\hspace{.5cm}Military & 20 gp & 30 lb.\\
\hspace{.5cm}Pack & 5 gp & 15 lb.\\
\hspace{.5cm}Riding & 10 gp & 25 lb.\\
Saddle (exotic)&&\\
\hspace{.5cm}Military & 60 gp & 40 lb.\\
\hspace{.5cm}Pack & 15 gp & 20 lb.\\
\hspace{.5cm}Riding & 30 gp & 30 lb.\\
Saddlebags & 4 gp & 8 lb.\\
Stabling (per day) & 5 sp & --\\
\hline
\end{tabu}}

\rowcolors{1}{colorone}{colortwo}
\caption{Mount Speed In Armor}
{\tabulinesep=1mm
\begin{tabu}to \linewidth{X c c c}
\header\textbf{Barding} & \textbf{40ft} & \textbf{50ft} & \textbf{60ft}\\ \hline
Medium & 30ft & 35ft & 40ft\\
Heavy & 30ft\textsuperscript{1} & 35ft\textsuperscript{1} & 40ft\textsuperscript{1}\\ \hline
\multicolumn{4}{p{\linewidth}}{\textsuperscript{1} A mount wearing heavy armor moves at only triple its normal speed when running, instead of quadruple.}\\ \hline
\end{tabu}}
\end{wraptable}

\textbf{Barding, Medium Creature and Large Creature:} Barding is a type of armor 
that covers the head, neck, chest, body, and possibly legs of a horse or other 
mount. Barding made of medium or heavy armor provides better protection than light 
barding, but at the expense of speed. Barding can be made of any of the armor types 
found on Table: Armor and Shields.

Armor for a horse (a Large nonhumanoid creature) costs four times as much as armor 
for a human (a Medium humanoid creature) and also weighs twice as much as the armor 
found on Table: Armor and Shields (see Armor for Unusual Creatures). If the barding 
is for a pony or other Medium mount, the cost is only double, and the weight is 
the same as for Medium armor worn by a humanoid. Medium or heavy barding slows 
a mount that wears it, as shown on the table below.

Flying mounts can't fly in medium or heavy barding.

Removing and fitting barding takes five times as long as the figures given on Table: 
Donning Armor. A barded animal cannot be used to carry any load other than the 
rider and normal saddlebags.

\textbf{Dog, Riding:} This Medium dog is specially trained to carry a Small humanoid 
rider. It is brave in combat like a warhorse. You take no damage when you fall 
from a riding dog.

\textbf{Donkey or Mule:} Donkeys and mules are stolid in the face of danger, hardy, 
surefooted, and capable of carrying heavy loads over vast distances. Unlike a horse, 
a donkey or a mule is willing (though not eager) to enter dungeons and other strange 
or threatening places.

\textbf{Feed:} Horses, donkeys, mules, and ponies can graze to sustain themselves, 
but providing feed for them is much better. If you have a riding dog, you have 
to feed it at least some meat.

\textbf{Horse:} A horse (other than a pony) is suitable as a mount for a human, 
dwarf, elf, half-elf, or half-orc. A pony is smaller than a horse and is a suitable 
mount for a gnome or halfling.

Warhorses and warponies can be ridden easily into combat. Light horses, ponies, 
and heavy horses are hard to control in combat.

\textbf{Saddle, Exotic:} An exotic saddle is like a normal saddle of the same sort 
except that it is designed for an unusual mount. Exotic saddles come in military, 
pack, and riding styles.

\textbf{Saddle, Military:} A military saddle braces the rider, providing a +2 circumstance 
bonus on Ride checks related to staying in the saddle. If you're knocked unconscious 
while in a military saddle, you have a 75\% chance to stay in the saddle (compared 
to 50\% for a riding saddle).

\textbf{Saddle, Pack:} A pack saddle holds gear and supplies, but not a rider. 
It holds as much gear as the mount can carry.

\textbf{Saddle, Riding:} The standard riding saddle supports a rider