\section{[Combat] Feats} \label{feats:combat}

	\begin{multicols}{2}

	\input{feats/combat/!combat}

	\end{multicols}

	\section{[Skill] Feats}

	\begin{multicols}{2}

	\input{feats/skill/!skill}

	\end{multicols}

\section{[Celestial] and [Fiend] Feats} \label{feats:outsider}

A feat with the [Celestial] or [Fiend] tag can only be taken by a creature who is an Outsider. For this purpose, any creature from any upper plane is a Celestial regardless of its alignment, while any creature from any lower plane is a Fiend regardless of its alignment. Further, any elemental or outsider with a Good alignment is a Celestial regardless of its plane of origin., while any elemental or outsider with an Evil alignment is a Fiend regardless of its plane of origin. The abilities granted by feats with the [Celestial] or [Fiend] tag are Extraordinary abilities unless otherwise stated. A Celestial does not gain a Fiendish trait from taking a [Celestial] feat that also has the [Fiend] tag.

	\begin{multicols}{2}

	\descfeat{Apprenticeship}
{New Mentor Types for the Lower Planes:}
While the race of a mentor is usually irrelevant, some mentors draw their knowledge and experience solely from their racial heritage and the magical radiations of their home plane. To choose one of these extraplanar mentors, the character must have at least 2 ranks in Knowledge (planes).\vspace*{\baselineskip}

	\shortability{Devil:}{A devil mentor is a powerful baatezu from the Nine Hells of Baator that has decided to share its knowledge with a worthy apprentice. An apprentice of this mentor gains an innate understanding of infernal contracts, and may use Knowledge(planes) to influence the attitude of any native of a plane that is aligned to law and evil, or any subject of a calling spell.}
	 \vspace*{-\baselineskip}\listone \item Knowledge (Planes) \item Knowledge (any one)\end{list} \vspace*{\baselineskip}

	\shortability{Demon:}{A demon mentor is a powerful tanar'ri from the Abyss, and it has forced his apprentices to hide a portion of his power. Once per month, you may use one of its spell-like abilities of a 2nd level effect or less.}
	 \vspace*{-\baselineskip}\listone \item Bluff \item Knowledge (Planes)\end{list} \vspace*{\baselineskip}

	\shortability{Yugoloth:}{While utterly evil, Yugoloth mentors are honorable in their own way and have been known to train apprentices in the dark arts. Apprentices of these fiends learn the true nature of evil, and may choose to count as evil for the prerequisites of feats or prestige classes, and for magical effects like spells or magic items.}
	 \vspace*{-\baselineskip}\listone \item Knowledge (Planes) \item Diplomacy\end{list} \vspace*{\baselineskip}

	\shortability{Demondand:}{A demondand mentor is a powerful fiend from Carceri. An apprentice of this mentor learns the arts of punishment at the hands of these extraplanar jailors, and may use Intimidate to influence a creature's attitude to Helpful by accepting a +10 to the DC of the check.}
	 \vspace*{-\baselineskip}\listone \item Knowledge (Planes) \item Intimidate\end{list} \vspace*{\baselineskip}
\descfeat{Attune Domain}{You incorporate the workings of a divine domain into your magic.}
	\shortability{Prerequisite:}{Caster level 1+, Must follow a god or philosophy consistent with the chosen domain.}
	\shortability{Benefit:}{Choose a domain when this feat is selected. Every spell from that domain is considered to be on your spell-list for any spellcasting classes you happen to have. These spells are considered to be spells known at the level they appear in the chosen domain. These spells are cast (and prepared, if appropriate) as normal for your class.}
	\shortability{Special:}{You may select this feat multiple times, its effects do not stack. Each time you may select a new domain, so long as your chosen god or philosophy can incorporate all of them. As usual, your DM must approve any god or philosophy. You may not have more than three attuned domains or spheres together.}
\descfeat{Attune Sphere}{You incorporate the workings of a sphere into your magic.}
	\shortability{Prerequisite:}{Caster level 1+, Must have bled from a wound inflicted by an outsider with access to the chosen sphere.}
	\shortability{Benefit:}{Choose a sphere when this feat is selected. Every spell from that sphere is considered to be on your spell-list for any spellcasting classes you happen to have. You are considered to know each of those spells at the level they appear in the chosen sphere. These spells are cast (and prepared, if appropriate) as normal for your class.}
	\shortability{Special:}{You may select this feat multiple times, its effects do not stack. Each time you may select a new sphere. You may not have more than three attuned domains or spheres together.}
\descfeat{Blood War Sorcerer}
{As a battle magician in the Blood War, you've learned killing arts that would amaze common spellcasters.}
	\shortability{Prerequisite:}{Blood War Squaddie, Caster level 5, must have fought in the Blood war for one year.}
	\shortability{Benefit:}{Each time one of your spells successfully damages a creature with Spell Resistance, they take a cumulative -1 penalty to SR. This penalty is reduced by 5 for every day of rest, and can be otherwise healed as ability damage. In addition, you may cast any spell that requires you to be of a fiendish race.}
\descfeat{Blood War Squaddie}
	{Due to your time during the Blood War, you've been tainted, honed, and hardened by the horrors you've seen.}
	\shortability{Prerequisite:}{Knowledge(planes) 2, must have fought in the Blood War for one year.}
	\shortability{Benefit:}{You are immune to fear, and actually gain a +2 Morale bonus to hit, damage, and saves when exposed to an enemy's fear effect (this bonus lasts one minute). In addition, you may treat any fiendish Exotic weapons as martial weapons.}
	\shortability{Special:}{This can only be taken at 1st level.}

	\begin{small}
	\begin{tabular}{ll}
	Spell   &CR\\
	Summon Monster I    &2\\
	Summon Monster II   &4\\
	Summon Monster III  &6\\
	Summon Monster IV   &8\\
	Summon Monster V    &10\\
	Summon Monster VI   &12\\
	Summon Monster VII  &14\\
	Summon Monster VIII &16\\
	Summon Monster IX   &18\\
	\end{tabular}
	\end{small}
\featname{Breath Weapon [Celestial], [Fiend]}
	\shortability{Prerequisites:}{Character level 6.}
	\shortability{Benefits:}{Choose a spell-like ability you possess with a duration of Instantaneous: this ability can be used as a Supernatural Breath Weapon with an area equal to a 10' per spell level of the spell-like ability used. Each use of this ability expends one use of the spell-like ability. Each time this breath weapon is used, it cannot be used again for 1d4 rounds.}
\descfeat{Broker of the Infernal}{Dues to the study of the Infernal laws, you have learned to harness the powers of True Names in your summoning magic.}
	\shortability{Prerequisites:}{Knowledge (Planes) 10, must be able to cast a spell of the [calling] subtype}
	\shortability{Benefits:}{When you possess the True Name of a creature, you may summon it with a Summon Monster spell. The version of the summon monster spell used must equal half their CR, as shown below. For all effects, this spell is a summoning spell, and functions as if the creature were a summoned monster, but if killed the creature is dead as normal and cannot be summoned again until it is returned from the dead.}
\descfeat{Carrier [Fiend]}
{You are a carrier of a dangerous disease, though you are immune to its effects}
	\shortability{Prerequisite:}{Must have one level of a Fiend class.}
	\shortability{Benefit:}{When you gain this ability choose a disease with a DC equal to the DC your disease would have(Half HD + Con mod). You disease does ability damage or special effects equal to the disease chosen. Once chosen, your disease type does not change, but your disease DC will increase when your HD or Con modifier increase. Unlike a normal disease, this is a supernatural disease, and its initial effects occur immediately.}
\descfeat{Constricting Fiend [Fiend]}
{Your legs merge into a long tail, and you gain the ability to squeeze the life from your foes.}
	\shortability{Prerequisites:}{Character level 6.}
	\shortability{Benefits:}{On a successful Grapple check, you can choose to do a 4d6 Constricting attack as a normal attack. Due to you change in form and body type, you can only use nonstandard-sized armor.}
\descfeat{Craft of the Soulstealer}
{By studying stolen souls, you have learned to fully tap their power for your magical creations.}
	\shortability{Prerequisites:}{Three or more item creation feats, caster level 6.}
	\shortability{Benefits:}{When creating magic items, you can bind a soul into the item by adding the actual receptacle of the soul into the item. In many cases, this is a gemstone that is added as decoration. A single soul is worth GP equal to its CR square, times 100 for magic item creation purposes, and is worth 1/5th of that value in XP. Only one soul may be added to an item, and any extra gold or XP provided by the soul above the cost of the item is wasted. Also, if the creature whose soul was taken had spell-like abilities, these spells may be used as prerequisites for the item's creation. Any item created by this art radiates the alignment of the soul inside the item, and it also radiates strong evil. If the receptacle containing the soul is removed from the item, the item is destroyed and the soul is released.}
\descfeat{Devour the Soul [Fiend]}
{As a fiend, you gain nourishment from devouring souls.}
	\shortability{Prerequisite:}{Must have one level of a Fiend class.}
	\shortability{Benefit:}{Each time a soul is consumed (either a receptacle or petitioner), you regain HPs equal to 10 times its CR, and heal ability damage or drain equal to its CR. Souls eaten in this fashion cannot be restored from the dead until you are killed.}
\descfeat{Dominions of the Infernal}
{When you call, armies of those you have defeated are forced to answer in service.}
	\shortability{Prerequisite:}{Must have the signature \spell{summon} ability of the great Fiendish Houses; must have a Leadership score.}
	\shortability{Benefit:}{If you successfully \spell{summon} a fiend with a CR less than your level, more than one creature may appear. The weaker the creatures are, the more are summoned.}

	\begin{small}
	\begin{tabular}{ll}%
	CR  &Number Appearing\\
	Level-2   &d2\\
	Level-3   &d3\\
	Level-4   &d4\\
	Level-5   &d6\\
	Level-6   &d8\\
	Level-7   &2d6\\
	Level-8   &2d10\\
	Level-9   &3d10\\
	Level-10  &7d6\\
	Level-11  &3d20\\
	Level-12  &7d12\\
	Level-13  &d100\\
	Level-14  &6d20\\
	Level-15  &25d6\\
	Level-16  &10d20\\
	Level-17  &40d6\\
	Level-18  &60d6\\
	Level-19  &80d6\\
	\end{tabular}
	\end{small}
\descfeat{Elemental Aura [Celestial], [Fiend]}
{Your close relationship with primal elemental forces has manifested in a damaging aura.}
	\shortability{Prerequisites:}{Character level 7, must have a subtype granting immunity to a form of elemental damage.}
	\shortability{Benefits:}{Choose one of your elemental subtypes granting immunity to a form of elemental damage. You radiate a damaging aura that does 4d6 of elemental damage of that type to any creature within 10' of you at the beginning of your turn.}
\descfeat{Essence Gourmand [Fiend]}
{Even among soul-eating fiends, you are a accomplished eater.}
	\shortability{Prerequisite:}{Must have one level of a Fiend class, Devour the Soul.}
	\shortability{Benefit:}{Whenever you devour a soul, you gain knowledge of your victim's personal history and important memories (not skills, levels, feats, etc), in addition to the normal effects. You also may cure any one status effect.}
\descfeat{Extra Arms [Celestial], [Fiend]}
{You have more arms than normal.}
	\shortability{Prerequisite:}{Character level 6 (per extra pair).}
	\shortability{Benefit:}{You have two extra humanoid arms. Each arm has your full strength and dexterity.}
	\shortability{Special:}{You may take this feat more than once, its effects stack. You must have a minimum of 6 levels for each iteration of this feat (so a 12th level character may have 2 sets of extra arms).}
\descfeat{Extra Summons [Celestial], [Fiend]}
{You may use your Summoning ability two extra times each day}
	\shortability{Prerequisite:}{Must have the signature \spell{summon} ability of the great Angelic or Fiendish Houses.}
	\shortability{Benefit:}{Your \spell{summon} ability may be used two extra times each day (the ability is normally usable once each day, so it could be used for 3 separate chances to conjure an ally.)}
\descfeat{Fiend Cabalist}
{You were trained in the mystic arts by a powerful fiend, and your magical power stems from a dark source.}
	\shortability{Prerequisite:}{Caster level 1}
	\shortability{Benefit:}{You gain Knowledge(planes) as a class skill, and all of your spells gain the [evil] subtype, and spells you cast and magic items you create radiate an evil aura equal to the strength of their normal magical aura. Any spells you cast that already have the Evil subtype gain a +4 caster level.}
	\shortability{Special:}{This can only be taken at 1st level.}
\descfeat{Fiendish Invisibility [Fiend]}
{You cannot be seen.}
	\shortability{Prerequisite:}{Character level 6}
	\shortability{Benefit:}{You are naturally invisible, as with the spell improved invisibility.}
	\shortability{Special:}{Fiendish Invisibility always has a flaw, something that will allow your character to be seen. Examples include:} \vspace*{-\baselineskip}
	\listone
	    \item\ability{Invisible in Light:}{If you are ever in shadowy illumination, you are visible.}
	    \item\ability{Visible by Breath:}{You are only invisible if you hold your breath for 3 rounds first. When you next exhale, you become visible again.}
	    \item\ability{Invisible on Stone:}{Your character is invisible when touching the ground. While standing on worked floors or flying, you can be seen.}
	\end{list}\vspace{\baselineskip}
\descfeat{Greater Teleport [Celestial], [Monstrous]}
{The extraplanar blood running through your veins allows you to use the signature travel methods of the outer planes.}
	\shortability{Prerequisite:}{Outsider, character level 5+}
	\shortability{Benefit:}{You may use \spell{greater teleport} at will as a spell-like ability. You may only transport yourself and 50 pounds of carried items.}
\descfeat{Harmless Form [Celestial], [Fiend]}
{You can assume the likeness of a mortal.}
	\shortability{Prerequisites:}{Character level 4}
	\shortability{Benefits:}{You can Change Shape into a medium-sized Humanoid appearance. You can use this ability to Disguise yourself as other people, and it gives a +10 to Disguise checks as normal. When using this ability, your reflection in mirrors is of your true form.}
\descfeat{Heighten Spell-like Ability [Celestial], [Fiend]}
{You can treat your spell-like abilities as more powerful spell effects.}
	\shortability{Benefit:}{When you use a spell-like ability, you may use it as if it were of a higher than normal spell level. You may not raise a spell-like ability to in this fashion to an effective spell-level higher than half your character level.}
\descfeat{Hellscarred}
{Having had your mind or body twisted by the essence of a fiend, you have gained some sensitivity and immunity to their power.}
	\shortability{Prerequisites:}{Must have failed a saving throw to a spell or effect associated with a fiend, and cannot be a fiend or have any feats with the [Fiend] subtype.}
	\shortability{Benefits:}{If you ever fail a save to a Special attack, Special Quality, or spell of a fiend, you may a reroll that save (once per save). This ability may be used a number of times per day equal to your Charisma modifier (minimum 1).
	In addition, you may cast detect fiends as a spell like ability at will. This spell functions as enlarged detect magic (120 foot range), but it only detects the presence of fiends and their magical effects. This effect also cannot determine school of magic, but instead will indicate the race of the fiend (baazetu, tanar'ri, yugoloth, or demondand subtype if appropriate).}
\descfeat{Huge Size [Celestial], [Fiend]}
{Your size increases to Huge.}
	\shortability{Prerequisites:}{Character level 10}
	\shortability{Benefits:}{If your size would normally be Large without this feat, it increases to Huge (with all the usual changes).}
\descfeat{Large Size [Celestial], [Fiend]}
{Your size increases to Large.}
	\shortability{Prerequisites:}{Character level 5.}
	\shortability{Benefits:}{If your size would normally be Medium without this feat, it increases to Large (with all the usual changes).}
\descfeat{Memories of Death}
{You retain your memories perfectly after you are slain and brought back from the dead.}
	\shortability{Prerequisite:}{Must be a native to the Prime Material Plane}
	\shortability{Benefit:}{When you die and are returned back from the dead by any means, you do not lose a level, any XP, or Constitution. Any other penalties associated with returning to life (such as being exhausted or waking up in a new body) are unchanged. Note that this means that you have flawless intelligence as to the alignment of whoever brought you back from the dead.}
\descfeat{Pincers [Fiend]}
{Two of your hands are converted into pincers.}
	\shortability{Benefit:}{Each Pincer is a natural weapon, and attacks made with the Pincer are considered to have the Improved Grab ability.}
\descfeat{Poison Sacs [Fiend]}
{One of your natural weapons is envenomed.}
	\shortability{Prerequisite:}{Must have one level of a Fiend class.}
	\shortability{Benefit:}{When you gain this ability choose any poison in Dungeons and Dragons with a DC equal or less to the DC your poison would have(Half HD + Con mod). You poison does ability damage or special effects equal to the poison chosen. Once chosen, your poison type does not change, but your poison DC will increase when your HD or Con modifier increase.}
\descfeat{Product of Celestial Dalliance}
{One of your recent ancestors was a Celestial Outsider or from a good-aligned plane. Maybe your parents play it off as a virgin birth, maybe your dad became a Saint.}
	\shortability{Benefits:}{You may take any [Celestial] feat. Additionally, you gain Resistance 5 to Acid, Cold, and  Electricity; the [Angel], [Archon], [Eladrin], or [Guardinal] subtype; and a Smite Evil attack usable at will that does bonus damage equal to \half of your strength modifier.}
	\shortability{Special:}{Can only be taken at 1st level.}
\descfeat{Product of Infernal Dalliance}
{One of your recent ancestors mated with an infernal creature, and now the tainted blood of a Lower Planar creature flows in your veins. Though you can resist the call of your evil heritage, it manifests itself in an inheritance of fiendish power.}
	\shortability{Benefits:}{You may take any feat with the [Fiend] subtype. In addition, you radiate faint evil, have either two claws or one bite natural weapon, and have Cold Resistance 5 or Fire Resistance 5. When this feat is gained, you also gain the [Baazetu], [Tanar'ri], [Yugoloth], or [Demondand] subtype.}
	\shortability{Special:}{Can only be taken at 1st level.}
\descfeat{Slime Trail [Fiend]}
{Your body secretes a slick mucus that dries quickly in contact with air, but you've learned to use this to your advantage.}
	\shortability{Prerequisites:}{Character level 2.}
	\shortability{Benefits:}{Your square counts as if the spell grease has been cast in it, and this effect ends when you leave a square and renews itself at the end of your turn. You are immune to this grease effect. You also gain a +4 bonus any checks to escape a Grapple.}
\descfeat{Sphere Focus [Monstrous]}{You can draw on the power of a specific Sphere more easily.}
	\shortability{Prerequisite:}{Access to at least one Sphere}
	\shortability{Benefit:}{Select a Sphere that you know.  The DC of any saving throw against spell-like abilities from that Sphere increases by 1.}
	\shortability{Special:}{You may select this feat multiple times.  Its effects do not stack.  Each time you take this feat, it applies to a different Sphere.}
\descfeat{Spines of Fury [Fiend]}
{Spines cover your body, and you may fire these spine at your enemies.}
	\shortability{Prerequisites:}{Character level 3.}
	\shortability{Benefits:}{You may fire up to two of your body's protruding spines per round as a standard action. You are proficient in these spines, and they have the same game effects as daggers. You may also remove them and use them as daggers, and they count as your natural weapons for purposes of damage reduction and spell effects. You body has a number of spines equal to twice your character level, and regenerate these amounts after one day of rest.}
\descfeat{Sting of the Scorpion [Fiend]}
{You have a viciously barbed tail that carries a lethal poison.}
	\shortability{Benefit:}{You have a stinger as a natural weapon that carries a poison that inflicts initial and secondary damage of 1d6 Con. The save DC is Constitution based. You may only inject a number of doses of poison per day equal to your Con bonus.}

\featname{Stolen Breath [Fiend]}
	\shortability{Prerequisites:}{Character level 3.}
	\shortability{Benefits:}{On a successful grapple check, your opponent may not speak or breathe for one round in addition to any normal effects of a successful Grapple check.}
\descfeat{Stoning Gaze [Fiend]}
{Your gaze petrifies the living and leaves them as statue to decorate your domain as a warning to others.}
	\shortability{Prerequisite:}{Character level 9}
	\shortability{Benefit:}{Once per round as a Free Action, you must designate one living creature within 60 feet of you. If that creature meets your gaze before your next turn, it must make a Fortitude save or be permanently transformed into Stone as by a stone to flesh spell. The save DC is Charisma based. The effects of this feat are a Supernatural Ability.}

\descfeat{Supernatural Virulence [Fiend]}
{Your poison is as much magical as it is biological.}
	\shortability{Prerequisite:}{Must have a poisonous natural weapon.}
	\shortability{Benefit:}{Choose one of your spell-like abilities of 3rd level or lower. Any time you successfully poison a victim, they are also targeted by this spell-like ability as if this effect was cast (this expends one use of the ability). While poisoned with your venom, the victim cannot be affected by your spell-like ability again.}
\descfeat{Wings of Evil [Fiend]}
{You have sinister bat-like wings growing from your back.}
	\shortability{Prerequisite:}{Character level 5.}
	\shortability{Benefit:}{You have a fly speed double that of your normal ground speed. You have good maneuverability, and you must be able to flap your wings to stay aloft (meaning that it requires very specialized armor or cloaks to permit flight).}
	\shortability{Special:}{If you would prefer to have insectile wings or feathered wings instead, you can do that. The maneuverability and speed are unchanged. Once the look and feel of the wings is selected it cannot be changed.}
\featname{Wings of Good [Celestial]}
	\shortability{Benefits:}{You gain wings and a fly speed equal to double your base land speed with good maneuverability. You must be able to flap your wings to stay aloft (meaning that it requires very specialized armor or cloaks to permit flight). These must be feathery or energy-based.}

	\end{multicols}

\section{[Elemental] Feats} \label{feats:elemental}

A feat with the [Elemental] tag can only be taken by a creature who is from an inner plane or any Elemental or Outsider with an elemental subtype. If the feat has another similar tag (such as [Celestial], [Fey], or [Fiend]), a creature who fulfills the criteria for the other tag may waive this requirement. The abilities granted by feats with the [Elemental] tag are Extraordinary abilities unless otherwise stated.

	\begin{multicols}{2}

		\descfeat{Abode of Earth [Elemental]}{You are at home within the earth.}
\shortability{Prerequisites:}{(Earth) subtype, Burrow speed, Character level 3+}
\shortability{Benefits:}{Your Burrow speed improves by 10' or to a minimum of 30', and you may burrow through rock.  You may leave a tunnel or leave the earth behind you undisturbed, as you choose.  If you leave the earth undisturbed, there is no sign of your passage unless you are in a square adjacent to a surface, except to creatures with Tremorsense or who make the Perception check to hear you.  The Perception check is not made more difficult by the earth you are in, just by distance through it.  Other rocks and earthen walls do interfere as normal.}



\descfeat{Adept Flyer [Elemental]}{You are a natural flyer.}
\shortability{Prerequisites:}{(Air) subtype, Fly speed, Character level 5+}
\shortability{Benefits:}{Your Fly speed improves to twice your base land speed (minimum 60'), or +10', whichever is more, and your maneuverability improves to Perfect.  Your Fly speed improves by 20' for every five character levels you gain beyond 5th.}



\descfeat{Binding Growth [Elemental]}{You grow on people.}
\shortability{Prerequisite:}{Wood Elemental Creature (e.g., Psuedoelemental Being (Wood) feat)}
\shortability{Benefit:}{After pinning or lifting a creature for a round, you may attempt to grow a Binding Growth on them with another grapple check against a DC of 10 + their Grapple modifier.  Once you do so, they are bound, losing their Dexterity bonus to AC and their ability to take physical actions other than try to escape, until they break the bonds.  The bonds can be broken by others with a slashing melee weapon capable of doing 5 + your hit dice points of damage against AC 5+your Constitution modifier, but a miss hurts the bound creature, or by a Strength check (DC 15 + your Constitution modifier) or Escape Artist check (DC 10 + your hit dice + your Constitution modifier).  Even once broken, they remain on for 1d4 rounds, entangling the bound creature.}



\descfeat{Ice Trail [Elemental]}{You leave a trail of ice wherever you go.}
\shortability{Prerequisite:}{Character Level 3+, (Cold) Subtype}
\shortability{Benefit:}{Your square counts as if it had the \emph{Grease} spell cast on it, except that the slick is made of ice and has the (Cold) descriptor.  Any square you leave has this effect on it, lasting until the end of your next turn.  You never slip on ice, making you immune to this effect.}



\descfeat{Infusion of Elemental Essence}{You have been infused with the power of one of the elemental planes, granting you an affinity for that element and a small degree of magical power.}
\shortability{Benefit:}{You may take any feat with the [Elemental] subtype that you qualify for; additionally, choose an elemental subtype (Air, Earth, Fire, or Water), and you may take [Elemental] feats as though you had that subtype. You also gain Resistance 10 to an energy type dependent on your element:

\begin{list}{}{\itemspace}
    \item Air or Earth: Acid or Electricity
    \item Fire: Fire
    \item Water: Acid or Cold
\end{list}
You may select this feat only once.}



\descfeat{Large Size [Celestial], [Fiend]}
{Your size increases to Large.}
	\shortability{Prerequisites:}{Character level 5.}
	\shortability{Benefits:}{If your size would normally be Medium without this feat, it increases to Large (with all the usual changes).}
\descfeat{Primal Armor [Elemental]}{Your body deflects blows off of itself.}
\shortability{Benefit:}{You gain impenetrable Damage Reduction equal to half your character level, rounded up (1/- at first level, 2/- at 3rd, 5/- at 9th, 10/- at 19th).}



\descfeat{Primal Fortification [Elemental], [Racial]}{Your body has become even more impenetrable.}
\shortability{Prerequisite:}{Elemental-Bodied, Hardiness of the Elements}
\shortability{Benefit:}{You gain immunity to Critical Hits.  You also cannot be flanked, as your undiffentiated body has no clear front or back.}



\descfeat{Psuedoelemental Being [Racial]}{You are a psuedoelemental being, with rare and unique powers.}
\shortability{Prerequisite:}{Elemental-Bodied.}
\shortability{Benefits:}{Instead of picking a normal elemental type as an elemental-bodied, select one of the following other planes: Ice, Magma, Shadow, or Wood.  You gain benefits as follows for the type you've picked:
\begin{list}{}{\itemspace}
    \item Ice: You gain the (Cold) subtype, a 30' base land speed, a 30' swim speed, and +2 to Str.  Your melee attacks do 1d6 bonus Cold damage.  You speak Aquan and Auran.
    \item Magma: You gain both the (Earth) and (Fire) subtypes. Your base land speed is 20', and you gain +2 Str. Otherwise you gain the full benefits of both elements.
    \item Shadow: You have a 30' base land speed and a Fly speed of 10', with good maneuverability, and gain +2 Dex. You are \emph{invisible} in any lighting less than bright light. You speak Common. Despite your affiliation to the Plane of Shadow instead of to the Inner Planes, you still qualify for [Elemental] feats.
    \item Wood: You have no elemental subtype, and gain +2 Con and a 10' Climb speed. You gain Regeneration 0, penetrable by Fire and Slashing weapons, which improves to Regeneration equal to your level in areas of natural daylight or equivalent brightness (such as a \emph{daylight} spell). You only gain natural healing if you spend at least 8 hours/day in such brightness. You count as a Plant, in addition to an Elemental, for all effects relating to type. You speak your choice of Sylvan or Treant, and any Elemental language.
\end{list}
Other Dual-element types than Magma, such as Ooze (Water and Earth), Smoke (Air and Fire), Vapor (Water and Air), and so on are possible.}
\shortability{Special:}{This feat can only be taken at 1st level.}



\featname{Stolen Breath [Fiend]}
	\shortability{Prerequisites:}{Character level 3.}
	\shortability{Benefits:}{On a successful grapple check, your opponent may not speak or breathe for one round in addition to any normal effects of a successful Grapple check.}
\descfeat{Tremorsense [Elemental]}{Your close connection to your home element gives you Tremorsense.}
\shortability{Prerequisites:}{(Earth) or (Water) Subtype, Character level 6+}
\shortability{Benefits:}{You gain Tremorsense out to 120'.  You gain Blindsight out to 30' against any creature you can Tremorsense.  If you have the (Water) subtype and not the (Earth) subtype, your Tremorsense works at its full range in liquids, but only to half range and you do not gain Blindsight through solids.}



\descfeat{Touch of Shadow [Elemental]}{Your shadowy touch can bypass armor.}
\shortability{Prerequisite:}{Shadow Elemental Creature (Psuedoelemental Being (Shadow), Shadow Genasi, or similar), Natural weapon, Character Level 3+}
\shortability{Benefit:}{You may choose to make natural weapon attacks as touch attacks.  Such attacks use your Dexterity bonus to hit, instead of Strength, and do not gain Strength to damage.}



\descfeat{Uncanny Flexibility [Elemental], [Racial]}{Your body, being made of a material other than flesh, bends in directions and places that flesh neither can  nor should.}
\shortability{Prerequisite:}{Airbodied, Firebodied, or Waterbodied; or Psuedoelemental Being (Magma or Shadow).}
\shortability{Benefit:}{You gain a +4 bonus to Escape Artist checks, and Escape Artist is always a class skill for you.  You can compress your head to about half area for purposes of slipping through tight spaces, and may attempt to slip manacles, ropes, webs, nets, grapplers, and similar bonds as a free action.}



\descfeat{Unstoppable Force [Elemental], [Racial]}{You cannot be stopped.}
\shortability{Prerequisites:}{Elemental-Bodied, Hardiness of the Elements}
\shortability{Benefit:}{You become immune to paralysis and stunning.}

%\end{multicols}


	\end{multicols}

\section{[Necromantic] Feats} \label{feats:necromantic}

\bolded{Necromantic Creation Feats:}

Any feat with the [Necromantic] tag is a necromantic creation feat. This means that it is merely one part of the dark tradition of necromancy; other means such as necromancy spells or other effects can create these undead, but this an easy path for the serious Necromancer. One trait shared by these feats is that each feat has a separate control pool for the undead it creates. For example, if a necromancer has the Path of Blood feat and the A Feast Unknown feat, he may control up to his unmodified charisma modifier in vampires or vampire spawn in addition to controlling up to his unmodified charisma modifier in ghouls. It is a move action to give commands any one undead creature. Any undead controlled by this feat cannot create undead or use the Spawn Undead ability.

The rituals are inexpensive, but require the flesh and blood of intelligent creatures as well as living creatures or fresh corpses as subjects. Any additional costs or conditions are listed in the individual feat. These rites take 1 hour per CR of the creature created, and can only be performed at night or in a location that has never been touched by the sun (such as a deep cave). The maximum CR of an undead creature created with these rites is two less than the creator's Character level.

Materials to create any undead always cost at least 25 gp per hit die. Creating undead by these method generally requires at least an hour.\\


	\begin{multicols}{2}

		\genfeat{A Feast Unknown}{[Necromantic]}
{You have partaken of the feast most foul and count yourself a king among the ghouls.}
\featprereq{You must have consumed the rapidly cooling flesh of an intelligent mortal creature. Must be evil.}
\featbenefit{You can create Ghouls or Ghasts from any dying person (at -1 to -9 hps). Any undead you create have the Scent special quality. In addition, any time you completely consume the flesh of a sentient creature, you regain 5 hps per HD.

You automatically control up to your Charisma modifier in undead created by this feat, but no undead can have a CR greater than two less your Character level.}


%\descfeat{Blood Painter}
{By painting magical diagrams out of your own blood, you can spontaneously cast spells using only your own life energy. This is especial use to casters who prepare spells, or to casters who have run out of spells.}
\shortability{Prerequisite:}{Path of Blood, Caster level 5, Spellcraft 4 ranks}
\shortability{Benefit:}{At any time, a caster with this feat can cast any spell he knows by painting a magical diagram on a flat 10' by 10' surface. This takes one minute per spell level, and deals two points of Constitution damage per spell level to the caster (or loses a like amount of Blood Pool if he has one). If the caster's current Con or Blood Pool is less than double the spell's level, the spell cannot be cast.\\
Any spells cast with this feat are Supernatural effects.}


%\descfeat{Body Assemblage [Necromantic]}
{The discarded husks of life are nothing more than a building material to you.}
\shortability{Prerequisite:}{Caster Level 1, ability to cast 1st level spells of the Necromancy school.}
\shortability{Benefit:}{You may create skeletons and zombies that serve you alone.  You automatically control up to your unmodified Charisma modifier in undead created by this feat, but no undead can have a CR greater than two less than your Character level.}
\shortability{Special:}{A first or second level character can create undead less than their own CR, but each undead creature counts as two for control purposes.}


%\descfeat{Boneblade Master}
{You have mastered the alchemic processes needed to create boneblades, as well as their use in combat.}
\shortability{Prerequisite:}{Craft(alchemy) 4, Craft 4 (scrimshaw)}
\shortability{Benefit:}{You are considered to be proficient in the use of any weapon made from the special material Boneblade, and you may craft weapons out of Boneblade. In addition, you are considered to have the Improved Critical feat for any boneblade weapons you use in melee combat.\\
You also gain a +2 to Initiative checks.}


%\descfeat{Child Necromancer}
{An obsession with death and experimentation with necromancy early in your childhood perverted your body and blossoming magical talent. As a result, your body never aged past childhood, and you are an adult in a child's body, magically powerful but physically weak.}
\shortability{Prerequisite:}{Caster level 1, must know at least one necromancy spell of each spell level you can cast.}
\shortability{Benefit:}{All Necromancy spells you cast are at +4 caster level, and you gain the effects of \hyperlink{feat:weaponfinesse}{Weapon Finesse} for all Necromancy touch attack spells you use (if you desire). You have -4 Strength, and appear to be a child despite your actual age category (this does not prevent penalties or bonuses from advancing in age categories, or stop the aging process). You are one size category smaller than normal for your race (do not further adjust ability modifiers). If you are a spontaneous caster, you may permanently exchange any spell known for any Necromancy spell you possess in written or scroll form. If you are a preparation caster, you may learn any Necromancy spell you possess in written or scroll form from any list, and you may not select Necromancy as a restricted school. These Necromancy spells may be from any list, can be exchanged at any time, and once gained are cast as spells of your spellcasting class. These spells remain as spells known even if you later lose this feat.}
\shortability{Special:}{This feat can only be taken at 1st level. If circumstances ever cause a character to no longer meet the prerequisites of this feat, they may choose any metamagic feat they qualify for to permanently replace this feat.}


%\descfeat{Devil Preparation}
{By learning dark culinary techniques, you have learned to consume the flesh of devils, demons, and other infernals, absorbing their taint and some of their power.}
\shortability{Prerequisite:}{A Feast Unknown, Character level 10, must have eaten the flesh of a Devil or Demon.}
\shortability{Benefit:}{You gain the ability to cast one spell from the Half-Fiend template per day as a spell-like ability (limited by your HD on the Half-fiend chart). In addition, all spells from the Evil Domain are considered spells known for you, you gain a +2 to Intimidate checks, and you can choose to count as a Tanari or Baazetu for the effects of spells, magic items, or prerequisites for feats or prestige classes.}


%\descfeat{Fairy Eater}
{By consuming the flesh of fairies, you have absorbed a fraction of their magic.}
\shortability{Prerequisite:}{A Feast Unknown, must have eaten the flesh of a creature with the Fey type.}
\shortability{Benefit:}{All figments and glamers you cast have their duration extended by two rounds. In addition, all spells from the Trickery Domain are considered spells known for you, you gain a +4 to Disguise checks, and you can choose to count as a Fey for the effects of spells, magic items, or prerequisites for feats or prestige classes.}


%\descfeat{Feed the Dark Gods [Necromantic]}
{You have attracted the attention of dark gods and demon lords, and they are willing to grant dark life to your creations in exchange for pain and power.}
\shortability{Prerequisite:}{Any two necromantic feats, Character level 7, 10 ranks in Knowledge(Religion)}
\shortability{Benefit:}{You may create any undead creature through the art of sacrifice. For every CR of the creature you wish to create, you must sacrifice one sentient soul (Int of 5 or better) and 500 gp. For example, if you wish to create a CR 8 Slaughterwight, you must sacrifice eight sentients and 4,000 gp. You cannot create any undead with a CR greater than two less than your Character level.\\
You automatically control up to your unmodified Charisma modifier in undead created by this feat, but no undead can have a CR greater than two less than your character level.}


%\descfeat{Ghost Cut Technique}
{Study of the ephemeral essence of incorporeal undead has taught you combat techniques that transcend the limitations of the flesh.}
\shortability{Prerequisite:}{Whispers of the Otherworld}
\shortability{Benefit:}{Each day, you can use the spell \spell{wraithstrike} as swift action spell-like ability a number of time equal to half your character level.\\
You also gain a +2 to initiative checks, a +4 to Move Silently checks, and Lifesight as a Special Quality.}


%\descfeat{Heavenly Desserts}
{By gorging on the sweet flesh of angels, you have digested a portion of their divine essence.}
\shortability{Prerequisite:}{A Feast Unknown, Character level 10, must have eaten the flesh of an Angel, Archon, Eladrin, or Deva.}
\shortability{Benefit:}{You gain the ability to cast one spell from the Half-Celestial template per day as a spell-like ability (limited by your HD on the Half-Celestial chart). In addition, all spells from the Gluttony Domain are considered spells known for you, you gain a +2 to Diplomacy checks, and you can choose to count as Good for the effects of spells or magic items.}


%\descfeat{Sleep of the Ages}
{Your mastery of ancient mummification techniques has revealed a secret technique for sleeping away the ages.}
\shortability{Prerequisite:}{Character level 8, Wrappings of the Ages, you must remove all of your internal organs and place them within canoptic jars during a magic ritual}
\shortability{Benefit:}{By arranging focuses worth 1,000 GP in a ritual manner and wrapping yourself in the funeral arrangements of a mummy, you can initiate the Sleep of the Ages. Until your focuses are disturbed, you will stay in suspended animation. In this state, you do not age, breath, need to eat, or are subject to any effect requiring a Fort Save.\\
As a side effect of learning this technique, you remove all of your internal organs and place them within canoptic jars during a magical ritual. This process does not harm you, and from this point onward you are no longer subject to critical hits or sneak attacks. Having your organs in canoptic jars has no other game effect, but if they are destroyed you no longer gain the effects of this feat (your organs magically return to your body and you must remove them again to regain the use of this feat.)}





\end{multicols}

\section{[Undead] Feats}

The powers of the undead are legendary, in part because they are so varied. A feat with the [Undead] tag can only be selected or used by a character who is undead.\\

\begin{multicols}{2}


%\descfeat{The Path of Blood [Necromantic]}
{You have learned the dark and selfish rites that create vampires, the legendary immortal blood drinkers of the night.}
\shortability{Prerequisite:}{Character level 5}
\shortability{Benefit:}{You can create Vampires and Vampire Spawn. Your unintelligent undead heal fully at the next sunset following them killing a living creature with a piercing or slashing attack. A spellcaster with this feat has access to any spell with a [blood] component. You automatically control up to your unmodified Charisma modifier in undead created by this feat, but no undead can have a CR greater than two less than your Character level.}


%\descfeat{Whispers of the Otherworld[Necromantic]}
{You have learned the tricks of torturing a soul past the veil of life, and into the shadow of death.}
\shortability{Prerequisite:}{Character level 4}
\shortability{Benefit:}{You may create incorporeal undead. In addition, any undead you create have a +2 to Initiative, +4 to Move Silently checks, and Lifesight as a Special Quality.\\
You automatically control up to your Charisma modifier in undead created by this feat, but no undead can have a CR greater than two less than your character level.}


%\descfeat{Wrappings of the Ages [Necromantic]}
{The ancient secrets by which unlife can be sustained in mummification have been unearthed.}
\shortability{Prerequisite:}{Character level 8}
\shortability{Benefit:}{You can create mummies. In addition, any undead you create has their natural armor increase by +3. Also, any time your undead rest (take no actions) in an enclosed space that has never been touched by the sun, the location counts as a Tomb for them as long as they inhabit it (see New Rules). In all other ways, the area is not a Tomb.\\
You automatically control up to your Charisma modifier in undead created by this feat, but no undead can have a CR greater than two less than your character level.}



%\descfeat{Enervating Touch [Undead]}
{Your undead nature allows you to drain the life out of living victims.}
\shortability{Benefit:}{Your unarmed attacks and natural weapons inflict one negative level. The DC to remove that negative level is Charisma based. You gain 5 temporary hit points every time you inflict a negative level on an intelligent creature in this way.}


%\descfeat{Control Spawn [Undead]}
{Your victims serve you eternally in death.}
\shortability{Prerequisite:}{Enervating Touch}
\shortability{Benefit:}{When a creature dies from the negative levels you inflict and rises as a Wight, it comes under your control. At any one time, you may control a number of Wights in this manner equal to your Charisma modifier (minimum of one). If you create additional Wights, you choose which spawn you lose control over.}


%\descfeat{Paralyzing Touch [Undead]}
{The touch of your clawed hand freezes the lifeblood of the hardiest of mortals.}
\shortability{Prerequisite:}{Ghoul}
\shortability{Benefit:}{Your unarmed strikes and natural attacks cause paralysis for one minute unless your victim makes a Fortitude save. This save is Charisma based.}

\end{multicols}



	\end{multicols}

\section{[Undead] Feats}

The powers of the undead are legendary, in part because they are so varied. A feat with the [Undead] tag can only be selected or used by a character who is undead.\\

	\begin{multicols}{2}

	\genfeat{A Feast Unknown}{[Necromantic]}
{You have partaken of the feast most foul and count yourself a king among the ghouls.}
\featprereq{You must have consumed the rapidly cooling flesh of an intelligent mortal creature. Must be evil.}
\featbenefit{You can create Ghouls or Ghasts from any dying person (at -1 to -9 hps). Any undead you create have the Scent special quality. In addition, any time you completely consume the flesh of a sentient creature, you regain 5 hps per HD.

You automatically control up to your Charisma modifier in undead created by this feat, but no undead can have a CR greater than two less your Character level.}


%\descfeat{Blood Painter}
{By painting magical diagrams out of your own blood, you can spontaneously cast spells using only your own life energy. This is especial use to casters who prepare spells, or to casters who have run out of spells.}
\shortability{Prerequisite:}{Path of Blood, Caster level 5, Spellcraft 4 ranks}
\shortability{Benefit:}{At any time, a caster with this feat can cast any spell he knows by painting a magical diagram on a flat 10' by 10' surface. This takes one minute per spell level, and deals two points of Constitution damage per spell level to the caster (or loses a like amount of Blood Pool if he has one). If the caster's current Con or Blood Pool is less than double the spell's level, the spell cannot be cast.\\
Any spells cast with this feat are Supernatural effects.}


%\descfeat{Body Assemblage [Necromantic]}
{The discarded husks of life are nothing more than a building material to you.}
\shortability{Prerequisite:}{Caster Level 1, ability to cast 1st level spells of the Necromancy school.}
\shortability{Benefit:}{You may create skeletons and zombies that serve you alone.  You automatically control up to your unmodified Charisma modifier in undead created by this feat, but no undead can have a CR greater than two less than your Character level.}
\shortability{Special:}{A first or second level character can create undead less than their own CR, but each undead creature counts as two for control purposes.}


%\descfeat{Boneblade Master}
{You have mastered the alchemic processes needed to create boneblades, as well as their use in combat.}
\shortability{Prerequisite:}{Craft(alchemy) 4, Craft 4 (scrimshaw)}
\shortability{Benefit:}{You are considered to be proficient in the use of any weapon made from the special material Boneblade, and you may craft weapons out of Boneblade. In addition, you are considered to have the Improved Critical feat for any boneblade weapons you use in melee combat.\\
You also gain a +2 to Initiative checks.}


%\descfeat{Child Necromancer}
{An obsession with death and experimentation with necromancy early in your childhood perverted your body and blossoming magical talent. As a result, your body never aged past childhood, and you are an adult in a child's body, magically powerful but physically weak.}
\shortability{Prerequisite:}{Caster level 1, must know at least one necromancy spell of each spell level you can cast.}
\shortability{Benefit:}{All Necromancy spells you cast are at +4 caster level, and you gain the effects of \hyperlink{feat:weaponfinesse}{Weapon Finesse} for all Necromancy touch attack spells you use (if you desire). You have -4 Strength, and appear to be a child despite your actual age category (this does not prevent penalties or bonuses from advancing in age categories, or stop the aging process). You are one size category smaller than normal for your race (do not further adjust ability modifiers). If you are a spontaneous caster, you may permanently exchange any spell known for any Necromancy spell you possess in written or scroll form. If you are a preparation caster, you may learn any Necromancy spell you possess in written or scroll form from any list, and you may not select Necromancy as a restricted school. These Necromancy spells may be from any list, can be exchanged at any time, and once gained are cast as spells of your spellcasting class. These spells remain as spells known even if you later lose this feat.}
\shortability{Special:}{This feat can only be taken at 1st level. If circumstances ever cause a character to no longer meet the prerequisites of this feat, they may choose any metamagic feat they qualify for to permanently replace this feat.}


%\descfeat{Devil Preparation}
{By learning dark culinary techniques, you have learned to consume the flesh of devils, demons, and other infernals, absorbing their taint and some of their power.}
\shortability{Prerequisite:}{A Feast Unknown, Character level 10, must have eaten the flesh of a Devil or Demon.}
\shortability{Benefit:}{You gain the ability to cast one spell from the Half-Fiend template per day as a spell-like ability (limited by your HD on the Half-fiend chart). In addition, all spells from the Evil Domain are considered spells known for you, you gain a +2 to Intimidate checks, and you can choose to count as a Tanari or Baazetu for the effects of spells, magic items, or prerequisites for feats or prestige classes.}


%\descfeat{Fairy Eater}
{By consuming the flesh of fairies, you have absorbed a fraction of their magic.}
\shortability{Prerequisite:}{A Feast Unknown, must have eaten the flesh of a creature with the Fey type.}
\shortability{Benefit:}{All figments and glamers you cast have their duration extended by two rounds. In addition, all spells from the Trickery Domain are considered spells known for you, you gain a +4 to Disguise checks, and you can choose to count as a Fey for the effects of spells, magic items, or prerequisites for feats or prestige classes.}


%\descfeat{Feed the Dark Gods [Necromantic]}
{You have attracted the attention of dark gods and demon lords, and they are willing to grant dark life to your creations in exchange for pain and power.}
\shortability{Prerequisite:}{Any two necromantic feats, Character level 7, 10 ranks in Knowledge(Religion)}
\shortability{Benefit:}{You may create any undead creature through the art of sacrifice. For every CR of the creature you wish to create, you must sacrifice one sentient soul (Int of 5 or better) and 500 gp. For example, if you wish to create a CR 8 Slaughterwight, you must sacrifice eight sentients and 4,000 gp. You cannot create any undead with a CR greater than two less than your Character level.\\
You automatically control up to your unmodified Charisma modifier in undead created by this feat, but no undead can have a CR greater than two less than your character level.}


%\descfeat{Ghost Cut Technique}
{Study of the ephemeral essence of incorporeal undead has taught you combat techniques that transcend the limitations of the flesh.}
\shortability{Prerequisite:}{Whispers of the Otherworld}
\shortability{Benefit:}{Each day, you can use the spell \spell{wraithstrike} as swift action spell-like ability a number of time equal to half your character level.\\
You also gain a +2 to initiative checks, a +4 to Move Silently checks, and Lifesight as a Special Quality.}


%\descfeat{Heavenly Desserts}
{By gorging on the sweet flesh of angels, you have digested a portion of their divine essence.}
\shortability{Prerequisite:}{A Feast Unknown, Character level 10, must have eaten the flesh of an Angel, Archon, Eladrin, or Deva.}
\shortability{Benefit:}{You gain the ability to cast one spell from the Half-Celestial template per day as a spell-like ability (limited by your HD on the Half-Celestial chart). In addition, all spells from the Gluttony Domain are considered spells known for you, you gain a +2 to Diplomacy checks, and you can choose to count as Good for the effects of spells or magic items.}


%\descfeat{Sleep of the Ages}
{Your mastery of ancient mummification techniques has revealed a secret technique for sleeping away the ages.}
\shortability{Prerequisite:}{Character level 8, Wrappings of the Ages, you must remove all of your internal organs and place them within canoptic jars during a magic ritual}
\shortability{Benefit:}{By arranging focuses worth 1,000 GP in a ritual manner and wrapping yourself in the funeral arrangements of a mummy, you can initiate the Sleep of the Ages. Until your focuses are disturbed, you will stay in suspended animation. In this state, you do not age, breath, need to eat, or are subject to any effect requiring a Fort Save.\\
As a side effect of learning this technique, you remove all of your internal organs and place them within canoptic jars during a magical ritual. This process does not harm you, and from this point onward you are no longer subject to critical hits or sneak attacks. Having your organs in canoptic jars has no other game effect, but if they are destroyed you no longer gain the effects of this feat (your organs magically return to your body and you must remove them again to regain the use of this feat.)}





\end{multicols}

\section{[Undead] Feats}

The powers of the undead are legendary, in part because they are so varied. A feat with the [Undead] tag can only be selected or used by a character who is undead.\\

\begin{multicols}{2}


%\descfeat{The Path of Blood [Necromantic]}
{You have learned the dark and selfish rites that create vampires, the legendary immortal blood drinkers of the night.}
\shortability{Prerequisite:}{Character level 5}
\shortability{Benefit:}{You can create Vampires and Vampire Spawn. Your unintelligent undead heal fully at the next sunset following them killing a living creature with a piercing or slashing attack. A spellcaster with this feat has access to any spell with a [blood] component. You automatically control up to your unmodified Charisma modifier in undead created by this feat, but no undead can have a CR greater than two less than your Character level.}


%\descfeat{Whispers of the Otherworld[Necromantic]}
{You have learned the tricks of torturing a soul past the veil of life, and into the shadow of death.}
\shortability{Prerequisite:}{Character level 4}
\shortability{Benefit:}{You may create incorporeal undead. In addition, any undead you create have a +2 to Initiative, +4 to Move Silently checks, and Lifesight as a Special Quality.\\
You automatically control up to your Charisma modifier in undead created by this feat, but no undead can have a CR greater than two less than your character level.}


%\descfeat{Wrappings of the Ages [Necromantic]}
{The ancient secrets by which unlife can be sustained in mummification have been unearthed.}
\shortability{Prerequisite:}{Character level 8}
\shortability{Benefit:}{You can create mummies. In addition, any undead you create has their natural armor increase by +3. Also, any time your undead rest (take no actions) in an enclosed space that has never been touched by the sun, the location counts as a Tomb for them as long as they inhabit it (see New Rules). In all other ways, the area is not a Tomb.\\
You automatically control up to your Charisma modifier in undead created by this feat, but no undead can have a CR greater than two less than your character level.}



%\descfeat{Enervating Touch [Undead]}
{Your undead nature allows you to drain the life out of living victims.}
\shortability{Benefit:}{Your unarmed attacks and natural weapons inflict one negative level. The DC to remove that negative level is Charisma based. You gain 5 temporary hit points every time you inflict a negative level on an intelligent creature in this way.}


%\descfeat{Control Spawn [Undead]}
{Your victims serve you eternally in death.}
\shortability{Prerequisite:}{Enervating Touch}
\shortability{Benefit:}{When a creature dies from the negative levels you inflict and rises as a Wight, it comes under your control. At any one time, you may control a number of Wights in this manner equal to your Charisma modifier (minimum of one). If you create additional Wights, you choose which spawn you lose control over.}


%\descfeat{Paralyzing Touch [Undead]}
{The touch of your clawed hand freezes the lifeblood of the hardiest of mortals.}
\shortability{Prerequisite:}{Ghoul}
\shortability{Benefit:}{Your unarmed strikes and natural attacks cause paralysis for one minute unless your victim makes a Fortitude save. This save is Charisma based.}

\end{multicols}



	\end{multicols}