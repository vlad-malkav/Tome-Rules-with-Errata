
\section{Unusual Races}

For a long time, there has been a definite pro-prettiness bias in the rules of D\&D. That is, elves (who are pretty) get a much better deal as player characters than do hobgoblins (who are ugly). This dates back to when races had mandatory alignments and people wanted to discourage Evil player characters from coming in and ruining games (which let's face it, a lot of evil PCs do). And while this has had the desired effect of keeping the number of orc player characters down and their impact minimal, it hasn't been good for game balance at all. Some people really want to be a gray skinned dude with shark's teeth, and they'll play whatever game mechanics are given to them. These players will be playing at the same level as other characters, and that means that they should be playing at the same power level! Really, all the unusual races are optional, so there's no purpose served in screwing them over. In the past, many races have simply been given insufficient goodies to be worth playing (Half-Orcs), or were given good enough abilities but then over-charged in levels for them so horribly as to make the character unplayable (Hobgoblins). We don't hold with that at all. If you don't want someone to play an ogre or goblin in your game, just don't let them play one. It's seriously not even a deal.

Furthermore, for some reason there has been a massive fear of giving straight statistic enhancements to characters without a level adjustment. I don't even understand that, because Halflings already get all that and more. Really, a character who gets +2 to two attributes and a total of +4 to skills and darkvision isn't even impressive compared to a Deep Halfling, so the ginormous fear that people have of letting Hobgoblins and Aasimar into games is perplexing. That being said, what follows are write-ups for the following races playable as normal starting characters in a 1st level game:

\subsection{Aasimar}
\vspace*{-8pt}
\quot{``My ancestors were more beautiful than you can imagine."}

Aasimar get a short stick from just about everyone. They get screwed as PCs by the Level Adjustment rules, and they get no respect from players. Frankly, Celestials just don't have a lot of dramatic tension most of the time. Sure you can have the occasional ``Legacy" scenario where you couldn't possibly live up to your awesome ancestors, but generally when it's important that someone has powerful family members it's so that you can introduce evil family members, not additional heroes.

So here's the deal: Aasimar are the great grandchildren of beautiful outsiders. They aren't just for being dudes with Devas as ancestors, the same game stats represent characters who come from Erinyes or Marilith stock.

\listone
    \item Medium Size
    \item 30' movement.
    \item Outsider Type (Native and Human subtype)
    \item Darkvision
    \item +2 Charisma, +2 Wisdom
    \item Aasimar with a Charisma of at least 10 may cast light as a spell-like ability with a caster level equal to their character level once per day.
    \item +2 bonus to Spot, and Listen checks.
    \item Favored Classes: Paladin and Sorcerer
    \item Automatic Languages: Common
    \item Bonus Languages: Abyssal, Aquan, Auran, Celestial, Formian, Ignan, Slaad, Sylvan, Terran.
\end{list}



\subsection{Drow}
\vspace*{-8pt}
\quot{``Time to die for the Spider Queen."}

The Drow are perhaps the most overused bunch of villains ever. Their entire ability set is one that is supposed to neutralize the advantages of player characters so that characters can have mirror matches against NPC parties without doubling their treasure. With magic items that turn off once they are brought out of Drow controlled regions, spell-resistance, and spell-like abilities designed to specifically negate common player-character tactical advantages, they can easily compete with Player Characters with massively more permanent magical equipment. And that means that they can be fought and killed several times without supercharging party treasure.

But if you want to play a Drow character, you don't want any of that crap. In fact, if you want a Drow character, probably the maxim you are looking for is ``WWDD?" and the answer is probably ``Fight with two scimitars." But more than that, there are a number of abilities that Drow characters in stories exhibit that people want. And then there are the game mechanical abilities in the rulebook that the characters in stories obviously don't have (like Touch of Fatigue, what's up with that?) So here it is, the LA +0 Drow that people actually want to play:

\listone
    \item Medium Size
    \item 30' movement.
    \item Humanoid Type (Elf subtype)
    \item Darkvision 120'
    \item +2 Dexterity, -2 Constitution
    \item Daylight Sensitivity: While in brightly lit surroundings (such as a daylight spell), a Drow suffers a -2 penalty to attack rolls and precision-based skill checks.
    \item Drow with a Charisma of at least 10 may cast deeper darkness (duration 4 hours), and fairie fire as spell-like abilities with a caster level equal to their character level once per day each.
    \item +2 bonus to saving throws against spells and spell-like abilities.
    \item +2 bonus to Spot, and Listen checks.
    \item Drow never sleep and are immune to sleep effects. Drow must still perform their 4 hour daily trance to stay coherent and rested.
    \item Drow live an exceedingly interesting life and every Drow has proficiency with the rapier and an exotic ranged weapon of their choice.
    \item Favored Classes: Cleric and Wizard
    \item Automatic Languages: Elvish
    \item Bonus Languages: Abyssal, Beholder, Common, Draconic, Drow Sign Language, Dwarvish, Gnome, Kuo-Toa, Terran, Undercommon.
\end{list}


\subsection{Feytouched}
\vspace*{-8pt}
\quot{``All my life, I have never fit in. Not in town, not in the forest. In some integral fashion I am unlike those around me, and I believe it is my fate to live and die alone."}

WotC has made no secret of the fact that they like doing bad things to the bad touched races. But if there was one Bad Touched race that is almost worth the level allowance, it would be the Feytouched. Every one of them looks completely different, but they all have the same power set: spell-like charm, and of course an Immunity to Mind Affecting effects. That goes a long way to being worth something, though of course there is no way you can survive as a character with a Level Allowance and a Constitution penalty -- the very idea is absurd. So of course, the race has to be overhauled, because that just isn't reasonable.

Now if you're one of the people who wonders why a product of fairies and humans, who both conspicuously lack an immunity to mind affecting magic, would have an immunity to mind affecting magic -- you aren't alone. That question comes up about as often as any other with regards to the Feytouched. Of course, not all of those born to fey and human stock are immune to mind affecting magic, as you might expect from a group so diverse that some have bug parts and others are simply beautiful humans, while still others look like crazy rock men with teeth sticking out all kinds of places, the powers that a fey-touched is born with are extremely random. The powers of the fairies are more than a little bit chaotic in nature, and no two babes born to these couplings are the same. Unfortunately, these mulish offspring are also interesting both in the general sense and, much more to their detriment, to other fairies in particular. The unmitigated interest of the fey is hard on a small child, so Feytouched who are not immune to compulsion effects are going to find themselves at the bottom of a pond or jumping out of a tall tree long before they reach adulthood. Indeed, Feytouched immune to compulsion effects are the only ones that ever reach maturity -- the well meaning but deadly interest of the fairy family members simply weeds out any other possible results.

That's not an excuse for the package presented in the Fiend Folio as a whole, that's simply unplayable. But it's close. Here's our version:

\listone
    \item Fey Type
    \item 30 foot movement rate
    \item Low-Light Vision
    \item +2 Dexterity, +2 Charisma, -2 Constitution. Feytouched are graceful and those which are not beautiful are terrifying, but they are fragile like flowers.
    \item Immunity to [Compulsion] Effects
    \item Magic Affinity: Every Feytouched is different, and marked by the signature magics of the fey in a different manner. Every Feytouched has one spell that can be used once per day as a spell-like ability. This spell is chosen at 1st level and cannot be changed. Any 1st level Illusion or Enchantment spell from the Sorcerer/Wizard list is fair game, and the save DC is Charisma-based.
    \item Favored Class: Bard
    \item Feytouched speak Common and Sylvan. Bonus Languages may be selected from the following list:
      Aquan, Auran, Elvish, Draconic, Dwarvish, Druidic, Goblin, Gnoll, Gnome, Halfling.
    \item Level Adjustment: +0
\end{list}

\subsection{Goblin}
\vspace*{-8pt}
\quot{``You weren't hired to think. You were hired because you have opposable thumbs."}

Goblins are the weakest and smallest of the Goblinoid races, and that means that in society in general they get a really crap deal. But that's not really important for a Player Character, because player characters get access to classes like Rogue, Knight, and Wizard for whom being small is not a huge problem. Indeed, Goblins have a number of saving graces that in the wild barely keep them alive that when used by a player character can make them very effective. Naturally adept at stealth, Goblins are virtually made to be a Rogue or Wizard, and indeed most Goblins who have class levels are one or the other.

But the Goblins are also extremely gifted mounted combatants. And why is that? Because they are the smallest and weakest of the Goblinoids, the Worgs long ago enslaved the Goblin people. That's right, the Worgs came in and imposed their dominion upon Goblins, not the other way around. But time does funny things\ldots\ Worgs are pretty stupid, and they don't have thumbs. So while they are individually powerful, eventually they were forced to have the Goblins do all the important stuff -- like keep records and make decisions.

So now, the Worgs have gone many generations doing pretty much whatever it is that their ``servants" tell them to do. Which means that really the Goblins are totally in control. And because of this, Goblin children are practically born into the saddle. Those rich enough to afford a wolf to ride (like well, player characters) can be devastatingly effective lancers.

\listone
    \item Small Size
    \item 30' movement (despite small size).
    \item Humanoid Type (Goblinoid subtype)
    \item Darkvision
    \item +2 Dexterity, -2 Strength, -2 Charisma
    \item +4 bonus to Move Silently and Ride checks.
    \item Bonus Feat: Mounted Combat
    \item Goblins benefit from an ancient pact with the Worgs, and every Goblin receives a +2 bonus to any Bluff, Diplomacy, Handle Animal, Sense Motive, or Survival check made with respect to a Worg.
    \item Favored Classes: Rogue and Wizard
    \item Automatic Languages: Common, Goblin
    \item Bonus Languages: Draconic, Elvish, Dwarvish, Giant, Gnoll, Infernal, Orcish, Undercommon, and Worg.
\end{list}

\subsection{Hobgoblin}
\vspace*{-8pt}
\quot{``That's some tough talk from a man who wears a basket on his head."}

Hobgoblins are totally awesome at everything they do. They don't have any telling weaknesses, and their strengths are general enough that they excel at everything they put their mind to. And like humans, this can seem like they are overpowered. But the thing is, each character is made separately. While many of the armies of the world are created of a single race, each player character can be something unique and crazy. So for the Hobgoblin people there is a substantial advantage to being good at any class. But a player character probably never sees that. A Hobgoblin [anything] is a viable character, but if you want your mouth to taste like velveeta you'd make your Rogue a Deep Halfling, you'd make your Wizard a Gray Elf, and you'd make your Fighter a Dwarf.

But there's more to being a Hobgoblin than being able to ably fill any party role without overpowering the world. You get to have orange or gray skin, sharp teeth, and depending upon which version of D\&D Hobgoblin you're using -- either radically more or radically less body hair than a human. So what does that mean? It means that an influential Hobgoblin character in your campaign is going to be played by Robin Williams. But while that means that Hobgoblins can be portrayed in a humorous light, chances are that the humor is going to be more like that in The Big White or Death to Smoochy. These guys have an incredibly baroque system of laws and an interlocking system of fealties that are actually a parody of Feudal Japan.

\listone
    \item Medium Size
    \item 30' movement.
    \item Humanoid Type (Goblinoid subtype)
    \item Darkvision
    \item +2 Dexterity, +2 Constitution
    \item +4 bonus to Move Silently checks.
    \item Favored Classes: Fighter and Samurai
    \item Automatic Languages: Common, Goblin
    \item Bonus Languages: Draconic, Elvish, Dwarvish, Giant, Gnoll, Ignan, Infernal, Orcish.
\end{list}

\subsection{Orc}
\vspace*{-8pt}
\quot{``Waaarrrggghhhh!"}

Orcs get the short end of the stick. They can eat pretty much anything and they have to because their race has lost every major war since\ldots\ well, forever. Orcs are extremely specialized, and rarely see play as anything except a Barbarian. However, some players will want to diversify the concept into say\ldots\  a Rogue, Assassin, or Fighter build. That works OK, but remember that an Orc always brings ``hitting things really hard" to the party. The Orcs other limitations are pretty severe, so taking a class combination that doesn't accentuate the narrow scope of Orc advantages is probably a mistake in the long run.

\listone
    \item Medium Size
    \item 30' movement.
    \item Humanoid Type (Orc subtype)
    \item Darkvision 60'
    \item +4 Strength, -2 Intelligence, -2 Charisma, -2 Wisdom
    \item Daylight Sensitivity: While in brightly lit surroundings (such as a daylight spell), an Orc suffers the dazzled condition and is thus at a -1 penalty to attack rolls and precision-based skill checks.
    \item +2 bonus to saving throws vs. Poison and Disease.
    \item Immunity to ingested poisons.
    \item +2 to Jump and Survival checks.
    \item Favored Classes: Barbarian and Cleric
    \item Automatic Languages: Orc, Common
    \item Bonus Languages: Dwarvish, Elvish, Giant, Gnoll, Goblin, Sylvan, Undercommon.
\end{list}

\subsection{Half-Orc}
\vspace*{-8pt}
\quot{``I don't fit in anywhere, but you may be surprised to know that this dagger fits all kinds of places."}

Ah, the Half-Orc. Has any race ever gotten quite as dusty a drumstick as they? The reason that we have half-orcs at all is because they were around in Tolkien. But they didn't really do much in those books, they were just easily deluded villains who were borderline racist stereotypes and made us want to forget them altogether. But time moves on, and where once the Half-Orcs were debased and pathetic pawns of The Dark One, now we have them as a legitimate playable race. And yet, their game mechanics have never really been compatible with that.

Here's what they're supposed to be: Half-Orcs have the smarts of a human and the strength of an Orc. If people didn't hate them so much, they'd rule everything. But people do hate them so much. And here's why: Human women are, compared to Orcs, weak; Orcish women are, compared to Humans, gullible. Making Half-Orcs is easy, and since the modern Orc looks like an Orc from World of Warcraft more than a pig-man, perfectly understandable.

With all the wars that Orcs and Humans have, even periods of relative peace are rarely considered periods of friendship. So any time a Half-Orc happens, both races tend to consider it an abomination. It doesn't matter that a Half-Orc is a better leader than any of the other Orcs. It doesn't matter that the Half-Orc is tougher than any of the other Humans -- he's hated for his talents. And that makes him perversely really good at finding out things he wants to know from people. He's dealt with prejudice all his life, and knows pretty much everything you'd want to know about working around it.

\listone
    \item Medium Size
    \item 30' movement
    \item Humanoid Type (Orc and Human subtypes)
    \item Darkvision
    \item +2 Strength
    \item +2 to Intimidate, Gather Information, and Survival checks.
    \item Favored Classes: Assassin and Barbarian
    \item Automatic Languages: Orc, Common
    \item Bonus Languages: Any.
\end{list}

\subsection{Tiefling}

Tieflings are the most popular of the bad touched races, and for good reason. They are awesome. Not mechanically, they're kind of unimpressive. But they have pizzazz as characters. They have fiendish ancestry, and that makes them great villains and great tortured heroes. What it doesn't make them is particularly powerful. Tieflings aren't actually that great. Darkness appears on some class lists as a cantrip, and that's not an accident. Fundamentally, darkness just isn't a good effect.

Tieflings are honestly somewhat less powerful than Aasimar are (having as they do, some reasonably annoying penalties), but they are descended from hideous monsters from all over the planes, and they are generally speaking more fun to play.

\listone
    \item Medium Size
    \item 30' movement.
    \item Outsider Type (Native and Human subtype)
    \item Darkvision
    \item +2 Dexterity, +2 Intelligence, -2 Charisma
    \item Tieflings with a Charisma of at least 10 may cast darkness as a spell-like ability with a caster level equal to their character level once per day.
    \item +2 bonus to Bluff, Hide, and Move Silently checks.
    \item Favored Classes: Rogue and True Fiend
    \item Automatic Languages: Common
    \item Bonus Languages: Abyssal, Aquan, Auran, Celestial, Formian, Ignan, Slaad, Sylvan, Terran.
\end{list}

\section{Powerful Races}

Level adjustments don't work at all. Characters end up with skill rank maximums that prevent them from taking prestige classes appropriate to their level and they have hit dice that are low enough that they end up getting caught by spells like cloudkill that are designed to keep the henchmen out of a climactic battle, and so on and so forth. Furthermore, while the concept is busted, the implementation is even worse. Characters end up getting LAs assigned to them based on the sum total of their abilities (disregarding hit dice) and then having them added on to the hit dice (disregarding abilities). Monstrous creatures end up paying for rather minor abilities more than once and the end result is that characters who really aren't good at anything end up being counted as being higher level than ``normal" characters who can outperform them in every way. That has to stop. In general, a monster that is built like a PC is about 1 CR better than one right out of the Monster Manual. CR really is supposed to equal Level, so we're going to be running with the races which are playable under that rubric:
Powerful Monsters As PCs, or ``Beholder Mages That Don't Make Us Cry."

Monsters need to be able to be easy blends of character levels and monster stats. We know that its completely awesome to fight evil mastermind wizards that might just be beholders or giants or some other big monster, and its equally neat to play a cursed vampire warriors who's trying to redeem his soul. Designers up to this point have attempted to stop players from doing both by making these options unplayable or ``the suck", so its time to right this wrong. To start, let us be perfectly honest about two things:

\begin{enumerate}\itemspace
   \item We want to play monsters.
   \item We know that the current ECL (Effective Character Level), LA (Level Adjustment), monster PrCs, and monster progression systems don't work\ldots\  like, at all.
\end{enumerate}

Ok, now that we've cleared that up, we can begin. [I could get into elaborate explanations of why these separate systems don't work, but lets just say that the flaws are self-evident if you put a PC frost giant with only Wizard levels up against an NPC frost giant with only Wizard levels and CRed at the PC's level, or you try to play a Vampire with its +8 level adjustment and minimum character level of 5. We don't even have to talk about the Beholder Mage, an atrocity against the D\&D community in both its incarnations.]

Monsters tend to be build along four kinds of design philosophies.

\begin{itemize}\itemspace
   \item \textbf{Characters:} This is the ``as a character" philosophy, which makes monsters at a certain CR where they are perfectly suited to fight parties of characters at that level, but might overpower a weaker party or single character or be a total pushover to more powerful individuals or parties. Giants, gnolls, yuan-ti, goblinoids and other monsters who are expect to use PC-level tactics and equipment fall into this category.
   \item \textbf{Glass Jaws and Sucker Punches AKA Suckers:} These monsters, which we'll just call ``Suckers" for their ability to suck and sucker punch. Usually they have an extremely powerful attack that can sucker punch a party, but they have some glaring weakness that means that they will go down extremely quickly if you exploit this weakness. Sprites, with their fabulously low HPs and powerful magic are a fine example of this monster. ``Closet trolls" like trolls and Pouncing dire animals fall into this category because they are extremely dangerous in enclosed spaces (better than any three fighting characters of their CR), but they die easily if you can attack them at range and stay at a distance.
   \item \textbf{Puzzle Monsters:} These monsters are in fact more puzzle than monster. They usually are unbeatable unless you know their one weakness, meaning that players who don't know the right Monster Manual by heart usually die to these things. Classic examples from old editions of D\&D like the Windwalkers would only die to a single spell from the spell list which you may or may not know or have on hand, but 3.x has from eased away from this level of arbitrariness. Now we have monsters like Swarms and incorporeal monsters who may be immune to all your normal weapon attacks (a killer for a party without a damage-capable spellcaster) and several kind of plants or oozes that seem to have random and crazy defenses when you attack them (like splitting into more monsters).
   \item \textbf{Awesome Because Its Awesome AKA Player Killers (PKs):} Some Monsters are just built to make players cry. Dragons are the classic example, as they are traditionally CRed about two to four lower than they should be, and some other monsters have also been unofficially given the [awesome] subtype, meaning that players will always remember these monsters for being Party Killers. Angels, beholders, monsters with PC spellcasting, and drow typically fall into this category.
\end{itemize}


Can you see the problem with making these creatures into playable and balanced characters? Character monsters and PKs can be easily modified into playable characters by modifying raw stats, but Suckers and Puzzle Monsters need massive rewrites before they can be playable characters.

\section{Converting Monsters Into Characters}

\subsection{Method 1: The Easy Way}
Assume that a monster is a character of its CR+1(modified if it is a monster with the [Awesome] tag), and that its stat modifiers are derived from the assumption that the base monster was built using the Elite Array (highest monster stat -- highest elite stat, then repeat for next lowest, etc). For level-dependant effects like skill point maxes, feat prereqs, etc, use the monster's CR+1. Round ability stat mods down to nearest multiple of 2(negative mods up to multiple of 2), and CRs down to nearest whole number.

The nice part of this method is that it is easy, fast, and you can get to playing a monster immediately without as lot of DM intervention or paperwork. The downside is that you might get an underpowered or overpowered monster character if you are not careful (like you forgot that Dragons are actually CRed two less than they should be, or that Sprites are unplayable).

Here's two examples:

\begin{itemize}\itemspace
   \item \textbf{Minotaur:} Its Base CR is 4, and add +1 for being a PC. Its stat mods are (monster-elite array) Str 19-15=+4, Con 15-14=+0(rounded down) Dex 10-13=-2 (rounded) Wis 10-12= -2 Int 10, Cha 8-10= -2 Int 8-7=+0, for a total of +4 Str, -2 Dex, -2 Int, -2, Cha -2 Wis, which is perfectly reasonable. It's a level 5 PC with skill rank maxes of 8 and 6 monster HD.
      Frankly, it's a warrior class with a little bit of punch from natural armor, small stat mods from its size, and some fun but not good noncombat abilities. It's nothing to write home about as a 5th level character, and that's much more reasonable than the ECL 8 the MM would have you play it at.
   \item \textbf{Succubus:} CR 7, +1 for being a PC. Stat mods equal Cha 26-15=+10(rounded), Int 16-14=+2, Wis 14-13=+0(rounded), Str 13- 12=+2, Con 12- 10= +2, Dex 12-8=+4 for a +10 Cha, +2 Int, +2 Str, +2 Con, +4 Dex.
      It's an 8th level character who is almost as good as a Warlock of its level. Generally, it's a far better 8th level character than the than the ECL 14 the MM would have you pay. The fact that its abilities will never grow in power is offset by the fact that it has a high Cha, and so good DCs on its spell-likes.
\end{itemize}

\subsection{Method 2}
This method is the same as Method 1, but it goes a bit further by converting HD to actually appropriate HD by giving the monster the HD that equals its CR and BAB. This corrects problems just as excess HD from giants and undead.

Basically, look that the monster's HD and BAB. What kind of HD would it need to keep about the same BAB and HPs, but would give it the appropriate number of HD to fit its CR/level (which also fixes Saves to reasonably levels). Assign it that HD, and move on with your life.

Here's an example:
Fire Giant. Ok, the Fire Giant is a CR 11 as a PC, and notice that it has a BAB of 11, Great! Normally, it has 15 HD which leads to some craziness if he ever gets a Con boost and it has saves that are a little too big, so lets convert it. Lets give it 11 Barbarian HD(d12s, +1 BAB, good Fort save). We see that he keeps his BAB of 11, his HPs change from 142 to 133, and its base saves are Fort +7, Will/Ref +3 like an actual 11th level character instead of Fort +9, Will/Ref +5.

\subsection{Method 3}
This Method is being saved for our upcoming Tome of Tiamat. Lets just say that is the version of monster progression classes that you actually wanted to be written.

\subsection{Sample conversions}

Here are some relatively simple character conversions:

\subsubsection{Gnoll (Minimum Level 2)}

Lazy Hyena men filled with awesome? Where do I sign!?

\listone
    \item Medium Size
    \item 30' movement
    \item Humanoid Type (Gnoll subtype)
    \item Darkvision 60'
    \item +4 Strength, +2 Constitution, -2 Intelligence, -2 Charisma
    \item Proficiency in Light Armor, Shields, Simple and Martial Weapons, and the Flindbar.
    \item +1 level in the first Divine Spellcasting class a Gnoll takes.
    \item Scent.
    \item +1 Natural Armor.
    \item Favored Classes: Ranger and Druid
    \item Automatic Languages: Gnoll, Common
    \item Bonus Languages: Abyssal, Blink Dog, Giant, Goblin, Infernal, Loxo, Orc, Sphinx, Sylvan, Worg.
    \item 2 Starting Hit Dice (2d8 HP; 4 + Int Bonus x 5 skill points; +3 Fort Save; +1 BAB)
\end{list}

\subsubsection{Bugbear (Minimum Level 3)}

\listone
    \item Medium Size
    \item 30' movement
    \item Humanoid Type (Goblinoid subtype)
    \item Darkvision 60'
    \item +4 Strength, +2 Constitution, +2 Dexterity, -2 Charisma
    \item Proficiency in Light Armor, Shields, Shuriken, and all Rogue Weapons.
    \item +2 levels in the first Sneak Attack or Sudden Strike class a Bugbear takes.
    \item +3 Natural Armor.
    \item +4 Racial bonus on Move Silently checks.
    \item Favored Classes: Rogue and Ninja
    \item Automatic Languages: Goblin, Common
    \item Bonus Languages: Abyssal, Draconic, Elvish, Giant, Gnoll, Orc, Undercommon.
    \item 3 Starting Hit Dice (3d8 HP; 4 + Int Bonus x 6 skill points; +1 Fort, +3 Reflex, +1 Will; +2 BAB)
\end{list}

\subsubsection{Ogre (Minimum Level 4)}

Giants, even the lowly Ogre, are very specialized creatures. They dominate melee at their level, and really suck at everything else. As monsters, that makes them dangerous. While their glass jaws often leave them in situations that they cannot survive or even put up a decent showing, their laser-like focus can allow them to brutalize characters higher level than themselves if the lighting is just right. As characters, though, this makes them somewhat underwhelming. The ability to win super hard in one encounter only to die horribly in the next is worth less than nothing in a campaign game. An Ogre is a vulnerable and weak character for his level, but he does shine brightly if he can sucker opponents into melee. As such, Ogres really only do well in large, highly varied parties. As long as the remaining characters have potential bases covered extremely well, the fact that a single Ogre can't always pull his weight won't matter as much. For this reason, an Ogre often makes a better cohort than he does a primary character.

\listone
    \item Large Size
    \item 40' movement
    \item Giant Type
    \item Low-light vision and Darkvision (60')
    \item +6 Strength, +2 Constitution, -2 Dexterity, -2 Intelligence, -4 Charisma.
    \item +5 Natural Armor
    \item Proficiency in Light Armor, Medium Armor, Martial Weapons, and Simple Weapons.
    \item Favored Classes: Barbarian and Ranger
    \item Automatic Languages: Giant, Common
    \item Bonus Languages: Draconic, Dwarvish, Goblin, Halfling, Orc, Terran.
    \item 4 Starting Hit Dice (4d10; 4 + Int Bonus x 7 skill points; +4 Fort, +1 Reflex, +1 Will; +4 BAB)
\end{list}

\subsubsection{Frost Giant (Minimum Level 10)}

Right out of the box, the Frost Giant is a bad dude capable of rescuing the head of state from ninjas. Based largely on Norse mythology, these bad boys are big and bad. In fact, at 15 feet tall, they are about as big as you can get and still count as a large creature. That makes it pretty hard for them to find mounts, or fit into small buildings, and do all kinds of other crap that adventurers want to do. But it's not impossible. A Frost Giant isn't a Cloud Giant, he doesn't need people to make new doors to accommodate him, he just needs special doors to get through without it being really inconvenient.

A frost giant gets by in human society mostly because most people wouldn't dare mess with him. And that makes for a decent enough 10th level character.

\listone
    \item Large Size
    \item 40' movement
    \item Giant Type (Cold subtype)
    \item Low-light vision
    \item +12 Strength, +8 Constitution, +2 Wisdom
    \item +9 Natural Armor
    \item Proficiency in Light Armor, Medium Armor, Shields, Simple Weapons, and Martial Weapons.
    \item Rock Throwing and Catching (a Frost Giant's rocks have a range increment of 120 feet).
    \item Favored Classes: Fighter and Barbarian
    \item Cold Immunity and Fire Vulnerability
    \item Automatic Languages: Giant, Common
    \item Bonus Languages: Abyssal, Aquan, Auran, Draconic, Dwarvish, Gnoll, Orc.
    \item 10 Starting Hit Dice (10d10; 4 + Int Bonus x 13 skill points; +7 Fort, +3 Reflex, +3 Will; +10 BAB)
\end{list}
