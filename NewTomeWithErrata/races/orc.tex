%%%%%%%%%%%%%%%%%%%%%%%%%%%%%%%%%%%%%%%%%%%%%%%%%%
\raceentry{Orc}
%%%%%%%%%%%%%%%%%%%%%%%%%%%%%%%%%%%%%%%%%%%%%%%%%%
\vspace*{-8pt}
\quot{``Waaarrrggghhhh!"}

Orcs get the short end of the stick. They can eat pretty much anything and they have to because their race has lost every major war since\ldots\ well, forever. Orcs are extremely specialized, and rarely see play as anything except a Barbarian. However, some players will want to diversify the concept into say\ldots\  a Rogue, Assassin, or Fighter build. That works OK, but remember that an Orc always brings ``hitting things really hard" to the party. The Orcs other limitations are pretty severe, so taking a class combination that doesn't accentuate the narrow scope of Orc advantages is probably a mistake in the long run.

\begin{itemize}
\item \racesize{Medium}
\item \racetype{Humanoid (Orc)}
\item \racemovement{30ft}
\item \racevision{Darkvision 60ft}
\item +4 Strength, -2 Intelligence, -2 Charisma, -2 Wisdom
\item \textbf{Daylight Sensitivity (Ex):} Orcs are \linkcondition{Dazzled} in bright sunlight or within the radius of a \linkspell{Daylight} spell.
\item +2 racial bonus to saving throws vs. Poison and Disease.
\item Immunity to ingested poisons of all kinds. Orcs can also eat moldy and rotten items without fear, though it still tastes as bad as you'd imagine.
\item +2 to \linkskill{Jump} and \linkskill{Survival} checks.
\item \racefavoredclass{Any class with full BAB}
\item \racelang{Orc, Common}
\item \raceextralang{Dwarvish, Elvish, Giant, Gnoll, Goblin, Sylvan, Undercommon}
\end{itemize}

